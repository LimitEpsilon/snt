%! TEX program = xelatex
\documentclass{article}
\usepackage[a4paper]{geometry}
\geometry{
  top=20mm,
  bottom=20mm,
  left=20mm,
  right=20mm
}
\usepackage[nodisplayskipstretch]{setspace}
\linespread{1.25}
\usepackage{graphicx}
\usepackage{xcolor}
\usepackage{kotex}
\usepackage{csquotes}
\usepackage{tabularray}
\usepackage{relsize}
\usepackage{titlesec}

%%% Math settings
\usepackage{amssymb,amsmath,amsthm,mathtools}
\usepackage[math-style=TeX,bold-style=TeX]{unicode-math}
\theoremstyle{definition}
\newtheorem{definition}{Definition}[section]
\newtheorem{example}{Example}[section]
\newtheorem{lemma}{Lemma}[section]
\newtheorem{theorem}{Theorem}[section]
\newtheorem{corollary}{Corollary}[section]
\newtheorem{claim}{Claim}[section]

%%% Font settings
\setmainfont{Libertinus Serif}
\setsansfont{Libertinus Sans}[Scale=MatchUppercase]

\usepackage{galois}
\usepackage{ebproofx}
\ebproofxset{center=false}

\newcommand*{\defeq}{\triangleq}


\title{Soundness of Bottom-Up Type-State Analysis}
\author{Joonhyup Lee}
\begin{document}
\maketitle
\section{Syntax and Semantics}
\subsection{Abstract Syntax}
\begin{figure}[htb]
  \centering
  \begin{tabular}{rrcll}
    Identifiers       & $x,y$ & $\in$         & $\ExprVar$                                            \\
    Allocation site   & $h$   & $\in$         & $\text{AllocSite}$                                    \\
    Primitive methods & $m$   & $\in$         & $\mathbb{M}$                                          \\
    Commands          & $C$   & $\rightarrow$ & $c$                           & atomic commands       \\
                      &       & $\vbar$       & $C+C$                         & branching             \\
                      &       & $\vbar$       & $C;C$                         & sequencing            \\
                      &       & $\vbar$       & $C^*$                         & iteration             \\
    Atomic commands   & $c$   & $\rightarrow$ & $x\:\texttt{=}\:\text{new }h$ & allocation            \\
                      &       & $\vbar$       & $x\:\texttt{=}\:y$            & assignment            \\
                      &       & $\vbar$       & $x.m()$                       & primitive method call
  \end{tabular}
  \caption{Abstract syntax of the language.}
  \label{fig:syntax}
\end{figure}
\subsection{Operational Semantics}
\begin{figure}[h!]
  \centering
  \begin{tabular}{rrcll}
    Location                  & $\ell$          & $\in$ & $\mathbb{L}\triangleq\text{AllocSite}\times\mathbb{N}$       \\
    Type-state                & $\tystate$      & $\in$ & $\mathbb{T}\triangleq\{\text{init, error, opened, closed}\}$ \\
    Interpretation of methods & $\underline{m}$ & $\in$ & $\mathbb{T}\rightarrow\mathbb{T}$                            \\
    Environment               & $\ctx$          & $\in$ & $\Ctx\triangleq\fin{\ExprVar}{\mathbb{L}}$                   \\
    Heap                      & $\mem$          & $\in$ & $\Mem\triangleq\fin{\mathbb{L}}{\mathbb{T}}$
  \end{tabular}
  \caption{Definition of the semantic domains.}
  \label{fig:domain}
\end{figure}
\begin{figure}[h!]
  \centering
  \footnotesize
  \begin{flushright}
    \fbox{$\ctx,\mem\vdash C\Rightarrow\ctx,\mem$}
  \end{flushright}
  \[
    \begin{InfRule}{Br-L}
      \hypo{\ctx,\mem\vdash C_1\Rightarrow\ctx',\mem'}
      \infer1{\ctx,\mem\vdash C_1+C_2\Rightarrow\ctx',\mem'}
    \end{InfRule}\quad
    \begin{InfRule}{Br-R}
      \hypo{\ctx,\mem\vdash C_2\Rightarrow\ctx',\mem'}
      \infer1{\ctx,\mem\vdash C_1+C_2\Rightarrow\ctx',\mem'}
    \end{InfRule}\quad
    \begin{InfRule}{Seq}
      \hypo{\ctx,\mem\vdash C_1\Rightarrow\ctx_1,\mem_1}
      \hypo{\ctx_1,\mem_1\vdash C_2\Rightarrow\ctx',\mem'}
      \infer2{\ctx,\mem\vdash C_1;C_2\Rightarrow\ctx',\mem'}
    \end{InfRule}
  \]

  \[
    \begin{InfRule}{Iter-0}
      \infer0{\ctx,\mem\vdash C^*\Rightarrow\ctx,\mem}
    \end{InfRule}\quad
    \begin{InfRule}{Iter-*}
      \hypo{\ctx,\mem\vdash C^*\Rightarrow\ctx_1,\mem_1}
      \hypo{\ctx_1,\mem_1\vdash C\Rightarrow\ctx',\mem'}
      \infer2{\ctx,\mem\vdash C^*\Rightarrow\ctx',\mem'}
    \end{InfRule}
  \]

  \[
    \begin{InfRule}{Alloc}
      \hypo{(h,n)=\ell}
      \hypo{\ell\not\in\dom(\mem)}
      \infer2{\ctx,\mem\vdash x\:\texttt{=}\:\text{new }h\Rightarrow\ctx[x\mapsto\ell],\mem[\ell\mapsto\text{init}]}
    \end{InfRule}\quad
    \begin{InfRule}{Assign}
      \hypo{\ctx(y)=\ell}
      \infer1{\ctx,\mem\vdash x\:\texttt{=}\:y\Rightarrow\ctx[x\mapsto\ell],\mem}
    \end{InfRule}\quad
    \begin{InfRule}{Method}
      \hypo{\ctx(x)=\ell}
      \hypo{\mem(\ell)=t}
      \infer2{\ctx,\mem\vdash v.m()\Rightarrow\ctx,\mem[\ell\mapsto\underline{m}(t)]}
    \end{InfRule}
  \]
  \caption{Operational semantics of the language.}
  \label{fig:semantics}
\end{figure}

\begin{figure}
  \centering
  \footnotesize
  \[
    \begin{InfRule}{Alloc-E}
      \hypo{(h,n)=\ell}
      \hypo{\ell\not\in\dom([\mem])}
      \infer2{[\ctx],[\mem],\Cstr\vdash x\:\texttt{=}\:\text{new }h\Rightarrow(x,\ell)\cons[\ctx],(\ell,\text{init})\cons[\mem],\Cstr}
    \end{InfRule}\quad
    \begin{InfRule}{Assign-E}
      \hypo{[\ctx](y)=\ell}
      \infer1{[\ctx],[\mem],\Cstr\vdash x\:\texttt{=}\:y\Rightarrow(x,\ell)\cons[\ctx],[\mem],\Cstr}
    \end{InfRule}
  \]

  \[
    \begin{InfRule}{Method1-E}
      \hypo{[\ctx](x)=[y]}
      \hypo{(t,\Cstr')\in[\mem]([y],\Cstr)}
      \infer2{[\ctx],[\mem],\Cstr\vdash v.m()\Rightarrow[\ctx],([y],\underline{m}(t))\cons[\mem],\Cstr'}
    \end{InfRule}\quad
    \begin{InfRule}{Method2-E}
      \hypo{[\ctx](x)=(h,n)}
      \hypo{[\mem](h,n)=t}
      \infer2{[\ctx],[\mem],\Cstr\vdash v.m()\Rightarrow[\ctx],((h,n),\underline{m}(t))\cons[\mem],\Cstr}
    \end{InfRule}
  \]

  \caption{Semantics for atomic commands, with read events.}
  \label{fig:eventsemantics}
\end{figure}

Definition for reading from the heap under the set of constraints $\Cstr$:
\begin{align*}
  (([y],t)\cons[\mem])([x],\Cstr)    & \triangleq\{(t,\{[x]\A{=}[y]\}\cup\Cstr)\}\cup[\mem]([y],\{[x]\A{\neq}[y]\}\cup\Cstr) \\
  (((h,n),\_)\cons[\mem])([x],\Cstr) & \triangleq[\mem]([x],\Cstr)                                                           \\
  ([])([x],\Cstr)                    & \triangleq\{(t,\{[[x]]\A{=}t\}\cup\Cstr)|t\in\mathbb{T}\}
\end{align*}
\end{document}
%%% Local Variables: 
%%% coding: utf-8
%%% mode: latex
%%% TeX-engine: xetex
%%% End:
