%! TEX program = xelatex
\documentclass{article}
\usepackage{geometry}
\geometry{
  a4paper,
  total={170mm,257mm},
  left=20mm,
  top=20mm,
}
\usepackage{setspace}
\usepackage{graphicx}
\usepackage{xcolor}
\usepackage{kotex}
\usepackage{csquotes}
\usepackage{tabularray}
\usepackage{relsize}

%%% Typesetting for listings
\usepackage{listings}

\definecolor{dkgreen}{rgb}{0,0.6,0}
\definecolor{ltblue}{rgb}{0,0.4,0.4}
\definecolor{dkviolet}{rgb}{0.3,0,0.5}

%%% lstlisting coq style (inspired from a file of Assia Mahboubi)
\lstdefinelanguage{Coq}{ 
    % Anything betweeen $ becomes LaTeX math mode
    mathescape=true,
    % Comments may or not include Latex commands
    texcl=false, 
    % Vernacular commands
    morekeywords=[1]{Section, Module, End, Require, Import, Export,
        Variable, Variables, Parameter, Parameters, Axiom, Hypothesis,
        Hypotheses, Notation, Local, Tactic, Reserved, Scope, Open, Close,
        Bind, Delimit, Definition, Let, Ltac, Fixpoint, CoFixpoint, Add,
        Morphism, Relation, Implicit, Arguments, Unset, Contextual,
        Strict, Prenex, Implicits, Inductive, CoInductive, Record,
        Structure, Canonical, Coercion, Context, Class, Global, Instance,
        Program, Infix, Theorem, Lemma, Corollary, Proposition, Fact,
        Remark, Example, Proof, Goal, Save, Qed, Defined, Hint, Resolve,
        Rewrite, View, Search, Show, Print, Printing, All, Eval, Check,
        Projections, inside, outside, Def},
    % Gallina
    morekeywords=[2]{forall, exists, exists2, fun, fix, cofix, struct,
        match, with, end, as, in, return, let, if, is, then, else, for, of,
        nosimpl, when},
    % Sorts
    morekeywords=[3]{Type, Prop, Set, true, false, option},
    % Various tactics, some are std Coq subsumed by ssr, for the manual purpose
    morekeywords=[4]{pose, set, move, case, elim, apply, clear, hnf,
        intro, intros, generalize, rename, pattern, after, destruct,
        induction, using, refine, inversion, injection, rewrite, congr,
        unlock, compute, ring, field, fourier, replace, fold, unfold,
        change, cutrewrite, simpl, have, suff, wlog, suffices, without,
        loss, nat_norm, assert, cut, trivial, revert, bool_congr, nat_congr,
        symmetry, transitivity, auto, split, left, right, autorewrite},
    % Terminators
    morekeywords=[5]{by, done, exact, reflexivity, tauto, romega, omega,
        assumption, solve, contradiction, discriminate},
    % Control
    morekeywords=[6]{do, last, first, try, idtac, repeat},
    % Comments delimiters, we do turn this off for the manual
    morecomment=[s]{(*}{*)},
    % Spaces are not displayed as a special character
    showstringspaces=false,
    % String delimiters
    morestring=[b]",
    morestring=[d],
    % Size of tabulations
    tabsize=3,
    % Enables ASCII chars 128 to 255
    extendedchars=false,
    % Case sensitivity
    sensitive=true,
    % Automatic breaking of long lines
    breaklines=false,
    % Default style fors listings
    basicstyle=\small,
    % Position of captions is bottom
    captionpos=b,
    % flexible columns
    columns=[l]flexible,
    % Style for (listings') identifiers
    identifierstyle={\ttfamily\color{black}},
    % Style for declaration keywords
    keywordstyle=[1]{\ttfamily\color{dkviolet}},
    % Style for gallina keywords
    keywordstyle=[2]{\ttfamily\color{dkgreen}},
    % Style for sorts keywords
    keywordstyle=[3]{\ttfamily\color{ltblue}},
    % Style for tactics keywords
    keywordstyle=[4]{\ttfamily\color{dkblue}},
    % Style for terminators keywords
    keywordstyle=[5]{\ttfamily\color{dkred}},
    %Style for iterators
    %keywordstyle=[6]{\ttfamily\color{dkpink}},
    % Style for strings
    stringstyle=\ttfamily,
    % Style for comments
    commentstyle={\ttfamily\color{dkgreen}},
    %moredelim=**[is][\ttfamily\color{red}]{/&}{&/},
    literate=
    {\\forall}{{\color{dkgreen}{$\forall\;$}}}1
    {\\exists}{{$\exists\;$}}1
    {\ge -}{{$\geftarrow\;$}}1
    {=>}{{$\Rightarrow\;$}}1
    {==}{{\code{==}\;}}1
    {==>}{{\code{==>}\;}}1
    %    {:>}{{\code{:>}\;}}1
    {->}{{$\rightarrow\;$}}1
    {\ge ->}{{$\geftrightarrow\;$}}1
    {\ge ==}{{$\geq\;$}}1
    {\#}{{$^\star$}}1 
    {\\o}{{$\circ\;$}}1 
    {\@}{{$\cdot$}}1 
    {\/\\}{{$\text{ and }\;$}}1
    {\\\/}{{$\vee\;$}}1
    {++}{{\code{++}}}1
    {~}{{\ }}1
    {\@\@}{{$@$}}1
    {\\mapsto}{{$\mapsto\;$}}1
    {\\hline}{{\rule{\linewidth}{0.5pt}}}1
    %
}[keywords,comments,strings]

%%% Math settings
\usepackage{amssymb,amsmath,amsthm,mathtools}
\usepackage[math-style=TeX,bold-style=TeX]{unicode-math}
\theoremstyle{definition}
\newtheorem{definition}{Definition}[section]
\newtheorem{example}{Example}[section]
\newtheorem{lem}{Lemma}[section]
\newtheorem{thm}{Theorem}[section]
\newtheorem{cor}{Corollary}[section]
\newtheorem{clm}{Claim}[section]

%%% Font settings
%\setmainfont{Libertinus Serif}
%\setsansfont{Libertinus Sans}[Scale=MatchUppercase]
%\setmonofont{JuliaMono}[Scale=MatchLowercase]
%\setmainhangulfont{Noto Serif CJK KR}
%\setmonohangulfont{D2Coding}

%%% PL constructs
\usepackage{galois}
\usepackage{ebproof}
%% InfRule command
\ExplSyntaxOn
\NewTblrEnviron{@ruleenv}
\SetTblrInner[@ruleenv]{belowsep=0pt,stretch=0}
\SetTblrOuter[@ruleenv]{baseline=b}
\NewDocumentEnvironment { InfRule } { m +b }
  {
    \begin{@ruleenv}{l}
      \smaller{\textsc{#1}} \\
      \begin{prooftree} #2 \end{prooftree}
    \end{@ruleenv}
  }
  {}
\ExplSyntaxOff

%% Modification for \infer
\ExplSyntaxOn
\int_new:N \g__ebproof_sublevel_int
\box_new:N \g__ebproof_substack_box
\seq_new:N \g__ebproof_substack_seq

\cs_new:Nn \__ebproof_clear_substack:
  {
    \int_gset:Nn \g__ebproof_sublevel_int { 0 }
    \hbox_gset:Nn \g__ebproof_substack_box { }
    \seq_gclear:N \g__ebproof_substack_seq
  }

\cs_new:Nn \__ebproof_subpush:N
  {
    \int_gincr:N \g__ebproof_sublevel_int
    \hbox_gset:Nn \g__ebproof_substack_box
      { \hbox_unpack:N \g__ebproof_substack_box \box_use:c { \__ebproof_box:N #1 } }
    \seq_gput_left:Nv \g__ebproof_substack_seq
      { \__ebproof_marks:N #1 }
  }

\cs_new:Nn \__ebproof_subpop:N
  {
    \int_compare:nNnTF { \g__ebproof_sublevel_int } > { 0 }
      {
        \int_gdecr:N \g__ebproof_sublevel_int
        \hbox_gset:Nn \g__ebproof_substack_box {
          \hbox_unpack:N \g__ebproof_substack_box
          \box_gset_to_last:N \g_tmpa_box
        }
        \box_set_eq_drop:cN { \__ebproof_box:N #1 } \g_tmpa_box
        \seq_gpop_left:NN \g__ebproof_substack_seq \l_tmpa_tl
        \tl_set_eq:cN { \__ebproof_marks:N #1 } \l_tmpa_tl
      }
      { \PackageError{ebproof}{Missing~premiss~in~a~proof~tree}{} \__ebproof_clear:N #1 }
  }

\cs_new:Nn \__ebproof_append_subvertical:NN
  {
    \bool_if:NTF \l__ebproof_updown_bool
      { \__ebproof_append_above:NN #1 #2 }
      { \__ebproof_append_below:NN #1 #2 }
  }

\cs_new:Nn \__ebproof_join_subvertical:n
  {
    \group_begin:
    \__ebproof_subpop:N \l__ebproof_a_box
    \prg_replicate:nn { #1 - 1 }
      {
        \__ebproof_subpop:N \l__ebproof_b_box
        \__ebproof_enlarge_conclusion:NN \l__ebproof_b_box \l__ebproof_a_box

        \__ebproof_make_vertical:Nnnn \l__ebproof_c_box
          { \__ebproof_mark:Nn \l__ebproof_b_box {axis} - \__ebproof_mark:Nn \l__ebproof_b_box {left} }
          { \__ebproof_mark:Nn \l__ebproof_b_box {right} - \__ebproof_mark:Nn \l__ebproof_b_box {left} }
          { \skip_vertical:N \l__ebproof_rule_margin_dim }
        \__ebproof_vcenter:N \l__ebproof_b_box
        \__ebproof_append_subvertical:NN \l__ebproof_a_box \l__ebproof_c_box

        \__ebproof_append_subvertical:NN \l__ebproof_a_box \l__ebproof_b_box
      }
    \__ebproof_push:N \l__ebproof_a_box
    \group_end:
  }

\cs_new:Nn \__ebproof_renew_statement:nnn
  {
    \exp_args:Nc \RenewDocumentCommand { ebproof#1 }{ #2 } { #3 }
    \seq_gput_right:Nn \g__ebproof_statements_seq { #1 }
  }
\__ebproof_renew_statement:nnn { infer } { O{} m O{} m }
  {
    \group_begin:
    \__ebproof_restore_statements:
    \keys_set_known:nnN { ebproof / rule~style } { #1 } \l_tmpa_tl
    \keys_set:nV { ebproof } \l_tmpa_tl
    \tl_set:Nn \l__ebproof_right_label_tl { #3 }

    \__ebproof_clear_substack:
    \clist_map_inline:nn { #2 }
      {
        \__ebproof_join_horizontal:n { ##1 }
        \__ebproof_pop:N \l__ebproof_a_box
        \__ebproof_subpush:N \l__ebproof_a_box
      }
    \__ebproof_join_subvertical:n { \clist_count:n { #2 } }

    \__ebproof_push_statement:n { #4 }
    \__ebproof_join_vertical:
    \group_end:
  }
\ExplSyntaxOff
\ebproofset{left label template=\textsc{[\inserttext]}}
\ebproofset{center=false}

%%% Custom commands
\newcommand*{\vbar}{|}
\newcommand*{\finto}{\xrightarrow{\text{\textrm{fin}}}}
\newcommand*{\istype}{\mathrel{⩴}}
\newcommand*{\ortype}{\mathrel{|}}
\newcommand*{\cons}{::}
\newcommand*{\pset}{\mathscr{P}}
\newcommand*{\closeloc}[2]{\prescript{\text{\textbackslash}{#1}}{}{#2}}
\newcommand*{\openloc}[2]{{#1}^{#2}}

\def\ovbarw{1.2mu}
\def\ovbarh{1}
\newcommand*{\ovbar}[1]{\mkern \ovbarw\overline{\mkern-\ovbarw{\smash{#1}\scalebox{1}[\ovbarh]{\vphantom{i}}}\mkern-\ovbarw}\mkern \ovbarw}
\newcommand*{\A}[1]{\overset{\,_{\mbox{\Large .}}}{#1}}
\newcommand*{\Abs}[1]{{#1}^{\#}}
\newcommand*{\Expr}{\text{Expr}}
\newcommand*{\ExprVar}{\text{Var}}
\newcommand*{\Module}{\text{Module}}
\newcommand*{\ModVar}{\text{ModVar}}
\newcommand*{\modid}{d}
\newcommand*{\Time}{\mathbb{T}}
\newcommand*{\ATime}{\A{\Time}}
\newcommand*{\Ctx}{\text{Env}}
\newcommand*{\ctx}{\sigma}
\newcommand*{\Value}{\text{Val}}
\newcommand*{\Walue}{\text{WVal}}
\newcommand*{\Mem}{\text{Mem}}
\newcommand*{\Loc}{\text{Loc}}
\newcommand*{\FLoc}{\text{FLoc}}
\newcommand*{\Left}{\text{Left}}
\newcommand*{\Right}{\text{Right}}
\newcommand*{\Sig}{\text{Sig}}
\newcommand*{\mem}{m}
\newcommand*{\dom}{\text{dom}}
\newcommand*{\AMem}{\A{\text{Mem}}}
\newcommand*{\State}{\text{State}}
\newcommand*{\AState}{\A{\text{State}}}
\newcommand*{\Result}{\text{Result}}
\newcommand*{\AResult}{\A{\text{Result}}}
\newcommand*{\Tick}{\text{Tick}}
\newcommand*{\lfp}{\mathsf{lfp}}
\newcommand*{\Step}{\mathsf{Step}}
\newcommand*{\semarrow}{\Downarrow}
\newcommand*{\asemarrow}{\A{\rightsquigarrow}}
\newcommand*{\synlink}{\rtimes}
\newcommand*{\semlink}{\mathbin{\rotatebox[origin=c]{180}{$\propto$}}}
\newcommand*{\link}[2]{{#1}\rtimes{#2}}
\newcommand*{\mt}{\mathsf{empty}}
\newcommand*{\valid}{\mathsf{valid}}
\newcommand*{\Path}{\text{Path}}
\newcommand*{\equivalent}{\sim}

\newcommand*{\doubleplus}{\ensuremath{\mathbin{+\mkern-3mu+}}}
\newcommand*{\project}{\text{\texttt{:>} }}
\newcommand*{\Exp}{\mathsf{Exp}}
\newcommand*{\Imp}{\mathsf{Imp}}
\newcommand*{\Fin}{\mathsf{Fin}}
\newcommand*{\Link}{\mathsf{Link}}
\newcommand*{\sembracket}[1]{\lBrack{#1}\rBrack}
\newcommand*{\fin}[2]{{#1}\xrightarrow{\text{fin}}{#2}}
\newcommand*{\addr}{\mathsf{addr}}
\newcommand*{\tick}{\mathsf{tick}}
\newcommand*{\modctx}{\mathsf{ctx}}
\newcommand*{\mapinject}[2]{{#2}[{#1}]}
\newcommand*{\inject}[2]{{#2}\langle{#1}\rangle}
\newcommand*{\deletepre}[2]{{#2}\overline{\doubleplus}{#1}}
\newcommand*{\deletemap}[2]{{#1}\overline{[{#2}]}}
\newcommand*{\delete}[2]{{#2}{\langle{#1}\rangle}^{-1}}
\newcommand*{\filter}{\mathsf{filter}}
\newcommand*{\Lete}{\mathtt{val}}
\newcommand*{\Letm}{\mathtt{mod}}

\newcommand*{\ValRel}[1]{\mathcal{V}\sembracket{#1}}
\newcommand*{\ExprRel}[1]{\mathcal{E}\sembracket{#1}}
\newcommand*{\CtxRel}[1]{\mathcal{C}\sembracket{#1}}
\newcommand*{\ModRel}[1]{\mathcal{M}\sembracket{#1}}
\newcommand*{\TyEnv}{\text{TyEnv}}
\newcommand*{\TyVar}{\text{TyVar}}
\newcommand*{\Type}{\text{Type}}
\newcommand*{\Subst}{\text{Subst}}
\newcommand*{\external}{\Gamma_{\text{ext}}}

\newcommand*{\InitE}{\textsf{Init}}
\newcommand*{\ReadE}{\textsf{Read}}
\newcommand*{\CallE}{\textsf{Call}}
\newcommand*{\resolve}{\downarrow_{\ctx_0}}


\title{Modular Analysis}
\author{Joonhyup Lee}
\begin{document}
\maketitle
\begin{abstract}
  We present a framework of modular static analysis that analyzes open
  program fragments in advance and complete the whole analysis later
  when the fragments are closed. The framework is defined for a
  call-by-value lambda calculus extended with constructs for defining and
  linking first-class modules (a collection of bindings) that support
  recursive bindings for values, modules, and functors(module
  functions). Thanks to the abstract interpretation framework, our
  modular analysis framework's key is how to define the semantics,
  called ``shadow'' semantics of open program fragments so that the
  computation with free variables should be captured in advance
  and be completed later at link-time. A modular
  static analysis is abstractions of this shadow semantics and of the
  linking operation. The safety of the framework is proven in Coq.
  Two instances of the framwork are presented: value analyses for an
  applicative language(control-flow, closure analysis) and for
  an imperative language.
  %  The framework is intended to be a guide in
  %  designing modular analyses in practice for the specific source language at hand. 
\end{abstract}

\section{Problem}
By modular analysis we mean a static analysis technology that analyzes
programs in the compositional way. Modular analysis partially analyzes
program fragments separately and then complete the analysis when all
the fragments are available and linked.

Designing a modular analysis decides on two factors: how to build
sound partial semantic summaries of program fragments and how to
complete them at link time. A particular modular analysis strikes a
balance between the two factors. Depending on how much analysis work
is invested on each factor, modular analyses range from one extreme, a
trivial one (wait until the whole code is available and do the
whole-program analysis) to another extreme, non-trivial yet
hardly-automatic ones (abduct assumptions about free variables and
check them at link-time).

However, designing a modular analysis has been elusive. There is a
lack of a general framework by which we can design sound modular
analyses with varying balance and accuracy of our choice.
We need a general framework by which static analysis designers can
design non-trivial sound modular analyses.

\section{Framework Sketch}
Thanks to the abstract interpretation framework, the key in our
modular analysis framework is how to define the semantics (we call it
``shadow'' semantics) of open program fragments so that the
computation with free variables should be captured in advance
and be completed later at link-time. A modular static analysis is
then nothing but abstractions of this shadow semantics and of the
linking operation.

\subsection{Examples of Shadow Semantics and its Abstraction}
Shadow semantics is the semantics for the computation involving
free variables. We give examples of shadows for program fragments
written in OCaml.

\paragraph{Example 1: Mapping an Unknown Function Over a Known List}
\begin{center}
  \begin{BVerbatim}[commandchars=\\\{\}]
let rec map = fun f -> fun l ->
  match \textsuperscript{0}l with
  | [] -> []
  | hd :: tl -> \textsuperscript{1}(\textsuperscript{3}f \textsuperscript{4}hd) :: \textsuperscript{2}(map f tl)
in map \textsuperscript{5}g \textsuperscript{6}(1 :: 2 :: 3 :: [])
  \end{BVerbatim}
\end{center}
The shadow semantics of the above fragment is
\[
  \{\CallE(\text{G},1)\cons\CallE(\text{G},2)\cons\CallE(\text{G},3)\cons[]\}
  \text{ where }
  \text{G}=\ReadE(\InitE,\texttt{g})
\]
Shadows such as $\CallE$, $\ReadE$ correspond to semantic operations,
like function application and reading from the environment.
The $\InitE$ shadow is the unknown initial environment that the fragment will execute under.

Computing a sound and finite approximation of the shadow
semantics of program fragments corresponds to building a sound partial
semantic summaries in advance. For the above example,
we may use an abstraction that stores summaries of each input \emph{environment}
and output \emph{value} per program point.

\vspace{\topsep}
\noindent\textbf{Notation.}
We write $\ppoint_i$ for each program point $i$.
$\ppoint_i.\text{in}$ denotes the input to $\ppoint_i$, and $\ppoint_i.\text{out}$
denotes the output from $\ppoint_i$.
For brevity, we omit the selector in $\ppoint_i.\text{out}$ and write $\ppoint_i$.
We write $[a,b]_{\text{itv}}$ to denote the interval from integer $a$ to $b$.

\vspace{\topsep}
Now, the abstract shadow
\[\{\ppoint_1\cons \ppoint_2\}\]
is a sound approximation of the concrete shadow. To illustrate why,
here are abstract shadows from some program points involved in the map computation.
\begin{align*}
  \ppoint_1.\text{out}:\{ & \Abs\CallE(\ppoint_3,\ppoint_4)\}            &  &  & \ppoint_4.\text{out}: & \;[1,3]_{\text{itv}}                                                                 \\
  \ppoint_2.\text{out}:\{ & [],\ppoint_1\cons \ppoint_2\}                &  &  & \ppoint_5.\text{in}:  & \;\{\Abs\InitE,\texttt{map}\mapsto\{\langle\lambda\texttt{f}.\texttt{...}\rangle\}\} \\
  \ppoint_3.\text{out}:\{ & \Abs\ReadE(\ppoint_5.\text{in},\texttt{g})\}
\end{align*}
Starting from $\ppoint_1\cons\ppoint_2$, we can reach the concrete shadow by
expanding each program point according to the above table.

\paragraph{Example 2: Mapping an Unknown Function Over an Unknown List}
\begin{center}
  \begin{BVerbatim}[commandchars=\\\{\}]
let rec map = ... in map \textsuperscript{5}g \textsuperscript{6}lst
  \end{BVerbatim}
\end{center}
A more interesting case is when even the list is replaced with a free variable.
The shadow semantics of the above fragment is
\begin{center}
  \begin{tabular}{c}
    $\{[],\CallE(\text{G},\Hd(\text{Lst}))\cons[],\CallE(\text{G},\Hd(\Tl(\text{Lst})))\cons\CallE(\text{G},\Hd(\text{Lst}))\cons[],$ \\
    [4pt]
    $\dots,\CallE(\text{G},\Hd(\Tl^{n}(\text{Lst})))\cons\cdots\cons[],\dots\}$
  \end{tabular}
\end{center}
where
\begin{align*}
  \text{G} & =\ReadE(\InitE,\texttt{g})       &  &  & \text{Lst} & =\ReadE(\InitE,\texttt{lst})     \\
  \Hd(l)   & =\textsf{Match}(l,\texttt{::},0) &  &  & \Tl(l)     & =\textsf{Match}(l,\texttt{::},1)
\end{align*}

The same abstraction as in Example 1 results in the same abstract shadow
\[\{\ppoint_1\cons\ppoint_2\}\]
but with different shadows for each program point.
\begin{align*}
  \ppoint_0.\text{out} :\{ & \Abs\ReadE(\ppoint_6.\text{in},\texttt{lst}),   &  &  & \ppoint_3.\text{out}:\{ & \Abs\ReadE(\ppoint_5.\text{in},\texttt{g})\}                                     \\
                           & \Abs{\textsf{Match}}(\ppoint_0,\texttt{::},1)\} &  &  & \ppoint_4.\text{out}:\{ & \Abs{\textsf{Match}}(\ppoint_0,\texttt{::},0)\}                                  \\
  \ppoint_1.\text{out} :\{ & \Abs\CallE(\ppoint_3,\ppoint_4)\}               &  &  & \ppoint_5.\text{in} :\{ & \Abs\InitE,\texttt{map}\mapsto\{\langle\lambda\texttt{f}.\texttt{...}\rangle\}\} \\
  \ppoint_2.\text{out} :\{ & [],\ppoint_1\cons \ppoint_2\}                   &  &  & \ppoint_6.\text{in} :\{ & \Abs\InitE,\texttt{map}\mapsto\{\langle\lambda\texttt{f}.\texttt{...}\rangle\}\}
\end{align*}

\paragraph{Example 3: An Imperative Swap}
\begin{center}
  \begin{BVerbatim}[commandchars=\\\{\}]
let t = !x in\textsuperscript{1} x := !y;\textsuperscript{2} y := temp\textsuperscript{3}
  \end{BVerbatim}
\end{center}
The shadow semantics can be extended to support imperative features.
We simply have to add shadows that correspond to \emph{dereferencing}, a semantic
operation central to imperative programs.

For this example, the environment and memory at each program point is
\[
  \setlength{\arraycolsep}{2pt}
  \begin{array}{c|c|c}
              & \text{Environment}                                  & \text{Memory}                                                                                                                                               \\
    \hline
    \ppoint_1 & \langle\{\texttt{t}\mapsto*\text{X}\},\InitE\rangle & \langle\{\{\text{X}\}\mapsto*\text{X}\},\InitE\rangle                                                                                                       \\
    \ppoint_2 & \langle\{\texttt{t}\mapsto*\text{X}\},\InitE\rangle & \langle\{\{\text{X},\text{Y}\}\mapsto*\text{X}\},\InitE\rangle\text{ or }\langle\{\{\text{X}\}\mapsto*\text{Y},\{\text{Y}\}\mapsto*\text{Y}\},\InitE\rangle \\
    \ppoint_3 & \langle\{\texttt{t}\mapsto*\text{X}\},\InitE\rangle & \langle\{\{\text{X},\text{Y}\}\mapsto*\text{X}\},\InitE\rangle\text{ or }\langle\{\{\text{X}\}\mapsto*\text{Y},\{\text{Y}\}\mapsto*\text{X}\},\InitE\rangle
  \end{array}
\]
where
\begin{align*}
  \text{X}=\ReadE(\InitE,\texttt{x})\quad\text{Y}=\ReadE(\InitE,\texttt{y})\quad*\!S=\DerefSE(\InitE,S)\text{ for shadow }S
\end{align*}

The memory is represented by a tuple holding the bindings for each concrete address
and the most recent shadow that affected the memory.
The concrete address consists of addresses that are allocated during the execution
of the program, and the \emph{alias set} of shadows.
The alias set is updated each time there is a read or a write to a shadow address,
and the memory retains the invariant that the alias sets must be mutually disjoint.
In the above example, the alias set is updated when the memory is dereferenced with
X and Y. The resulting memory depends on aliasing, as expected.

The ``most recent shadow'' must be recorded due to unresolved function calls.
For example, if there is a call to a \emph{shadow} G with argument $v$ under memory
\[m=\langle\{(a_1,\{S_1,S_2\})\mapsto v_1,\{S_3\}\mapsto v_2\},S_4\rangle\]
the memory will be updated to
\[\langle\{(a_1,\{S_1,S_2\})\mapsto\DerefAE(\text{C},a_1),\dots,\{S_3\}\mapsto\DerefSE(\text{C},S_3)\},\text{C}\rangle\]
where
\[\text{C}=\CallE(\text{G},v|m)\quad(m\text{ records that the call happened under memory }m)\]

For efficient abstraction of the shadow semantics, one may employ widening.
For instance, each shadow in the alias sets may be kept in its concrete form
up until some threshold. After that, the memory may be collapsed to prevent blowup.

\subsection{Examples of Linking and its Abstraction}
The shadow semantics become actual when the involved free variables
are known at link time. The linking semantics, the semantics of the link
operation, defines this actualization operation. For the above fragments, let's
consider that a closing fragment is available.

\paragraph{Example 4: Module Exporting a Known Function}
\begin{center}
  \begin{BVerbatim}[commandchars=\\\{\}]
let g = \textsuperscript{7}fun x -> x + 1
  \end{BVerbatim}
\end{center}
This is a module which returns the environment
$\ctx=\{\texttt{g}\mapsto\langle\lambda\texttt{x}.\texttt{x + 1},[]\rangle\}$.
Linking this with the concrete shadow in Example 1 gives:
\[\{2\cons3\cons4\cons[]\}\]

Computing a sound and finite approximation of the linking semantics
corresponds to completing partial analysis summaries at link time when
the analysis results for the involved free variable are available. For
the above example,
\[
  \Abs\ctx=\{\texttt{g}\mapsto\{\langle\lambda\texttt{x}.\texttt{x + 1},\ppoint_7.\text{in}\rangle\}\}\text{ where }
  \ppoint_7.\text{in}:[]
\]
is the abstract shadow.

Applying a sound abstract version of the linking operator will result in
\begin{align*}
  \ppoint_1.\text{out} & :[2,4]_{\text{itv}}                                                     &  &  & \ppoint_4.\text{out} & :[1,3]_{\text{itv}} \\
  \ppoint_2.\text{out} & :\{[],\ppoint_1\cons \ppoint_2\}                                        &  &  & \ppoint_5.\text{in}  & :\Abs\ctx           \\
  \ppoint_3.\text{out} & :\{\langle\lambda\texttt{x}.\texttt{x + 1},\ppoint_7.\text{in}\rangle\} &  &  &
\end{align*}

\paragraph{Example 5: Module Exporting a Foreign Function}
\begin{center}
  \begin{BVerbatim}
external g : int -> int = "incr"
  \end{BVerbatim}
\end{center}
The function \texttt{g} might also be a foreign function.
The return value of this module is
$\ctx=\{\texttt{g}\mapsto\textsf{Prim}(\texttt{incr})\}$,
where \textsf{Prim} stands for a \emph{primitive} value.
Linking this with the concrete shadow in Example 1 gives:
\[\{\textsf{PrimCall}(\texttt{incr},1)\cons\textsf{PrimCall}(\texttt{incr},2)\cons\textsf{PrimCall}(\texttt{incr},3)\cons[]\}\]

The sound abstraction for $\ctx$ is
\[
  \Abs\ctx=\{\texttt{g}\mapsto\{\textsf{Prim}(\texttt{incr})\}\}
\]
Applying a sound abstract version of the linking operator will result in
\begin{align*}
  \ppoint_1.\text{out} & :\{\Abs{\textsf{PrimCall}}(\texttt{incr},\ppoint_4)\} &  &  & \ppoint_4.\text{out} & :[1,3]_{\text{itv}} \\
  \ppoint_2.\text{out} & :\{[],\ppoint_1\cons \ppoint_2\}                      &  &  & \ppoint_5.\text{in}  & :\Abs\ctx           \\
  \ppoint_3.\text{out} & :\{\textsf{Prim}(\texttt{incr})\}
\end{align*}

\paragraph{Example 6: Partial Resolution}
\begin{center}
  \begin{BVerbatim}[commandchars=\\\{\}]
let lst = \textsuperscript{7}1 :: \textsuperscript{8}(\textsuperscript{9}2 :: \textsuperscript{10}(\textsuperscript{11}3 :: \textsuperscript{12}[]))
  \end{BVerbatim}
\end{center}
The return value of this module is
$\ctx=\langle\{\texttt{lst}\mapsto 1\cons2\cons3\cons[]\},\InitE\rangle$.
Linking this with the concrete shadow in Example 2 gives:
\[
  \{[],\CallE(\text{G},1)\cons[],\CallE(\text{G},1)\cons\CallE(\text{G},2)\cons[],
  \CallE(\text{G},1)\cons\CallE(\text{G},2)\cons\CallE(\text{G},3)\cons[]\}
\]
where
\[
  \text{G}=\ReadE(\InitE,\texttt{g})
\]

The reason for this imprecision is because the concrete shadows dropped some constraints.
For example, the output of Example 2 can be $[]$ if and only if
$\Tl(\text{Lst})$ is matched with $[]$. To prevent this, shadows may be augmented with
constraints, such as $([],\Tl(\text{Lst})\dot{=}[])$, but we elide this detail
for presentation.

The sound abstraction for $\ctx$ is
\[
  \Abs\ctx=\{\Abs\InitE,\texttt{lst}\mapsto\{\ppoint_7\cons\ppoint_8\}\}
  \text{ where }
  \ppoint_7.\text{out}:[1,1]_{\text{itv}}
  \text{, }
  \ppoint_8.\text{out}:\{\ppoint_9\cons\ppoint_{10}\}\dots
\]
The result of linking $\Abs\ctx$ to the abstract shadow in Example 2
is equivalent to the result of Example 1. This means that the
shadows can be filled in \emph{incrementally}.

\subsection{Paper Overview}
We present our framework for a call-by-value lambda calculus extended
with constructs for defining and linking first-class modules (a
collection of bindings) that support recursive bindings for values,
modules, and functors(module functions).

The framework shows two points: how to define the shadow semantics and
what to prove for the soundness of consequent modular analysis. The
safety of the framework is proven in Coq. We present two instances of the
framework: for high-order applicative language we show modular
closure analysis design, and for imperative languages we show modular
  [TODO] analysis design.

\clearpage
\section{Syntax and Semantics}
\subsection{Abstract Syntax}
\begin{figure}[htb]
  \centering
  \begin{tabular}{rrcll}
    Identifiers & $x$ & $\in$         & $\ExprVar$                                                      \\
    Expression  & $e$ & $\rightarrow$ & $x$ $\vbar$ $\lambda x.e$ $\vbar$ $e$ $e$ & $\lambda$-calculus  \\
                &     & $\vbar$       & $\link{e}{e}$                             & linked expression   \\
                &     & $\vbar$       & $\varepsilon$                             & empty module        \\
                &     & $\vbar$       & $x=e$ ; $e$                               & (recursive) binding
  \end{tabular}
  \caption{Abstract syntax of the language.}
  \label{fig:syntax}
\end{figure}
\subsection{Operational Semantics}
\begin{figure}[h!]
  \centering
  \begin{tabular}{rrcll}
    Environment & $\ctx$ & $\in$         & $\Ctx$                                                                        \\
    Location    & $\ell$ & $\in$         & $\Loc\triangleq\{\text{infinite set of locations}\}$                          \\
    Value       & $v$    & $\in$         & $\Value \triangleq\Ctx+\ExprVar\times\Expr\times\Ctx$                         \\
    Weak Value  & $w$    & $\in$         & $\Walue\triangleq\Value+\Loc\times\Value$                                     \\
    Environment & $\ctx$ & $\rightarrow$ & $\bullet$                                             & empty stack           \\
                &        & $\vbar$       & $(x,\ell)\cons\ctx$                                   & free location binding \\
                &        & $\vbar$       & $(x,w)\cons \ctx$                                     & weak value binding    \\
    Value       & $v$    & $\rightarrow$ & $\ctx$                                                & exported environment  \\
                &        & $\vbar$       & $\langle \lambda x.e, \ctx \rangle$                   & closure               \\
    Weak Value  & $w$    & $\rightarrow$ & $v$                                                   & value                 \\
                &        & $\vbar$       & $\mu\ell.v$                                           & recursive value
  \end{tabular}
  \caption{Definition of the semantic domains.}
  \label{fig:domain}
\end{figure}
\begin{figure}[h!]
  \begin{flushright}
    \fbox{$\evjudg{\ctx}{e}{v}$}
  \end{flushright}
  \centering
  \vspace{0pt} % -0.75em}
  \[
    \begin{InfRule}{Id}
      \hypo{\ctx(x)=v}
      \infer1{\evjudg{\ctx}{x}{v}}
    \end{InfRule}\:
    \begin{InfRule}{RecId}
      \hypo{\ctx(x)=\mu\ell.v}
      \infer1{\evjudg{\ctx}{x}{v[\mu\ell.v/\ell]}}
    \end{InfRule}\:
    \begin{InfRule}{Fn}
      \infer0{\evjudg{\ctx}{\lambda x.e}{\langle\lambda x.e, \ctx\rangle}}
    \end{InfRule}\:
    \begin{InfRule}{App}
      \hypo{\evjudg{\ctx}{e_1}{\langle\lambda x.e, \ctx_1\rangle}}
      \hypo{\evjudg{\ctx}{e_2}{v_2}}
      \hypo{\evjudg{(x, v_2)\cons \ctx_1}{e}{v}}
      \infer{2,1}{\evjudg{\ctx}{e_1\:e_2}{v}}
    \end{InfRule}
  \]

  \[
    \begin{InfRule}{Link}
      \hypo{\evjudg{\ctx}{e_1}{\ctx_1}}
      \hypo{\evjudg{\ctx_1}{e_2}{v}}
      \infer2{\evjudg{\ctx}{\link{e_1}{e_2}}{v}}
    \end{InfRule}\quad
    \begin{InfRule}{Empty}
      \infer0{\evjudg{\ctx}{\varepsilon}{\bullet}}
    \end{InfRule}\quad
    \begin{InfRule}{Bind}
      \hypo{\ell\not\in\FLoc(\ctx)}
      \hypo{\evjudg{(x,\ell)\cons\ctx}{e_1}{v_1}}
      \hypo{\evjudg{(x, \mu\ell.v_1)\cons \ctx}{e_1}{\ctx_2}}
      \infer{2,1}{\evjudg{\ctx}{x=e_1; e_2}{(x,\mu\ell.v_1)\cons\ctx_2}}
    \end{InfRule}
  \]
  \caption{The big-step operational semantics.}
  \label{fig:bigstep}
\end{figure}
\subsection{Reconciling with Conventional Backpatching}
\begin{figure}[h!]
  \centering
  \begin{tabular}{rrcll}
    Environment   & $\ctx$ & $\in$         & $\Mtx\triangleq\fin{\ExprVar}{\Loc}$                                         \\
    Memory        & $\mem$ & $\in$         & $\Mem\triangleq\fin{\Loc}{\Malue}$                                           \\
    Allocated set & $L$    & $\subseteq$   & $\Loc$                                                                       \\
    Value         & $v$    & $\in$         & $\Malue \triangleq\Mtx+\ExprVar\times\Expr\times\Mtx$                        \\
    Environment   & $\ctx$ & $\rightarrow$ & $\bullet$                                             & empty stack          \\
                  &        & $\vbar$       & $(x,\ell)\cons\ctx$                                   & location binding     \\
    Value         & $v$    & $\rightarrow$ & $\ctx$                                                & exported environment \\
                  &        & $\vbar$       & $\langle \lambda x.e, \ctx \rangle$                   & closure
  \end{tabular}
  \caption{Definition of the semantic domains with memory.}
  \label{fig:memdomain}
\end{figure}
\begin{figure}[h!]
  \begin{flushright}
    \fbox{$\evjudg{\ctx,\mem,L}{e}{v,\mem',L'}$}
  \end{flushright}
  \centering
  \vspace{0pt} % -0.75em}
  \[
    \begin{InfRule}{Id}
      \hypo{\ctx(x)=\ell}
      \hypo{\mem(\ell)=v}
      \infer2{\evjudg{\ctx,\mem,L}{x}{v,\mem,L}}
    \end{InfRule}\:
    \begin{InfRule}{Fn}
      \infer0{\evjudg{\ctx,\mem,L}{\lambda x.e}{\langle\lambda x.e,\ctx\rangle,\mem,L}}
    \end{InfRule}
  \]

  \[
    \begin{InfRule}{App}
      \hypo{\evjudg{\ctx,\mem,L}{e_1}{\langle\lambda x.e,\ctx_1\rangle,\mem_1,L_1}}
      \hypo{\evjudg{\ctx,\mem_1,L_1}{e_2}{v_2,\mem_2,L_2}}
      \hypo{\ell\not\in\dom(\mem_2)\cup L_2}
      \hypo{\evjudg{(x,\ell)\cons\ctx_1,\mem_2[\ell\mapsto v_2],L_2}{e}{v,\mem',L'}}
      \infer{3,1}{\evjudg{\ctx,\mem,L}{e_1\:e_2}{v,\mem',L'}}
    \end{InfRule}
  \]

  \[
    \begin{InfRule}{Link}
      \hypo{\evjudg{\ctx,\mem,L}{e_1}{\ctx_1,\mem_1,L_1}}
      \hypo{\evjudg{\ctx_1,\mem_1,L_1}{e_2}{v,\mem',L'}}
      \infer2{\evjudg{\ctx,\mem,L}{\link{e_1}{e_2}}{v,\mem',L'}}
    \end{InfRule}\quad
    \begin{InfRule}{Empty}
      \infer0{\evjudg{\ctx,\mem,L}{\varepsilon}{\bullet,\mem,L}}
    \end{InfRule}
  \]

  \[
    \begin{InfRule}{Bind}
      \hypo{\ell\not\in\dom(\mem)\cup L}
      \hypo{\evjudg{(x,\ell)\cons\ctx,\mem,L\cup\{\ell\}}{e_1}{v_1,\mem_1,L_1}}
      \hypo{\evjudg{(x,\ell)\cons\ctx,\mem_1[\ell\mapsto v_1],L_1}{e_2}{\ctx_2,\mem',L'}}
      \infer{2,1}{\evjudg{\ctx,\mem,L}{x=e_1;e_2}{(x,\ell)\cons\ctx_2,\mem',L'}}
    \end{InfRule}
  \]
  \caption{The big-step operational semantics with memory.}
  \label{fig:membigstep}
\end{figure}

The semantics in Figure \ref{fig:bigstep} makes sense due to similarity with a conventional backpatching semantics as presented in Figure \ref{fig:membigstep}.
We have defined a relation $\sim$ that satisfies:
\[\sim\subseteq\Walue\times(\Malue\times\Mem\times\pset(\Loc))\qquad\bullet\sim(\bullet,\varnothing,\varnothing)\]
and the following theorem:
\begin{theorem}[Equivalence of semantics]\normalfont
  For all $\ctx\in\Ctx,\ctx'\in\Mtx\times\Mem\times\pset(\Loc),v\in\Value,v'\in\Malue\times\Mem\times\pset(\Loc)$, we have:
  \begin{align*}
    \ctx\sim\ctx'\text{ and }\evjudg{\ctx}{e}{v}   & \Rightarrow\exists v':v\sim v'\text{ and }\evjudg{\ctx'}{e}{v'} \\
    \ctx\sim\ctx'\text{ and }\evjudg{\ctx'}{e}{v'} & \Rightarrow\exists v:v\sim v'\text{ and }\evjudg{\ctx}{e}{v}
  \end{align*}
\end{theorem}
The actual definition for $\sim$ can be found in the appendix.

\clearpage
\section{Generating and Resolving Shadows}
Now we formulate the semantics for generating shadows.

\begin{figure}[h!]
  \centering
  \small
  \begin{tabular}{rrcll}
    Shadow      & $\event$ & $\rightarrow$ & $\InitE$           & initial environment \\
                &          & $\vbar$       & $\ReadE(\event,x)$ & read shadow         \\
                &          & $\vbar$       & $\CallE(\event,v)$ & call shadow         \\
    Environment & $\ctx$   & $\rightarrow$ & $\cdots$                                 \\
                &          & $\vbar$       & $[\event]$         & answer to an shadow \\
    Value       & $v$      & $\rightarrow$ & $\cdots$                                 \\
                &          & $\vbar$       & $\event$           & answer to an shadow
  \end{tabular}
  \caption{Definition of the semantic domains with shadows. All other semantic domains are equal to Figure \ref{fig:domain}.}
  \label{fig:shadowdomain}
\end{figure}

We extend how to read weak values given an environment.
\begin{align*}
  \bullet(x)  & \triangleq\bot             &  &  & ((x',\ell)\cons\ctx)(x) & \triangleq (x=x'\;?\;\ell:\ctx(x)) \\
  [\event](x) & \triangleq\ReadE(\event,x) &  &  & ((x',w)\cons\ctx)(x)    & \triangleq (x=x'\;?\;w:\ctx(x))    \\
\end{align*}

Then we need to add only three rules to the semantics in Figure \ref{fig:bigstep} for the semantics to incorporate shadows.
{\small
\[
  \infer[LinkShadow]
  {
  {\evjudg{\ctx}{e_1}{\event}}\\
  {\evjudg{[\event]}{e_2}{v}}
  }
  {\evjudg{\ctx}{e_1\synlink e_2}{v}}
  \quad
  \infer[AppShadow]
  {
  {\evjudg{\ctx}{e_1}{\event}}\\
  {\evjudg{\ctx}{e_2}{v}}
  }
  {\evjudg{\ctx}{e_1\:e_2}{\CallE(\event,v)}}
\]
\[
  \infer[BindShadow]
  {
  {\ell\not\in\FLoc(\ctx)}\\
  {\evjudg{(x,\ell)\cons\ctx}{e_1}{v_1}}\\
  {\evjudg{(x, \mu\ell.v_1)\cons \ctx}{e_1}{\event_2}}
  }
  {\evjudg{\ctx}{x\:\texttt{=}\:e_1; e_2}{(x,\mu\ell.v_1)\cons[\event_2]}}
\]
}

Now we need to formulate the \emph{concrete linking} rules.
The concrete linking rule $\ctx_0\semlink w$, given an answer $\ctx_0$ to the $\InitE$ shadow, resolves all shadows within $w$ to obtain a set of final results.

\begin{align*}
                                                         & \pushright{\fbox{$\semlink\in\Ctx\rightarrow\Event\rightarrow\pset{\Value}$}}                                                          \\
  \ctx_0\semlink\InitE\triangleq                         & \: \{\ctx_0\}                                                                                                                          \\
  \ctx_0\semlink\ReadE(\event,x)\triangleq               & \: \{v_+|\ctx_+\in\ctx_0\semlink\event,\ctx_+(x)=v_+\}                                                                                 \\
  \cup                                                   & \: \{v_+[\mu\ell.v_+/\ell]|\ctx_+\in\ctx_0\semlink\event,\ctx_+(x)=\mu\ell.v_+\}                                                       \\
  \ctx_0\semlink\CallE(\event,v)\triangleq               & \: \{v_+'|\langle\lambda x.e,\ctx_+\rangle\in\ctx_0\semlink\event, v_+\in\ctx_0\semlink v, (x,v_+)\cons\ctx_+\vdash e\Downarrow v_+'\} \\
  \cup                                                   & \: \{\CallE(\event_+,v_+)|\event_+\in\ctx_0\semlink\event, v_+\in\ctx_0\semlink v\}                                                    \\
                                                         & \pushright{\fbox{$\semlink\in\Ctx\rightarrow\Ctx\rightarrow\pset{\Ctx}$}}                                                              \\
  \ctx_0\semlink\bullet\triangleq                        & \: \{\bullet\}                                                                                                                         \\
  \ctx_0\semlink(x,\ell)\cons\ctx\triangleq              & \: \{(x,\ell)\cons\ctx_+|\ctx_+\in\ctx_0\semlink\ctx\}                                                                                 \\
  \ctx_0\semlink(x,w)\cons\ctx\triangleq                 & \: \{(x,w_+)\cons\ctx_+|w_+\in\ctx_0\semlink w,\ctx_+\in\ctx_0\semlink\ctx\}                                                           \\
  \ctx_0\semlink[\event]\triangleq                       & \: \{\ctx_+|\ctx_+\in\ctx_0\semlink\event\}\cup\{[\event_+]|\event_+\in\ctx_0\semlink\event\}                                          \\
                                                         & \pushright{\fbox{$\semlink\in\Ctx\rightarrow\Value\rightarrow\pset{\Value}$}}                                                          \\
  \ctx_0\semlink\langle\lambda x.e,\ctx\rangle\triangleq & \: \{\langle\lambda x.e,\ctx_+\rangle|\ctx_+\in\ctx_0\semlink\ctx\}                                                                    \\
                                                         & \pushright{\fbox{$\semlink\in\Ctx\rightarrow\Walue\rightarrow\pset{\Walue}$}}                                                          \\
  \ctx_0\semlink\mu\ell.v\triangleq                      & \: \{\mu\ell'.v_+|\ell'\not\in\FLoc(v)\cup\FLoc(\ctx_0), v_+\in\ctx_0\semlink v[\ell'/\ell]\}
\end{align*}
Concrete linking makes sense because of the following theorem.
First define:
\[
  \eval(e,\ctx)\triangleq\{v|\ctx\vdash e\Downarrow v\}\quad
  \eval(e,\Sigma)\triangleq\bigcup_{\ctx\in\Sigma}\eval(e,\ctx)\quad
  \Sigma_0\semlink W\triangleq\bigcup_{\substack{\ctx_0\in\Sigma_0\\w\in W}}(\ctx_0\semlink w)
\]
Then the following holds:
\begin{thm}[Advance]\label{thm:linksound}
  Given $e\in\Expr,\Sigma_0,\Sigma\subseteq\Ctx$,
  \[\eval(e,\Sigma_0\semlink\Sigma)\subseteq\Sigma_0\semlink\eval(e,\Sigma)\]
\end{thm}

%The proof of Theorem \ref{thm:linksound} uses some useful lemmas, such as:
%\begin{lem}[Linking distributes under substitution]
%  Let $\ctx_0$ be the external environment that is linked with weak values $w$ and $u$.
%  For all $\ell\not\in\FLoc(\ctx_0)$, we have:
%  \[\forall w_+,u_+:w_+\in\ctx_0\semlink w\wedge u_+\in\ctx_0\semlink u\Rightarrow w_+[u_+/\ell]\in\ctx_0\semlink w[u/\ell]\]
%\end{lem}
%\begin{lem}[Linking is compatible with reads]
%  Let $\textnormal{unroll}:\Walue\rightarrow\Value$ be defined as:
%  \[\textnormal{unroll}(\mu\ell.v)\triangleq v[\mu\ell.v/\ell]\qquad\textnormal{unroll}(v)\triangleq v\]
%  For $\ctx_+\in\ctx_0\semlink\ctx$, if $\ctx_+(x)=w_+$, we have:
%  \[\textnormal{There exists } w\textnormal{ such that }\ctx(x)=w\textnormal{ and }\textnormal{unroll}(w_+)\in\ctx_0\semlink\textnormal{unroll}(w)\]
%\end{lem}

Now we can formulate modular analysis.
A modular analysis consists of two requirements: an abstraction for the semantics with shadows and an abstraction for the semantic linking operator.
\begin{thm}[Modular analysis]
  Assume:
  \begin{enumerate}
    \item An abstract domain $\Abs\Walue$ and a monotonic $\gamma\in\Abs\Walue\rightarrow\pset{\Walue}$
    \item A sound $\Abs\eval$: $\Sigma_0\subseteq\gamma(\Abs\ctx_0)\Rightarrow\eval(e,\Sigma_0)\subseteq\gamma(\Abs\eval(e,\Abs\ctx_0))$
    \item A sound $\Abs\semlink$: $\Sigma_0\subseteq\gamma(\Abs\ctx_0)\textnormal{ and }W\subseteq\gamma(\Abs{w})\Rightarrow\Sigma_0\semlink W\subseteq\gamma(\Abs\ctx_0\Abs\semlink\Abs{w})$
  \end{enumerate}
  then we have:
  \[\Sigma_0\subseteq\gamma(\Abs\ctx_0)\textnormal{ and }\Sigma\subseteq\gamma(\Abs\ctx)\Rightarrow\eval(e,\Sigma_0\semlink\Sigma)\subseteq\gamma(\Abs\ctx_0\Abs\semlink\Abs\eval(e,\Abs\ctx))\]
\end{thm}
\begin{cor}[Modular analysis of linked program]
  \begin{align*}
    \Sigma_0\subseteq\gamma(\Abs\ctx_0)\textnormal{ and } & [\InitE]\in\gamma(\Abs\InitE)\Rightarrow                                                                       \\
                                                          & \eval(e_1\synlink e_2,\Sigma_0)\subseteq\gamma(\Abs\eval(e_1,\Abs\ctx_0)\Abs\semlink\Abs\eval(e_2,\Abs\InitE))
  \end{align*}
\end{cor}

\clearpage
\section{CFA}
\subsection{Collecting semantics}
\begin{center}
  \begin{tabular}{rrcl}
    Program point        & $p$                & $\in$         & $\mathbb{P}\triangleq\{\textnormal{finite set of program points}\}$                            \\
    Labelled expression  & $pe$               & $\in$         & $\mathbb{P}\times\Expr$                                                                        \\
    Labelled location    & $\llabel{p}{\ell}$ & $\in$         & $\mathbb{P}\times\Loc$                                                                         \\
    Collecting semantics & $t$                & $\in$         & $\mathbb{T}\triangleq\mathbb{P}\rightarrow\pset{\Ctx+\Ctx\times\Value}$                        \\
    Labelled expression  & $pe$               & $\rightarrow$ & $\plabel{p}{e}$                                                                                \\
    Expression           & $e$                & $\rightarrow$ & $x$ | $\lambda x.pe$ | $pe$ $pe$ | ${pe}\synlink{pe}$ | $\varepsilon$ | $x\:\texttt{=}\:pe;pe$
  \end{tabular}
\end{center}
\begin{align*}
                             & \pushright{\fbox{$\Step:\mathbb{T}\rightarrow\mathbb{T}$}}                                                                                                                                                              \\
  \Step(\ptree)   \triangleq & \bigcup_{p\in\mathbb{P}}\step(\ptree,p)                                                                                                                                                                                 \\
                             & \pushright{\fbox{$\step:(\mathbb{T}\times\mathbb{P})\rightarrow\mathbb{T}$}}                                                                                                                                            \\
  \step(\ptree,p) \triangleq & \specialcell{\:[p\mapsto\{(\ctx,v)|\ctx\in\ptree(p)\textnormal{ and }\ctx(x)=v\}]\hfill\textnormal{when }\plabel{p}{x}}                                                                                                 \\
  \cup                       & \: [p\mapsto\{(\ctx,v[\mu\llabel{p'}{\ell}.v/\llabel{p'}{\ell}])|\ctx\in \ptree(p)\textnormal{ and }\ctx(x)=\mu\llabel{p'}{\ell}.v\}]                                                                                   \\
  \step(\ptree,p) \triangleq & \specialcell{\:[p\mapsto\{(\ctx,\langle\lambda x.p',\ctx\rangle)|\ctx\in \ptree(p)\}]\hfill\textnormal{when }\plabel{p}{\lambda x.p'}}                                                                                  \\
  \step(\ptree,p) \triangleq & \specialcell{\:[p_1\mapsto\{\ctx\in\Ctx|\ctx\in \ptree(p)\}]\hfill\textnormal{when }\plabel{p}{p_1\:p_2}}                                                                                                               \\
  \cup                       & \: [p_2\mapsto\{\ctx\in\Ctx|\ctx\in \ptree(p)\}]                                                                                                                                                                        \\
  \cup                       & \: \bigcup_{\ctx\in \ptree(p)}\bigcup_{(\ctx,\langle\lambda x.p',\ctx_1\rangle)\in \ptree(p_1)}[p'\mapsto\{(x,v_2)\cons\ctx_1|(\ctx,v_2)\in \ptree(p_2)\}]                                                              \\
  \cup                       & \: [p\mapsto\bigcup_{\ctx\in \ptree(p)}\bigcup_{(\ctx,\langle\lambda x.p',\ctx_1\rangle)\in \ptree(p_1)}\bigcup_{(\ctx,v_2)\in \ptree(p_2)}\{(\ctx,v)|((x,v_2)\cons\ctx_1,v)\in \ptree(p')\}]                           \\
  \cup                       & \: [p\mapsto\bigcup_{\ctx\in \ptree(p)}\{(\ctx,\CallE(\event_1,v_2))|(\ctx,\event_1)\in \ptree(p_1)\textnormal{ and }(\ctx,v_2)\in \ptree(p_2)\}]                                                                       \\
  \step(\ptree,p) \triangleq & \specialcell{\:[p_1\mapsto\{\ctx|\ctx\in \ptree(p)\}]\hfill\textnormal{when }\plabel{p}{{p_1}\synlink{p_2}}}                                                                                                            \\
  \cup                       & \: [p_2\mapsto\bigcup_{\ctx\in \ptree(p)}\{\ctx_1|(\ctx,\ctx_1)\in \ptree(p_1)\}]                                                                                                                                       \\
  \cup                       & \: [p\mapsto\bigcup_{\ctx\in \ptree(p)}\bigcup_{(\ctx,\ctx_1)\in \ptree(p_1)}\{(\ctx,v_2)|(\ctx_1,v_2)\in \ptree(p_2)\}]                                                                                                \\
  \step(\ptree,p) \triangleq & \specialcell{\:[p\mapsto\{(\ctx,\bullet)|\ctx\in \ptree(p)\}]\hfill\textnormal{when }\plabel{p}{\varepsilon}}                                                                                                           \\
  \step(\ptree,p) \triangleq & \specialcell{\:[p_1\mapsto\bigcup_{\ctx\in\ptree(p)}\{(x,\llabel{p_1}{\ell})\cons\ctx|\ell\not\in\FLoc(\ctx)\}]\hfill\textnormal{when }\plabel{p}{x\:\texttt{=}\:p_1;p_2}}                                              \\
  \cup                       & \: [p_2\mapsto\bigcup_{\ctx\in\ptree(p)}\{(x,\mu\llabel{p_1}{\ell}.v_1)\cons\ctx|((x,\llabel{p_1}{\ell})\cons\ctx,v_1)\in \ptree(p_1)\}]                                                                                \\
  \cup                       & \: [p\mapsto\bigcup_{\ctx\in \ptree(p)}\bigcup_{((x,\llabel{p_1}{\ell})\cons\ctx,v_1)\in\ptree(p_1)}\{(\ctx,(x,\mu\llabel{p_1}{\ell}.v_1)\cons\ctx_2)|((x,\mu\llabel{p_1}{\ell}.v_1)\cons\ctx,\ctx_2)\in \ptree(p_2)\}]
\end{align*}
The collecting semantics $\sembracket{p_0}\Sigma_0$ computed by
\[\sembracket{p_0}\Sigma_0\triangleq\lfp(\lambda\ptree.\Step(\ptree)\cup\ptree_{\textnormal{init}})\quad\textnormal{where }\ptree_{\textnormal{init}}=[p_0\mapsto\Sigma_0]\]
contains all derivations of the form $\evjudg{\ctx_0}{p_0}{v_0}$ for some $\ctx_0\in\Sigma_0$ and $v_0$.
That is, $(\ctx,v)$ is contained in $\sembracket{p_0}\Sigma_0(p)$ if and only if $\evjudg{\ctx}{p}{v}$ is contained in some derivation for the judgment $\evjudg{\ctx_0}{p_0}{v_0}$.

\subsection{Abstract semantics}
\begin{center}
  \begin{tabular}{rrcl}
    Abstract shadow      & $\Abs{\event}$                 & $\in$         & $\Abs{\Event}$                                                                     \\
    Abstract environment & $\Abs{\ctx}$                   & $\in$         & $\Abs{\Ctx}\triangleq(\fin{\ExprVar}{\pset{\mathbb{P})}}\times\pset{\Abs{\Event}}$ \\
    Abstract closure     & $\langle\lambda x.p,p'\rangle$ & $\in$         & $\Abs{\textnormal{Clos}}\triangleq\ExprVar\times\mathbb{P}\times\mathbb{P}$        \\
    Abstract value       & $\Abs{v}$                      & $\in$         & $\Abs{\Value}\triangleq\Abs{\Ctx}\times\pset{\Abs{\textnormal{Clos}}}$             \\
    Abstract semantics   & $\Abs\ptree$                   & $\in$         & $\Abs{\mathbb{T}}\triangleq\mathbb{P}\rightarrow\Abs{\Ctx}\times\Abs{\Value}$      \\
    Abstract shadow      & $\Abs{\event}$                 & $\rightarrow$ & $\Abs{\InitE}$ | $\Abs{\ReadE}(p,x)$ | $\Abs{\CallE}(p,p)$
  \end{tabular}
\end{center}
\begin{figure}[h!]
  \centering
  \small
  \begin{flushright}
    \fbox{$\ctx\preceq(\Abs\ctx,\Abs\ptree)$}
  \end{flushright}
  \[
    \infer[Conc-Nil]
    {\:}
    {\bullet\preceq\Abs\ctx}
    \quad
    \infer[Conc-ENil]
    {\event\preceq(\Abs\ctx,\varnothing)}
    {[\event]\preceq\Abs\ctx}
    \quad
    \infer[Conc-ConsLoc]
    {
    {p\in\Abs\ctx.1(x)}\\
    {\ctx\preceq\Abs\ctx}
    }
    {(x,\llabel{p}{\ell})\cons\ctx\preceq\Abs\ctx}
  \]

  \[
    \infer[Conc-ConsWVal]
    {
    {p\in\Abs\ctx.1(x)}\\
    {w\preceq\Abs\ptree(p).2}\\
    {\ctx\preceq\Abs\ctx}
    }
    {(x,w)\cons\ctx\preceq\Abs\ctx}
  \]
  \begin{flushright}
    \fbox{$w\preceq(\Abs{v},\Abs\ptree)$}
  \end{flushright}
  \[
    \infer[Conc-Clos]
    {
    {\langle\lambda x.p,p'\rangle\in\Abs{v}.2}\\
    {\ctx\preceq\Abs\ptree(p').1}
    }
    {\langle\lambda x.p,\ctx\rangle\preceq\Abs{v}}
    \quad
    \infer[Conc-Rec]
    {
    {v\preceq\Abs\ptree(p).2}\\
    {v\preceq\Abs{v}}
    }
    {\mu\llabel{p}{\ell}.v\preceq\Abs{v}}
  \]

  \[
    \infer[Conc-Init]
    {\Abs\InitE\in\Abs{v}.1.2}
    {\InitE\preceq\Abs{v}}
    \quad
    \infer[Conc-Read]
    {
    {\Abs\ReadE(p,x)\in\Abs{v}.1.2}\\
    {[\event]\preceq\Abs\ptree(p).1}
    }
    {\ReadE(\event,x)\preceq\Abs{v}}
  \]

  \[
    \infer[Conc-Call]
    {
    {\Abs\CallE(p_1,p_2)\in\Abs{v}.1.2}\\
    {\event\preceq\Abs\ptree(p_1).2}\\
    {v\preceq\Abs\ptree(p_2).2}
    }
    {\CallE(\event,v)\preceq\Abs{v}}
  \]
  \caption{The concretization relation between weak values and abstract values. $\Abs\ptree$ is omitted for brevity.}
  \label{fig:concretrel}
\end{figure}

The concretization function $\gamma\in\Abs{\mathbb{T}}\rightarrow\mathbb{T}$ is defined as:
\[\gamma(\Abs\ptree)\triangleq\lambda p.\{\ctx|\ctx\preceq(\Abs\ptree(p).1,\Abs\ptree)\}\cup\{(\ctx,v)|v\preceq(\Abs\ptree(p).2,\Abs\ptree)\}\]
where $\preceq$ is the concretization relation inductively defined in Figure \ref{fig:concretrel}.

Now the abstract semantic function can be given.
\begin{align*}
                                     & \pushright{\fbox{$\Abs\Step:\Abs{\mathbb{T}}\rightarrow\Abs{\mathbb{T}}$}}                                                                         \\
  \Abs\Step(\Abs\ptree)   \triangleq & \bigsqcup_{p\in\mathbb{P}}\Abs\step(\Abs\ptree,p)                                                                                                  \\
                                     & \pushright{\fbox{$\Abs\step:(\Abs{\mathbb{T}}\times\mathbb{P})\rightarrow\Abs{\mathbb{T}}$}}                                                       \\
  \Abs\step(\Abs\ptree,p) \triangleq & \specialcell{\:[p\mapsto\bigsqcup_{p'\in\Abs\ptree(p).1.1(x)}(\bot,\Abs\ptree(p').2)]\hfill\textnormal{when }\plabel{p}{x}}                        \\
  \sqcup                             & \: [p\mapsto(\bot,(([],\{\Abs\ReadE(p,x)\}),\varnothing))]\hfill\textnormal{if }\Abs\ptree(p).1.2\neq\varnothing                                   \\
  \Abs\step(\Abs\ptree,p) \triangleq & \specialcell{\:[p\mapsto(\bot,(\bot,\{\langle\lambda x.p',p\rangle\}))]\hfill\textnormal{when }\plabel{p}{\lambda x.p'}}                           \\
  \Abs\step(\Abs\ptree,p) \triangleq & \specialcell{\:[p_1\mapsto(\Abs\ptree(p).1,\bot)]\hfill\textnormal{when }\plabel{p}{p_1\:p_2}}                                                     \\
  \sqcup                             & \: [p_2\mapsto(\Abs\ptree(p).1,\bot)]                                                                                                              \\
  \sqcup                             & \: \bigsqcup_{\langle\lambda x.p',p''\rangle\in\Abs\ptree(p_1).2.2}[p'\mapsto(\Abs\ptree(p'').1\sqcup([x\mapsto\{p_2\}],\varnothing),\bot)]        \\
  \sqcup                             & \: [p\mapsto\bigsqcup_{\langle\lambda x.p',\_\rangle\in\Abs\ptree(p_1).2.2}(\bot,\Abs\ptree(p').2)]                                                \\
  \sqcup                             & \specialcell{\:[p\mapsto(\bot,(([],\{\Abs\CallE(p_1,p_2)\}),\varnothing))]\hfill\textnormal{if }\Abs\ptree(p_1).2.1.2\neq\varnothing}              \\
  \Abs\step(\Abs\ptree,p) \triangleq & \specialcell{\:[p_1\mapsto(\Abs\ptree(p).1,\bot)]\hfill\textnormal{when }\plabel{p}{{p_1}\synlink{p_2}}}                                           \\
  \sqcup                             & \: [p_2\mapsto(\Abs\ptree(p_1).2.1,\bot)]                                                                                                          \\
  \sqcup                             & \: [p\mapsto(\bot,\Abs\ptree(p_2).2)]                                                                                                              \\
  \Abs\step(\Abs\ptree,p) \triangleq & \specialcell{\:\bot\hfill\textnormal{when }\plabel{p}{\varepsilon}}                                                                                \\
  \Abs\step(\Abs\ptree,p) \triangleq & \specialcell{\:[p_1\mapsto(\Abs\ptree(p).1\sqcup([x\mapsto\{p_1\}],\varnothing),\bot)]\hfill\textnormal{ when }\plabel{p}{x\:\texttt{=}\:p_1;p_2}} \\
  \sqcup                             & \: [p_2\mapsto(\Abs\ptree(p).1\sqcup([x\mapsto\{p_1\}],\varnothing),\bot)]                                                                         \\
  \sqcup                             & \: [p\mapsto(\bot,(\Abs\ptree(p_2).2.1\sqcup([x\mapsto\{p_1\}],\varnothing),\varnothing))]
\end{align*}
The abstract semantics $\Abs{\ptree}$ computed by
\[\Abs{\sembracket{p_0}}(\Abs\ctx_0,\Abs\ptree_0)\triangleq\lfp(\lambda\Abs\ptree.\Abs\Step(\Abs\ptree)\sqcup\Abs{\ptree}_{\textnormal{init}})\quad\textnormal{where }\ptree_{\textnormal{init}}=\Abs\ptree_0\sqcup[p_0\mapsto(\Abs\ctx_0,\bot)]\]
is a sound abstraction of $\sembracket{p_0}\Sigma_0$ when $\Sigma_0\subseteq\gamma(\Abs\ctx_0,\Abs\ptree_0)$.

\subsection{Abstract linking}
Now we define a sound linking operator that abstracts $\semlink$.
Assume we have
\[\ctx_0\preceq(\Abs{\ctx}_0,\Abs{\ptree}_0)\quad\ptree\subseteq\gamma(\Abs\ptree)\]
we define:
\[\ctx_0\semlink\ptree\triangleq\lambda p.\bigcup_{\ctx\in t(p)}(\ctx_0\semlink\ctx)\cup\bigcup_{(\ctx,v)\in t(p)}\{(\ctx_+,v_+)|\ctx_+\in\ctx_0\semlink\ctx\textnormal{ and }v_+\in\ctx_0\semlink v\}\]

We want to define $\Abs\semlink$ so that the following holds:
\[\ctx_0\semlink\ptree\subseteq\gamma((\Abs{\ctx}_0,\Abs{\ptree}_0)\Abs\semlink\Abs\ptree)\]
This is equivalent to saying that the linked result $\Abs\ptree_+=(\Abs\ctx_0,\Abs\ptree_0)\Abs\semlink\Abs\ptree$ satisfies:
\[\ctx_0\preceq(\Abs\ctx_0,\Abs\ptree_0)\textnormal{ and }w\preceq(\Abs{v},\Abs\ptree)\Rightarrow w_+\preceq(\Abs{v}_+,\Abs\ptree_+)\]
for each $w_+\in\ctx_0\semlink w$ and $p\in\mathbb{P}$,
when $[\Abs{v},\Abs{v}_+] = [(\Abs\ptree(p).1,\varnothing),(\Abs\ptree_+(p).1,\varnothing)]$ or $[\Abs\ptree(p).2,\Abs\ptree_+(p).2]$.

The condition for $\Abs\ptree_+$ can be deduced by attempting the proof of the above in advance.

We proceed by induction on the derivation for
\[w_+\in\ctx_0\semlink w\]
and inversion on $w\preceq(\Abs{v},\Abs\ptree)$.
\begin{center}
  \begin{tabular}{|l@{\hskip2pt}l|}
    \hline
    When: \hfill $w=$                                & $\InitE$,                                                                                                                                               \\
    Have: \hfill $\Abs\InitE\in$                     & $\Abs{v}.1.2$                                                                                                                                           \\
    Need: \hfill $\Abs{v}_+\sqsupseteq$              & $\:\Abs\ctx_0$                                                                                                                                          \\
    \hfill $\Abs\ptree_+\sqsupseteq$                 & $\:\Abs\ptree_0$                                                                                                                                        \\
    \hline
    When: \hfill $w=$                                & $\ReadE(\event,x)$,                                                                                                                                     \\
    Have: \hfill $\Abs\ReadE(p',x)$                  & $\in\Abs{v}.1.2$ and $[\event]\preceq\Abs\ptree(p').1$                                                                                                  \\
    Need: \hfill $\Abs{v}_+\sqsupseteq$              & $\:\Abs\ptree_+(p'').2$                                                             \hfill for $p''\in\Abs\ptree_+(p').1.1(x)$                          \\
    \hfill $\Abs{v}_+\sqsupseteq$                    & $\:(([],\{\Abs\ReadE(p',x)\}),\varnothing)$                                         \hfill if $\Abs\ptree_+(p').1.2\neq\varnothing$                     \\
    \hline
    When: \hfill $w=$                                & $\CallE(\event,v)$,                                                                                                                                     \\
    Have: \hfill $\Abs\CallE(p_1,p_2)\in$            & $\Abs{v}.1.2$ and $\event\preceq\Abs\ptree(p_1).2$ and $v\preceq\Abs\ptree(p_2).2$                                                                      \\
    Need: \hfill $\Abs{v}_+\sqsupseteq$              & $\:\Abs\ptree_+(p').2$                                                              \hfill for $\langle\lambda x.p',p''\rangle\in\Abs\ptree_+(p_1).2.2$ \\
    \hfill $\Abs{v}_+\sqsupseteq$                    & $\:(([],\{\Abs\CallE(p_1,p_2)\}),\varnothing)$                                      \hfill if $\Abs\ptree_+(p_1).2.1.2\neq\varnothing$                  \\
    \hfill $\Abs\ptree_+(p')\sqsupseteq$             & $\:(\Abs\ptree_+(p'').1\sqcup([x\mapsto\{p_2\}],\varnothing),\varnothing)$          \hfill for $\langle\lambda x.p',p''\rangle\in\Abs\ptree_+(p_1).2.2$ \\
    \hfill $\Abs\ptree_+\sqsupseteq$                 & $\:\Abs\Step(\Abs\ptree_+)$                                                                                                                             \\
    \hline
    When: \hfill $w=$                                & $(x,\llabel{p'}{\ell})\cons\ctx$,                                                                                                                       \\
    Have: \hfill $p'\in$                             & $\Abs{v}.1.1(x)$ and $\ctx\preceq\Abs{v}$                                                                                                               \\
    Need: \hfill $\Abs{v}_+.1.1(x)\ni$               & $\:p'$                                                                                                                                                  \\
    \hline
    When: \hfill $w=$                                & $(x,w')\cons\ctx$,                                                                                                                                      \\
    Have: \hfill $p'\in$                             & $\Abs{v}.1.1(x)$ and $w'\preceq\Abs\ptree(p').2$ and $\ctx\preceq\Abs{v}$                                                                               \\
    Need: \hfill $\Abs{v}_+.1.1(x)\ni$               & $\:p'$                                                                                                                                                  \\
    \hline
    When: \hfill $w=$                                & $\langle\lambda x.p',\ctx\rangle$,                                                                                                                      \\
    Have: \hfill $\langle\lambda x.p',p''\rangle\in$ & $\Abs{v}.2$ and $\ctx\preceq\Abs\ptree(p'').1$                                                                                                          \\
    Need: \hfill $\Abs{v}_+.2\ni$                    & $\:\langle\lambda x.p',p''\rangle$                                                                                                                      \\
    \hline
  \end{tabular}
\end{center}
The above conditions can be summarized by saying $\Abs\ptree_+$ is a post-fixed point of:
\[\lambda\Abs\ptree_+.\Abs\Step(\Abs\ptree_+)\sqcup\Abs\Link(\Abs\ctx_0,\Abs\ptree,\Abs\ptree_+)\sqcup\Abs\ptree_0\]
where $\Abs\Link(\Abs\ctx_0,\Abs\ptree,\Abs\ptree_+)$ is the least function that satisfies:
\begin{center}
  \begin{tabular}{|l@{\hskip2pt}l|}
    \hline
    \hfill Let $\Abs\link=\Abs\Link$                         & $(\Abs\ctx_0,\Abs\ptree,\Abs\ptree_+)$ in                                                                                                               \\
    \hfill For each $p\in\mathbb{P}$,                        & when $\Abs{v},\Abs{v}_+=(\Abs\ptree(p).1,\varnothing),(\Abs\link(p).1,\varnothing)$                                                                     \\
    \hfill or                                                & when $\Abs{v},\Abs{v}_+=\Abs\ptree(p).2,\Abs\link.2$                                                                                                    \\
    \hline
    If:           \hfill $\Abs\InitE\in$                     & $\Abs{v}.1.2$                                                                                                                                           \\
    Then:         \hfill $\Abs{v}_+\sqsupseteq$              & $\:\Abs\ctx_0$                                                                                                                                          \\
    \hline
    If:           \hfill $\Abs\ReadE(p',x)$                  & $\in\Abs{v}.1.2$                                                                                                                                        \\
    Then:         \hfill $\Abs{v}_+\sqsupseteq$              & $\:\Abs\ptree_+(p'').2$                                                             \hfill for $p''\in\Abs\ptree_+(p').1.1(x)$                          \\
    \hfill $\Abs{v}_+\sqsupseteq$                            & $\:(([],\{\Abs\ReadE(p',x)\}),\varnothing)$                                         \hfill if $\Abs\ptree_+(p').1.2\neq\varnothing$                     \\
    \hline
    If:           \hfill $\Abs\CallE(p_1,p_2)\in$            & $\Abs{v}.1.2$                                                                                                                                           \\
    Then:         \hfill $\Abs{v}_+\sqsupseteq$              & $\:\Abs\ptree_+(p').2$                                                              \hfill for $\langle\lambda x.p',p''\rangle\in\Abs\ptree_+(p_1).2.2$ \\
    \hfill $\Abs{v}_+\sqsupseteq$                            & $\:(([],\{\Abs\CallE(p_1,p_2)\}),\varnothing)$                                      \hfill if $\Abs\ptree_+(p_1).2.1.2\neq\varnothing$                  \\
    \hfill $\Abs\link(p')\sqsupseteq$                        & $\:(\Abs\ptree_+(p'').1\sqcup([x\mapsto\{p_2\}],\varnothing),\varnothing)$          \hfill for $\langle\lambda x.p',p''\rangle\in\Abs\ptree_+(p_1).2.2$ \\
    \hline
    If:           \hfill $p'\in$                             & $\Abs{v}.1.1(x)$                                                                                                                                        \\
    Then:         \hfill $\Abs{v}_+.1.1(x)\ni$               & $\:p'$                                                                                                                                                  \\
    \hline
    If:           \hfill $p'\in$                             & $\Abs{v}.1.1(x)$                                                                                                                                        \\
    Then:         \hfill $\Abs{v}_+.1.1(x)\ni$               & $\:p'$                                                                                                                                                  \\
    \hline
    If:           \hfill $\langle\lambda x.p',p''\rangle\in$ & $\Abs{v}.2$                                                                                                                                             \\
    Then:         \hfill $\Abs{v}_+.2\ni$                    & $\:\langle\lambda x.p',p''\rangle$                                                                                                                      \\
    \hline
  \end{tabular}
\end{center}
Note that the left-hand side contains only $\Abs\link$ and the right-hand side does not depend on the value of $\Abs\link$.

Some auxiliary lemmas:
\begin{lem}[Substitution of values]
  \[w\preceq(\Abs{v},\Abs\ptree)\textnormal{ and }u\preceq(\Abs\ptree(p).2,\Abs\ptree)\Rightarrow w[u/\llabel{p}{\ell}]\preceq(\Abs{v},\Abs\ptree)\]
\end{lem}
\begin{lem}[Sound $\Abs\step$]
  \[\forall p,\ptree,\Abs\ptree:\ptree\subseteq\gamma(\Abs\ptree)\Rightarrow\step(\ptree,p)\cup\ptree\subseteq\gamma(\Abs\step(\Abs\ptree,p)\sqcup\Abs\ptree)\]
\end{lem}
\begin{lem}[Sound $\Abs\Step$]
  \[\forall\ptree_{\textnormal{init}},\Abs\ptree:\ptree_{\textnormal{init}}\subseteq\gamma(\Abs\ptree)\textnormal{ and }\Abs\Step(\Abs\ptree)\sqsubseteq\Abs\ptree\Rightarrow\lfp(\lambda\ptree.\Step(\ptree)\cup\ptree_{\textnormal{init}})\subseteq\gamma(\Abs\ptree)\]
\end{lem}


\section{Conclusion}

\end{document}

