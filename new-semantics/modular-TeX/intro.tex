\section{Introduction}
``There's a library for everything'' is one way of summarizing modern software development.
A library is a reusable component of software that is packaged into a module, which can be imported and used by others that require similar features.
Since libraries promote code reuse and modular code design, both recommended practices, the number of libraries continues to grow.

Libraries, however, mean trouble for static program analyzers.
Using values from libraries that cannot be resolved from analyzing the local code prevent the analyzer from creating accurate summaries of the program behavior.
Realistic program analyzers resort to making either angelic assumptions about unknown functions (such as \href{https://fbinfer.com/docs/next/checker-pulse/#unknown-functions}{Infer}), which may lead to false negatives,
or demonic assumptions, which greatly drops the precision of the analysis.

Thus, a method of analysis that allows uninterpreted function symbols is needed.
Even in the presence of unresolved variables, a program analyzer should be able to produce a result that can be reused when the names are resolved \emph{separately}.

We present such a framework powered by \emph{abstract interpretation}.
What abstract interpretation teaches us is that given a concrete semantics of a program and semantic operators that act on the semantic domains, approximating each with a sound abstract version leads to a sound analysis.
Thus we present a semantics of programs that works even when the initial environment the program is evaluated under is not fully given.
We call this the ``shadow semantics'', since it computes a shadow of the expected program execution.
Also, we present a semantic linking operator that can fill in the missing parts later.

\begin{center}
  \begin{tabular}{p{200pt}|p{200pt}}
    \hline
    Example 1: Error handling                                   & Example 2: React-like state management                                                                     \\
    \hline
    \begin{verbatim}
let _ =
  match f 42 with
  | Ok result -> result
  | Err trace -> trace - 1
\end{verbatim}
                                                                &
    \begin{verbatim}
let make = fun cur ->
  let x, setX =
    useState { cur with val = 0 }
  in setX (x + 1)
\end{verbatim}
    \\\hline
    Shadow 1: value of the \texttt{let} binding                 & Shadow 2: body of the \texttt{make} function
    \\\hline
    \\
    $\textsf{Match}(\textsf{ret},\texttt{Ok})$ or               & $\CallE(\textsf{Get}(\textsf{ret},1),\textsf{Add}(\textsf{Get}(\textsf{ret},0), 1))$ \\
    $\textsf{Sub}(\textsf{Match}(\textsf{ret},\texttt{Err}),1)$ & where $\textsf{ret}=\CallE(\textsf{useState},(\texttt{val},0)\cons[\textsf{cur}])$   \\
    where $\textsf{ret}=\CallE(\textsf{f},42)$                  & and $\textsf{useState}=\ReadE(\InitE,\texttt{useState})$                             \\
    and $\textsf{f}=\ReadE(\InitE,\texttt{f})$                  & and $\textsf{cur}=\ReadE(\InitE,\texttt{cur})$                                       \\
    \\\hline
  \end{tabular}
\end{center}

As a preview, see the examples.
The first example calls an unknown function \texttt{f} with argument \texttt{42}, and performs different actions depending on the result of the function call.
There are two shadows for this program, one from the \texttt{Ok} branch and another from the \texttt{Err} branch.
When the code for the \texttt{f} function is given in the initial environment, the shadow can be colored in to a specific branch.
For example, when
\[\ctx_0=(\texttt{f},\langle\lambda\texttt{x}.\texttt{if x = 0 then Err -1 else Ok (42 / x)},[]\rangle)\cons[]\]
is given as an answer to the $\InitE$ shadow, the linked result will contain only $42/42=1$.

The second example is adapted from how JavaScript programmers use \href{https://react.dev/}{React}, a library for building reactive web applications.
In the example, the function \texttt{make} is called with the \emph{current state} of the reactive component \texttt{x} each time the web page is rendered, and updates \texttt{x} with its incremented value.
What is noticable is that idioms such as functions that returns functions (\texttt{useState} returning \texttt{setX}) is used to hide how the library manipulates the local state.
To analyze this program, one has to record what calls affect the result in what order.
The answer can be obtained by linking the details of how React schedules state updates, which can change over library versions.

The example also shows how shadows are no different from ``normal'' values.
Note the record $(\texttt{val},0)\cons[\textsf{cur}]$ that is given as the argument to \texttt{useState}.
\textsf{cur} is a shadow, yet it is consed with a concrete value $(\texttt{val},0)$.
That is, the shadows we create live in a semantic domain, which can be subjected to regular techniques in abstract interpretation.

Our modular analysis framework consists of two parts; abstracting the shadow semantics and abstracting the semantic linking operator.

