\section{Generating and Resolving Events}
Now we formulate the semantics for generating events.

\begin{figure}[h!]
  \centering
  \small
  \begin{tabular}{rrcll}
    Event       & $\event$ & $\rightarrow$ & $\InitE$           & initial environment \\
                &          & $\vbar$       & $\ReadE(\event,x)$ & read event          \\
                &          & $\vbar$       & $\CallE(\event,v)$ & call event          \\
    Environment & $\ctx$   & $\rightarrow$ & $\cdots$                                 \\
                &          & $\vbar$       & $[\event]$         & answer to an event  \\
    Value       & $v$      & $\rightarrow$ & $\cdots$                                 \\
                &          & $\vbar$       & $\event$           & answer to an event
  \end{tabular}
  \caption{Definition of the semantic domains with events. All other semantic domains are equal to Figure \ref{fig:domain}.}
  \label{fig:eventdomain}
\end{figure}

We extend how to read weak values given an environment.
\begin{align*}
  \bullet(x)  & \triangleq\bot             &  &  & ((x',\ell)\cons\ctx)(x) & \triangleq (x=x'?\ell:\ctx(x)) \\
  [\event](x) & \triangleq\ReadE(\event,x) &  &  & ((x',w)\cons\ctx)(x)    & \triangleq (x=x'?w:\ctx(x))    \\
\end{align*}

Then we need to add only three rules to the semantics in Figure \ref{fig:bigstep} for the semantics to incorporate events.
  {\small
    \[
      \begin{InfRule}{LinkEvent}
        \hypo{\evjudg{\ctx}{e_1}{\event}}
        \hypo{\evjudg{[\event]}{e_2}{v}}
        \infer2{\evjudg{\ctx}{e_1\synlink e_2}{v}}
      \end{InfRule}\:
      \begin{InfRule}{AppEvent}
        \hypo{\evjudg{\ctx}{e_1}{\event}}
        \hypo{\evjudg{\ctx}{e_2}{v}}
        \infer2{\evjudg{\ctx}{e_1\:e_2}{\CallE(\event,v)}}
      \end{InfRule}\:
      \begin{InfRule}{BindEvent}
        \hypo{\ell\not\in\FLoc(\ctx)}
        \hypo{\evjudg{(x,\ell)\cons\ctx}{e_1}{v_1}}
        \hypo{\evjudg{(x, \mu\ell.v_1)\cons \ctx}{e_1}{\event_2}}
        \infer{2,1}{\evjudg{\ctx}{x\:\texttt{=}\:e_1; e_2}{(x,\mu\ell.v_1)\cons[\event_2]}}
      \end{InfRule}
    \]
  }

Now we need to formulate the \emph{concrete linking} rules.
The concrete linking rule $\ctx_0\semlink w$, given an answer $\ctx_0$ to the $\InitE$ event, resolves all events within $w$ to obtain a set of final results.
\begin{align*}
  \fbox{$\semlink\in\Ctx\rightarrow\Event\rightarrow\pset(\Value)$}                                                                                                                           \\
  \ctx_0\semlink\InitE\triangleq                         & \: \{\ctx_0\}                                                                                                                      \\
  \ctx_0\semlink\ReadE(\event,x)\triangleq               & \: \{v_+|\ctx_+\in\ctx_0\semlink\event,\ctx_+(x)=v_+\}                                                                             \\
  \cup                                                   & \: \{v_+[\mu\ell.v_+/\ell]|\ctx_+\in\ctx_0\semlink E,\ctx_+(x)=\mu\ell.v_+\}                                                       \\
  \ctx_0\semlink\CallE(\event,v)\triangleq               & \: \{v_+'|\langle\lambda x.e,\ctx_+\rangle\in\ctx_0\semlink E, v_+\in\ctx_0\semlink v, (x,v_+)\cons\ctx_+\vdash e\Downarrow v_+'\} \\
  \cup                                                   & \: \{\CallE(\event_+,v_+)|\event_+\in\ctx_0\semlink E, v_+\in\ctx_0\semlink v\}                                                    \\
  \fbox{$\semlink\in\Ctx\rightarrow\Ctx\rightarrow\pset(\Ctx)$}                                                                                                                               \\
  \ctx_0\semlink\bullet\triangleq                        & \: \{\bullet\}                                                                                                                     \\
  \ctx_0\semlink(x,\ell)\cons\ctx\triangleq              & \: \{(x,\ell)\cons\ctx_+|\ctx_+\in\ctx_0\semlink\ctx\}                                                                             \\
  \ctx_0\semlink(x,w)\cons\ctx\triangleq                 & \: \{(x,w_+)\cons\ctx_+|w_+\in\ctx_0\semlink w,\ctx_+\in\ctx_0\semlink\ctx\}                                                       \\
  \ctx_0\semlink[E]\triangleq                            & \: \{\ctx_+|\ctx_+\in\ctx_0\semlink\event\}\cup\{[\event_+]|\event_+\in\ctx_0\semlink\event\}                                      \\
  \fbox{$\semlink\in\Ctx\rightarrow\Value\rightarrow\pset(\Value)$}                                                                                                                           \\
  \ctx_0\semlink\langle\lambda x.e,\ctx\rangle\triangleq & \: \{\langle\lambda x.e,\ctx_+\rangle|\ctx_+\in\ctx_0\semlink\ctx\}                                                                \\
  \fbox{$\semlink\in\Ctx\rightarrow\Walue\rightarrow\pset(\Walue)$}                                                                                                                           \\
  \ctx_0\semlink\mu\ell.v\triangleq                      & \: \{\mu\ell'.v_+|\ell'\not\in\FLoc(v)\cup\FLoc(\ctx_0), v_+\in\ctx_0\semlink v[\ell'/\ell]\}
\end{align*}
Concrete linking makes sense because of the following theorem.
First define:
\[\eval(e,\ctx)\triangleq\{v|\ctx\vdash e\Downarrow v\}\qquad\eval(e,\Sigma)\triangleq\bigcup_{\ctx\in\Sigma}\eval(e,\ctx)\qquad\Sigma_0\semlink W\triangleq\bigcup_{\substack{\ctx_0\in\Sigma_0\\w\in W}}(\ctx_0\semlink w)\]
Then the following holds:
\begin{thm}[Advance]\label{thm:linksound}
  Given $e\in\Expr,\Sigma_0,\Sigma\subseteq\Ctx$,
  \[\eval(e,\Sigma_0\semlink\Sigma)\subseteq\Sigma_0\semlink\eval(e,\Sigma)\]
\end{thm}

The proof of Theorem \ref{thm:linksound} uses some useful lemmas, such as:
\begin{lem}[Linking distributes under substitution]
  Let $\ctx_0$ be the external environment that is linked with weak values $w$ and $u$.
  For all $\ell\not\in\FLoc(\ctx_0)$, we have:
  \[\forall w_+,u_+:w_+\in\ctx_0\semlink w\wedge u_+\in\ctx_0\semlink u\Rightarrow w_+[u_+/\ell]\in\ctx_0\semlink w[u/\ell]\]
\end{lem}
\begin{lem}[Linking is compatible with reads]
  Let $\ctx_0$ be the external environment that is linked with some environment $\ctx$.
  Let $w$ be the value obtained from reading $x$ from $\ctx$.
  Let $\text{unfold}:\Walue\rightarrow\Value$ be defined as:
  \[\text{unfold}(\mu\ell.v)\triangleq v[\mu\ell.v/\ell]\qquad\text{unfold}(v)\triangleq v\]
  Then for all $\ctx_+\in\ctx_0\semlink\ctx$, we have:
  \[\exists w_+\in\Walue:\ctx_+(x)=w_+\land\text{unfold}(w_+)\in\ctx_0\semlink\text{unfold}(w)\]
\end{lem}

Now we can formulate modular analysis.
A modular analysis consists of two requirements: an abstraction for the semantics with events and an abstraction for the semantic linking operator.
\begin{thm}[Modular analysis]
  Assume:
  \begin{enumerate}
    \item An abstract domain $\Abs\Walue$ that is concretized by a monotonic $\gamma\in\Abs\Walue\rightarrow\pset(\Walue)$
    \item A sound $\Abs\eval$: $\Sigma_0\subseteq\gamma(\Abs\ctx_0)\Rightarrow\eval(e,\Sigma_0)\subseteq\gamma(\Abs\eval(e,\Abs\ctx_0))$
    \item A sound $\Abs\semlink$: $\Sigma_0\subseteq\gamma(\Abs\ctx_0)\text{ and }W\subseteq\gamma(\Abs{w})\Rightarrow\Sigma_0\semlink W\subseteq\gamma(\Abs\ctx_0\Abs\semlink\Abs{w})$
  \end{enumerate}
  then we have:
  \[\Sigma_0\subseteq\gamma(\Abs\ctx_0)\text{ and }\Sigma\subseteq\gamma(\Abs\ctx)\Rightarrow\eval(e,\Sigma_0\semlink\Sigma)\subseteq\gamma(\Abs\ctx_0\Abs\semlink\Abs\eval(e,\Abs\ctx))\]
\end{thm}
\begin{cor}[Modular analysis of linked program]
  \[\Sigma_0\subseteq\gamma(\Abs\ctx_0)\text{ and }[\InitE]\in\gamma(\Abs\InitE)\Rightarrow\eval(e_1\synlink e_2,\Sigma_0)\subseteq\gamma(\Abs\eval(e_1,\Abs\ctx_0)\Abs\semlink\Abs\eval(e_2,\Abs\InitE))\]
\end{cor}
