\section{Generating and Resolving Shadows}
Now we formulate the semantics for generating shadows.

\begin{figure}[h!]
  \centering
  \small
  \begin{tabular}{rrcll}
    Shadow      & $\event$ & $\rightarrow$ & $\InitE$           & initial environment \\
                &          & $\vbar$       & $\ReadE(\event,x)$ & read shadow         \\
                &          & $\vbar$       & $\CallE(\event,v)$ & call shadow         \\
    Environment & $\ctx$   & $\rightarrow$ & $\cdots$                                 \\
                &          & $\vbar$       & $[\event]$         & answer to an shadow \\
    Value       & $v$      & $\rightarrow$ & $\cdots$                                 \\
                &          & $\vbar$       & $\event$           & answer to an shadow
  \end{tabular}
  \caption{Definition of the semantic domains with shadows. All other semantic domains are equal to Figure \ref{fig:domain}.}
  \label{fig:shadowdomain}
\end{figure}

We extend how to read weak values given an environment.
\begin{align*}
  \bullet(x)  & \triangleq\bot             &  &  & ((x',\ell)\cons\ctx)(x) & \triangleq (x=x'\;?\;\ell:\ctx(x)) \\
  [\event](x) & \triangleq\ReadE(\event,x) &  &  & ((x',w)\cons\ctx)(x)    & \triangleq (x=x'\;?\;w:\ctx(x))    \\
\end{align*}

Then we need to add only three rules to the semantics in Figure \ref{fig:bigstep} for the semantics to incorporate shadows.
{\small
\[
  \infer[LinkShadow]
  {
  {\evjudg{\ctx}{e_1}{\event}}\\
  {\evjudg{[\event]}{e_2}{v}}
  }
  {\evjudg{\ctx}{e_1\synlink e_2}{v}}
  \quad
  \infer[AppShadow]
  {
  {\evjudg{\ctx}{e_1}{\event}}\\
  {\evjudg{\ctx}{e_2}{v}}
  }
  {\evjudg{\ctx}{e_1\:e_2}{\CallE(\event,v)}}
\]
\[
  \infer[BindShadow]
  {
  {\ell\not\in\FLoc(\ctx)}\\
  {\evjudg{(x,\ell)\cons\ctx}{e_1}{v_1}}\\
  {\evjudg{(x, \mu\ell.v_1)\cons \ctx}{e_1}{\event_2}}
  }
  {\evjudg{\ctx}{x\:\texttt{=}\:e_1; e_2}{(x,\mu\ell.v_1)\cons[\event_2]}}
\]
}

Now we need to formulate the \emph{concrete linking} rules.
The concrete linking rule $\ctx_0\semlink w$, given an answer $\ctx_0$ to the $\InitE$ shadow, resolves all shadows within $w$ to obtain a set of final results.

\begin{align*}
                                                         & \pushright{\fbox{$\semlink\in\Ctx\rightarrow\Event\rightarrow\pset{\Value}$}}                                                      \\
  \ctx_0\semlink\InitE\triangleq                         & \: \{\ctx_0\}                                                                                                                      \\
  \ctx_0\semlink\ReadE(\event,x)\triangleq               & \: \{v_+|\ctx_+\in\ctx_0\semlink\event,\ctx_+(x)=v_+\}                                                                             \\
  \cup                                                   & \: \{v_+[\mu\ell.v_+/\ell]|\ctx_+\in\ctx_0\semlink E,\ctx_+(x)=\mu\ell.v_+\}                                                       \\
  \ctx_0\semlink\CallE(\event,v)\triangleq               & \: \{v_+'|\langle\lambda x.e,\ctx_+\rangle\in\ctx_0\semlink E, v_+\in\ctx_0\semlink v, (x,v_+)\cons\ctx_+\vdash e\Downarrow v_+'\} \\
  \cup                                                   & \: \{\CallE(\event_+,v_+)|\event_+\in\ctx_0\semlink E, v_+\in\ctx_0\semlink v\}                                                    \\
                                                         & \pushright{\fbox{$\semlink\in\Ctx\rightarrow\Ctx\rightarrow\pset{\Ctx}$}}                                                          \\
  \ctx_0\semlink\bullet\triangleq                        & \: \{\bullet\}                                                                                                                     \\
  \ctx_0\semlink(x,\ell)\cons\ctx\triangleq              & \: \{(x,\ell)\cons\ctx_+|\ctx_+\in\ctx_0\semlink\ctx\}                                                                             \\
  \ctx_0\semlink(x,w)\cons\ctx\triangleq                 & \: \{(x,w_+)\cons\ctx_+|w_+\in\ctx_0\semlink w,\ctx_+\in\ctx_0\semlink\ctx\}                                                       \\
  \ctx_0\semlink[\event]\triangleq                       & \: \{\ctx_+|\ctx_+\in\ctx_0\semlink\event\}\cup\{[\event_+]|\event_+\in\ctx_0\semlink\event\}                                      \\
                                                         & \pushright{\fbox{$\semlink\in\Ctx\rightarrow\Value\rightarrow\pset{\Value}$}}                                                      \\
  \ctx_0\semlink\langle\lambda x.e,\ctx\rangle\triangleq & \: \{\langle\lambda x.e,\ctx_+\rangle|\ctx_+\in\ctx_0\semlink\ctx\}                                                                \\
                                                         & \pushright{\fbox{$\semlink\in\Ctx\rightarrow\Walue\rightarrow\pset{\Walue}$}}                                                      \\
  \ctx_0\semlink\mu\ell.v\triangleq                      & \: \{\mu\ell'.v_+|\ell'\not\in\FLoc(v)\cup\FLoc(\ctx_0), v_+\in\ctx_0\semlink v[\ell'/\ell]\}
\end{align*}
Concrete linking makes sense because of the following theorem.
First define:
\[
  \eval(e,\ctx)\triangleq\{v|\ctx\vdash e\Downarrow v\}\quad
  \eval(e,\Sigma)\triangleq\bigcup_{\ctx\in\Sigma}\eval(e,\ctx)\quad
  \Sigma_0\semlink W\triangleq\bigcup_{\substack{\ctx_0\in\Sigma_0\\w\in W}}(\ctx_0\semlink w)
\]
Then the following holds:
\begin{thm}[Advance]\label{thm:linksound}
  Given $e\in\Expr,\Sigma_0,\Sigma\subseteq\Ctx$,
  \[\eval(e,\Sigma_0\semlink\Sigma)\subseteq\Sigma_0\semlink\eval(e,\Sigma)\]
\end{thm}

%The proof of Theorem \ref{thm:linksound} uses some useful lemmas, such as:
%\begin{lem}[Linking distributes under substitution]
%  Let $\ctx_0$ be the external environment that is linked with weak values $w$ and $u$.
%  For all $\ell\not\in\FLoc(\ctx_0)$, we have:
%  \[\forall w_+,u_+:w_+\in\ctx_0\semlink w\wedge u_+\in\ctx_0\semlink u\Rightarrow w_+[u_+/\ell]\in\ctx_0\semlink w[u/\ell]\]
%\end{lem}
%\begin{lem}[Linking is compatible with reads]
%  Let $\textnormal{unroll}:\Walue\rightarrow\Value$ be defined as:
%  \[\textnormal{unroll}(\mu\ell.v)\triangleq v[\mu\ell.v/\ell]\qquad\textnormal{unroll}(v)\triangleq v\]
%  For $\ctx_+\in\ctx_0\semlink\ctx$, if $\ctx_+(x)=w_+$, we have:
%  \[\textnormal{There exists } w\textnormal{ such that }\ctx(x)=w\textnormal{ and }\textnormal{unroll}(w_+)\in\ctx_0\semlink\textnormal{unroll}(w)\]
%\end{lem}

Now we can formulate modular analysis.
A modular analysis consists of two requirements: an abstraction for the semantics with shadows and an abstraction for the semantic linking operator.
\begin{thm}[Modular analysis]
  Assume:
  \begin{enumerate}
    \item An abstract domain $\Abs\Walue$ and a monotonic $\gamma\in\Abs\Walue\rightarrow\pset{\Walue}$
    \item A sound $\Abs\eval$: $\Sigma_0\subseteq\gamma(\Abs\ctx_0)\Rightarrow\eval(e,\Sigma_0)\subseteq\gamma(\Abs\eval(e,\Abs\ctx_0))$
    \item A sound $\Abs\semlink$: $\Sigma_0\subseteq\gamma(\Abs\ctx_0)\textnormal{ and }W\subseteq\gamma(\Abs{w})\Rightarrow\Sigma_0\semlink W\subseteq\gamma(\Abs\ctx_0\Abs\semlink\Abs{w})$
  \end{enumerate}
  then we have:
  \[\Sigma_0\subseteq\gamma(\Abs\ctx_0)\textnormal{ and }\Sigma\subseteq\gamma(\Abs\ctx)\Rightarrow\eval(e,\Sigma_0\semlink\Sigma)\subseteq\gamma(\Abs\ctx_0\Abs\semlink\Abs\eval(e,\Abs\ctx))\]
\end{thm}
\begin{cor}[Modular analysis of linked program]
  \begin{align*}
    \Sigma_0\subseteq\gamma(\Abs\ctx_0)\textnormal{ and } & [\InitE]\in\gamma(\Abs\InitE)\Rightarrow                                                                       \\
                                                          & \eval(e_1\synlink e_2,\Sigma_0)\subseteq\gamma(\Abs\eval(e_1,\Abs\ctx_0)\Abs\semlink\Abs\eval(e_2,\Abs\InitE))
  \end{align*}
\end{cor}
