\section{Syntax and Semantics}
\subsection{Abstract Syntax}
\begin{figure}[h!]
  \centering
  \small
  \begin{tabular}{rrcll}
    Identifiers & $x$ & $\in$         & $\ExprVar$                                                      \\
    Expression  & $e$ & $\rightarrow$ & $x$ $\vbar$ $\lambda x.e$ $\vbar$ $e$ $e$ & $\lambda$-calculus  \\
                &     & $\vbar$       & $e\synlink e$                             & linked expression   \\
                &     & $\vbar$       & $\varepsilon$                             & empty module        \\
                &     & $\vbar$       & $x\:\texttt{=}\:e$ ; $e$                  & (recursive) binding
  \end{tabular}
  \caption{Abstract syntax of the language.}
  \label{fig:syntax}
\end{figure}
\subsection{Operational Semantics}
\begin{figure}[h!]
  \centering
  \small
  \begin{tabular}{rrcll}
    Environment & $\ctx$ & $\in$         & $\Ctx\triangleq\{\bullet\}+\ExprVar\times(\Loc+\Walue)\times\Ctx$  \\
    Location    & $\ell$ & $\in$         & $\Loc$                                                             \\
    Value       & $v$    & $\in$         & $\Value \triangleq\Ctx+\ExprVar\times\Expr\times\Ctx$              \\
    Weak Value  & $w$    & $\in$         & $\Walue\triangleq\Value+\underline\Value$                          \\
    Environment & $\ctx$ & $\rightarrow$ & $\bullet$                             \hfill empty stack           \\
                &        & $\vbar$       & $(x,w)\cons \ctx$                     \hfill weak value binding    \\
                &        & $\vbar$       & $(x,\ell)\cons\ctx$                   \hfill free location binding \\
    Value       & $v$    & $\rightarrow$ & $\ctx$                                \hfill exported environment  \\
                &        & $\vbar$       & $\langle \lambda x.e, \ctx \rangle$   \hfill closure               \\
    Weak Value  & $w$    & $\rightarrow$ & $v$                                   \hfill value                 \\
                &        & $\vbar$       & $\mu\ell.v$                           \hfill recursive value
  \end{tabular}
  \caption{Definition of the semantic domains.}
  \label{fig:domain}
\end{figure}
\begin{figure}[h!]
  \small
  \begin{flushright}
    \fbox{$\evjudg{\ctx}{e}{v}$}
  \end{flushright}
  \centering
  \vspace{0pt} % -0.75em}
  \[
    \infer[Id]
    {\ctx(x)=v}
    {\evjudg{\ctx}{x}{v}}
    \quad
    \infer[RecId]
    {\ctx(x)=\mu\ell.v}
    {\evjudg{\ctx}{x}{v[\mu\ell.v/\ell]}}
    \quad
    \infer[Fn]
    {\:}
    {\evjudg{\ctx}{\lambda x.e}{\langle\lambda x.e, \ctx\rangle}}
  \]

  \[
    \infer[App]
    {
    {\evjudg{\ctx}{e_1}{\langle\lambda x.e, \ctx_1\rangle}}\\
    {\evjudg{\ctx}{e_2}{v_2}}\\
    {\evjudg{(x, v_2)\cons \ctx_1}{e}{v}}
    }
    {\evjudg{\ctx}{e_1\:e_2}{v}}
  \]

  \[
    \infer[Link]
    {
    {\evjudg{\ctx}{e_1}{\ctx_1}}\\
    {\evjudg{\ctx_1}{e_2}{v}}
    }
    {\evjudg{\ctx}{{e_1}\synlink{e_2}}{v}}
    \quad
    \infer[Empty]
    {\:}
    {\evjudg{\ctx}{\varepsilon}{\bullet}}
  \]

  \[
    \infer[Bind]
    {
    {\ell\not\in\FLoc(\ctx)}\\
    {\evjudg{(x,\ell)\cons\ctx}{e_1}{v_1}}\\
    {\evjudg{(x, \mu\ell.v_1)\cons \ctx}{e_1}{\ctx_2}}
    }
    {\evjudg{\ctx}{x\:\texttt{=}\:e_1; e_2}{(x,\mu\ell.v_1)\cons\ctx_2}}
  \]
  \caption{The big-step operational semantics.}
  \label{fig:bigstep}
\end{figure}
The big-step operational semantics is \emph{deterministic} up to $\alpha$-equivalence.

\subsection{Reconciling with Conventional Backpatching}
\begin{figure}[h!]
  \centering
  \small
  \begin{tabular}{rrcll}
    Environment   & $\ctx$ & $\in$         & $\Mtx\triangleq\fin{\ExprVar}{\Loc}$                            \\
    Memory        & $\mem$ & $\in$         & $\Mem\triangleq\fin{\Loc}{\Malue}$                              \\
    Allocated set & $L$    & $\subseteq$   & $\Loc$                                                          \\
    Value         & $v$    & $\in$         & $\Malue \triangleq\Mtx+\ExprVar\times\Expr\times\Mtx$           \\
    Environment   & $\ctx$ & $\rightarrow$ & $\bullet$                           \hfill empty stack          \\
                  &        & $\vbar$       & $(x,\ell)\cons\ctx$                 \hfill location binding     \\
    Value         & $v$    & $\rightarrow$ & $\ctx$                              \hfill exported environment \\
                  &        & $\vbar$       & $\langle \lambda x.e, \ctx \rangle$ \hfill closure
  \end{tabular}
  \caption{Definition of the semantic domains with memory.}
  \label{fig:memdomain}
\end{figure}
\begin{figure}[h!]
  \small
  \begin{flushright}
    \fbox{$\evjudg{\ctx,\mem,L}{e}{v,\mem',L'}$}
  \end{flushright}
  \centering
  \vspace{0pt} % -0.75em}
  \[
    \infer[Id]
    {
    {\ctx(x)=\ell}\\
    {\mem(\ell)=v}
    }
    {\evjudg{\ctx,\mem,L}{x}{v,\mem,L}}
    \quad
    \infer[Fn]
    {\:}
    {\evjudg{\ctx,\mem,L}{\lambda x.e}{\langle\lambda x.e,\ctx\rangle,\mem,L}}
  \]

  \[
    \infer[App]
    {
    {\evjudg{\ctx,\mem,L}{e_1}{\langle\lambda x.e,\ctx_1\rangle,\mem_1,L_1}}\\
    {\evjudg{\ctx,\mem_1,L_1}{e_2}{v_2,\mem_2,L_2}}\\
    {\ell\not\in\dom(\mem_2)\cup L_2}\\
    {\evjudg{(x,\ell)\cons\ctx_1,\mem_2[\ell\mapsto v_2],L_2}{e}{v,\mem',L'}}
    }
    {\evjudg{\ctx,\mem,L}{e_1\:e_2}{v,\mem',L'}}
  \]

  \[
    \infer[Link]
    {
    {\evjudg{\ctx,\mem,L}{e_1}{\ctx_1,\mem_1,L_1}}\\
    {\evjudg{\ctx_1,\mem_1,L_1}{e_2}{v,\mem',L'}}
    }
    {\evjudg{\ctx,\mem,L}{{e_1}\synlink{e_2}}{v,\mem',L'}}
    \quad
    \infer[Empty]
    {\:}
    {\evjudg{\ctx,\mem,L}{\varepsilon}{\bullet,\mem,L}}
  \]

  \[
    \infer[Bind]
    {
    {\ell\not\in\dom(\mem)\cup L}\\
    {\evjudg{(x,\ell)\cons\ctx,\mem,L\cup\{\ell\}}{e_1}{v_1,\mem_1,L_1}}\\
    {\evjudg{(x,\ell)\cons\ctx,\mem_1[\ell\mapsto v_1],L_1}{e_2}{\ctx_2,\mem',L'}}
    }
    {\evjudg{\ctx,\mem,L}{x=e_1;e_2}{(x,\ell)\cons\ctx_2,\mem',L'}}
  \]
  \caption{The big-step operational semantics with memory.}
  \label{fig:membigstep}
\end{figure}
\begin{figure}[h!]
  \centering
  \small
  \begin{flushright}
    \fbox{$w\sim_f v,\mem$}
  \end{flushright}
  \[
    \infer[Eq-Nil]
    {\:}
    {\bullet\sim_f\bullet}
    \quad
    \infer[Eq-ConsFree]
    {
    {\ell\not\in\dom(f)}\\
    {\ell\not\in\dom(\mem)}\\
    {\ctx\sim_f\ctx'}
    }
    {(x,\ell)\cons\ctx\sim_f(x,\ell)\cons\ctx'}
  \]

  \[
    \infer[Eq-ConsBound]
    {
    {f(\ell)=\ell'}\\
    {\ell'\in\dom(\mem)}\\
    {\ctx\sim_f\ctx'}
    }
    {(x,\ell)\cons\ctx\sim_f(x,\ell')\cons\ctx'}
  \]

  \[
    \infer[Eq-ConsWVal]
    {
    {\mem(\ell')=v'}\\
    {w\sim_f v'}\\
    {\ctx\sim_f\ctx'}
    }
    {(x,w)\cons\ctx\sim_f(x,\ell')\cons\ctx'}
    \quad
    \infer[Eq-Clos]
    {\ctx\sim_f\ctx'}
    {\langle\lambda x.e,\ctx\rangle\sim_f\langle\lambda x.e,\ctx'\rangle}
  \]

  \[
    \infer[Eq-Rec]
    {
    {L\text{ finite}}\\
    {\mem(\ell')=v'}\\
    {\forall\nu\not\in L,\:v[\nu/\ell]\sim_{f[\nu\mapsto\ell']}v'}
    }
    {\mu\ell.v\sim_f v'}
  \]
  \caption{The equivalence relation between weak values in the original semantics and values in the semantics with memory.
    $f\in\fin{\Loc}{\Loc}$ tells what the free locations in $w$ that were \emph{opened} should be mapped to in memory.
    $\mem$ is omitted for brevity.}
  \label{fig:equivrel}
\end{figure}


The semantics in Figure \ref{fig:bigstep} makes sense due to similarity with a conventional backpatching semantics as presented in Figure \ref{fig:membigstep}.
We have defined a relation $\sim$ that satisfies:
\[\sim\subseteq\Walue\times(\Malue\times\Mem\times\pset{\Loc})\qquad\bullet\sim(\bullet,\varnothing,\varnothing)\]
The following theorem holds:
\begin{thm}[Equivalence of semantics]\label{thm:equivsem}
  For all $\ctx\in\Ctx,\ctx'\in\Mtx\times\Mem\times\pset{\Loc},v\in\Value,v'\in\Malue\times\Mem\times\pset{\Loc}$, we have:
  \begin{align*}
    \ctx\sim\ctx'\text{ and }\evjudg{\ctx}{e}{v}   & \Rightarrow\exists v':v\sim v'\text{ and }\evjudg{\ctx'}{e}{v'} \\
    \ctx\sim\ctx'\text{ and }\evjudg{\ctx'}{e}{v'} & \Rightarrow\exists v:v\sim v'\text{ and }\evjudg{\ctx}{e}{v}
  \end{align*}
\end{thm}

The definition for $w\sim(\ctx,\mem,L)$ is:
\[w\sim_\bot(\ctx,\mem)\text{ and }\FLoc(w)\subseteq L\]
where the definition for $\sim_f$ is given in Figure \ref{fig:equivrel}.

