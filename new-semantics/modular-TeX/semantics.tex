\section{Syntax and Semantics}
\subsection{Abstract Syntax}
\begin{figure}[htb]
	\centering
	\small
	\begin{tabular}{rrcll}
		Identifiers & $x$ & $\in$         & $\ExprVar$                                                      \\
		Expression  & $e$ & $\rightarrow$ & $x$ $\vbar$ $\lambda x.e$ $\vbar$ $e$ $e$ & $\lambda$-calculus  \\
		            &     & $\vbar$       & $e\synlink e$                             & linked expression   \\
		            &     & $\vbar$       & $\varepsilon$                             & empty module        \\
		            &     & $\vbar$       & $x\:\texttt{=}\:e$ ; $e$                  & (recursive) binding
	\end{tabular}
	\caption{Abstract syntax of the language.}
	\label{fig:syntax}
\end{figure}
\subsection{Operational Semantics}
\begin{figure}[h!]
	\centering
	\small
	\begin{tabular}{rrcll}
		Environment & $\ctx$ & $\in$         & $\Ctx$                                                                        \\
		Location    & $\ell$ & $\in$         & $\Loc$                                                                        \\
		Value       & $v$    & $\in$         & $\Value \triangleq\Ctx+\ExprVar\times\Expr\times\Ctx$                         \\
		Weak Value  & $w$    & $\in$         & $\Walue\triangleq\Value+\underline\Value$                                     \\
		Environment & $\ctx$ & $\rightarrow$ & $\bullet$                                             & empty stack           \\
		            &        & $\vbar$       & $(x,w)\cons \ctx$                                     & weak value binding    \\
		            &        & $\vbar$       & $(x,\ell)\cons\ctx$                                   & free location binding \\
		Value       & $v$    & $\rightarrow$ & $\ctx$                                                & exported environment  \\
		            &        & $\vbar$       & $\langle \lambda x.e, \ctx \rangle$                   & closure               \\
		Weak Value  & $w$    & $\rightarrow$ & $v$                                                   & value                 \\
		            &        & $\vbar$       & $\mu\ell.v$                                           & recursive value
	\end{tabular}
	\caption{Definition of the semantic domains.}
	\label{fig:domain}
\end{figure}
\begin{figure}[h!]
	\small
	\begin{flushright}
		\fbox{$\evjudg{\ctx}{e}{v}$}
	\end{flushright}
	\centering
	\vspace{0pt} % -0.75em}
	\[
		\begin{InfRule}{Id}
			\hypo{\ctx(x)=v}
			\infer1{\evjudg{\ctx}{x}{v}}
		\end{InfRule}\:
		\begin{InfRule}{RecId}
			\hypo{\ctx(x)=\mu\ell.v}
			\infer1{\evjudg{\ctx}{x}{v[\mu\ell.v/\ell]}}
		\end{InfRule}\:
		\begin{InfRule}{Fn}
			\infer0{\evjudg{\ctx}{\lambda x.e}{\langle\lambda x.e, \ctx\rangle}}
		\end{InfRule}\:
		\begin{InfRule}{App}
			\hypo{\evjudg{\ctx}{e_1}{\langle\lambda x.e, \ctx_1\rangle}}
			\hypo{\evjudg{\ctx}{e_2}{v_2}}
			\hypo{\evjudg{(x, v_2)\cons \ctx_1}{e}{v}}
			\infer{2,1}{\evjudg{\ctx}{e_1\:e_2}{v}}
		\end{InfRule}
	\]

	\[
		\begin{InfRule}{Link}
			\hypo{\evjudg{\ctx}{e_1}{\ctx_1}}
			\hypo{\evjudg{\ctx_1}{e_2}{v}}
			\infer2{\evjudg{\ctx}{{e_1}\synlink{e_2}}{v}}
		\end{InfRule}\quad
		\begin{InfRule}{Empty}
			\infer0{\evjudg{\ctx}{\varepsilon}{\bullet}}
		\end{InfRule}\quad
		\begin{InfRule}{Bind}
			\hypo{\ell\not\in\FLoc(\ctx)}
			\hypo{\evjudg{(x,\ell)\cons\ctx}{e_1}{v_1}}
			\hypo{\evjudg{(x, \mu\ell.v_1)\cons \ctx}{e_1}{\ctx_2}}
			\infer{2,1}{\evjudg{\ctx}{x\:\texttt{=}\:e_1; e_2}{(x,\mu\ell.v_1)\cons\ctx_2}}
		\end{InfRule}
	\]
	\caption{The big-step operational semantics.}
	\label{fig:bigstep}
\end{figure}
The big-step operational semantics is \emph{deterministic} up to $\alpha$-equivalence.

\subsection{Reconciling with Conventional Backpatching}
\begin{figure}[h!]
	\centering
	\small
	\begin{tabular}{rrcll}
		Environment   & $\ctx$ & $\in$         & $\Mtx\triangleq\fin{\ExprVar}{\Loc}$                                         \\
		Memory        & $\mem$ & $\in$         & $\Mem\triangleq\fin{\Loc}{\Malue}$                                           \\
		Allocated set & $L$    & $\subseteq$   & $\Loc$                                                                       \\
		Value         & $v$    & $\in$         & $\Malue \triangleq\Mtx+\ExprVar\times\Expr\times\Mtx$                        \\
		Environment   & $\ctx$ & $\rightarrow$ & $\bullet$                                             & empty stack          \\
		              &        & $\vbar$       & $(x,\ell)\cons\ctx$                                   & location binding     \\
		Value         & $v$    & $\rightarrow$ & $\ctx$                                                & exported environment \\
		              &        & $\vbar$       & $\langle \lambda x.e, \ctx \rangle$                   & closure
	\end{tabular}
	\caption{Definition of the semantic domains with memory.}
	\label{fig:memdomain}
\end{figure}
\begin{figure}[h!]
	\small
	\begin{flushright}
		\fbox{$\evjudg{\ctx,\mem,L}{e}{v,\mem',L'}$}
	\end{flushright}
	\centering
	\vspace{0pt} % -0.75em}
	\[
		\begin{InfRule}{Id}
			\hypo{\ctx(x)=\ell}
			\hypo{\mem(\ell)=v}
			\infer2{\evjudg{\ctx,\mem,L}{x}{v,\mem,L}}
		\end{InfRule}\:
		\begin{InfRule}{Fn}
			\infer0{\evjudg{\ctx,\mem,L}{\lambda x.e}{\langle\lambda x.e,\ctx\rangle,\mem,L}}
		\end{InfRule}
	\]

	\[
		\begin{InfRule}{App}
			\hypo{\evjudg{\ctx,\mem,L}{e_1}{\langle\lambda x.e,\ctx_1\rangle,\mem_1,L_1}}
			\hypo{\evjudg{\ctx,\mem_1,L_1}{e_2}{v_2,\mem_2,L_2}}
			\hypo{\ell\not\in\dom(\mem_2)\cup L_2}
			\hypo{\evjudg{(x,\ell)\cons\ctx_1,\mem_2[\ell\mapsto v_2],L_2}{e}{v,\mem',L'}}
			\infer{3,1}{\evjudg{\ctx,\mem,L}{e_1\:e_2}{v,\mem',L'}}
		\end{InfRule}
	\]

	\[
		\begin{InfRule}{Link}
			\hypo{\evjudg{\ctx,\mem,L}{e_1}{\ctx_1,\mem_1,L_1}}
			\hypo{\evjudg{\ctx_1,\mem_1,L_1}{e_2}{v,\mem',L'}}
			\infer2{\evjudg{\ctx,\mem,L}{{e_1}\synlink{e_2}}{v,\mem',L'}}
		\end{InfRule}\quad
		\begin{InfRule}{Empty}
			\infer0{\evjudg{\ctx,\mem,L}{\varepsilon}{\bullet,\mem,L}}
		\end{InfRule}
	\]

	\[
		\begin{InfRule}{Bind}
			\hypo{\ell\not\in\dom(\mem)\cup L}
			\hypo{\evjudg{(x,\ell)\cons\ctx,\mem,L\cup\{\ell\}}{e_1}{v_1,\mem_1,L_1}}
			\hypo{\evjudg{(x,\ell)\cons\ctx,\mem_1[\ell\mapsto v_1],L_1}{e_2}{\ctx_2,\mem',L'}}
			\infer{2,1}{\evjudg{\ctx,\mem,L}{x=e_1;e_2}{(x,\ell)\cons\ctx_2,\mem',L'}}
		\end{InfRule}
	\]
	\caption{The big-step operational semantics with memory.}
	\label{fig:membigstep}
\end{figure}
\begin{figure}[h!]
	\centering
	\small
	\begin{flushright}
		\fbox{$w\sim_f v,\mem$}
	\end{flushright}
	\[
		\begin{InfRule}{Eq-Nil}
			\infer0{\bullet\sim_f\bullet}
		\end{InfRule}\:
		\begin{InfRule}{Eq-ConsFree}
			\hypo{\ell\not\in\dom(f)}
			\hypo{\ell\not\in\dom(\mem)}
			\hypo{\ctx\sim_f\ctx'}
			\infer3{(x,\ell)\cons\ctx\sim_f(x,\ell)\cons\ctx'}
		\end{InfRule}\:
		\begin{InfRule}{Eq-ConsBound}
			\hypo{f(\ell)=\ell'}
			\hypo{\ell'\in\dom(\mem)}
			\hypo{\ctx\sim_f\ctx'}
			\infer3{(x,\ell)\cons\ctx\sim_f(x,\ell')\cons\ctx'}
		\end{InfRule}
	\]

	\[
		\begin{InfRule}{Eq-ConsWVal}
			\hypo{\mem(\ell')=v'}
			\hypo{w\sim_f v'}
			\hypo{\ctx\sim_f\ctx'}
			\infer3{(x,w)\cons\ctx\sim_f(x,\ell')\cons\ctx'}
		\end{InfRule}\:
		\begin{InfRule}{Eq-Clos}
			\hypo{\ctx\sim_f\ctx'}
			\infer1{\langle\lambda x.e,\ctx\rangle\sim_f\langle\lambda x.e,\ctx'\rangle}
		\end{InfRule}\:
		\begin{InfRule}{Eq-Rec}
			\hypo{L\text{ finite}}
			\hypo{\mem(\ell')=v'}
			\hypo{\forall\nu\not\in L,\:v[\nu/\ell]\sim_{f[\nu\mapsto\ell']}v'}
			\infer3{\mu\ell.v\sim_f v'}
		\end{InfRule}
	\]
	\caption{The equivalence relation between weak values in the original semantics and values in the semantics with memory.
		$f\in\fin{\Loc}{\Loc}$ tells what the free locations in $w$ that were \emph{opened} should be mapped to in memory.
		$\mem$ is omitted for brevity.}
	\label{fig:equivrel}
\end{figure}


The semantics in Figure \ref{fig:bigstep} makes sense due to similarity with a conventional backpatching semantics as presented in Figure \ref{fig:membigstep}.
We have defined a relation $\sim$ that satisfies:
\[\sim\subseteq\Walue\times(\Malue\times\Mem\times\pset(\Loc))\qquad\bullet\sim(\bullet,\varnothing,\varnothing)\]
The following theorem holds:
\begin{thm}[Equivalence of semantics]\label{thm:equivsem}
	For all $\ctx\in\Ctx,\ctx'\in\Mtx\times\Mem\times\pset(\Loc),v\in\Value,v'\in\Malue\times\Mem\times\pset(\Loc)$, we have:
	\begin{align*}
		\ctx\sim\ctx'\text{ and }\evjudg{\ctx}{e}{v}   & \Rightarrow\exists v':v\sim v'\text{ and }\evjudg{\ctx'}{e}{v'} \\
		\ctx\sim\ctx'\text{ and }\evjudg{\ctx'}{e}{v'} & \Rightarrow\exists v:v\sim v'\text{ and }\evjudg{\ctx}{e}{v}
	\end{align*}
\end{thm}

The definition for $w\sim(\ctx,\mem,L)$ is:
\[w\sim_\bot(\ctx,\mem)\text{ and }\FLoc(w)\subseteq L\]
where the definition for $\sim_f$ is given in Figure \ref{fig:equivrel}.

The proof of Theorem \ref{thm:equivsem} uses some useful lemmas, such as:
\begin{lem}[Free locations not in $f$ are free in memory]
	\[w\sim_f v',\mem\Rightarrow m|_{\FLoc(w)-\dom(f)}=\bot\]
\end{lem}

\begin{lem}[Equivalence is preserved by extension of memory]
	\[w\sim_f v',\mem\text{ and }m\sqsubseteq m'\text{ and }m'|_{\FLoc(w)-\dom(f)}=\bot\Rightarrow w\sim_f v',m\]
\end{lem}

\begin{lem}[Equivalence only cares about $f$ on free locations]
	\[w\sim_f v',\mem\text{ and }f|_{\FLoc(w)}=f|_{\FLoc(w)}\Rightarrow w\sim_{f'}v',m\]
\end{lem}

\begin{lem}[Extending equivalence on free locations]
	\[w\sim_f v',\mem\text{ and }\ell\not\in\dom(f)\text{ and }\ell\not\in\dom(\mem)\Rightarrow\forall u',w\sim_{f[\ell\mapsto\ell]}v',m[\ell\mapsto u']\]
\end{lem}

\begin{lem}[Substitution of values]
	\[w\sim_f v',\mem\text{ and }f(\ell)=\ell'\text{ and }\mem(\ell')=u'\text{ and }u\sim_{f-\ell}u',\mem\Rightarrow w[u/\ell]\sim_{f-\ell}v',\mem\]
\end{lem}

\begin{lem}[Substitution of locations]
	\[w\sim_f v',\mem\text{ and }\ell\in\dom(f)\text{ and }\nu\not\in\FLoc(w)\Rightarrow w[\nu/\ell]\sim_{f\circ(\nu\leftrightarrow\ell)}v',\mem\]
\end{lem}
