%! TEX program = xelatex
\documentclass{article}
\usepackage[a4paper]{geometry}
\geometry{
  top=20mm,
  bottom=20mm,
  left=20mm,
  right=20mm
}
\usepackage[nodisplayskipstretch]{setspace}
\linespread{1.25}
\usepackage{graphicx}
\usepackage{xcolor}
\usepackage{kotex}
\usepackage{csquotes}
\usepackage{tabularray}
\usepackage{relsize}
\usepackage{titlesec}

%%% Math settings
\usepackage{amssymb,amsmath,amsthm,mathtools}
\usepackage[math-style=TeX,bold-style=TeX]{unicode-math}
\theoremstyle{definition}
\newtheorem{definition}{Definition}[section]
\newtheorem{example}{Example}[section]
\newtheorem{lemma}{Lemma}[section]
\newtheorem{theorem}{Theorem}[section]
\newtheorem{corollary}{Corollary}[section]
\newtheorem{claim}{Claim}[section]

%%% Font settings
\setmainfont{Libertinus Serif}
\setsansfont{Libertinus Sans}[Scale=MatchUppercase]

\usepackage{galois}
\usepackage{ebproofx}
\ebproofxset{center=false}

\newcommand*{\defeq}{\triangleq}


\title{Modular Analysis}
\author{Joonhyup Lee}
\begin{document}
\maketitle
\section{Syntax and Semantics}
\subsection{Abstract Syntax}
\begin{figure}[htb]
  \centering
  \begin{tabular}{rrcll}
    Identifiers & $x$ & $\in$         & $\ExprVar$                                                      \\
    Expression  & $e$ & $\rightarrow$ & $x$ $\vbar$ $\lambda x.e$ $\vbar$ $e$ $e$ & $\lambda$-calculus  \\
                &     & $\vbar$       & $\link{e}{e}$                             & linked expression   \\
                &     & $\vbar$       & $\varepsilon$                             & empty module        \\
                &     & $\vbar$       & $x\texttt{=}e$ ; $e$                      & (recursive) binding
  \end{tabular}
  \caption{Abstract syntax of the language.}
  \label{fig:syntax}
\end{figure}
\subsection{Operational Semantics}
\begin{figure}[h!]
  \centering
  \begin{tabular}{rrcll}
    Environment     & $\ctx$ & $\in$         & $\Ctx$                                                                         \\
    Location        & $\ell$ & $\in$         & $\Loc$                                                                         \\
    de Bruijn Index & $n$    & $\in$         & $\mathbb{N}$                                                                   \\
    Value           & $v$    & $\in$         & $\Value \triangleq\Ctx+\ExprVar\times\Expr\times\Ctx$                          \\
    Weak Value      & $w$    & $\in$         & $\Walue\triangleq\Value+\underline\Value$                                      \\
    Environment     & $\ctx$ & $\rightarrow$ & $\bullet$                                             & empty stack            \\
                    &        & $\vbar$       & $(x,w)\cons \ctx$                                     & weak value binding     \\
                    &        & $\vbar$       & $(x,\ell)\cons\ctx$                                   & free location binding  \\
                    &        & $\vbar$       & $(x,n)\cons\ctx$                                      & bound location binding \\
    Value           & $v$    & $\rightarrow$ & $\ctx$                                                & exported environment   \\
                    &        & $\vbar$       & $\langle \lambda x.e, \ctx \rangle$                   & closure                \\
    Weak Value      & $w$    & $\rightarrow$ & $v$                                                   & value                  \\
                    &        & $\vbar$       & $\mu.v$                                               & recursive value
  \end{tabular}
  \caption{Definition of the semantic domains.}
  \label{fig:domain}
\end{figure}
\begin{figure}[h!]
  \begin{flushright}
    \fbox{$\evjudg{\ctx}{e}{v}$}
  \end{flushright}
  \centering
  \vspace{0pt} % -0.75em}
  \[
    \begin{InfRule}{Id}
      \hypo{\ctx(x)=v}
      \infer1{\evjudg{\ctx}{x}{v}}
    \end{InfRule}\:
    \begin{InfRule}{RecId}
      \hypo{\ctx(x)=\mu.v}
      \infer1{\evjudg{\ctx}{x}{\openloc{\mu.v}{v}}}
    \end{InfRule}\:
    \begin{InfRule}{Fn}
      \infer0{\evjudg{\ctx}{\lambda x.e}{\langle\lambda x.e, \ctx\rangle}}
    \end{InfRule}\:
    \begin{InfRule}{App}
      \hypo{\evjudg{\ctx}{e_1}{\langle\lambda x.e, \ctx_1\rangle}}
      \hypo{\evjudg{\ctx}{e_2}{v_2}}
      \hypo{\evjudg{(x, v_2)\cons \ctx_1}{e}{v}}
      \infer{2,1}{\evjudg{\ctx}{e_1\:e_2}{v}}
    \end{InfRule}
  \]

  \[
    \begin{InfRule}{Link}
      \hypo{\evjudg{\ctx}{e_1}{\ctx_1}}
      \hypo{\evjudg{\ctx_1}{e_2}{v}}
      \infer2{\evjudg{\ctx}{\link{e_1}{e_2}}{v}}
    \end{InfRule}\quad
    \begin{InfRule}{Empty}
      \infer0{\evjudg{\ctx}{\varepsilon}{\bullet}}
    \end{InfRule}\quad
    \begin{InfRule}{Bind}
      \hypo{\ell\not\in\FLoc(\ctx)}
      \hypo{\evjudg{(x,\ell)\cons\ctx}{e_1}{v_1}}
      \hypo{\evjudg{(x, \mu.\closeloc{\ell}{v_1})\cons \ctx}{e_1}{\ctx_2}}
      \infer{2,1}{\evjudg{\ctx}{x=e_1; e_2}{(x,\mu.\closeloc{\ell}{v_1})\cons\ctx_2}}
    \end{InfRule}
  \]
  \caption{The big-step operational semantics.}
  \label{fig:bigstep}
\end{figure}
We use the locally nameless representation, and enforce that all values be \emph{locally closed}.
As a consequence, the big-step operational semantics will be \emph{deterministic}, no matter what $\ell$ is chosen in the Bind rule.

\subsection{Reconciling with Conventional Backpatching}
\begin{figure}[h!]
  \centering
  \small
  \begin{tabular}{rrcll}
    Environment   & $\ctx$ & $\in$         & $\Mtx\triangleq\fin{\ExprVar}{\Loc}$                                         \\
    Memory        & $\mem$ & $\in$         & $\Mem\triangleq\fin{\Loc}{\Malue}$                                           \\
    Allocated set & $L$    & $\subseteq$   & $\Loc$                                                                       \\
    Value         & $v$    & $\in$         & $\Malue \triangleq\Mtx+\ExprVar\times\Expr\times\Mtx$                        \\
    Environment   & $\ctx$ & $\rightarrow$ & $\bullet$                                             & empty stack          \\
                  &        & $\vbar$       & $(x,\ell)\cons\ctx$                                   & location binding     \\
    Value         & $v$    & $\rightarrow$ & $\ctx$                                                & exported environment \\
                  &        & $\vbar$       & $\langle \lambda x.e, \ctx \rangle$                   & closure
  \end{tabular}
  \caption{Definition of the semantic domains with memory.}
  \label{fig:memdomain}
\end{figure}
\begin{figure}[h!]
  \small
  \begin{flushright}
    \fbox{$\evjudg{\ctx,\mem,L}{e}{v,\mem',L'}$}
  \end{flushright}
  \centering
  \vspace{0pt} % -0.75em}
  \[
    \begin{InfRule}{Id}
      \hypo{\ctx(x)=\ell}
      \hypo{\mem(\ell)=v}
      \infer2{\evjudg{\ctx,\mem,L}{x}{v,\mem,L}}
    \end{InfRule}\:
    \begin{InfRule}{Fn}
      \infer0{\evjudg{\ctx,\mem,L}{\lambda x.e}{\langle\lambda x.e,\ctx\rangle,\mem,L}}
    \end{InfRule}
  \]

  \[
    \begin{InfRule}{App}
      \hypo{\evjudg{\ctx,\mem,L}{e_1}{\langle\lambda x.e,\ctx_1\rangle,\mem_1,L_1}}
      \hypo{\evjudg{\ctx,\mem_1,L_1}{e_2}{v_2,\mem_2,L_2}}
      \hypo{\ell\not\in\dom(\mem_2)\cup L_2}
      \hypo{\evjudg{(x,\ell)\cons\ctx_1,\mem_2[\ell\mapsto v_2],L_2}{e}{v,\mem',L'}}
      \infer{3,1}{\evjudg{\ctx,\mem,L}{e_1\:e_2}{v,\mem',L'}}
    \end{InfRule}
  \]

  \[
    \begin{InfRule}{Link}
      \hypo{\evjudg{\ctx,\mem,L}{e_1}{\ctx_1,\mem_1,L_1}}
      \hypo{\evjudg{\ctx_1,\mem_1,L_1}{e_2}{v,\mem',L'}}
      \infer2{\evjudg{\ctx,\mem,L}{\link{e_1}{e_2}}{v,\mem',L'}}
    \end{InfRule}\quad
    \begin{InfRule}{Empty}
      \infer0{\evjudg{\ctx,\mem,L}{\varepsilon}{\bullet,\mem,L}}
    \end{InfRule}
  \]

  \[
    \begin{InfRule}{Bind}
      \hypo{\ell\not\in\dom(\mem)\cup L}
      \hypo{\evjudg{(x,\ell)\cons\ctx,\mem,L\cup\{\ell\}}{e_1}{v_1,\mem_1,L_1}}
      \hypo{\evjudg{(x,\ell)\cons\ctx,\mem_1[\ell\mapsto v_1],L_1}{e_2}{\ctx_2,\mem',L'}}
      \infer{2,1}{\evjudg{\ctx,\mem,L}{x=e_1;e_2}{(x,\ell)\cons\ctx_2,\mem',L'}}
    \end{InfRule}
  \]
  \caption{The big-step operational semantics with memory.}
  \label{fig:membigstep}
\end{figure}
\begin{figure}[h!]
  \centering
  \small
  \begin{flushright}
    \fbox{$w\sim_f v,\mem$}
  \end{flushright}
  \[
    \begin{InfRule}{Eq-Nil}
      \infer0{\bullet\sim_f\bullet}
    \end{InfRule}\:
    \begin{InfRule}{Eq-ConsFree}
      \hypo{\ell\not\in\dom(f)}
      \hypo{\ell\not\in\dom(\mem)}
      \hypo{\ctx\sim_f\ctx'}
      \infer3{(x,\ell)\cons\ctx\sim_f(x,\ell)\cons\ctx'}
    \end{InfRule}\:
    \begin{InfRule}{Eq-ConsBound}
      \hypo{f(\ell)=\ell'}
      \hypo{\ell'\in\dom(\mem)}
      \hypo{\ctx\sim_f\ctx'}
      \infer3{(x,\ell)\cons\ctx\sim_f(x,\ell')\cons\ctx'}
    \end{InfRule}
  \]

  \[
    \begin{InfRule}{Eq-ConsWVal}
      \hypo{\mem(\ell')=v'}
      \hypo{w\sim_f v'}
      \hypo{\ctx\sim_f\ctx'}
      \infer3{(x,w)\cons\ctx\sim_f(x,\ell')\cons\ctx'}
    \end{InfRule}\:
    \begin{InfRule}{Eq-Clos}
      \hypo{\ctx\sim_f\ctx'}
      \infer1{\langle\lambda x.e,\ctx\rangle\sim_f\langle\lambda x.e,\ctx'\rangle}
    \end{InfRule}\:
    \begin{InfRule}{Eq-Rec}
      \hypo{L\text{ finite}}
      \hypo{\mem(\ell')=v'}
      \hypo{\forall\ell\not\in L,\:\openloc{\ell}{v}\sim_{f[\ell\mapsto\ell']}v'}
      \infer3{\mu.v\sim_f v'}
    \end{InfRule}
  \]
  \caption{The equivalence relation between weak values in the original semantics and values in the semantics with memory.
    $f\in\fin{\Loc}{\Loc}$ tells what the free locations in $w$ that were \emph{opened} should be mapped to in memory.}
  \label{fig:equivrel}
\end{figure}


The semantics in Figure \ref{fig:bigstep} makes sense due to similarity with a conventional backpatching semantics as presented in Figure \ref{fig:membigstep}.
We have defined a relation $\sim$ that satisfies:
\[\sim\subseteq\Walue\times(\Malue\times\Mem\times\pset(\Loc))\qquad\bullet\sim(\bullet,\varnothing,\varnothing)\]
and the following theorem:
\begin{thm}[Equivalence of semantics]\label{thm:equivsem}
  For all $\ctx\in\Ctx,\ctx'\in\Mtx\times\Mem\times\pset(\Loc),v\in\Value,v'\in\Malue\times\Mem\times\pset(\Loc)$, we have:
  \begin{align*}
    \ctx\sim\ctx'\text{ and }\evjudg{\ctx}{e}{v}   & \Rightarrow\exists v':v\sim v'\text{ and }\evjudg{\ctx'}{e}{v'} \\
    \ctx\sim\ctx'\text{ and }\evjudg{\ctx'}{e}{v'} & \Rightarrow\exists v:v\sim v'\text{ and }\evjudg{\ctx}{e}{v}
  \end{align*}
\end{thm}

The actual definition for $\sim$ is given in Figure \ref{fig:equivrel}.

The proof of Theorem \ref{thm:equivsem} uses some useful lemmas, such as:
\begin{lem}[Free locations not in $f$ are free in memory]
  \[w\sim_f v',\mem\Rightarrow m|_{\FLoc(w)-\dom(f)}=\bot\]
\end{lem}

\begin{lem}[Equivalence is preserved by extension of memory]
  \[w\sim_f v',\mem\text{ and }m\sqsubseteq m'\text{ and }m'|_{\FLoc(w)-\dom(f)}=\bot\Rightarrow w\sim_f v',m\]
\end{lem}

\begin{lem}[Equivalence only cares about $f$ on free locations]
  \[w\sim_f v',\mem\text{ and }f|_{\FLoc(w)}=f|_{\FLoc(w)}\Rightarrow w\sim_{f'}v',m\]
\end{lem}

\begin{lem}[Extending equivalence on free locations]
  \[w\sim_f v',\mem\text{ and }\ell\not\in\dom(f)\text{ and }\ell\not\in\dom(\mem)\Rightarrow\forall u',w\sim_{f[\ell\mapsto\ell]}v',m[\ell\mapsto u']\]
\end{lem}

\begin{lem}[Substitution of values]
  \[w\sim_f v',\mem\text{ and }f(\ell)=\ell'\text{ and }\mem(\ell')=u'\text{ and }u\sim_{f-\ell}u',\mem\Rightarrow w[u/\ell]\sim_{f-\ell}v',\mem\]
\end{lem}

\begin{lem}[Substitution of locations]
  \[w\sim_f v',\mem\text{ and }\ell\in\dom(f)\text{ and }\nu\not\in\FLoc(w)\Rightarrow w[\nu/\ell]\sim_{f\circ(\nu\leftrightarrow\ell)}v',\mem\]
\end{lem}

\section{Generating and Resolving Events}
Now we formulate the semantics for generating events.

\begin{figure}[h!]
  \centering
  \begin{tabular}{rrcll}
    Event       & $\event$ & $\rightarrow$ & $\InitE$           & initial environment \\
                &          & $\vbar$       & $\ReadE(\event,x)$ & read event          \\
                &          & $\vbar$       & $\CallE(\event,v)$ & call event          \\
    Environment & $\ctx$   & $\rightarrow$ & $\cdots$                                 \\
                &          & $\vbar$       & $[\event]$         & answer to an event  \\
    Value       & $v$      & $\rightarrow$ & $\cdots$                                 \\
                &          & $\vbar$       & $\event$           & answer to an event
  \end{tabular}
  \caption{Definition of the semantic domains with events. All other semantic domains are equal to Figure \ref{fig:domain}.}
  \label{fig:eventdomain}
\end{figure}

We extend how to read weak values given an environment.
\begin{align*}
  \bullet(x)  & \triangleq\bot             &  &  & ((x',\ell)\cons\ctx)(x) & \triangleq (x=x'?\ell:\ctx(x)) \\
  [\event](x) & \triangleq\ReadE(\event,x) &  &  & ((x',w)\cons\ctx)(x)    & \triangleq (x=x'?w:\ctx(x))    \\
\end{align*}

Then we need to add only one rule to the semantics in Figure \ref{fig:bigstep} for the semantics to incorporate events.
\[
  \begin{InfRule}{AppEvent}
    \hypo{
      \ctx\vdash e_{1}
      \Downarrow
      \event
    }
    \hypo{
      \ctx\vdash e_{2}
      \Downarrow
      v
    }
    \infer2{
      \ctx\vdash e_{1}\:e_{2}
      \Downarrow
      \CallE(\event,v)
    }
  \end{InfRule}
\]

Now we need to formulate the \emph{concrete linking} rules.
The concrete linking rule $\ctx_0\semlink w$, given an answer $\ctx_0$ to the $\InitE$ event, resolves all events within $w$ to obtain a set of final results.

\begin{figure}
  \begin{align*}
    \fbox{$\semlink\in\Ctx\rightarrow\Event\rightarrow\pset(\Value)$}                                                                                                                               \\
    \ctx_0\semlink\InitE\triangleq                         & \{\ctx_0\}                                                                                                                             \\
    \ctx_0\semlink\ReadE(\event,x)\triangleq               & \{v_+|\ctx_+\in\ctx_0\semlink\event\land\ctx_+(x)=v_+\}                                                                                \\
    \cup                                                   & \{\openloc{\mu.v_+}{v_+}|\ctx_+\in\ctx_0\semlink E\land\ctx_+(x)=\mu.v_+\}                                                             \\
    \ctx_0\semlink\CallE(\event,v)\triangleq               & \{v_+'|\langle\lambda x.e,\ctx_+\rangle\in\ctx_0\semlink E\land v_+\in\ctx_0\semlink v\land(x,v_+)\cons\ctx_+\vdash e\Downarrow v_+'\} \\
    \cup                                                   & \{\CallE(\event_+,v_+)|\event_+\in\ctx_0\semlink\event\land v_+\in\ctx_0\semlink v\}                                                   \\
    \fbox{$\semlink\in\Ctx\rightarrow\Ctx\rightarrow\pset(\Ctx)$}                                                                                                                                   \\
    \ctx_0\semlink\bullet\triangleq                        & \{\bullet\}                                                                                                                            \\
    \ctx_0\semlink(x,\ell)\cons\ctx\triangleq              & \{(x,\ell)\cons\ctx_+|\ctx_+\in\ctx_0\semlink\ctx\}                                                                                    \\
    \ctx_0\semlink(x,w)\cons\ctx\triangleq                 & \{(x,w_+)\cons\ctx_+|w_+\in\ctx_0\semlink w\land\ctx_+\in\ctx_0\semlink\ctx\}                                                          \\
    \ctx_0\semlink[E]\triangleq                            & \{\ctx_+|\ctx_+\in\ctx_0\semlink\event\}\cup\{[\event_+]|\event_+\in\ctx_0\semlink\event\}                                             \\
    \fbox{$\semlink\in\Ctx\rightarrow\Value\rightarrow\pset(\Value)$}                                                                                                                               \\
    \ctx_0\semlink\langle\lambda x.e,\ctx\rangle\triangleq & \{\langle\lambda x.e,\ctx_+\rangle|\ctx_+\in\ctx_0\semlink\ctx\}                                                                       \\
    \fbox{$\semlink\in\Ctx\rightarrow\Walue\rightarrow\pset(\Walue)$}                                                                                                                               \\
    \ctx_0\semlink\mu.v\triangleq                          & \{\mu.\closeloc{\ell}{v_+}|\ell\not\in\FLoc(v)\cup\FLoc(\ctx_0)\land v_+\in\ctx_0\semlink\openloc{\ell}{v}\}
  \end{align*}
  \caption{Definition for concrete linking.}
  \label{fig:conclink}
\end{figure}

Concrete linking makes sense because of the following theorem.
First define:
\[\eval(e,\ctx)\triangleq\{v|\ctx\vdash e\Downarrow v\}\qquad\eval(e,\Sigma)\triangleq\bigcup_{\ctx\in\Sigma}\eval(e,\ctx)\qquad\ctx_0\semlink W\triangleq\bigcup_{w\in W}(\ctx_0\semlink w)\]
Then the following holds:
\begin{thm}[Soundness of concrete linking]\label{thm:linksound}
  Given $e\in\Expr,\ctx\in\Ctx,v\in\Value$,
  \[\forall\ctx_0\in\Ctx:\eval(e,\ctx_0\semlink\ctx)\subseteq\ctx_0\semlink\eval(e,\ctx)\]
\end{thm}

The proof of Theorem \ref{thm:linksound} uses some useful lemmas, such as:
\begin{lem}[Linking distributes under substitution]
  Let $\ctx_0$ be the external environment that is linked with locally closed weak values $w$ and $u$.
  For all $\ell\not\in\FLoc(\ctx_0)$, we have:
  \[\forall w_+,u_+:w_+\in\ctx_0\semlink w\wedge u_+\in\ctx_0\semlink u\Rightarrow\{u_+\leftarrow\ell\}w_+\in\ctx_0\semlink\{u\leftarrow\ell\}w\]
\end{lem}
\begin{lem}[Linking is compatible with reads]
  Let $\ctx_0$ be the external environment that is linked with some environment $\ctx$.
  Let $v$ be the value obtained from reading $x$ from $\ctx$.
  Let $\text{unfold}:\Walue\rightarrow\Value$ be defined as:
  \[\text{unfold}(\mu.v)\triangleq\openloc{\mu.v}{v}\qquad\text{unfold}(v)\triangleq v\]
  Then for all $\ctx_+\in\ctx_0\semlink\ctx$, we have:
  \[\exists w_+\in\Walue:\ctx_+(x)=w_+\land\text{unfold}(w_+)\in\ctx_0\semlink v\]
\end{lem}
\clearpage
\section{CFA}
\begin{center}
  \begin{tabular}{rrcl}
    Program point        & $p$  & $\in$         & $\mathbb{P}\triangleq\{\text{finite set of program points}\}$                           \\
    Labelled expression  & $pe$ & $\in$         & $\mathbb{P}\times\Expr$                                                                 \\
    Collecting semantics & $t$  & $\in$         & $\mathbb{T}\triangleq\mathbb{P}\rightarrow\pset(\Ctx+\Ctx\times\Value)$                 \\
    Labelled expression  & $pe$ & $\rightarrow$ & $\plabel{p}{e}$                                                                         \\
    Expression           & $e$  & $\rightarrow$ & $x$ | $\lambda x.pe$ | $pe$ $pe$ | $\link{pe}{pe}$ | $\varepsilon$ | $x\texttt{=}pe;pe$
  \end{tabular}
\end{center}
\begin{align*}
                             &                                                                                                                                                                                                       & \fbox{$\Step:\mathbb{T}\rightarrow\mathbb{T}$}                   \\
  \step(\ptree)   \triangleq & \bigcup_{p\in\mathbb{P}}\step(\ptree,p)                                                                                                                                                                                                                                  \\
                             &                                                                                                                                                                                                       & \fbox{$\step:(\mathbb{T}\times\mathbb{P})\rightarrow\mathbb{T}$} \\
  \step(\ptree,p) \triangleq & [p\mapsto\{(\ctx,v)|\ctx\in \ptree(p)\text{ and }\ctx(x)=v\}]                                                                                                                                         & \plabel{p}{x}                                                    \\
  \cup                       & [p\mapsto\{(\ctx,\openloc{\mu.v}{v})|\ctx\in \ptree(p)\text{ and }\ctx(x)=\mu.v\}]                                                                                                                                                                                       \\
  \step(\ptree,p) \triangleq & [p\mapsto\{(\ctx,\langle\lambda x.p',\ctx\rangle)|\ctx\in \ptree(p)\}]                                                                                                                                & \plabel{p}{\lambda x.p'}                                         \\
  \step(\ptree,p) \triangleq & [p_1\mapsto\{\ctx|\ctx\in \ptree(p)\}]                                                                                                                                                                & \plabel{p}{p_1\:p_2}                                             \\
  \cup                       & [p_2\mapsto\{\ctx|\ctx\in \ptree(p)\}]                                                                                                                                                                                                                                   \\
  \cup                       & \bigcup_{\ctx\in \ptree(p)}\bigcup_{(\ctx,\langle\lambda x.p',\ctx_1\rangle)\in \ptree(p_1)}[p'\mapsto\{(x,v_2)\cons\ctx_1|(\ctx,v_2)\in \ptree(p_2)\}]                                                                                                                  \\
  \cup                       & [p\mapsto\bigcup_{\ctx\in \ptree(p)}\bigcup_{(\ctx,\langle\lambda x.p',\ctx_1\rangle)\in \ptree(p_1)}\bigcup_{(\ctx,v_2)\in \ptree(p_2)}\{(\ctx,v)|((x,v_2)\cons\ctx_1,v)\in \ptree(p')\}]                                                                               \\
  \cup                       & [p\mapsto\bigcup_{\ctx\in \ptree(p)}\{(\ctx,\CallE(\event_1,v_2))|(\ctx,\event_1)\in \ptree(p_1)\text{ and }(\ctx,v_2)\in \ptree(p_2)\}]                                                                                                                                 \\
  \step(\ptree,p) \triangleq & [p_1\mapsto\{\ctx|\ctx\in \ptree(p)\}]                                                                                                                                                                & \plabel{p}{\link{p_1}{p_2}}                                      \\
  \cup                       & [p_2\mapsto\bigcup_{\ctx\in \ptree(p)}\{\ctx_1|(\ctx,\ctx_1)\in \ptree(p_1)\}]                                                                                                                                                                                           \\
  \cup                       & [p\mapsto\bigcup_{\ctx\in \ptree(p)}\bigcup_{(\ctx,\ctx_1)\in \ptree(p_1)}\{(\ctx,v_2)|(\ctx_1,v_2)\in \ptree(p_2)\}]                                                                                                                                                    \\
  \step(\ptree,p) \triangleq & [p\mapsto\{(\ctx,\bullet)|\ctx\in \ptree(p)\}]                                                                                                                                                        & \plabel{p}{\varepsilon}                                          \\
  \step(\ptree,p) \triangleq & [p_1\mapsto\{(x,\ell)\cons\ctx|\ctx\in \ptree(p)\text{ and }\ell\not\in\FLoc(\ctx)\}]                                                                                                                 & \plabel{p}{x\texttt{=}p_1;p_2}                                   \\
  \cup                       & [p_2\mapsto\{(x,\mu.\closeloc{\ell}{v_1})\cons\ctx|\ctx\in \ptree(p)\text{ and }((x,\ell)\cons\ctx,v_1)\in \ptree(p_1)\}]                                                                                                                                                \\
  \cup                       & [p\mapsto\bigcup_{\ctx\in \ptree(p)}\bigcup_{((x,\ell)\cons\ctx,v_1)\in \ptree(p_1)}\{(\ctx,(x,\mu.\closeloc{\ell}{v_1})\cons\ctx_2)|((x,\mu.\closeloc{\ell}{v_1})\cons\ctx,\ctx_2)\in \ptree(p_2)\}]
\end{align*}
The proof tree $t_0$ computed by
\[t_0\triangleq\lfp(\lambda t.\step(\ptree)\cup[p_0\mapsto\{\ctx_0\}])\]
contains all derivations of the form $\evjudg{\ctx_0}{p_0}{v_0}$ for some $v_0$.
That is, $(\ctx,v)$ is contained in $t_0(p)$ if and only if $\evjudg{\ctx}{p}{v}$ must be contained in a valid derivation for the judgment $\evjudg{\ctx_0}{p_0}{v_0}$.
\begin{center}
  \begin{tabular}{rrcl}
    Abstract event       & $\Abs{\event}$                 & $\in$         & $\Abs{\Event}$                                                                     \\
    Abstract environment & $\Abs{\ctx}$                   & $\in$         & $\Abs{\Ctx}\triangleq(\fin{\ExprVar}{\pset(\mathbb{P})})\times\pset(\Abs{\Event})$ \\
    Abstract closure     & $\langle\lambda x.p,p'\rangle$ & $\in$         & $\Abs{\text{Clos}}\triangleq\ExprVar\times\mathbb{P}\times\mathbb{P}$              \\
    Abstract value       & $\Abs{v}$                      & $\in$         & $\Abs{\Value}\triangleq\Abs{\Ctx}\times\pset(\Abs{\text{Clos}})$                   \\
    Abstract semantics   & $\Abs\ptree$                   & $\in$         & $\Abs{\mathbb{T}}\triangleq\mathbb{P}\rightarrow\Abs{\Ctx}\times\Abs{\Value}$      \\
    Abstract event       & $\Abs{\event}$                 & $\rightarrow$ & $\Abs{\InitE}$ | $\Abs{\ReadE}(p,x)$ | $\Abs{\CallE}(p,p)$
  \end{tabular}
\end{center}
\begin{align*}
                                     &                                                                                                                                          & \fbox{$\Abs\Step:\Abs{\mathbb{T}}\rightarrow\Abs{\mathbb{T}}$}                   \\
  \Abs\Step(\Abs\ptree)   \triangleq & \bigsqcup_{p\in\mathbb{P}}\Abs\step(\Abs\ptree,p)                                                                                                                                                                           \\
                                     &                                                                                                                                          & \fbox{$\Abs\step:(\Abs{\mathbb{T}}\times\mathbb{P})\rightarrow\Abs{\mathbb{T}}$} \\
  \Abs\step(\Abs\ptree,p) \triangleq & [p\mapsto\bigsqcup_{p'\in\Abs\ptree(p).1.1(x)}(\Abs\ptree(p).1,\Abs\ptree(p').2)]                                                        & \plabel{p}{x}                                                                    \\
  \sqcup                             & [p\mapsto\bigsqcup_{\Abs\ptree(p).1.2\neq\varnothing}(\Abs\ptree(p).1,(([],\{\Abs\ReadE(p,x)\}),\varnothing))]                                                                                                              \\
  \Abs\step(\Abs\ptree,p) \triangleq & [p\mapsto(\Abs\ptree(p).1,(\bot,\{\langle\lambda x.p',p\rangle\}))]                                                                      & \plabel{p}{\lambda x.p'}                                                         \\
  \Abs\step(\Abs\ptree,p) \triangleq & [p_1\mapsto(\Abs\ptree(p).1,\bot)]                                                                                                       & \plabel{p}{p_1\:p_2}                                                             \\
  \sqcup                             & [p_2\mapsto(\Abs\ptree(p).1,\bot)]                                                                                                                                                                                          \\
  \sqcup                             & \bigsqcup_{\langle\lambda x.p',p''\rangle\in\Abs\ptree(p_1).2.2}[p'\mapsto(\Abs\ptree(p'').1\sqcup([x\mapsto\{p_2\}],\varnothing),\bot)]                                                                                    \\
  \sqcup                             & [p\mapsto\bigsqcup_{\langle\lambda x.p',\_\rangle\in\Abs\ptree(p_1).2.2}(\Abs\ptree(p).1,\Abs\ptree(p').2)]                                                                                                                 \\
  \sqcup                             & [p\mapsto\bigsqcup_{\Abs\ptree(p_1).2.1.2\neq\varnothing}(\Abs\ptree(p).1,(([],\{\Abs\CallE(p_1,p_2)\}),\varnothing))]                                                                                                      \\
  \Abs\step(\Abs\ptree,p) \triangleq & [p_1\mapsto(\Abs\ptree(p).1,\bot)]                                                                                                       & \plabel{p}{\link{p_1}{p_2}}                                                      \\
  \sqcup                             & [p_2\mapsto(\Abs\ptree(p_1).2.1,\bot)]                                                                                                                                                                                      \\
  \sqcup                             & [p\mapsto(\Abs\ptree(p).1,\Abs\ptree(p_2).2)]                                                                                                                                                                               \\
  \Abs\step(\Abs\ptree,p) \triangleq & [p\mapsto(\Abs\ptree(p).1,\bot)]                                                                                                         & \plabel{p}{\varepsilon}                                                          \\
  \Abs\step(\Abs\ptree,p) \triangleq & [p_1\mapsto(\Abs\ptree(p).1\sqcup([x\mapsto\{p_1\}],\varnothing),\bot)]                                                                  & \plabel{p}{x\texttt{=}p_1;p_2}                                                   \\
  \sqcup                             & [p_2\mapsto(\Abs\ptree(p).1\sqcup([x\mapsto\{p_1\}],\varnothing),\bot)]                                                                                                                                                     \\
  \sqcup                             & [p\mapsto(\Abs\ptree(p).1,(\Abs\ptree(p_2).2.1\sqcup([x\mapsto\{p_1\}],\varnothing),\varnothing))]
\end{align*}
\end{document}
%%% Local Variables: 
%%% coding: utf-8
%%% mode: latex
%%% TeX-engine: xetex
%%% End:
