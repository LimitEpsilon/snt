%! TEX program = xelatex
\documentclass{article}
\usepackage{geometry}
\geometry{
  a4paper,
  total={170mm,257mm},
  left=20mm,
  top=20mm,
}
\usepackage{setspace}
\usepackage{graphicx}
\usepackage{xcolor}
\usepackage{kotex}
\usepackage{csquotes}
\usepackage{tabularray}
\usepackage{relsize}

%%% Typesetting for listings
\usepackage{listings}

\definecolor{dkgreen}{rgb}{0,0.6,0}
\definecolor{ltblue}{rgb}{0,0.4,0.4}
\definecolor{dkviolet}{rgb}{0.3,0,0.5}

%%% lstlisting coq style (inspired from a file of Assia Mahboubi)
\lstdefinelanguage{Coq}{ 
    % Anything betweeen $ becomes LaTeX math mode
    mathescape=true,
    % Comments may or not include Latex commands
    texcl=false, 
    % Vernacular commands
    morekeywords=[1]{Section, Module, End, Require, Import, Export,
        Variable, Variables, Parameter, Parameters, Axiom, Hypothesis,
        Hypotheses, Notation, Local, Tactic, Reserved, Scope, Open, Close,
        Bind, Delimit, Definition, Let, Ltac, Fixpoint, CoFixpoint, Add,
        Morphism, Relation, Implicit, Arguments, Unset, Contextual,
        Strict, Prenex, Implicits, Inductive, CoInductive, Record,
        Structure, Canonical, Coercion, Context, Class, Global, Instance,
        Program, Infix, Theorem, Lemma, Corollary, Proposition, Fact,
        Remark, Example, Proof, Goal, Save, Qed, Defined, Hint, Resolve,
        Rewrite, View, Search, Show, Print, Printing, All, Eval, Check,
        Projections, inside, outside, Def},
    % Gallina
    morekeywords=[2]{forall, exists, exists2, fun, fix, cofix, struct,
        match, with, end, as, in, return, let, if, is, then, else, for, of,
        nosimpl, when},
    % Sorts
    morekeywords=[3]{Type, Prop, Set, true, false, option},
    % Various tactics, some are std Coq subsumed by ssr, for the manual purpose
    morekeywords=[4]{pose, set, move, case, elim, apply, clear, hnf,
        intro, intros, generalize, rename, pattern, after, destruct,
        induction, using, refine, inversion, injection, rewrite, congr,
        unlock, compute, ring, field, fourier, replace, fold, unfold,
        change, cutrewrite, simpl, have, suff, wlog, suffices, without,
        loss, nat_norm, assert, cut, trivial, revert, bool_congr, nat_congr,
        symmetry, transitivity, auto, split, left, right, autorewrite},
    % Terminators
    morekeywords=[5]{by, done, exact, reflexivity, tauto, romega, omega,
        assumption, solve, contradiction, discriminate},
    % Control
    morekeywords=[6]{do, last, first, try, idtac, repeat},
    % Comments delimiters, we do turn this off for the manual
    morecomment=[s]{(*}{*)},
    % Spaces are not displayed as a special character
    showstringspaces=false,
    % String delimiters
    morestring=[b]",
    morestring=[d],
    % Size of tabulations
    tabsize=3,
    % Enables ASCII chars 128 to 255
    extendedchars=false,
    % Case sensitivity
    sensitive=true,
    % Automatic breaking of long lines
    breaklines=false,
    % Default style fors listings
    basicstyle=\small,
    % Position of captions is bottom
    captionpos=b,
    % flexible columns
    columns=[l]flexible,
    % Style for (listings') identifiers
    identifierstyle={\ttfamily\color{black}},
    % Style for declaration keywords
    keywordstyle=[1]{\ttfamily\color{dkviolet}},
    % Style for gallina keywords
    keywordstyle=[2]{\ttfamily\color{dkgreen}},
    % Style for sorts keywords
    keywordstyle=[3]{\ttfamily\color{ltblue}},
    % Style for tactics keywords
    keywordstyle=[4]{\ttfamily\color{dkblue}},
    % Style for terminators keywords
    keywordstyle=[5]{\ttfamily\color{dkred}},
    %Style for iterators
    %keywordstyle=[6]{\ttfamily\color{dkpink}},
    % Style for strings
    stringstyle=\ttfamily,
    % Style for comments
    commentstyle={\ttfamily\color{dkgreen}},
    %moredelim=**[is][\ttfamily\color{red}]{/&}{&/},
    literate=
    {\\forall}{{\color{dkgreen}{$\forall\;$}}}1
    {\\exists}{{$\exists\;$}}1
    {\ge -}{{$\geftarrow\;$}}1
    {=>}{{$\Rightarrow\;$}}1
    {==}{{\code{==}\;}}1
    {==>}{{\code{==>}\;}}1
    %    {:>}{{\code{:>}\;}}1
    {->}{{$\rightarrow\;$}}1
    {\ge ->}{{$\geftrightarrow\;$}}1
    {\ge ==}{{$\geq\;$}}1
    {\#}{{$^\star$}}1 
    {\\o}{{$\circ\;$}}1 
    {\@}{{$\cdot$}}1 
    {\/\\}{{$\text{ and }\;$}}1
    {\\\/}{{$\vee\;$}}1
    {++}{{\code{++}}}1
    {~}{{\ }}1
    {\@\@}{{$@$}}1
    {\\mapsto}{{$\mapsto\;$}}1
    {\\hline}{{\rule{\linewidth}{0.5pt}}}1
    %
}[keywords,comments,strings]

%%% Math settings
\usepackage{amssymb,amsmath,amsthm,mathtools}
\usepackage[math-style=TeX,bold-style=TeX]{unicode-math}
\theoremstyle{definition}
\newtheorem{definition}{Definition}[section]
\newtheorem{example}{Example}[section]
\newtheorem{lem}{Lemma}[section]
\newtheorem{thm}{Theorem}[section]
\newtheorem{cor}{Corollary}[section]
\newtheorem{clm}{Claim}[section]

%%% Font settings
%\setmainfont{Libertinus Serif}
%\setsansfont{Libertinus Sans}[Scale=MatchUppercase]
%\setmonofont{JuliaMono}[Scale=MatchLowercase]
%\setmainhangulfont{Noto Serif CJK KR}
%\setmonohangulfont{D2Coding}

%%% PL constructs
\usepackage{galois}
\usepackage{ebproof}
%% InfRule command
\ExplSyntaxOn
\NewTblrEnviron{@ruleenv}
\SetTblrInner[@ruleenv]{belowsep=0pt,stretch=0}
\SetTblrOuter[@ruleenv]{baseline=b}
\NewDocumentEnvironment { InfRule } { m +b }
  {
    \begin{@ruleenv}{l}
      \smaller{\textsc{#1}} \\
      \begin{prooftree} #2 \end{prooftree}
    \end{@ruleenv}
  }
  {}
\ExplSyntaxOff

%% Modification for \infer
\ExplSyntaxOn
\int_new:N \g__ebproof_sublevel_int
\box_new:N \g__ebproof_substack_box
\seq_new:N \g__ebproof_substack_seq

\cs_new:Nn \__ebproof_clear_substack:
  {
    \int_gset:Nn \g__ebproof_sublevel_int { 0 }
    \hbox_gset:Nn \g__ebproof_substack_box { }
    \seq_gclear:N \g__ebproof_substack_seq
  }

\cs_new:Nn \__ebproof_subpush:N
  {
    \int_gincr:N \g__ebproof_sublevel_int
    \hbox_gset:Nn \g__ebproof_substack_box
      { \hbox_unpack:N \g__ebproof_substack_box \box_use:c { \__ebproof_box:N #1 } }
    \seq_gput_left:Nv \g__ebproof_substack_seq
      { \__ebproof_marks:N #1 }
  }

\cs_new:Nn \__ebproof_subpop:N
  {
    \int_compare:nNnTF { \g__ebproof_sublevel_int } > { 0 }
      {
        \int_gdecr:N \g__ebproof_sublevel_int
        \hbox_gset:Nn \g__ebproof_substack_box {
          \hbox_unpack:N \g__ebproof_substack_box
          \box_gset_to_last:N \g_tmpa_box
        }
        \box_set_eq_drop:cN { \__ebproof_box:N #1 } \g_tmpa_box
        \seq_gpop_left:NN \g__ebproof_substack_seq \l_tmpa_tl
        \tl_set_eq:cN { \__ebproof_marks:N #1 } \l_tmpa_tl
      }
      { \PackageError{ebproof}{Missing~premiss~in~a~proof~tree}{} \__ebproof_clear:N #1 }
  }

\cs_new:Nn \__ebproof_append_subvertical:NN
  {
    \bool_if:NTF \l__ebproof_updown_bool
      { \__ebproof_append_above:NN #1 #2 }
      { \__ebproof_append_below:NN #1 #2 }
  }

\cs_new:Nn \__ebproof_join_subvertical:n
  {
    \group_begin:
    \__ebproof_subpop:N \l__ebproof_a_box
    \prg_replicate:nn { #1 - 1 }
      {
        \__ebproof_subpop:N \l__ebproof_b_box
        \__ebproof_enlarge_conclusion:NN \l__ebproof_b_box \l__ebproof_a_box

        \__ebproof_make_vertical:Nnnn \l__ebproof_c_box
          { \__ebproof_mark:Nn \l__ebproof_b_box {axis} - \__ebproof_mark:Nn \l__ebproof_b_box {left} }
          { \__ebproof_mark:Nn \l__ebproof_b_box {right} - \__ebproof_mark:Nn \l__ebproof_b_box {left} }
          { \skip_vertical:N \l__ebproof_rule_margin_dim }
        \__ebproof_vcenter:N \l__ebproof_b_box
        \__ebproof_append_subvertical:NN \l__ebproof_a_box \l__ebproof_c_box

        \__ebproof_append_subvertical:NN \l__ebproof_a_box \l__ebproof_b_box
      }
    \__ebproof_push:N \l__ebproof_a_box
    \group_end:
  }

\cs_new:Nn \__ebproof_renew_statement:nnn
  {
    \exp_args:Nc \RenewDocumentCommand { ebproof#1 }{ #2 } { #3 }
    \seq_gput_right:Nn \g__ebproof_statements_seq { #1 }
  }
\__ebproof_renew_statement:nnn { infer } { O{} m O{} m }
  {
    \group_begin:
    \__ebproof_restore_statements:
    \keys_set_known:nnN { ebproof / rule~style } { #1 } \l_tmpa_tl
    \keys_set:nV { ebproof } \l_tmpa_tl
    \tl_set:Nn \l__ebproof_right_label_tl { #3 }

    \__ebproof_clear_substack:
    \clist_map_inline:nn { #2 }
      {
        \__ebproof_join_horizontal:n { ##1 }
        \__ebproof_pop:N \l__ebproof_a_box
        \__ebproof_subpush:N \l__ebproof_a_box
      }
    \__ebproof_join_subvertical:n { \clist_count:n { #2 } }

    \__ebproof_push_statement:n { #4 }
    \__ebproof_join_vertical:
    \group_end:
  }
\ExplSyntaxOff
\ebproofset{left label template=\textsc{[\inserttext]}}
\ebproofset{center=false}

%%% Custom commands
\newcommand*{\vbar}{|}
\newcommand*{\finto}{\xrightarrow{\text{\textrm{fin}}}}
\newcommand*{\istype}{\mathrel{⩴}}
\newcommand*{\ortype}{\mathrel{|}}
\newcommand*{\cons}{::}
\newcommand*{\pset}{\mathscr{P}}
\newcommand*{\closeloc}[2]{\prescript{\text{\textbackslash}{#1}}{}{#2}}
\newcommand*{\openloc}[2]{{#1}^{#2}}

\def\ovbarw{1.2mu}
\def\ovbarh{1}
\newcommand*{\ovbar}[1]{\mkern \ovbarw\overline{\mkern-\ovbarw{\smash{#1}\scalebox{1}[\ovbarh]{\vphantom{i}}}\mkern-\ovbarw}\mkern \ovbarw}
\newcommand*{\A}[1]{\overset{\,_{\mbox{\Large .}}}{#1}}
\newcommand*{\Abs}[1]{{#1}^{\#}}
\newcommand*{\Expr}{\text{Expr}}
\newcommand*{\ExprVar}{\text{Var}}
\newcommand*{\Module}{\text{Module}}
\newcommand*{\ModVar}{\text{ModVar}}
\newcommand*{\modid}{d}
\newcommand*{\Time}{\mathbb{T}}
\newcommand*{\ATime}{\A{\Time}}
\newcommand*{\Ctx}{\text{Env}}
\newcommand*{\ctx}{\sigma}
\newcommand*{\Value}{\text{Val}}
\newcommand*{\Walue}{\text{WVal}}
\newcommand*{\Mem}{\text{Mem}}
\newcommand*{\Loc}{\text{Loc}}
\newcommand*{\FLoc}{\text{FLoc}}
\newcommand*{\Left}{\text{Left}}
\newcommand*{\Right}{\text{Right}}
\newcommand*{\Sig}{\text{Sig}}
\newcommand*{\mem}{m}
\newcommand*{\dom}{\text{dom}}
\newcommand*{\AMem}{\A{\text{Mem}}}
\newcommand*{\State}{\text{State}}
\newcommand*{\AState}{\A{\text{State}}}
\newcommand*{\Result}{\text{Result}}
\newcommand*{\AResult}{\A{\text{Result}}}
\newcommand*{\Tick}{\text{Tick}}
\newcommand*{\lfp}{\mathsf{lfp}}
\newcommand*{\Step}{\mathsf{Step}}
\newcommand*{\semarrow}{\Downarrow}
\newcommand*{\asemarrow}{\A{\rightsquigarrow}}
\newcommand*{\synlink}{\rtimes}
\newcommand*{\semlink}{\mathbin{\rotatebox[origin=c]{180}{$\propto$}}}
\newcommand*{\link}[2]{{#1}\rtimes{#2}}
\newcommand*{\mt}{\mathsf{empty}}
\newcommand*{\valid}{\mathsf{valid}}
\newcommand*{\Path}{\text{Path}}
\newcommand*{\equivalent}{\sim}

\newcommand*{\doubleplus}{\ensuremath{\mathbin{+\mkern-3mu+}}}
\newcommand*{\project}{\text{\texttt{:>} }}
\newcommand*{\Exp}{\mathsf{Exp}}
\newcommand*{\Imp}{\mathsf{Imp}}
\newcommand*{\Fin}{\mathsf{Fin}}
\newcommand*{\Link}{\mathsf{Link}}
\newcommand*{\sembracket}[1]{\lBrack{#1}\rBrack}
\newcommand*{\fin}[2]{{#1}\xrightarrow{\text{fin}}{#2}}
\newcommand*{\addr}{\mathsf{addr}}
\newcommand*{\tick}{\mathsf{tick}}
\newcommand*{\modctx}{\mathsf{ctx}}
\newcommand*{\mapinject}[2]{{#2}[{#1}]}
\newcommand*{\inject}[2]{{#2}\langle{#1}\rangle}
\newcommand*{\deletepre}[2]{{#2}\overline{\doubleplus}{#1}}
\newcommand*{\deletemap}[2]{{#1}\overline{[{#2}]}}
\newcommand*{\delete}[2]{{#2}{\langle{#1}\rangle}^{-1}}
\newcommand*{\filter}{\mathsf{filter}}
\newcommand*{\Lete}{\mathtt{val}}
\newcommand*{\Letm}{\mathtt{mod}}

\newcommand*{\ValRel}[1]{\mathcal{V}\sembracket{#1}}
\newcommand*{\ExprRel}[1]{\mathcal{E}\sembracket{#1}}
\newcommand*{\CtxRel}[1]{\mathcal{C}\sembracket{#1}}
\newcommand*{\ModRel}[1]{\mathcal{M}\sembracket{#1}}
\newcommand*{\TyEnv}{\text{TyEnv}}
\newcommand*{\TyVar}{\text{TyVar}}
\newcommand*{\Type}{\text{Type}}
\newcommand*{\Subst}{\text{Subst}}
\newcommand*{\external}{\Gamma_{\text{ext}}}

\newcommand*{\InitE}{\textsf{Init}}
\newcommand*{\ReadE}{\textsf{Read}}
\newcommand*{\CallE}{\textsf{Call}}
\newcommand*{\resolve}{\downarrow_{\ctx_0}}


\title{Modular Analysis}
\author{Joonhyup Lee}
\begin{document}
\maketitle
\section{Syntax and Semantics}
\subsection{Abstract Syntax}
\begin{figure}[htb]
	\centering
	\small
	\begin{tabular}{rrcll}
		Identifiers & $x$ & $\in$         & $\ExprVar$                                                      \\
		Expression  & $e$ & $\rightarrow$ & $x$ $\vbar$ $\lambda x.e$ $\vbar$ $e$ $e$ & $\lambda$-calculus  \\
		            &     & $\vbar$       & $e\synlink e$                             & linked expression   \\
		            &     & $\vbar$       & $\varepsilon$                             & empty module        \\
		            &     & $\vbar$       & $x\:\texttt{=}\:e$ ; $e$                  & (recursive) binding
	\end{tabular}
	\caption{Abstract syntax of the language.}
	\label{fig:syntax}
\end{figure}
\subsection{Operational Semantics}
\begin{figure}[h!]
	\centering
	\small
	\begin{tabular}{rrcll}
		Environment     & $\ctx$ & $\in$         & $\Ctx$                                                                         \\
		Location        & $\ell$ & $\in$         & $\Loc$                                                                         \\
		de Bruijn Index & $n$    & $\in$         & $\mathbb{N}$                                                                   \\
		Value           & $v$    & $\in$         & $\Value \triangleq\Ctx+\ExprVar\times\Expr\times\Ctx$                          \\
		Weak Value      & $w$    & $\in$         & $\Walue\triangleq\Value+\underline\Value$                                      \\
		Environment     & $\ctx$ & $\rightarrow$ & $\bullet$                                             & empty stack            \\
		                &        & $\vbar$       & $(x,w)\cons \ctx$                                     & weak value binding     \\
		                &        & $\vbar$       & $(x,\ell)\cons\ctx$                                   & free location binding  \\
		                &        & $\vbar$       & $(x,n)\cons\ctx$                                      & bound location binding \\
		Value           & $v$    & $\rightarrow$ & $\ctx$                                                & exported environment   \\
		                &        & $\vbar$       & $\langle \lambda x.e, \ctx \rangle$                   & closure                \\
		Weak Value      & $w$    & $\rightarrow$ & $v$                                                   & value                  \\
		                &        & $\vbar$       & $\mu.v$                                               & recursive value
	\end{tabular}
	\caption{Definition of the semantic domains.}
	\label{fig:domain}
\end{figure}
\begin{figure}[h!]
	\small
	\begin{flushright}
		\fbox{$\evjudg{\ctx}{e}{v}$}
	\end{flushright}
	\centering
	\vspace{0pt} % -0.75em}
	\[
		\begin{InfRule}{Id}
			\hypo{\ctx(x)=v}
			\infer1{\evjudg{\ctx}{x}{v}}
		\end{InfRule}\:
		\begin{InfRule}{RecId}
			\hypo{\ctx(x)=\mu.v}
			\infer1{\evjudg{\ctx}{x}{\openloc{\mu.v}{v}}}
		\end{InfRule}\:
		\begin{InfRule}{Fn}
			\infer0{\evjudg{\ctx}{\lambda x.e}{\langle\lambda x.e, \ctx\rangle}}
		\end{InfRule}\:
		\begin{InfRule}{App}
			\hypo{\evjudg{\ctx}{e_1}{\langle\lambda x.e, \ctx_1\rangle}}
			\hypo{\evjudg{\ctx}{e_2}{v_2}}
			\hypo{\evjudg{(x, v_2)\cons \ctx_1}{e}{v}}
			\infer{2,1}{\evjudg{\ctx}{e_1\:e_2}{v}}
		\end{InfRule}
	\]

	\[
		\begin{InfRule}{Link}
			\hypo{\evjudg{\ctx}{e_1}{\ctx_1}}
			\hypo{\evjudg{\ctx_1}{e_2}{v}}
			\infer2{\evjudg{\ctx}{{e_1}\synlink{e_2}}{v}}
		\end{InfRule}\quad
		\begin{InfRule}{Empty}
			\infer0{\evjudg{\ctx}{\varepsilon}{\bullet}}
		\end{InfRule}\quad
		\begin{InfRule}{Bind}
			\hypo{\ell\not\in\FLoc(\ctx)}
			\hypo{\evjudg{(x,\ell)\cons\ctx}{e_1}{v_1}}
			\hypo{\evjudg{(x, \mu.\closeloc{\ell}{v_1})\cons \ctx}{e_1}{\ctx_2}}
			\infer{2,1}{\evjudg{\ctx}{x\:\texttt{=}\:e_1; e_2}{(x,\mu.\closeloc{\ell}{v_1})\cons\ctx_2}}
		\end{InfRule}
	\]
	\caption{The big-step operational semantics.}
	\label{fig:bigstep}
\end{figure}
We use the locally nameless representation, and enforce that all values be \emph{locally closed}.
As a consequence, the big-step operational semantics will be \emph{deterministic}, no matter what $\ell$ is chosen in the Bind rule.

\subsection{Reconciling with Conventional Backpatching}
\begin{figure}[h!]
	\centering
	\small
	\begin{tabular}{rrcll}
		Environment   & $\ctx$ & $\in$         & $\Mtx\triangleq\fin{\ExprVar}{\Loc}$                                         \\
		Memory        & $\mem$ & $\in$         & $\Mem\triangleq\fin{\Loc}{\Malue}$                                           \\
		Allocated set & $L$    & $\subseteq$   & $\Loc$                                                                       \\
		Value         & $v$    & $\in$         & $\Malue \triangleq\Mtx+\ExprVar\times\Expr\times\Mtx$                        \\
		Environment   & $\ctx$ & $\rightarrow$ & $\bullet$                                             & empty stack          \\
		              &        & $\vbar$       & $(x,\ell)\cons\ctx$                                   & location binding     \\
		Value         & $v$    & $\rightarrow$ & $\ctx$                                                & exported environment \\
		              &        & $\vbar$       & $\langle \lambda x.e, \ctx \rangle$                   & closure
	\end{tabular}
	\caption{Definition of the semantic domains with memory.}
	\label{fig:memdomain}
\end{figure}
\begin{figure}[h!]
	\small
	\begin{flushright}
		\fbox{$\evjudg{\ctx,\mem,L}{e}{v,\mem',L'}$}
	\end{flushright}
	\centering
	\vspace{0pt} % -0.75em}
	\[
		\begin{InfRule}{Id}
			\hypo{\ctx(x)=\ell}
			\hypo{\mem(\ell)=v}
			\infer2{\evjudg{\ctx,\mem,L}{x}{v,\mem,L}}
		\end{InfRule}\:
		\begin{InfRule}{Fn}
			\infer0{\evjudg{\ctx,\mem,L}{\lambda x.e}{\langle\lambda x.e,\ctx\rangle,\mem,L}}
		\end{InfRule}
	\]

	\[
		\begin{InfRule}{App}
			\hypo{\evjudg{\ctx,\mem,L}{e_1}{\langle\lambda x.e,\ctx_1\rangle,\mem_1,L_1}}
			\hypo{\evjudg{\ctx,\mem_1,L_1}{e_2}{v_2,\mem_2,L_2}}
			\hypo{\ell\not\in\dom(\mem_2)\cup L_2}
			\hypo{\evjudg{(x,\ell)\cons\ctx_1,\mem_2[\ell\mapsto v_2],L_2}{e}{v,\mem',L'}}
			\infer{3,1}{\evjudg{\ctx,\mem,L}{e_1\:e_2}{v,\mem',L'}}
		\end{InfRule}
	\]

	\[
		\begin{InfRule}{Link}
			\hypo{\evjudg{\ctx,\mem,L}{e_1}{\ctx_1,\mem_1,L_1}}
			\hypo{\evjudg{\ctx_1,\mem_1,L_1}{e_2}{v,\mem',L'}}
			\infer2{\evjudg{\ctx,\mem,L}{{e_1}\synlink{e_2}}{v,\mem',L'}}
		\end{InfRule}\quad
		\begin{InfRule}{Empty}
			\infer0{\evjudg{\ctx,\mem,L}{\varepsilon}{\bullet,\mem,L}}
		\end{InfRule}
	\]

	\[
		\begin{InfRule}{Bind}
			\hypo{\ell\not\in\dom(\mem)\cup L}
			\hypo{\evjudg{(x,\ell)\cons\ctx,\mem,L\cup\{\ell\}}{e_1}{v_1,\mem_1,L_1}}
			\hypo{\evjudg{(x,\ell)\cons\ctx,\mem_1[\ell\mapsto v_1],L_1}{e_2}{\ctx_2,\mem',L'}}
			\infer{2,1}{\evjudg{\ctx,\mem,L}{x=e_1;e_2}{(x,\ell)\cons\ctx_2,\mem',L'}}
		\end{InfRule}
	\]
	\caption{The big-step operational semantics with memory.}
	\label{fig:membigstep}
\end{figure}
\begin{figure}[h!]
	\centering
	\small
	\begin{flushright}
		\fbox{$w\sim_f v,\mem$}
	\end{flushright}
	\[
		\begin{InfRule}{Eq-Nil}
			\infer0{\bullet\sim_f\bullet}
		\end{InfRule}\:
		\begin{InfRule}{Eq-ConsFree}
			\hypo{\ell\not\in\dom(f)}
			\hypo{\ell\not\in\dom(\mem)}
			\hypo{\ctx\sim_f\ctx'}
			\infer3{(x,\ell)\cons\ctx\sim_f(x,\ell)\cons\ctx'}
		\end{InfRule}\:
		\begin{InfRule}{Eq-ConsBound}
			\hypo{f(\ell)=\ell'}
			\hypo{\ell'\in\dom(\mem)}
			\hypo{\ctx\sim_f\ctx'}
			\infer3{(x,\ell)\cons\ctx\sim_f(x,\ell')\cons\ctx'}
		\end{InfRule}
	\]

	\[
		\begin{InfRule}{Eq-ConsWVal}
			\hypo{\mem(\ell')=v'}
			\hypo{w\sim_f v'}
			\hypo{\ctx\sim_f\ctx'}
			\infer3{(x,w)\cons\ctx\sim_f(x,\ell')\cons\ctx'}
		\end{InfRule}\:
		\begin{InfRule}{Eq-Clos}
			\hypo{\ctx\sim_f\ctx'}
			\infer1{\langle\lambda x.e,\ctx\rangle\sim_f\langle\lambda x.e,\ctx'\rangle}
		\end{InfRule}\:
		\begin{InfRule}{Eq-Rec}
			\hypo{L\text{ finite}}
			\hypo{\mem(\ell')=v'}
			\hypo{\forall\ell\not\in L,\:\openloc{\ell}{v}\sim_{f[\ell\mapsto\ell']}v'}
			\infer3{\mu.v\sim_f v'}
		\end{InfRule}
	\]
	\caption{The equivalence relation between weak values in the original semantics and values in the semantics with memory.
		$f\in\fin{\Loc}{\Loc}$ tells what the free locations in $w$ that were \emph{opened} should be mapped to in memory.
    $\mem$ is omitted for brevity.}
	\label{fig:equivrel}
\end{figure}


The semantics in Figure \ref{fig:bigstep} makes sense due to similarity with a conventional backpatching semantics as presented in Figure \ref{fig:membigstep}.
We have defined a relation $\sim$ that satisfies:
\[\sim\subseteq\Walue\times(\Malue\times\Mem\times\pset(\Loc))\qquad\bullet\sim(\bullet,\varnothing,\varnothing)\]
and the following theorem:
\begin{thm}[Equivalence of semantics]\label{thm:equivsem}
	For all $\ctx\in\Ctx,\ctx'\in\Mtx\times\Mem\times\pset(\Loc),v\in\Value,v'\in\Malue\times\Mem\times\pset(\Loc)$, we have:
	\begin{align*}
		\ctx\sim\ctx'\text{ and }\evjudg{\ctx}{e}{v}   & \Rightarrow\exists v':v\sim v'\text{ and }\evjudg{\ctx'}{e}{v'} \\
		\ctx\sim\ctx'\text{ and }\evjudg{\ctx'}{e}{v'} & \Rightarrow\exists v:v\sim v'\text{ and }\evjudg{\ctx}{e}{v}
	\end{align*}
\end{thm}

The definition for $w\sim(\ctx,\mem,L)$ is:
\[w\sim_\bot(\ctx,\mem)\text{ and }\FLoc(w)\subseteq L\]
where the definition for $\sim_f$ is given in Figure \ref{fig:equivrel}.

The proof of Theorem \ref{thm:equivsem} uses some useful lemmas, such as:
\begin{lem}[Free locations not in $f$ are free in memory]
	\[w\sim_f v',\mem\Rightarrow m|_{\FLoc(w)-\dom(f)}=\bot\]
\end{lem}

\begin{lem}[Equivalence is preserved by extension of memory]
	\[w\sim_f v',\mem\text{ and }m\sqsubseteq m'\text{ and }m'|_{\FLoc(w)-\dom(f)}=\bot\Rightarrow w\sim_f v',m\]
\end{lem}

\begin{lem}[Equivalence only cares about $f$ on free locations]
	\[w\sim_f v',\mem\text{ and }f|_{\FLoc(w)}=f|_{\FLoc(w)}\Rightarrow w\sim_{f'}v',m\]
\end{lem}

\begin{lem}[Extending equivalence on free locations]
	\[w\sim_f v',\mem\text{ and }\ell\not\in\dom(f)\text{ and }\ell\not\in\dom(\mem)\Rightarrow\forall u',w\sim_{f[\ell\mapsto\ell]}v',m[\ell\mapsto u']\]
\end{lem}

\begin{lem}[Substitution of values]
	\[w\sim_f v',\mem\text{ and }f(\ell)=\ell'\text{ and }\mem(\ell')=u'\text{ and }u\sim_{f-\ell}u',\mem\Rightarrow w[u/\ell]\sim_{f-\ell}v',\mem\]
\end{lem}

\begin{lem}[Substitution of locations]
	\[w\sim_f v',\mem\text{ and }\ell\in\dom(f)\text{ and }\nu\not\in\FLoc(w)\Rightarrow w[\nu/\ell]\sim_{f\circ(\nu\leftrightarrow\ell)}v',\mem\]
\end{lem}

\section{Generating and Resolving Events}
Now we formulate the semantics for generating events.

\begin{figure}[h!]
	\centering
	\small
	\begin{tabular}{rrcll}
		Event       & $\event$ & $\rightarrow$ & $\InitE$           & initial environment \\
		            &          & $\vbar$       & $\ReadE(\event,x)$ & read event          \\
		            &          & $\vbar$       & $\CallE(\event,v)$ & call event          \\
		Environment & $\ctx$   & $\rightarrow$ & $\cdots$                                 \\
		            &          & $\vbar$       & $[\event]$         & answer to an event  \\
		Value       & $v$      & $\rightarrow$ & $\cdots$                                 \\
		            &          & $\vbar$       & $\event$           & answer to an event
	\end{tabular}
	\caption{Definition of the semantic domains with events. All other semantic domains are equal to Figure \ref{fig:domain}.}
	\label{fig:eventdomain}
\end{figure}

We extend how to read weak values given an environment.
\begin{align*}
	\bullet(x)  & \triangleq\bot             &  &  & ((x',\ell)\cons\ctx)(x) & \triangleq (x=x'?\ell:\ctx(x)) \\
	[\event](x) & \triangleq\ReadE(\event,x) &  &  & ((x',w)\cons\ctx)(x)    & \triangleq (x=x'?w:\ctx(x))    \\
\end{align*}

Then we need to add only one rule to the semantics in Figure \ref{fig:bigstep} for the semantics to incorporate events.
	{\small
		\[
			\begin{InfRule}{AppEvent}
				\hypo{
					\ctx\vdash e_{1}
					\Downarrow
					\event
				}
				\hypo{
					\ctx\vdash e_{2}
					\Downarrow
					v
				}
				\infer2{
					\ctx\vdash e_{1}\:e_{2}
					\Downarrow
					\CallE(\event,v)
				}
			\end{InfRule}
		\]
	}

Now we need to formulate the \emph{concrete linking} rules.
The concrete linking rule $\ctx_0\semlink w$, given an answer $\ctx_0$ to the $\InitE$ event, resolves all events within $w$ to obtain a set of final results.

\begin{figure}
	\begin{align*}
		\fbox{$\semlink\in\Ctx\rightarrow\Event\rightarrow\pset(\Value)$}                                                                                                                               \\
		\ctx_0\semlink\InitE\triangleq                         & \{\ctx_0\}                                                                                                                             \\
		\ctx_0\semlink\ReadE(\event,x)\triangleq               & \{v_+|\ctx_+\in\ctx_0\semlink\event\land\ctx_+(x)=v_+\}                                                                                \\
		\cup                                                   & \{\openloc{\mu.v_+}{v_+}|\ctx_+\in\ctx_0\semlink E\land\ctx_+(x)=\mu.v_+\}                                                             \\
		\ctx_0\semlink\CallE(\event,v)\triangleq               & \{v_+'|\langle\lambda x.e,\ctx_+\rangle\in\ctx_0\semlink E\land v_+\in\ctx_0\semlink v\land(x,v_+)\cons\ctx_+\vdash e\Downarrow v_+'\} \\
		\cup                                                   & \{\CallE(\event_+,v_+)|\event_+\in\ctx_0\semlink E\land v_+\in\ctx_0\semlink v\}                                                       \\
		\fbox{$\semlink\in\Ctx\rightarrow\Ctx\rightarrow\pset(\Ctx)$}                                                                                                                                   \\
		\ctx_0\semlink\bullet\triangleq                        & \{\bullet\}                                                                                                                            \\
		\ctx_0\semlink(x,\ell)\cons\ctx\triangleq              & \{(x,\ell)\cons\ctx_+|\ctx_+\in\ctx_0\semlink\ctx\}                                                                                    \\
		\ctx_0\semlink(x,w)\cons\ctx\triangleq                 & \{(x,w_+)\cons\ctx_+|w_+\in\ctx_0\semlink w\land\ctx_+\in\ctx_0\semlink\ctx\}                                                          \\
		\ctx_0\semlink[E]\triangleq                            & \{\ctx_+|\ctx_+\in\ctx_0\semlink\event\}\cup\{[\event_+]|\event_+\in\ctx_0\semlink\event\}                                             \\
		\fbox{$\semlink\in\Ctx\rightarrow\Value\rightarrow\pset(\Value)$}                                                                                                                               \\
		\ctx_0\semlink\langle\lambda x.e,\ctx\rangle\triangleq & \{\langle\lambda x.e,\ctx_+\rangle|\ctx_+\in\ctx_0\semlink\ctx\}                                                                       \\
		\fbox{$\semlink\in\Ctx\rightarrow\Walue\rightarrow\pset(\Walue)$}                                                                                                                               \\
		\ctx_0\semlink\mu.v\triangleq                          & \{\mu.\closeloc{\ell}{v_+}|\ell\not\in\FLoc(v)\cup\FLoc(\ctx_0)\land v_+\in\ctx_0\semlink\openloc{\ell}{v}\}
	\end{align*}
	\caption{Definition for concrete linking.}
	\label{fig:conclink}
\end{figure}

Concrete linking makes sense because of the following theorem.
First define:
\[\eval(e,\ctx)\triangleq\{v|\ctx\vdash e\Downarrow v\}\qquad\eval(e,\Sigma)\triangleq\bigcup_{\ctx\in\Sigma}\eval(e,\ctx)\qquad\ctx_0\semlink W\triangleq\bigcup_{w\in W}(\ctx_0\semlink w)\]
Then the following holds:
\begin{thm}[Soundness of concrete linking]\label{thm:linksound}
	Given $e\in\Expr,\ctx\in\Ctx,v\in\Value$,
	\[\forall\ctx_0\in\Ctx:\eval(e,\ctx_0\semlink\ctx)\subseteq\ctx_0\semlink\eval(e,\ctx)\]
\end{thm}

The proof of Theorem \ref{thm:linksound} uses some useful lemmas, such as:
\begin{lem}[Linking distributes under substitution]
	Let $\ctx_0$ be the external environment that is linked with locally closed weak values $w$ and $u$.
	For all $\ell\not\in\FLoc(\ctx_0)$, we have:
	\[\forall w_+,u_+:w_+\in\ctx_0\semlink w\wedge u_+\in\ctx_0\semlink u\Rightarrow\{u_+\leftarrow\ell\}w_+\in\ctx_0\semlink\{u\leftarrow\ell\}w\]
\end{lem}
\begin{lem}[Linking is compatible with reads]
	Let $\ctx_0$ be the external environment that is linked with some environment $\ctx$.
	Let $v$ be the value obtained from reading $x$ from $\ctx$.
	Let $\text{unfold}:\Walue\rightarrow\Value$ be defined as:
	\[\text{unfold}(\mu.v)\triangleq\openloc{\mu.v}{v}\qquad\text{unfold}(v)\triangleq v\]
	Then for all $\ctx_+\in\ctx_0\semlink\ctx$, we have:
	\[\exists w_+\in\Walue:\ctx_+(x)=w_+\land\text{unfold}(w_+)\in\ctx_0\semlink v\]
\end{lem}
\clearpage
\section{CFA}
\begin{center}
	\begin{tabular}{rrcl}
		Program point        & $p$                & $\in$         & $\mathbb{P}\triangleq\{\text{finite set of program points}\}$                                  \\
		Labelled expression  & $pe$               & $\in$         & $\mathbb{P}\times\Expr$                                                                        \\
		Labelled location    & $\llabel{p}{\ell}$ & $\in$         & $\mathbb{P}\times\Loc$                                                                         \\
		Collecting semantics & $t$                & $\in$         & $\mathbb{T}\triangleq\mathbb{P}\rightarrow\pset(\Ctx+\Ctx\times\Value)$                        \\
		Labelled expression  & $pe$               & $\rightarrow$ & $\plabel{p}{e}$                                                                                \\
		Expression           & $e$                & $\rightarrow$ & $x$ | $\lambda x.pe$ | $pe$ $pe$ | ${pe}\synlink{pe}$ | $\varepsilon$ | $x\:\texttt{=}\:pe;pe$
	\end{tabular}
\end{center}
\begin{align*}
	                           &                                                                                                                                                                                            & \fbox{$\Step:\mathbb{T}\rightarrow\mathbb{T}$}                   \\
	\Step(\ptree)   \triangleq & \bigcup_{p\in\mathbb{P}}\step(\ptree,p)                                                                                                                                                                                                                       \\
	                           &                                                                                                                                                                                            & \fbox{$\step:(\mathbb{T}\times\mathbb{P})\rightarrow\mathbb{T}$} \\
	\step(\ptree,p) \triangleq & [p\mapsto\{(\ctx,v)|\ctx\in\ptree(p)\text{ and }\ctx(x)=v\}]                                                                                                                               & \text{when }\plabel{p}{x}                                        \\
	\cup                       & [p\mapsto\{(\ctx,\openloc{\mu.v}{v})|\ctx\in \ptree(p)\text{ and }\ctx(x)=\mu.v\}]                                                                                                                                                                            \\
	\step(\ptree,p) \triangleq & [p\mapsto\{(\ctx,\langle\lambda x.p',\ctx\rangle)|\ctx\in \ptree(p)\}]                                                                                                                     & \text{when }\plabel{p}{\lambda x.p'}                             \\
	\step(\ptree,p) \triangleq & [p_1\mapsto\{\ctx\in\Ctx|\ctx\in \ptree(p)\}]                                                                                                                                              & \text{when }\plabel{p}{p_1\:p_2}                                 \\
	\cup                       & [p_2\mapsto\{\ctx\in\Ctx|\ctx\in \ptree(p)\}]                                                                                                                                                                                                                 \\
	\cup                       & \bigcup_{\ctx\in \ptree(p)}\bigcup_{(\ctx,\langle\lambda x.p',\ctx_1\rangle)\in \ptree(p_1)}[p'\mapsto\{(x,v_2)\cons\ctx_1|(\ctx,v_2)\in \ptree(p_2)\}]                                                                                                       \\
	\cup                       & [p\mapsto\bigcup_{\ctx\in \ptree(p)}\bigcup_{(\ctx,\langle\lambda x.p',\ctx_1\rangle)\in \ptree(p_1)}\bigcup_{(\ctx,v_2)\in \ptree(p_2)}\{(\ctx,v)|((x,v_2)\cons\ctx_1,v)\in \ptree(p')\}]                                                                    \\
	\cup                       & [p\mapsto\bigcup_{\ctx\in \ptree(p)}\{(\ctx,\CallE(\event_1,v_2))|(\ctx,\event_1)\in \ptree(p_1)\text{ and }(\ctx,v_2)\in \ptree(p_2)\}]                                                                                                                      \\
	\step(\ptree,p) \triangleq & [p_1\mapsto\{\ctx|\ctx\in \ptree(p)\}]                                                                                                                                                     & \text{when }\plabel{p}{{p_1}\synlink{p_2}}                       \\
	\cup                       & [p_2\mapsto\bigcup_{\ctx\in \ptree(p)}\{\ctx_1|(\ctx,\ctx_1)\in \ptree(p_1)\}]                                                                                                                                                                                \\
	\cup                       & [p\mapsto\bigcup_{\ctx\in \ptree(p)}\bigcup_{(\ctx,\ctx_1)\in \ptree(p_1)}\{(\ctx,v_2)|(\ctx_1,v_2)\in \ptree(p_2)\}]                                                                                                                                         \\
	\step(\ptree,p) \triangleq & [p\mapsto\{(\ctx,\bullet)|\ctx\in \ptree(p)\}]                                                                                                                                             & \text{when }\plabel{p}{\varepsilon}                              \\
	\step(\ptree,p) \triangleq & [p_1\mapsto\{(x,\llabel{p_1}{\ell})\cons\ctx|\ctx\in \ptree(p)\text{ and }\ell\not\in\FLoc(\ctx)\}]                                                                                        & \text{when }\plabel{p}{x\:\texttt{=}\:p_1;p_2}                   \\
	\cup                       & [p_2\mapsto\{(x,\llabel{p_1}{\mu}.\closeloc{\ell}{v_1})\cons\ctx|\ctx\in \ptree(p)\text{ and }((x,\llabel{p_1}{\ell})\cons\ctx,v_1)\in \ptree(p_1)\}]                                                                                                         \\
	\cup                       & [p\mapsto\bigcup_{\ctx\in \ptree(p)}\{(\ctx,(x,\llabel{p_1}{\mu}.\closeloc{\ell}{v_1})\cons\ctx_2)|((x,\llabel{p_1}{\mu}.\closeloc{\ell}{v_1})\cons\ctx,\ctx_2)\in \ptree(p_2)\}]
\end{align*}
The proof tree $\ptree$ computed by
\[\ptree\triangleq\lfp(\lambda\ptree.\Step(\ptree)\cup\ptree_{\text{init}})\quad\text{where }\ptree_{\text{init}}=[p_0\mapsto\{\ctx_0\}]\]
contains all derivations of the form $\evjudg{\ctx_0}{p_0}{v_0}$ for some $v_0$.
That is, $(\ctx,v)$ is contained in $t_0(p)$ if and only if $\evjudg{\ctx}{p}{v}$ must be contained in a valid derivation for the judgment $\evjudg{\ctx_0}{p_0}{v_0}$.
\begin{center}
	\begin{tabular}{rrcl}
		Abstract event       & $\Abs{\event}$                 & $\in$         & $\Abs{\Event}$                                                                     \\
		Abstract environment & $\Abs{\ctx}$                   & $\in$         & $\Abs{\Ctx}\triangleq(\fin{\ExprVar}{\pset(\mathbb{P})})\times\pset(\Abs{\Event})$ \\
		Abstract closure     & $\langle\lambda x.p,p'\rangle$ & $\in$         & $\Abs{\text{Clos}}\triangleq\ExprVar\times\mathbb{P}\times\mathbb{P}$              \\
		Abstract value       & $\Abs{v}$                      & $\in$         & $\Abs{\Value}\triangleq\Abs{\Ctx}\times\pset(\Abs{\text{Clos}})$                   \\
		Abstract semantics   & $\Abs\ptree$                   & $\in$         & $\Abs{\mathbb{T}}\triangleq\mathbb{P}\rightarrow\Abs{\Ctx}\times\Abs{\Value}$      \\
		Abstract event       & $\Abs{\event}$                 & $\rightarrow$ & $\Abs{\InitE}$ | $\Abs{\ReadE}(p,x)$ | $\Abs{\CallE}(p,p)$
	\end{tabular}
\end{center}
\begin{figure}[h!]
	\centering
	\small
	\begin{flushright}
    \fbox{$\ctx\le(\Abs\ctx,\Abs\ptree)$}
	\end{flushright}
	\[
		\begin{InfRule}{Conc-Nil}
      \infer0{\bullet\le\Abs\ctx}
		\end{InfRule}\:
    \begin{InfRule}{Conc-ENil}
      \hypo{\event\le(\Abs\ctx,\varnothing)}
      \infer1{[\event]\le\Abs\ctx}
    \end{InfRule}\:
		\begin{InfRule}{Conc-ConsLoc}
			\hypo{p\in\Abs\ctx.1(x)}
			\hypo{\ctx\le\Abs\ctx}
      \infer2{(x,\llabel{p}{\ell})\cons\ctx\le\Abs\ctx}
		\end{InfRule}\:
    \begin{InfRule}{Conc-ConsWVal}
      \hypo{p\in\Abs\ctx.1(x)}
      \hypo{w\le\Abs\ptree(p).2}
      \hypo{\ctx\le\Abs\ctx}
			\infer3{(x,w)\cons\ctx\le\Abs\ctx}
		\end{InfRule}
	\]
  \begin{flushright}
    \fbox{$w\le(\Abs{v},\Abs\ptree)$}
	\end{flushright}
	\[
		\begin{InfRule}{Conc-Clos}
      \hypo{\langle\lambda x.p,p'\rangle\in\Abs{v}.2}
			\hypo{\ctx\le\Abs\ptree(p').1}
      \infer2{\langle\lambda x.p,\ctx\rangle\le\Abs{v}}
		\end{InfRule}\:
		\begin{InfRule}{Conc-Rec}
      \hypo{\forall\ell,\:\openloc{\llabel{p}{\ell}}{v}\le\Abs{v}}
      \infer1{\llabel{p}{\mu}.v\le\Abs{v}}
		\end{InfRule}
  \]

	\[
    \begin{InfRule}{Conc-Init}
      \hypo{\Abs\InitE\in\Abs{v}.1.2}
      \infer1{\InitE\le\Abs{v}}
    \end{InfRule}\:
    \begin{InfRule}{Conc-Read}
      \hypo{\Abs\ReadE(p,x)\in\Abs{v}.1.2}
      \hypo{[\event]\le\Abs\ptree(p).1}
      \infer2{\ReadE(\event,x)\le\Abs{v}}
    \end{InfRule}\:
    \begin{InfRule}{Conc-Call}
      \hypo{\Abs\CallE(p_1,p_2)\in\Abs{v}.1.2}
      \hypo{\event\le\Abs\ptree(p_1).2}
      \hypo{v\le\Abs\ptree(p_2).2}
      \infer3{\CallE(\event,v)\le\Abs{v}}
    \end{InfRule}
  \]
  \caption{The concretization relation between weak values and abstract values. $\Abs\ptree$ is omitted.}
	\label{fig:concretrel}
\end{figure}

The concretization function $\gamma$ that sends an element of $\Abs{\mathbb{T}}$ to $\mathbb{T}$ is defined as:
\[\gamma(\Abs\ptree)\triangleq\lambda p.\{\ctx|\ctx\le(\Abs\ptree(p).1,\Abs\ptree)\}\cup\{(\ctx,v)|\ctx\le(\Abs\ptree(p).1,\Abs\ptree)\text{ and }v\le(\Abs\ptree(p).2,\Abs\ptree)\}\]
where $\le$ is the concretization relation that is inductively defined in Figure \ref{fig:concretrel}.

Now the abstract semantic function can be given.
\begin{align*}
	                                   &                                                                                                                                          & \fbox{$\Abs\Step:\Abs{\mathbb{T}}\rightarrow\Abs{\mathbb{T}}$}                   \\
	\Abs\Step(\Abs\ptree)   \triangleq & \bigsqcup_{p\in\mathbb{P}}\Abs\step(\Abs\ptree,p)                                                                                                                                                                           \\
	                                   &                                                                                                                                          & \fbox{$\Abs\step:(\Abs{\mathbb{T}}\times\mathbb{P})\rightarrow\Abs{\mathbb{T}}$} \\
	\Abs\step(\Abs\ptree,p) \triangleq & [p\mapsto\bigsqcup_{p'\in\Abs\ptree(p).1.1(x)}(\bot,\Abs\ptree(p').2)]                                                                   & \text{when }\plabel{p}{x}                                                        \\
	\sqcup                             & [p\mapsto(\bot,(([],\{\Abs\ReadE(p,x)\}),\varnothing))]                                                                                  & \text{if }\Abs\ptree(p).1.2\neq\varnothing                                       \\
	\Abs\step(\Abs\ptree,p) \triangleq & [p\mapsto(\bot,(\bot,\{\langle\lambda x.p',p\rangle\}))]                                                                                 & \text{when }\plabel{p}{\lambda x.p'}                                             \\
	\Abs\step(\Abs\ptree,p) \triangleq & [p_1\mapsto(\Abs\ptree(p).1,\bot)]                                                                                                       & \text{when }\plabel{p}{p_1\:p_2}                                                 \\
	\sqcup                             & [p_2\mapsto(\Abs\ptree(p).1,\bot)]                                                                                                                                                                                          \\
	\sqcup                             & \bigsqcup_{\langle\lambda x.p',p''\rangle\in\Abs\ptree(p_1).2.2}[p'\mapsto(\Abs\ptree(p'').1\sqcup([x\mapsto\{p_2\}],\varnothing),\bot)]                                                                                    \\
	\sqcup                             & [p\mapsto\bigsqcup_{\langle\lambda x.p',\_\rangle\in\Abs\ptree(p_1).2.2}(\bot,\Abs\ptree(p').2)]                                                                                                                            \\
	\sqcup                             & [p\mapsto(\bot,(([],\{\Abs\CallE(p_1,p_2)\}),\varnothing))]                                                                              & \text{if }\Abs\ptree(p_1).2.1.2\neq\varnothing                                   \\
	\Abs\step(\Abs\ptree,p) \triangleq & [p_1\mapsto(\Abs\ptree(p).1,\bot)]                                                                                                       & \text{when }\plabel{p}{{p_1}\synlink{p_2}}                                       \\
	\sqcup                             & [p_2\mapsto(\Abs\ptree(p_1).2.1,\bot)]                                                                                                                                                                                      \\
	\sqcup                             & [p\mapsto(\bot,\Abs\ptree(p_2).2)]                                                                                                                                                                                          \\
	\Abs\step(\Abs\ptree,p) \triangleq & \bot                                                                                                                                     & \text{when }\plabel{p}{\varepsilon}                                              \\
	\Abs\step(\Abs\ptree,p) \triangleq & [p_1\mapsto(\Abs\ptree(p).1\sqcup([x\mapsto\{p_1\}],\varnothing),\bot)]                                                                  & \text{when }\plabel{p}{x\:\texttt{=}\:p_1;p_2}                                   \\
	\sqcup                             & [p_2\mapsto(\Abs\ptree(p).1\sqcup([x\mapsto\{p_1\}],\varnothing),\bot)]                                                                                                                                                     \\
	\sqcup                             & [p\mapsto(\bot,(\Abs\ptree(p_2).2.1\sqcup([x\mapsto\{p_1\}],\varnothing),\varnothing))]
\end{align*}
The abstract proof tree $\Abs{\ptree}$ computed by
\[\Abs{\ptree}\triangleq\lfp(\lambda\Abs\ptree.\Abs\Step(\Abs\ptree)\sqcup\Abs{\ptree}_{\text{init}})\quad\text{where }\ptree_{\text{init}}\sqsubseteq\gamma(\Abs{\ptree}_{\text{init}})\]
is a sound abstraction of $\ptree$.

Now we define a sound linking operator that abstracts $\semlink$.
Assume we have
\[\ctx_0\le(\Abs{\ctx}_0,\Abs{\ptree}_0)\quad\ptree\sqsubseteq\gamma(\Abs\ptree)\]
we define:
\[\ctx_0\semlink\ptree\triangleq\lambda p.\left(\ctx_0\semlink\ptree(p)\right)\]

We want to define $\Abs\semlink$ so that the following holds:
\[\ctx_0\semlink\ptree\sqsubseteq\gamma((\Abs{\ctx}_0,\Abs{\ptree}_0)\Abs\semlink\Abs\ptree)\]

This is defined by
\[(\Abs{\ctx}_0,\Abs{\ptree}_0)\Abs\semlink\Abs\ptree\triangleq\lfp(\lambda\Abs{\ptree}_+.\Abs\Step(\Abs{\ptree}_+)\sqcup\Abs\Link(\Abs{\ctx}_0,\mathsf{\event}(\Abs\ptree),\Abs{\ptree}_+)\sqcup\Abs{\ptree}_0\sqcup\mathsf{V}(\Abs\ptree))\]
where
\[\mathsf{\event}(\Abs\ptree)\in\mathbb{P}\rightarrow\pset(\Abs\Event)^2\quad\mathsf{V}(\Abs\ptree)\in\Abs{\mathbb{T}}\quad\Abs\Link(\Abs\ctx,\mathcal{E},\Abs\ptree)\in\Abs{\mathbb{T}}\]
are defined by
\[
	\mathsf{\event}(\Abs\ptree)\triangleq\lambda p.(\Abs\ptree(p).1.2,\Abs\ptree(p).2.1.2)
	\quad
	\mathsf{V}(\Abs\ptree)\triangleq\lambda p.((\Abs\ptree(p).1.1,\varnothing),((\Abs\ptree(p).2.1.1,\varnothing),\Abs\ptree(p).2.2))
\]
and
\[\Abs\Link(\Abs\ctx,\mathcal{E},\Abs\ptree)\triangleq\bigsqcup_{\Abs\event\in\mathcal{\event}(p).1}\Abs\link_1(\Abs\ctx,\Abs\event,\Abs\ptree,p)\sqcup\bigsqcup_{\Abs\event\in\mathcal{E}(p).2}\Abs\link_2(\Abs\ctx,\Abs\event,\Abs\ptree,p)\]
where
\[\Abs\link_1(\Abs\ctx,\Abs\event,\Abs\ptree,p)\in\Abs{\mathbb{T}}\quad\Abs\link_2(\Abs\ctx,\Abs\event,\Abs\ptree,p)\in\Abs{\mathbb{T}}\]
are defined by
\begin{align*}
	\Abs\link_1(\Abs\ctx,\Abs\event,\Abs\ptree,p) \triangleq & [p\mapsto(\Abs\ctx,\bot)]                                                                                                                & \text{when }\Abs\event=\Abs\InitE              \\
	\Abs\link_1(\Abs\ctx,\Abs\event,\Abs\ptree,p) \triangleq & [p\mapsto\bigsqcup_{p''\in\Abs\ptree(p').1.1(x)}(\Abs\ptree(p'').2.1,\bot)]                                                              & \text{when }\Abs\event=\Abs\ReadE(p',x)        \\
	\sqcup                                                   & [p\mapsto(([],\{\Abs\ReadE(p,x)\}),\bot)]                                                                                                & \text{if }\Abs\ptree(p).1.2\neq\varnothing     \\
	\Abs\link_1(\Abs\ctx,\Abs\event,\Abs\ptree,p) \triangleq & \bigsqcup_{\langle\lambda x.p',p''\rangle\in\Abs\ptree(p_1).2.2}[p'\mapsto(\Abs\ptree(p'').1\sqcup([x\mapsto\{p_2\}],\varnothing),\bot)] & \text{when }\Abs\event=\Abs\CallE(p_1,p_2)     \\
	\sqcup                                                   & [p\mapsto\bigsqcup_{\langle\lambda x.p',\_\rangle\in\Abs\ptree(p_1).2.2}(\Abs\ptree(p').2.1,\bot)]                                                                                        \\
	\sqcup                                                   & [p\mapsto(([],\{\Abs\CallE(p_1,p_2)\}),\bot)]                                                                                            & \text{if }\Abs\ptree(p_1).2.1.2\neq\varnothing \\
	\Abs\link_2(\Abs\ctx,\Abs\event,\Abs\ptree,p) \triangleq & [p\mapsto(\bot,(\Abs\ctx,\varnothing))]                                                                                                  & \text{when }\Abs\event=\Abs\InitE              \\
	\Abs\link_2(\Abs\ctx,\Abs\event,\Abs\ptree,p) \triangleq & [p\mapsto\bigsqcup_{p''\in\Abs\ptree(p').1.1(x)}(\bot,\Abs\ptree(p'').2)]                                                                & \text{when }\Abs\event=\Abs\ReadE(p',x)        \\
	\sqcup                                                   & [p\mapsto(\bot,(([],\{\Abs\ReadE(p,x)\}),\varnothing))]                                                                                  & \text{if }\Abs\ptree(p).1.2\neq\varnothing     \\
	\Abs\link_2(\Abs\ctx,\Abs\event,\Abs\ptree,p) \triangleq & \bigsqcup_{\langle\lambda x.p',p''\rangle\in\Abs\ptree(p_1).2.2}[p'\mapsto(\Abs\ptree(p'').1\sqcup([x\mapsto\{p_2\}],\varnothing),\bot)] & \text{when }\Abs\event=\Abs\CallE(p_1,p_2)     \\
	\sqcup                                                   & [p\mapsto\bigsqcup_{\langle\lambda x.p',\_\rangle\in\Abs\ptree(p_1).2.2}(\bot,\Abs\ptree(p').2)]                                                                                          \\
	\sqcup                                                   & [p\mapsto(\bot,(([],\{\Abs\CallE(p_1,p_2)\}),\varnothing))]                                                                              & \text{if }\Abs\ptree(p_1).2.1.2\neq\varnothing
\end{align*}

\end{document}

