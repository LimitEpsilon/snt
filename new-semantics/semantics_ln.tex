%! TEX program = xelatex
\documentclass{article}
\usepackage{geometry}
\geometry{
  a4paper,
  total={170mm,257mm},
  left=20mm,
  top=20mm,
}
\usepackage{setspace}
\usepackage{graphicx}
\usepackage{xcolor}
\usepackage{kotex}
\usepackage{csquotes}
\usepackage{tabularray}
\usepackage{relsize}

%%% Typesetting for listings
\usepackage{listings}

\definecolor{dkgreen}{rgb}{0,0.6,0}
\definecolor{ltblue}{rgb}{0,0.4,0.4}
\definecolor{dkviolet}{rgb}{0.3,0,0.5}

%%% lstlisting coq style (inspired from a file of Assia Mahboubi)
\lstdefinelanguage{Coq}{ 
    % Anything betweeen $ becomes LaTeX math mode
    mathescape=true,
    % Comments may or not include Latex commands
    texcl=false, 
    % Vernacular commands
    morekeywords=[1]{Section, Module, End, Require, Import, Export,
        Variable, Variables, Parameter, Parameters, Axiom, Hypothesis,
        Hypotheses, Notation, Local, Tactic, Reserved, Scope, Open, Close,
        Bind, Delimit, Definition, Let, Ltac, Fixpoint, CoFixpoint, Add,
        Morphism, Relation, Implicit, Arguments, Unset, Contextual,
        Strict, Prenex, Implicits, Inductive, CoInductive, Record,
        Structure, Canonical, Coercion, Context, Class, Global, Instance,
        Program, Infix, Theorem, Lemma, Corollary, Proposition, Fact,
        Remark, Example, Proof, Goal, Save, Qed, Defined, Hint, Resolve,
        Rewrite, View, Search, Show, Print, Printing, All, Eval, Check,
        Projections, inside, outside, Def},
    % Gallina
    morekeywords=[2]{forall, exists, exists2, fun, fix, cofix, struct,
        match, with, end, as, in, return, let, if, is, then, else, for, of,
        nosimpl, when},
    % Sorts
    morekeywords=[3]{Type, Prop, Set, true, false, option},
    % Various tactics, some are std Coq subsumed by ssr, for the manual purpose
    morekeywords=[4]{pose, set, move, case, elim, apply, clear, hnf,
        intro, intros, generalize, rename, pattern, after, destruct,
        induction, using, refine, inversion, injection, rewrite, congr,
        unlock, compute, ring, field, fourier, replace, fold, unfold,
        change, cutrewrite, simpl, have, suff, wlog, suffices, without,
        loss, nat_norm, assert, cut, trivial, revert, bool_congr, nat_congr,
        symmetry, transitivity, auto, split, left, right, autorewrite},
    % Terminators
    morekeywords=[5]{by, done, exact, reflexivity, tauto, romega, omega,
        assumption, solve, contradiction, discriminate},
    % Control
    morekeywords=[6]{do, last, first, try, idtac, repeat},
    % Comments delimiters, we do turn this off for the manual
    morecomment=[s]{(*}{*)},
    % Spaces are not displayed as a special character
    showstringspaces=false,
    % String delimiters
    morestring=[b]",
    morestring=[d],
    % Size of tabulations
    tabsize=3,
    % Enables ASCII chars 128 to 255
    extendedchars=false,
    % Case sensitivity
    sensitive=true,
    % Automatic breaking of long lines
    breaklines=false,
    % Default style fors listings
    basicstyle=\small,
    % Position of captions is bottom
    captionpos=b,
    % flexible columns
    columns=[l]flexible,
    % Style for (listings') identifiers
    identifierstyle={\ttfamily\color{black}},
    % Style for declaration keywords
    keywordstyle=[1]{\ttfamily\color{dkviolet}},
    % Style for gallina keywords
    keywordstyle=[2]{\ttfamily\color{dkgreen}},
    % Style for sorts keywords
    keywordstyle=[3]{\ttfamily\color{ltblue}},
    % Style for tactics keywords
    keywordstyle=[4]{\ttfamily\color{dkblue}},
    % Style for terminators keywords
    keywordstyle=[5]{\ttfamily\color{dkred}},
    %Style for iterators
    %keywordstyle=[6]{\ttfamily\color{dkpink}},
    % Style for strings
    stringstyle=\ttfamily,
    % Style for comments
    commentstyle={\ttfamily\color{dkgreen}},
    %moredelim=**[is][\ttfamily\color{red}]{/&}{&/},
    literate=
    {\\forall}{{\color{dkgreen}{$\forall\;$}}}1
    {\\exists}{{$\exists\;$}}1
    {\ge -}{{$\geftarrow\;$}}1
    {=>}{{$\Rightarrow\;$}}1
    {==}{{\code{==}\;}}1
    {==>}{{\code{==>}\;}}1
    %    {:>}{{\code{:>}\;}}1
    {->}{{$\rightarrow\;$}}1
    {\ge ->}{{$\geftrightarrow\;$}}1
    {\ge ==}{{$\geq\;$}}1
    {\#}{{$^\star$}}1 
    {\\o}{{$\circ\;$}}1 
    {\@}{{$\cdot$}}1 
    {\/\\}{{$\text{ and }\;$}}1
    {\\\/}{{$\vee\;$}}1
    {++}{{\code{++}}}1
    {~}{{\ }}1
    {\@\@}{{$@$}}1
    {\\mapsto}{{$\mapsto\;$}}1
    {\\hline}{{\rule{\linewidth}{0.5pt}}}1
    %
}[keywords,comments,strings]

%%% Math settings
\usepackage{amssymb,amsmath,amsthm,mathtools}
\usepackage[math-style=TeX,bold-style=TeX]{unicode-math}
\theoremstyle{definition}
\newtheorem{definition}{Definition}[section]
\newtheorem{example}{Example}[section]
\newtheorem{lem}{Lemma}[section]
\newtheorem{thm}{Theorem}[section]
\newtheorem{cor}{Corollary}[section]
\newtheorem{clm}{Claim}[section]

%%% Font settings
%\setmainfont{Libertinus Serif}
%\setsansfont{Libertinus Sans}[Scale=MatchUppercase]
%\setmonofont{JuliaMono}[Scale=MatchLowercase]
%\setmainhangulfont{Noto Serif CJK KR}
%\setmonohangulfont{D2Coding}

%%% PL constructs
\usepackage{galois}
\usepackage{ebproof}
%% InfRule command
\ExplSyntaxOn
\NewTblrEnviron{@ruleenv}
\SetTblrInner[@ruleenv]{belowsep=0pt,stretch=0}
\SetTblrOuter[@ruleenv]{baseline=b}
\NewDocumentEnvironment { InfRule } { m +b }
  {
    \begin{@ruleenv}{l}
      \smaller{\textsc{#1}} \\
      \begin{prooftree} #2 \end{prooftree}
    \end{@ruleenv}
  }
  {}
\ExplSyntaxOff

%% Modification for \infer
\ExplSyntaxOn
\int_new:N \g__ebproof_sublevel_int
\box_new:N \g__ebproof_substack_box
\seq_new:N \g__ebproof_substack_seq

\cs_new:Nn \__ebproof_clear_substack:
  {
    \int_gset:Nn \g__ebproof_sublevel_int { 0 }
    \hbox_gset:Nn \g__ebproof_substack_box { }
    \seq_gclear:N \g__ebproof_substack_seq
  }

\cs_new:Nn \__ebproof_subpush:N
  {
    \int_gincr:N \g__ebproof_sublevel_int
    \hbox_gset:Nn \g__ebproof_substack_box
      { \hbox_unpack:N \g__ebproof_substack_box \box_use:c { \__ebproof_box:N #1 } }
    \seq_gput_left:Nv \g__ebproof_substack_seq
      { \__ebproof_marks:N #1 }
  }

\cs_new:Nn \__ebproof_subpop:N
  {
    \int_compare:nNnTF { \g__ebproof_sublevel_int } > { 0 }
      {
        \int_gdecr:N \g__ebproof_sublevel_int
        \hbox_gset:Nn \g__ebproof_substack_box {
          \hbox_unpack:N \g__ebproof_substack_box
          \box_gset_to_last:N \g_tmpa_box
        }
        \box_set_eq_drop:cN { \__ebproof_box:N #1 } \g_tmpa_box
        \seq_gpop_left:NN \g__ebproof_substack_seq \l_tmpa_tl
        \tl_set_eq:cN { \__ebproof_marks:N #1 } \l_tmpa_tl
      }
      { \PackageError{ebproof}{Missing~premiss~in~a~proof~tree}{} \__ebproof_clear:N #1 }
  }

\cs_new:Nn \__ebproof_append_subvertical:NN
  {
    \bool_if:NTF \l__ebproof_updown_bool
      { \__ebproof_append_above:NN #1 #2 }
      { \__ebproof_append_below:NN #1 #2 }
  }

\cs_new:Nn \__ebproof_join_subvertical:n
  {
    \group_begin:
    \__ebproof_subpop:N \l__ebproof_a_box
    \prg_replicate:nn { #1 - 1 }
      {
        \__ebproof_subpop:N \l__ebproof_b_box
        \__ebproof_enlarge_conclusion:NN \l__ebproof_b_box \l__ebproof_a_box

        \__ebproof_make_vertical:Nnnn \l__ebproof_c_box
          { \__ebproof_mark:Nn \l__ebproof_b_box {axis} - \__ebproof_mark:Nn \l__ebproof_b_box {left} }
          { \__ebproof_mark:Nn \l__ebproof_b_box {right} - \__ebproof_mark:Nn \l__ebproof_b_box {left} }
          { \skip_vertical:N \l__ebproof_rule_margin_dim }
        \__ebproof_vcenter:N \l__ebproof_b_box
        \__ebproof_append_subvertical:NN \l__ebproof_a_box \l__ebproof_c_box

        \__ebproof_append_subvertical:NN \l__ebproof_a_box \l__ebproof_b_box
      }
    \__ebproof_push:N \l__ebproof_a_box
    \group_end:
  }

\cs_new:Nn \__ebproof_renew_statement:nnn
  {
    \exp_args:Nc \RenewDocumentCommand { ebproof#1 }{ #2 } { #3 }
    \seq_gput_right:Nn \g__ebproof_statements_seq { #1 }
  }
\__ebproof_renew_statement:nnn { infer } { O{} m O{} m }
  {
    \group_begin:
    \__ebproof_restore_statements:
    \keys_set_known:nnN { ebproof / rule~style } { #1 } \l_tmpa_tl
    \keys_set:nV { ebproof } \l_tmpa_tl
    \tl_set:Nn \l__ebproof_right_label_tl { #3 }

    \__ebproof_clear_substack:
    \clist_map_inline:nn { #2 }
      {
        \__ebproof_join_horizontal:n { ##1 }
        \__ebproof_pop:N \l__ebproof_a_box
        \__ebproof_subpush:N \l__ebproof_a_box
      }
    \__ebproof_join_subvertical:n { \clist_count:n { #2 } }

    \__ebproof_push_statement:n { #4 }
    \__ebproof_join_vertical:
    \group_end:
  }
\ExplSyntaxOff
\ebproofset{left label template=\textsc{[\inserttext]}}
\ebproofset{center=false}

%%% Custom commands
\newcommand*{\vbar}{|}
\newcommand*{\finto}{\xrightarrow{\text{\textrm{fin}}}}
\newcommand*{\istype}{\mathrel{⩴}}
\newcommand*{\ortype}{\mathrel{|}}
\newcommand*{\cons}{::}
\newcommand*{\pset}{\mathscr{P}}
\newcommand*{\closeloc}[2]{\prescript{\text{\textbackslash}{#1}}{}{#2}}
\newcommand*{\openloc}[2]{{#1}^{#2}}

\def\ovbarw{1.2mu}
\def\ovbarh{1}
\newcommand*{\ovbar}[1]{\mkern \ovbarw\overline{\mkern-\ovbarw{\smash{#1}\scalebox{1}[\ovbarh]{\vphantom{i}}}\mkern-\ovbarw}\mkern \ovbarw}
\newcommand*{\A}[1]{\overset{\,_{\mbox{\Large .}}}{#1}}
\newcommand*{\Abs}[1]{{#1}^{\#}}
\newcommand*{\Expr}{\text{Expr}}
\newcommand*{\ExprVar}{\text{Var}}
\newcommand*{\Module}{\text{Module}}
\newcommand*{\ModVar}{\text{ModVar}}
\newcommand*{\modid}{d}
\newcommand*{\Time}{\mathbb{T}}
\newcommand*{\ATime}{\A{\Time}}
\newcommand*{\Ctx}{\text{Env}}
\newcommand*{\ctx}{\sigma}
\newcommand*{\Value}{\text{Val}}
\newcommand*{\Walue}{\text{WVal}}
\newcommand*{\Mem}{\text{Mem}}
\newcommand*{\Loc}{\text{Loc}}
\newcommand*{\FLoc}{\text{FLoc}}
\newcommand*{\Left}{\text{Left}}
\newcommand*{\Right}{\text{Right}}
\newcommand*{\Sig}{\text{Sig}}
\newcommand*{\mem}{m}
\newcommand*{\dom}{\text{dom}}
\newcommand*{\AMem}{\A{\text{Mem}}}
\newcommand*{\State}{\text{State}}
\newcommand*{\AState}{\A{\text{State}}}
\newcommand*{\Result}{\text{Result}}
\newcommand*{\AResult}{\A{\text{Result}}}
\newcommand*{\Tick}{\text{Tick}}
\newcommand*{\lfp}{\mathsf{lfp}}
\newcommand*{\Step}{\mathsf{Step}}
\newcommand*{\semarrow}{\Downarrow}
\newcommand*{\asemarrow}{\A{\rightsquigarrow}}
\newcommand*{\synlink}{\rtimes}
\newcommand*{\semlink}{\mathbin{\rotatebox[origin=c]{180}{$\propto$}}}
\newcommand*{\link}[2]{{#1}\rtimes{#2}}
\newcommand*{\mt}{\mathsf{empty}}
\newcommand*{\valid}{\mathsf{valid}}
\newcommand*{\Path}{\text{Path}}
\newcommand*{\equivalent}{\sim}

\newcommand*{\doubleplus}{\ensuremath{\mathbin{+\mkern-3mu+}}}
\newcommand*{\project}{\text{\texttt{:>} }}
\newcommand*{\Exp}{\mathsf{Exp}}
\newcommand*{\Imp}{\mathsf{Imp}}
\newcommand*{\Fin}{\mathsf{Fin}}
\newcommand*{\Link}{\mathsf{Link}}
\newcommand*{\sembracket}[1]{\lBrack{#1}\rBrack}
\newcommand*{\fin}[2]{{#1}\xrightarrow{\text{fin}}{#2}}
\newcommand*{\addr}{\mathsf{addr}}
\newcommand*{\tick}{\mathsf{tick}}
\newcommand*{\modctx}{\mathsf{ctx}}
\newcommand*{\mapinject}[2]{{#2}[{#1}]}
\newcommand*{\inject}[2]{{#2}\langle{#1}\rangle}
\newcommand*{\deletepre}[2]{{#2}\overline{\doubleplus}{#1}}
\newcommand*{\deletemap}[2]{{#1}\overline{[{#2}]}}
\newcommand*{\delete}[2]{{#2}{\langle{#1}\rangle}^{-1}}
\newcommand*{\filter}{\mathsf{filter}}
\newcommand*{\Lete}{\mathtt{val}}
\newcommand*{\Letm}{\mathtt{mod}}

\newcommand*{\ValRel}[1]{\mathcal{V}\sembracket{#1}}
\newcommand*{\ExprRel}[1]{\mathcal{E}\sembracket{#1}}
\newcommand*{\CtxRel}[1]{\mathcal{C}\sembracket{#1}}
\newcommand*{\ModRel}[1]{\mathcal{M}\sembracket{#1}}
\newcommand*{\TyEnv}{\text{TyEnv}}
\newcommand*{\TyVar}{\text{TyVar}}
\newcommand*{\Type}{\text{Type}}
\newcommand*{\Subst}{\text{Subst}}
\newcommand*{\external}{\Gamma_{\text{ext}}}

\newcommand*{\InitE}{\textsf{Init}}
\newcommand*{\ReadE}{\textsf{Read}}
\newcommand*{\CallE}{\textsf{Call}}
\newcommand*{\resolve}{\downarrow_{\ctx_0}}


\title{Modular Analysis}
\author{Joonhyup Lee}
\begin{document}
\maketitle
\section{Syntax and Semantics}
\subsection{Abstract Syntax}
\begin{figure}[htb]
  \centering
  \begin{tabular}{rrcll}
    Identifiers & $x$ & $\in$         & $\ExprVar$                                                      \\
    Expression  & $e$ & $\rightarrow$ & $x$ $\vbar$ $\lambda x.e$ $\vbar$ $e$ $e$ & $\lambda$-calculus  \\
                &     & $\vbar$       & $\link{e}{e}$                             & linked expression   \\
                &     & $\vbar$       & $\varepsilon$                             & empty module        \\
                &     & $\vbar$       & $x=e$ ; $e$                               & (recursive) binding
  \end{tabular}
  \caption{Abstract syntax of the language.}
  \label{fig:syntax}
\end{figure}
\subsection{Operational Semantics}
\begin{figure}[h!]
  \centering
  \begin{tabular}{rrcll}
    Environment     & $\ctx$ & $\in$         & $\Ctx$                                                                         \\
    Location        & $\ell$ & $\in$         & $\Loc$                                                                         \\
    de Bruijn Index & $n$    & $\in$         & $\mathbb{N}$                                                                   \\
    Value           & $v$    & $\in$         & $\Value \triangleq\Ctx+\ExprVar\times\Expr\times\Ctx$                          \\
    Weak Value      & $w$    & $\in$         & $\Walue\triangleq\Value+\underline\Value$                                      \\
    Environment     & $\ctx$ & $\rightarrow$ & $\bullet$                                             & empty stack            \\
                    &        & $\vbar$       & $(x,w)\cons \ctx$                                     & weak value binding     \\
                    &        & $\vbar$       & $(x,\ell)\cons\ctx$                                   & free location binding  \\
                    &        & $\vbar$       & $(x,n)\cons\ctx$                                      & bound location binding \\
    Value           & $v$    & $\rightarrow$ & $\ctx$                                                & exported environment   \\
                    &        & $\vbar$       & $\langle \lambda x.e, \ctx \rangle$                   & closure                \\
    Weak Value      & $w$    & $\rightarrow$ & $v$                                                   & value                  \\
                    &        & $\vbar$       & $\mu.v$                                               & recursive value
  \end{tabular}
  \caption{Definition of the semantic domains.}
  \label{fig:domain}
\end{figure}
\begin{figure}[h!]
  \begin{flushright}
    \fbox{$(e,\ctx)\Downarrow v$}
  \end{flushright}
  \centering
  \vspace{0pt} % -0.75em}
  \[
    \begin{InfRule}{Id}
      \hypo{v=\ctx(x)}
      \infer1{
        (x, \ctx)
        \Downarrow
        v
      }
    \end{InfRule}\:
    \begin{InfRule}{RecId}
      \hypo{\mu.v=\ctx(x)}
      \infer1{
        (x, \ctx)
        \Downarrow
        \openloc{v}{\mu.v}
      }
    \end{InfRule}\:
    \begin{InfRule}{Fn}
      \infer0{
        (\lambda x.e, \ctx)
        \Downarrow
        \langle\lambda x.e, \ctx\rangle
      }
    \end{InfRule}\:
    \begin{InfRule}{App}
      \hypo{
        (e_{1}, \ctx)
        \Downarrow
        \langle\lambda x.e, \ctx_1\rangle
      }
      \hypo{
        (e_{2}, \ctx)
        \Downarrow
        v_2
      }
      \hypo{
        (e, (x, v_2)\cons \ctx_1)
        \Downarrow
        v
      }
      \infer3{
        (e_{1}\:e_{2}, \ctx)
        \Downarrow
        v
      }
    \end{InfRule}
  \]

  \[
    \begin{InfRule}{Link}
      \hypo{
        (e_{1}, \ctx)
        \Downarrow
        \ctx_1
      }
      \hypo{
        (e_{2}, \ctx_1)
        \Downarrow
        v
      }
      \infer2{
        (\link{e_{1}}{e_{2}}, \ctx)
        \Downarrow
        v
      }
    \end{InfRule}\quad
    \begin{InfRule}{Empty}
      \infer0{
        (\varepsilon, \ctx)
        \Downarrow
        \bullet
      }
    \end{InfRule}\quad
    \begin{InfRule}{Bind}
      \hypo{
        \ell\not\in\FLoc(\ctx)
      }
      \hypo{
        (e_{1}, (x,\ell)\cons\ctx)
        \Downarrow
        v_1
      }
      \hypo{
        (e_{2}, (x, \mu.\closeloc{\ell}{v_1})\cons \ctx)
        \Downarrow
        \ctx_2
      }
      \infer3{
        (x=e_1; e_2, \ctx)
        \Downarrow
        (x,\mu.\closeloc{\ell}{v_1})\cons\ctx_2
      }
    \end{InfRule}
  \]
  \caption{The big-step operational semantics.}
  \label{fig:bigstep}
\end{figure}
We use the locally nameless representation, and enforce that all values be \emph{locally closed}.
As a consequence, the big-step operational semantics will be \emph{deterministic}, no matter what $\ell$ is chosen in the Bind rule.

\begin{figure}[h!]
  \centering
  \begin{tabular}{rclr}
                                                              &               &                                                                                   & \fbox{$e,\ctx,K\rightarrow e,\ctx,K$} \\
    $e_1\:e_2,\ctx,K$                                         & $\rightarrow$ & $e_1,\ctx,K\circ(\_\:(e_2,\ctx))$                                                                                         \\
    $\link{e_1}{e_2},\ctx,K$                                  & $\rightarrow$ & $e_1,\ctx,K\circ(\link{\_}{e_2})$                                                                                         \\
    $x=e_1;e_2,\ctx,K$                                        & $\rightarrow$ & $e_1,(x,\ell)\cons\ctx,K\circ(x=\ell;(e_2,\ctx))$                                 & $\ell\not\in\FLoc(\ctx)$              \\
    \\
                                                              &               &                                                                                   & \fbox{$v,K\rightarrow e,\ctx,K$}      \\
    $\langle\lambda x.e,\ctx_1\rangle,K\circ(\_\:(e_2,\ctx))$ & $\rightarrow$ & $e_2,\ctx,K\circ(\langle\lambda x.e,\ctx_1\rangle\:\_)$                                                                   \\
    $\ctx_1,K\circ(\link{\_}{e_2})$                           & $\rightarrow$ & $e_2,\ctx_1,K$                                                                                                            \\
    $v_1,K\circ(x=\ell;(e_2,\ctx))$                           & $\rightarrow$ & $e_2,(x,\mu.\closeloc{\ell}{v_1})\cons\ctx,K\circ(x=\mu.\closeloc{\ell}{v_1};\_)$                                         \\
    $v_2,K\circ(\langle\lambda x.e,\ctx_1\rangle\:\_)$        & $\rightarrow$ & $e,(x,v_2)\cons\ctx_1,K$                                                                                                  \\
    \\
                                                              &               &                                                                                   & \fbox{$v,K\rightarrow v,K$}           \\
    $\ctx_2,K\circ(x=w_1;\_)$                                 & $\rightarrow$ & $(x,w_1)\cons\ctx_2,K$                                                                                                    \\
    \\
                                                              &               &                                                                                   & \fbox{$e,\ctx,K\rightarrow v,K$}      \\
    $x,\ctx,K$                                                & $\rightarrow$ & $v,K$                                                                             & $v=\ctx(x)$                           \\
    $x,\ctx,K$                                                & $\rightarrow$ & $v^{\mu.v},K$                                                                     & $\mu.v=\ctx(x)$                       \\
    $\lambda x.e,\ctx,K$                                      & $\rightarrow$ & $\langle\lambda x.e,\ctx\rangle,K$                                                                                        \\
    $\varepsilon,\ctx,K$                                      & $\rightarrow$ & $\bullet,K$
  \end{tabular}
  \caption{The equivalent small-step operational semantics.}
  \label{fig:smallstep}
\end{figure}

\subsection{Adding Memory}

The first step towards abstraction is reformulating the semantics into a version with memory.

\begin{figure}[h!]
  \centering
  \begin{tabular}{rrcll}
    Environment & $\ctx$ & $\in$         & $\Ctx$                                                                       \\
    Location    & $\ell$ & $\in$         & $\Loc$                                                                       \\
    Memory      & $\mem$ & $\in$         & $\Mem\triangleq\fin{\Loc}{\Value}$                                           \\
    Value       & $v$    & $\in$         & $\Value \triangleq\Ctx+\ExprVar\times\Expr\times\Ctx$                        \\
    Environment & $\ctx$ & $\rightarrow$ & $\bullet$                                             & empty stack          \\
                &        & $\vbar$       & $(x,\ell)\cons\ctx$                                   & location binding     \\
    Value       & $v$    & $\rightarrow$ & $\ctx$                                                & exported environment \\
                &        & $\vbar$       & $\langle \lambda x.e, \ctx \rangle$                   & closure
  \end{tabular}
  \caption{Definition of the semantic domains with memory.}
  \label{fig:memdomain}
\end{figure}
\begin{figure}[h!]
  \centering
  \begin{tabular}{rclr}
                                                                   &               &                                                                 & \fbox{$e,\ctx,\mem,K\rightarrow e,\ctx,\mem,K$} \\
    $e_1\:e_2,\ctx,\mem,K$                                         & $\rightarrow$ & $e_1,\ctx,\mem,K\circ(\_\:(e_2,\ctx))$                                                                            \\
    $\link{e_1}{e_2},\ctx,\mem,K$                                  & $\rightarrow$ & $e_1,\ctx,\mem,K\circ(\link{\_}{e_2})$                                                                            \\
    $x=e_1;e_2,\ctx,\mem,K$                                        & $\rightarrow$ & $e_1,(x,\ell)\cons\ctx,\mem,K\circ(x=\ell;(e_2,\ctx))$          & $\ell\not\in\dom(m)\cup\FLoc(K)$                \\
    \\
                                                                   &               &                                                                 & \fbox{$v,\mem,K\rightarrow e,\ctx,\mem,K$}      \\
    $\langle\lambda x.e,\ctx_1\rangle,\mem,K\circ(\_\:(e_2,\ctx))$ & $\rightarrow$ & $e_2,\ctx,\mem,K\circ(\langle\lambda x.e,\ctx_1\rangle\:\_)$                                                      \\
    $\ctx_1,\mem,K\circ(\link{\_}{e_2})$                           & $\rightarrow$ & $e_2,\ctx_1,\mem,K$                                                                                               \\
    $v_1,\mem,K\circ(x=\ell;(e_2,\ctx))$                           & $\rightarrow$ & $e_2,(x,\ell)\cons\ctx,\mem[\ell\mapsto v_1],K\circ(x=\ell;\_)$                                                   \\
    $v_2,\mem,K\circ(\langle\lambda x.e,\ctx_1\rangle\:\_)$        & $\rightarrow$ & $e,(x,\ell)\cons\ctx_1,\mem[\ell\mapsto v_2],K$                 & $\ell\not\in\dom(m)\cup\FLoc(K)$                \\
    \\
                                                                   &               &                                                                 & \fbox{$v,\mem,K\rightarrow v,\mem,K$}           \\
    $\ctx_2,\mem,K\circ(x=\ell;\_)$                                & $\rightarrow$ & $(x,\ell)\cons\ctx_2,\mem,K$                                                                                      \\
    \\
                                                                   &               &                                                                 & \fbox{$e,\ctx,\mem,K\rightarrow v,\mem,K$}      \\
    $x,\ctx,\mem,K$                                                & $\rightarrow$ & $v,\mem,K$                                                      & $\ell=\ctx(x),v=\mem(\ell)$                     \\
    $\lambda x.e,\ctx,\mem,K$                                      & $\rightarrow$ & $\langle\lambda x.e,\ctx\rangle,\mem,K$                                                                           \\
    $\varepsilon,\ctx,\mem,K$                                      & $\rightarrow$ & $\bullet,\mem,K$
  \end{tabular}
  \caption{The small-step operational semantics with memory.}
  \label{fig:memsmallstep}
\end{figure}
\subsection{Reconciling the Two Semantics}
We need to prove that the two semantics simulate each other.
Thus, we need to define a notion of equivalence between the two semantic domains.
\begin{figure}[h!]
  \centering
  \begin{flushright}
    \fbox{$w\sim_f v,\mem$}
  \end{flushright}
  \[
    \begin{InfRule}{Eq-Nil}
      \infer0{\bullet\sim_f\bullet}
    \end{InfRule}\:
    \begin{InfRule}{Eq-ConsFree}
      \hypo{\ell\not\in\dom(f)}
      \hypo{\ell\not\in\dom(\mem)}
      \hypo{\ctx\sim_f\ctx'}
      \infer3{(x,\ell)\cons\ctx\sim_f(x,\ell)\cons\ctx'}
    \end{InfRule}\:
    \begin{InfRule}{Eq-ConsBound}
      \hypo{f(\ell)=\ell'}
      \hypo{\ell'\in\dom(\mem)}
      \hypo{\ctx\sim_f\ctx'}
      \infer3{(x,\ell)\cons\ctx\sim_f(x,\ell')\cons\ctx'}
    \end{InfRule}
  \]

  \[
    \begin{InfRule}{Eq-ConsWVal}
      \hypo{\mem(\ell')=v'}
      \hypo{w\sim_f v'}
      \hypo{\ctx\sim_f\ctx'}
      \infer3{(x,w)\cons\ctx\sim_f(x,\ell')\cons\ctx'}
    \end{InfRule}\:
    \begin{InfRule}{Eq-Clos}
      \hypo{\ctx\sim_f\ctx'}
      \infer1{\langle\lambda x.e,\ctx\rangle\sim_f\langle\lambda x.e,\ctx'\rangle}
    \end{InfRule}\:
    \begin{InfRule}{Eq-Rec}
      \hypo{L\text{ finite}}
      \hypo{\mem(\ell')=v'}
      \hypo{\forall\ell\not\in L,\:\openloc{v}{\ell}\sim_{f[\ell\mapsto\ell']}v'}
      \infer3{\mu.v\sim_f v'}
    \end{InfRule}
  \]
  \caption{The equivalence relation between weak values in the original semantics and values in the semantics with memory.
    $f\in\fin{\Loc}{\Loc}$ tells what the free locations in $w$ that were \emph{opened} should be mapped to in memory.}
  \label{fig:equivrel}
\end{figure}

Some useful lemmas.

\begin{lem}[Free locations not in $f$ are free in memory]
  \[w\sim_f v',\mem\Rightarrow m|_{\FLoc(w)-\dom(f)}=\bot\]
\end{lem}

\begin{lem}[Equivalence is preserved by extension of memory]
  \[w\sim_f v',\mem\text{ and }m\sqsubseteq m'\text{ and }m'|_{\FLoc(w)-\dom(f)}=\bot\Rightarrow w\sim_f v',m\]
\end{lem}

\begin{lem}[Equivalence only cares about $f$ on free locations]
  \[w\sim_f v',\mem\text{ and }f|_{\FLoc(w)}=f|_{\FLoc(w)}\Rightarrow w\sim_{f'}v',m\]
\end{lem}

\begin{lem}[Extending equivalence on free locations]
  \[w\sim_f v',\mem\text{ and }\ell\not\in\dom(f)\text{ and }\ell\not\in\dom(\mem)\Rightarrow\forall u',w\sim_{f[\ell\mapsto\ell]}v',m[\ell\mapsto u']\]
\end{lem}

\begin{lem}[Substitution of values]
  \[w\sim_f v',\mem\text{ and }f(\ell)=\ell'\text{ and }\mem(\ell')=u'\text{ and }u\sim_{f-\ell}u',\mem\Rightarrow w[u/\ell]\sim_{f-\ell}v',\mem\]
\end{lem}

\begin{lem}[Substitution of locations]
  \[w\sim_f v',\mem\text{ and }\ell\in\dom(f)\text{ and }\nu\not\in\FLoc(w)\Rightarrow w[\nu/\ell]\sim_{f\circ(\nu\leftrightarrow\ell)}v',\mem\]
\end{lem}

It turns out, we only need to reason about what free locations are in the continuation of the semantics of Figure \ref{fig:memsmallstep}.
Thus, we can formulate a big-step semantics and reason about the equivalence between the big-step semantics directly.

\begin{figure}[h!]
  \begin{flushright}
    \fbox{$(e,\ctx,\mem,L)\Downarrow (v,\mem,L)$}
  \end{flushright}
  \centering
  \vspace{0pt} % -0.75em}
  \[
    \begin{InfRule}{Id}
      \hypo{\ell=\ctx(x)}
      \hypo{v=\mem(\ell)}
      \infer2{
        (x,\ctx,\mem,L)
        \Downarrow
        (v,\mem,L)
      }
    \end{InfRule}\:
    \begin{InfRule}{Fn}
      \infer0{
        (\lambda x.e, \ctx,\mem,L)
        \Downarrow
        (\langle\lambda x.e, \ctx\rangle,\mem,L)
      }
    \end{InfRule}
  \]

  \[
    \begin{InfRule}{App}
      \hypo{
        (e_{1}, \ctx,\mem,L)
        \Downarrow
        (\langle\lambda x.e, \ctx_1\rangle,\mem_1,L_1)
      }
      \hypo{
        (e_{2}, \ctx,\mem_1,L_1)
        \Downarrow
        (v_2,\mem_2,L_2)
      }
      \hypo{
        \ell\not\in\dom(\mem_2)\cup L_2
      }
      \hypo{
        (e, (x, \ell)\cons \ctx_1,\mem_2[\ell\mapsto v_2],L_2)
        \Downarrow
        (v,\mem',L')
      }
      \infer{1,3}{
        (e_{1}\:e_{2}, \ctx,\mem,L)
        \Downarrow
        (v,\mem',L')
      }
    \end{InfRule}
  \]

  \[
    \begin{InfRule}{Link}
      \hypo{
        (e_{1}, \ctx,\mem,L)
        \Downarrow
        (\ctx_1,\mem_1,L_1)
      }
      \hypo{
        (e_{2}, \ctx_1,\mem_1,L_1)
        \Downarrow
        (v,\mem',L')
      }
      \infer2{
        (\link{e_{1}}{e_{2}}, \ctx)
        \Downarrow
        (v,\mem',L')
      }
    \end{InfRule}\quad
    \begin{InfRule}{Empty}
      \infer0{
        (\varepsilon, \ctx,\mem,L)
        \Downarrow
        (\bullet,\mem,L)
      }
    \end{InfRule}\quad
  \]

  \[
    \begin{InfRule}{Bind}
      \hypo{
        \ell\not\in\dom(\mem)\cup L
      }
      \hypo{
        (e_{1}, (x,\ell)\cons\ctx,\mem,L\cup\{\ell\})
        \Downarrow
        (v_1,\mem_1,L_1)
      }
      \hypo{
        (e_{2}, (x, \ell)\cons \ctx,\mem_1[\ell\mapsto v_1],L_1-\{\ell\})
        \Downarrow
        (\ctx_2,\mem',L')
      }
      \infer3{
        (x=e_1; e_2, \ctx,\mem,L)
        \Downarrow
        ((x,\ell)\cons\ctx_2,\mem',L')
      }
    \end{InfRule}
  \]
  \caption{The big-step operational semantics with memory.}
  \label{fig:membigstep}
\end{figure}

Then we can prove the following.

\begin{definition}[Equivalence of configurations]
  \[\ctx\sim\ctx',\mem,L\triangleq\ctx\sim_\bot\ctx',\mem\text{ and }\FLoc(\ctx)\subseteq L\]
\end{definition}

\begin{thm}[Equivalence of semantics]
  \begin{align*}
    \ctx\sim\ctx',\mem,L\text{ and }(e,\ctx)\semarrow v                    & \Rightarrow\exists v',\mem',L':v\sim v',\mem',L'\text{ and }(e,\ctx',\mem,L)\semarrow(v',\mem',L') \\
    \ctx\sim\ctx',\mem,L\text{ and }(e,\ctx',\mem,L)\semarrow(v',\mem',L') & \Rightarrow\exists v:v\sim v',\mem',L'\text{ and }(e,\ctx)\semarrow v
  \end{align*}
\end{thm}

\section{Generating and Resolving Events}

\subsection{Without Memory}
Now we formulate the semantics for generating events.
The point of this semantics is that we are expressing a function that interacts with a stream of answers to external events as a data structure.

\begin{figure}[h!]
  \centering
  \begin{tabular}{rrcll}
    Environment & $\ctx$ & $\rightarrow$ & $\cdots$                            \\
                &        & $\vbar$       & $[E]$         & answer to an event  \\
    Value       & $v$    & $\rightarrow$ & $\cdots$                            \\
                &        & $\vbar$       & $E$           & answer to an event  \\
    Event       & $E$    & $\rightarrow$ & $\InitE$      & initial environment \\
                &        & $\vbar$       & $\ReadE(E,x)$ & read event          \\
                &        & $\vbar$       & $\CallE(E,v)$ & call event
  \end{tabular}
  \caption{Definition of the semantic domains with events. All other semantic domains are equal to Figure \ref{fig:domain}.}
  \label{fig:eventdomain}
\end{figure}

We redefine how to read weak values given an environment.
\begin{align*}
  \bullet(x)             & \triangleq\bot    &            &  & ((x,w)\cons\ctx)(x) & \triangleq w                                                                   \\
  ((x,\ell)\cons\ctx)(x) & \triangleq\bot    &            &  & ((x,n)\cons\ctx)(x) & \triangleq\bot        & (\text{impossible when }\ctx\text{ is locally closed}) \\
  ((x',\_)\cons\ctx)(x)  & \triangleq\ctx(x) & (x\neq x') &  & [E](x)              & \triangleq\ReadE(E,x)
\end{align*}

Then we need to add only one rule to the semantics in Figure \ref{fig:bigstep} for the semantics to incorporate events.
\[
  \begin{InfRule}{AppEvent}
    \hypo{
      (e_{1}, \ctx)
      \Downarrow
      E
    }
    \hypo{
      (e_{2}, \ctx)
      \Downarrow
      v
    }
    \infer2{
      (e_{1}\:e_{2}, \ctx)
      \Downarrow
      \CallE(E,v)
    }
  \end{InfRule}
\]

Now we need to formulate the \emph{event resolution} rules.
The event resolution rule $w\resolve W$, given an answer $\ctx_0$ to the $\InitE$ event, resolves all events within $w$ to obtain $W$.
Note that $\FLoc(\ctx_0)\cap\FLoc(w)=\varnothing$ is required for the resolution to work.

For the event resolution to be sound, we expect the following theorem to hold.

\begin{thm}[Soundness of resolution]
  \[\ctx\resolve\Sigma\text{ and }(e,\ctx)\semarrow v\text{ and }(e,\Sigma)\semarrow V\Rightarrow v\resolve V\]
\end{thm}

Now we present the definition for resolution.

\begin{figure}[h!]
  \begin{flushright}
    \fbox{$w\downarrow W$}
  \end{flushright}
  \centering
  \vspace{0pt} % -0.75em}
  \[
    \begin{InfRule}{R-Init}
      \infer0{
        \InitE
        \downarrow
        \ctx_0
      }
    \end{InfRule}\:
    \begin{InfRule}{R-Read}
      \hypo{E\downarrow\Sigma}
      \hypo{V=\Sigma(x)}
      \infer2{
        \ReadE(E,x)
        \downarrow
        V
      }
    \end{InfRule}\:
    \begin{InfRule}{R-ReadRec}
      \hypo{E\downarrow\Sigma}
      \hypo{\mu.V=\Sigma(x)}
      \infer2{
        \ReadE(E,x)
        \downarrow
        \openloc{V}{\mu.V}
      }
    \end{InfRule}\:
    \begin{InfRule}{R-CallV}
      \hypo{E\downarrow\langle\lambda x.e,\Sigma\rangle}
      \hypo{v\downarrow V}
      \hypo{(e,(x,V)\cons\Sigma)\semarrow V'}
      \infer3{
        \CallE(E,v)
        \downarrow
        V'
      }
    \end{InfRule}
  \]

  \[
    \begin{InfRule}{R-CallE}
      \hypo{E\downarrow E'}
      \hypo{v\downarrow V}
      \infer2{
        \CallE(E,v)
        \downarrow
        \CallE(E',V)
      }
    \end{InfRule}\:
    \begin{InfRule}{R-Nil}
      \infer0{\bullet\downarrow\bullet}
    \end{InfRule}\:
    \begin{InfRule}{R-ConsLoc}
      \hypo{\ctx\downarrow\Sigma}
      \infer1{
        (x,\ell)\cons\ctx
        \downarrow
        (x,\ell)\cons\Sigma
      }
    \end{InfRule}\:
    \begin{InfRule}{R-ConsWVal}
      \hypo{w\downarrow W}
      \hypo{\ctx\downarrow\Sigma}
      \infer2{
        (x,w)\cons\ctx
        \downarrow
        (x,W)\cons\Sigma
      }
    \end{InfRule}
  \]

  \[
    \begin{InfRule}{R-HoleE}
      \hypo{E\downarrow E'}
      \infer1{
        [E]
        \downarrow
        [E']
      }
    \end{InfRule}\:
    \begin{InfRule}{R-HoleEnv}
      \hypo{E\downarrow \Sigma}
      \infer1{
        [E]
        \downarrow
        \Sigma
      }
    \end{InfRule}\:
    \begin{InfRule}{R-Clos}
      \hypo{\ctx\downarrow\Sigma}
      \infer1{
        \langle\lambda x.e,\ctx\rangle
        \downarrow
        \langle\lambda x.e,\Sigma\rangle
      }
    \end{InfRule}\:
    \begin{InfRule}{R-Rec}
      \hypo{\ell\not\in\FLoc(v)\cup\FLoc(\ctx_0)}
      \hypo{\openloc{v}{\ell}\downarrow V}
      \infer2{
        \mu.v
        \downarrow
        \mu.\closeloc{\ell}{V}
      }
    \end{InfRule}
  \]
  \caption{The resolution relation. The subscript $\ctx_0$ in $\resolve$ is omitted for brevity.}
  \label{fig:resolution}
\end{figure}

Note that substitution of free locations that happens when unfolding a recursive value may generate multiple copies of the same event.
In our simple module language, resolution is not impacted by neither the order nor the redundancy of resolution.
However, in the future, to support stateful events which depend on both the order and redundancy, we must assign \emph{unique numbers} to the events.
Furthermore, for abstraction of the data structure, it is much easier if everything is threaded through the memory through memory locations.
This motivates us to:
\begin{enumerate}
  \item Lift $\downarrow$ in the case for memory in a way that is \emph{compatible} with the equivalence relation $\sim$.
  \item Formulate a semantics where events are assigned a unique number, and their dependencies are tracked by a \emph{memory for events}.
\end{enumerate}

We expect by doing so, we may obtain a sound approximation for the semantics by simply abstracting the \emph{evaluation relation} and the \emph{resolution relation}.

\subsection{With Memory}

\subsection{With Event Memory}

\end{document}
%%% Local Variables: 
%%% coding: utf-8
%%% mode: latex
%%% TeX-engine: xetex
%%% End:
