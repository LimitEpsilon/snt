% Gemini theme
% https://github.com/anishathalye/gemini

\documentclass[final]{beamer}

% ====================
% Packages
% ====================

\usepackage[a4paper]{geometry}
\geometry{
  top=20mm,
  bottom=20mm,
  left=20mm,
  right=20mm
}
\usepackage[nodisplayskipstretch]{setspace}
\linespread{1.25}
\usepackage{graphicx}
\usepackage{xcolor}
\usepackage{kotex}
\usepackage{csquotes}
\usepackage{tabularray}
\usepackage{relsize}
\usepackage{titlesec}

%%% Math settings
\usepackage{amssymb,amsmath,amsthm,mathtools}
\usepackage[math-style=TeX,bold-style=TeX]{unicode-math}
\theoremstyle{definition}
\newtheorem{definition}{Definition}[section]
\newtheorem{example}{Example}[section]
\newtheorem{lemma}{Lemma}[section]
\newtheorem{theorem}{Theorem}[section]
\newtheorem{corollary}{Corollary}[section]
\newtheorem{claim}{Claim}[section]

%%% Font settings
\setmainfont{Libertinus Serif}
\setsansfont{Libertinus Sans}[Scale=MatchUppercase]

\usepackage{galois}
\usepackage{ebproofx}
\ebproofxset{center=false}

\newcommand*{\defeq}{\triangleq}


% ====================
% Lengths
% ====================

% If you have N columns, choose \sepwidth and \colwidth such that
% (N+1)*\sepwidth + N*\colwidth = \paperwidth
\newlength{\sepwidth}
\newlength{\colwidth}
\setlength{\sepwidth}{0.025\paperwidth}
\setlength{\colwidth}{0.4625\paperwidth}

\newcommand{\separatorcolumn}{\begin{column}{\sepwidth}\end{column}}

% ====================
% Title
% ====================

\title{프로그램 따로분석의 이론적 기틀: 재귀적 모듈 지원하기}

\author{이준협 \and 이광근}

\institute[shortinst]{서울대학교 프로그래밍 연구실 (ROPAS)}

% ====================
% Footer (optional)
% ====================

%\footercontent{
%  \hfill
%  2023년 SIGPL 여름학교 포스터 발표
%  \hfill
%}
% (can be left out to remove footer)

% ====================
% Logo (optional)
% ====================

% use this to include logos on the left and/or right side of the header:
\logoright{\includegraphics[height=5cm]{ropas-symbol.png}}
\logoleft{\includegraphics[height=5cm]{snu-symbol.png}}

% ====================
% Body
% ====================

\begin{document}

\begin{frame}[t]
  \begin{columns}[t]
    \separatorcolumn

    \begin{column}{\colwidth}
      \begin{block}{문제 소개}
        \heading{무엇을 하고 싶은가?}
        \begin{itemize}
          \item 프로그램 외부의 값을 몰라도 분석하고 싶다.\\
                \textbf{ex. }외부 라이브러리 함수를 부르는 경우
          \item 이름이 ``복잡한 방식''으로 알려질 때에도 분석하고 싶다.\\
                \textbf{ex. }First-class modules, Functors, Recursive modules
        \end{itemize}
        \heading{어떻게 할 것인가?}
        \begin{itemize}
          \item ``Algebraic Effects'': 환경을 받는 \textbf{함수}를 \textbf{자료구조}로 나타냄.
          \item 환경과의 상호작용을 요약하고, 나중에 환경이 들어오면 풀자!
        \end{itemize}
      \end{block}

      \begin{block}{모듈이 있는 언어의 정의}
        \heading{겉모습 (Untyped $\lambda$+Modules)}
        \begin{figure}[h!]
          \large
          \begin{tabular}{rrcll}
            Identifiers & $x$ & $\in$         & $\ExprVar$                                                      \\
            Expression  & $e$ & $\rightarrow$ & $x$ $\vbar$ $\lambda x.e$ $\vbar$ $e$ $e$ & $\lambda$-calculus  \\
                        &     & $\vbar$       & $\link{e}{e}$                             & linked expression   \\
                        &     & $\vbar$       & $\varepsilon$                             & empty module        \\
                        &     & $\vbar$       & $x=e$ ; $e$                               & (recursive) binding
          \end{tabular}
        \end{figure}
        \heading{속내용}
        \begin{figure}[h!]
          \centering
          \large
          \begin{tabular}{rrcll}
            Environment & $\ctx$ & $\in$         & $\Ctx$                                                                        \\
            Location    & $\ell$ & $\in$         & $\Loc$                                                                        \\
            Value       & $v$    & $\in$         & $\Value \triangleq\Ctx+\ExprVar\times\Expr\times\Ctx$                         \\
            Weak Value  & $w$    & $\in$         & $\Walue\triangleq\Value+\Loc\times\Value$                                     \\
            Environment & $\ctx$ & $\rightarrow$ & $\bullet$                                             & empty stack           \\
                        &        & $\vbar$       & $(x,w)\cons \ctx$                                     & weak value binding    \\
                        &        & $\vbar$       & $(x,\ell)\cons\ctx$                                   & free location binding \\
            Value       & $v$    & $\rightarrow$ & $\ctx$                                                & exported environment  \\
                        &        & $\vbar$       & $\langle \lambda x.e, \ctx \rangle$                   & closure               \\
            Weak Value  & $w$    & $\rightarrow$ & $v$                                                   & value                 \\
                        &        & $\vbar$       & $\mu\ell.v$                                           & recursive value
          \end{tabular}
        \end{figure}
        \heading{실행의미}
        \begin{figure}[h!]
          \centering\large
          \begin{flushright}
            \fbox{$(e,\ctx)\Downarrow v$}
          \end{flushright}
          \vspace{0pt} % -0.75em}
          \[
            \begin{InfRule}{Id}
              \hypo{\ctx(x)=v}
              \infer1{
                (x, \ctx)
                \Downarrow
                v
              }
            \end{InfRule}\quad
            \fbox{
              \begin{InfRule}{RecId}
                \hypo{\ctx(x)=\mu\ell.v}
                \infer1{
                  (x, \ctx)
                  \Downarrow
                  v[\mu\ell.v/\ell]
                }
              \end{InfRule}}\quad
            \begin{InfRule}{Fn}
              \infer0{
                (\lambda x.e, \ctx)
                \Downarrow
                \langle\lambda x.e, \ctx\rangle
              }
            \end{InfRule}
          \]

          \[
            \begin{InfRule}{App}
              \hypo{
                (e_{1}, \ctx)
                \Downarrow
                \langle\lambda x.e, \ctx_1\rangle
              }
              \hypo{
                (e_{2}, \ctx)
                \Downarrow
                v_2
              }
              \hypo{
                (e, (x, v_2)\cons \ctx_1)
                \Downarrow
                v
              }
              \infer3{
                (e_{1}\:e_{2}, \ctx)
                \Downarrow
                v
              }
            \end{InfRule}
          \]

          \[
            \fbox{
              \begin{InfRule}{Link}
                \hypo{
                  (e_{1}, \ctx)
                  \Downarrow
                  \ctx_1
                }
                \hypo{
                  (e_{2}, \ctx_1)
                  \Downarrow
                  v
                }
                \infer2{
                  (\link{e_{1}}{e_{2}}, \ctx)
                  \Downarrow
                  v
                }
              \end{InfRule}}\quad
            \begin{InfRule}{Empty}
              \infer0{
                (\varepsilon, \ctx)
                \Downarrow
                \bullet
              }
            \end{InfRule}
          \]

          \[
            \fbox{
              \begin{InfRule}{RecBind}
                \hypo{
                  \ell\not\in\FLoc(\ctx)
                }
                \hypo{
                  (e_{1}, (x,\ell)\cons\ctx)
                  \Downarrow
                  v_1
                }
                \hypo{
                  (e_{2}, (x, \mu\ell.v_1)\cons \ctx)
                  \Downarrow
                  \ctx_2
                }
                \infer3{
                  (x=e_1; e_2, \ctx)
                  \Downarrow
                  (x,\mu\ell.v_1)\cons\ctx_2
                }
              \end{InfRule}}
          \]
        \end{figure}
      \end{block}

    \end{column}

    \separatorcolumn

    \begin{column}{\colwidth}
      \begin{block}{상호작용 기록하고 나중에 풀기}
        \heading{상호작용 기록}
        \begin{figure}[h!]
          \centering
          \large
          \begin{tabular}{rrcll}
            Environment & $\ctx$ & $\rightarrow$ & $\cdots$                            \\
                        &        & $\vbar$       & $[E]$         & answer to an event  \\
            Value       & $v$    & $\rightarrow$ & $\cdots$                            \\
                        &        & $\vbar$       & $E$           & answer to an event  \\
            Event       & $E$    & $\rightarrow$ & $\InitE$      & initial environment \\
                        &        & $\vbar$       & $\ReadE(E,x)$ & read event          \\
                        &        & $\vbar$       & $\CallE(E,v)$ & call event
          \end{tabular}

          \[
            [E](x)\triangleq\ReadE(E,x)\qquad
            \begin{InfRule}{AppEvent}
              \hypo{
                (e_{1}, \ctx)
                \Downarrow
                E
              }
              \hypo{
                (e_{2}, \ctx)
                \Downarrow
                v
              }
              \infer2{
                (e_{1}\:e_{2}, \ctx)
                \Downarrow
                \CallE(E,v)
              }
            \end{InfRule}
          \]
        \end{figure}

        \heading{상호작용 풀기 (일부 규칙들)}
        \begin{figure}[h!]
          \centering
          \large
          \begin{flushright}
            \fbox{$w\resolve W$}
          \end{flushright}
          \vspace{0pt} % -0.75em}
          \[
            \begin{InfRule}{R-Init}
              \infer0{
                \InitE
                \downarrow
                \ctx_0
              }
            \end{InfRule}\quad
            \begin{InfRule}{R-Read}
              \hypo{E\downarrow\Sigma}
              \hypo{\Sigma(x)=V}
              \infer2{
                \ReadE(E,x)
                \downarrow
                V
              }
            \end{InfRule}\quad
            \begin{InfRule}{R-ReadRec}
              \hypo{E\downarrow\Sigma}
              \hypo{\Sigma(x)=\mu\ell.V}
              \infer2{
                \ReadE(E,x)
                \downarrow
                V[\mu\ell.V/\ell]
              }
            \end{InfRule}
          \]

          \[
            \begin{InfRule}{R-CallV}
              \hypo{E\downarrow\langle\lambda x.e,\Sigma\rangle}
              \hypo{v\downarrow V}
              \hypo{(e,(x,V)\cons\Sigma)\semarrow V'}
              \infer3{
                \CallE(E,v)
                \downarrow
                V'
              }
            \end{InfRule}\quad
            \begin{InfRule}{R-CallE}
              \hypo{E\downarrow E'}
              \hypo{v\downarrow V}
              \infer2{
                \CallE(E,v)
                \downarrow
                \CallE(E',V)
              }
            \end{InfRule}
          \]
        \end{figure}

        \heading{Conjecture (상호작용 풀기는 안전하다)}
        \[\ctx\resolve\Sigma\text{ and }(e,\ctx)\semarrow v\text{ and }(e,\Sigma)\semarrow V\Rightarrow v\resolve V\]

        \heading{예시}
        {\ttfamily
          \begin{tabular}{rl}
            (\textcolor{dkviolet}{Top}=          & (\textcolor{dkgreen}{Tree}=(\textcolor{blue}{max}=$\lambda$x.f (Top$\synlink$Forest$\synlink$max) x; \textcolor{blue}{$\varepsilon$}); \\
                                                 & \textcolor{dkgreen}{Forest}=(\textcolor{red}{max}=$\lambda$x.(Top$\synlink$Tree$\synlink$max) x; \textcolor{red}{$\varepsilon$});      \\
                                                 & \textcolor{dkgreen}{$\varepsilon$});                                                                                                   \\
            \textcolor{dkviolet}{$\varepsilon$}) & $\synlink$((\textcolor{dkviolet}{Top}$\synlink$\textcolor{dkgreen}{Tree}$\synlink$\textcolor{blue}{max}) id)
          \end{tabular}
        }

        $\semarrow\CallE(\CallE(\ReadE(\InitE,\text{``f''}),\langle\lambda\texttt{x.Top}\synlink\texttt{Tree}\synlink\texttt{max x},\dots\rangle),\texttt{id})$
      \end{block}
      \begin{block}{요약하기}
        \heading{메모리의 등장}
        \begin{itemize}
          \item 메모리가 있으면 요약이 편해진다.
          \item $\Ctx\triangleq\fin{\ExprVar}{\Loc}$, $\Mem\triangleq\fin{\Loc}{\Value}$이면, $\Loc$만 요약.
          \item 그렇다면 실행의미가 동일하다는 것부터 보여야 한다.
        \end{itemize}
        \begin{align*}
          \ctx\sim\ctx_0,\mem_0\land(e,\ctx)\semarrow v                    & \Rightarrow\exists v_1,\mem_1\sim v:(e,\ctx_0,\mem_0)\semarrow(v_1,\mem_1) \\
          \ctx\sim\ctx_0,\mem_0\land(e,\ctx_0,\mem_0)\semarrow(v_1,\mem_1) & \Rightarrow\exists v\sim v_1,\mem_1:(e,\ctx)\semarrow v
        \end{align*}
        \heading{$\semarrow$와 $\resolve$의 요약 (진행 중)}
        \begin{itemize}
          \item 첫번째: $\downarrow$를 메모리가 있을 때 ``안전''하게 정의하기
                \[\text{``안전'':}w\sim v,\mem\land w\downarrow W\Rightarrow\exists V,M:W\sim V,M\land v,\mem\downarrow V,M\]
          \item 두번째: 메모리의 요약에 대해 ``안전''하게 $\downarrow$, $\Downarrow$를 요약
        \end{itemize}
      \end{block}
    \end{column}

    \separatorcolumn
  \end{columns}
\end{frame}

\end{document}
