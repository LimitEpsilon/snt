%! TEX program = xelatex
\documentclass{beamer}
\usepackage[a4paper]{geometry}
\geometry{
  top=20mm,
  bottom=20mm,
  left=20mm,
  right=20mm
}
\usepackage[nodisplayskipstretch]{setspace}
\linespread{1.25}
\usepackage{graphicx}
\usepackage{xcolor}
\usepackage{kotex}
\usepackage{csquotes}
\usepackage{tabularray}
\usepackage{relsize}
\usepackage{titlesec}

%%% Math settings
\usepackage{amssymb,amsmath,amsthm,mathtools}
\usepackage[math-style=TeX,bold-style=TeX]{unicode-math}
\theoremstyle{definition}
\newtheorem{definition}{Definition}[section]
\newtheorem{example}{Example}[section]
\newtheorem{lemma}{Lemma}[section]
\newtheorem{theorem}{Theorem}[section]
\newtheorem{corollary}{Corollary}[section]
\newtheorem{claim}{Claim}[section]

%%% Font settings
\setmainfont{Libertinus Serif}
\setsansfont{Libertinus Sans}[Scale=MatchUppercase]

\usepackage{galois}
\usepackage{ebproofx}
\ebproofxset{center=false}

\newcommand*{\defeq}{\triangleq}


\title{프로그램 따로분석의 이론적 기틀}
%\subtitle{}
\author{이준협}
\date{2024년 4월 19일}
\institute{ROPAS S\&T}

\titlegraphic{%
  \begin{tikzpicture}[overlay,remember picture]
    \node at (current page.145) [xshift=3em, yshift=-1.3em] {
      \includegraphics[width=3em]{snu-symbol.png}
    };
    \node at (current page.35) [xshift=-3em, yshift=-1.3em] {
      \includegraphics[width=2.5em]{ropas-symbol.png}
    };
  \end{tikzpicture}%
}

\begin{document}
\maketitle
\begin{frame}[c,fragile]
	\frametitle{따로분석?}
	\begin{itemize}
		\item 따로분석이란,
		\item 모든 것이 알려지기 전에,
		\item 프로그램의 의미를 포섭하는 분석이다.
	\end{itemize}
\end{frame}
\begin{frame}[c,fragile]
	\frametitle{왜 필요한가?}
	\begin{tabular}{c}
		\begin{lstlisting}[language=Simple]
int *x;
void f(void) {
  if (*x == *x) g(); /* call to unknown function */
  *x = 42; /* what happens here? */
}
  \end{lstlisting}
	\end{tabular}
	\begin{itemize}
		\item 코드 전체가 주어지지 않는 일이 빈번하다.\\
		      \textbf{ex. }외부 라이브러리 함수를 부르는 경우
		\item \textbf{미리} 최대한 분석하고 싶다.
	\end{itemize}
\end{frame}
\begin{frame}[c,fragile]
	\frametitle{미리보기}
	\begin{center}
		$\sembracket{e}(\Sigma_0\semlink\Sigma)\subseteq\Sigma_0\semlink\sembracket{e}\Sigma$
	\end{center}
	\begin{itemize}
		\item $\Sigma_0:\text{내가 모르던 외부 환경}$
		\item $\sembracket{e}\Sigma:\text{몰라도 최선을 다해 실행한 결과}$
		\item $\semlink:\text{합치기(linking) 연산}$
	\end{itemize}
	\[\text{따로분석}=\text{안전한 }\Abs{\sembracket{\cdot}}+\text{안전한 }\Abs\semlink\]
	\[\text{분석의 안전성}:\sembracket{e}(\Sigma_0\semlink\Sigma)\subseteq\gamma(\overbracket{\Abs\ctx_0\Abs\semlink\underbracket{\Abs{\sembracket{e}}\Abs\ctx}_{\text{따로분석}}}^{\text{나중에 합치기}})\]
\end{frame}
\begin{frame}[c,fragile]
	\frametitle{언어 소개}
	{
		\small
		\begin{tabular}{rrcll}
			Identifiers & $x$ & $\in$         & $\ExprVar$                                                      \\
			Expression  & $e$ & $\rightarrow$ & $x$ $\vbar$ $\lambda x.e$ $\vbar$ $e$ $e$ & $\lambda$-calculus  \\
			            &     & $\vbar$       & $e\synlink e$                             & linked expression   \\
			            &     & $\vbar$       & $\varepsilon$                             & empty module        \\
			            &     & $\vbar$       & $x\:\texttt{=}\:e$ \texttt{;} $e$         & (recursive) binding
		\end{tabular}
	}
\end{frame}
\begin{frame}[c,fragile]
	\frametitle{의미공간 소개}
	{
		\small
		\begin{tabular}{rrcll}
			Environment & $\ctx$ & $\in$         & $\Ctx$                                                                        \\
			Location    & $\ell$ & $\in$         & $\Loc$                                                                        \\
			Value       & $v$    & $\in$         & $\Value \triangleq\Ctx+\ExprVar\times\Expr\times\Ctx$                         \\
			Weak Value  & $w$    & $\in$         & $\Walue\triangleq\Value+\underline\Value$                                     \\
			Environment & $\ctx$ & $\rightarrow$ & $\bullet$                                             & empty stack           \\
			            &        & $\vbar$       & $(x,w)\cons \ctx$                                     & weak value binding    \\
			            &        & $\vbar$       & $(x,\ell)\cons\ctx$                                   & free location binding \\
			Value       & $v$    & $\rightarrow$ & $\ctx$                                                & exported environment  \\
			            &        & $\vbar$       & $\langle \lambda x.e, \ctx \rangle$                   & closure               \\
			Weak Value  & $w$    & $\rightarrow$ & $v$                                                   & value                 \\
			            &        & $\vbar$       & $\mu\ell.v$                                           & recursive value
		\end{tabular}
	}
\end{frame}
\begin{frame}[c,fragile]
	\frametitle{의미공간 확장}
	{
		\small
		\begin{tabular}{rrcll}
			Event       & $\event$ & $\rightarrow$ & $\InitE$           & initial environment \\
			            &          & $\vbar$       & $\ReadE(\event,x)$ & read event          \\
			            &          & $\vbar$       & $\CallE(\event,v)$ & call event          \\
			Environment & $\ctx$   & $\rightarrow$ & $\cdots$                                 \\
			            &          & $\vbar$       & $[\event]$         & answer to an event  \\
			Value       & $v$      & $\rightarrow$ & $\cdots$                                 \\
			            &          & $\vbar$       & $\event$           & answer to an event
		\end{tabular}
	}
\end{frame}
\begin{frame}[c,fragile]
	\frametitle{실행의미 소개}
	{
		\footnotesize
		\begin{flushright}
			\fbox{$\evjudg{\ctx}{e}{v}$}
		\end{flushright}
		\centering
		\[
			\begin{InfRule}{Id}
				\hypo{\ctx(x)=v}
				\infer1{\evjudg{\ctx}{x}{v}}
			\end{InfRule}\:
			\begin{InfRule}{RecId}
				\hypo{\ctx(x)=\mu\ell.v}
				\infer1{\evjudg{\ctx}{x}{v[\mu\ell.v/\ell]}}
			\end{InfRule}\:
			\begin{InfRule}{Fn}
				\infer0{\evjudg{\ctx}{\lambda x.e}{\langle\lambda x.e, \ctx\rangle}}
			\end{InfRule}\:
			\begin{InfRule}{App}
				\hypo{\evjudg{\ctx}{e_1}{\langle\lambda x.e, \ctx_1\rangle}}
				\hypo{\evjudg{\ctx}{e_2}{v_2}}
				\hypo{\evjudg{(x, v_2)\cons \ctx_1}{e}{v}}
				\infer{2,1}{\evjudg{\ctx}{e_1\:e_2}{v}}
			\end{InfRule}
		\]

		\[
			\begin{InfRule}{Link}
				\hypo{\evjudg{\ctx}{e_1}{\ctx_1}}
				\hypo{\evjudg{\ctx_1}{e_2}{v}}
				\infer2{\evjudg{\ctx}{{e_1}\synlink{e_2}}{v}}
			\end{InfRule}\:
			\begin{InfRule}{Empty}
				\infer0{\evjudg{\ctx}{\varepsilon}{\bullet}}
			\end{InfRule}\:
			\begin{InfRule}{Bind}
				\hypo{\ell\not\in\FLoc(\ctx)}
				\hypo{\evjudg{(x,\ell)\cons\ctx}{e_1}{v_1}}
				\hypo{\evjudg{(x, \mu\ell.v_1)\cons \ctx}{e_1}{\ctx_2}}
				\infer{2,1}{\evjudg{\ctx}{x\:\texttt{=}\:e_1; e_2}{(x,\mu\ell.v_1)\cons\ctx_2}}
			\end{InfRule}
		\]
	}
\end{frame}
\begin{frame}
	\frametitle{실행의미 확장}
	{\footnotesize
		\[
			\begin{InfRule}{LinkEvent}
				\hypo{\evjudg{\ctx}{e_1}{\event}}
				\hypo{\evjudg{[\event]}{e_2}{v}}
				\infer2{\evjudg{\ctx}{e_1\synlink e_2}{v}}
			\end{InfRule}\:
			\begin{InfRule}{AppEvent}
				\hypo{\evjudg{\ctx}{e_1}{\event}}
				\hypo{\evjudg{\ctx}{e_2}{v}}
				\infer2{\evjudg{\ctx}{e_1\:e_2}{\CallE(\event,v)}}
			\end{InfRule}
		\]

		\[
			\begin{InfRule}{BindEvent}
				\hypo{\ell\not\in\FLoc(\ctx)}
				\hypo{\evjudg{(x,\ell)\cons\ctx}{e_1}{v_1}}
				\hypo{\evjudg{(x, \mu\ell.v_1)\cons \ctx}{e_1}{\event_2}}
				\infer{2,1}{\evjudg{\ctx}{x\:\texttt{=}\:e_1; e_2}{(x,\mu\ell.v_1)\cons[\event_2]}}
			\end{InfRule}
		\]
	}
\end{frame}
\begin{frame}
	\frametitle{합치기}
	\footnotesize
	\begin{align*}
		\fbox{$\ctx_0\semlink\cdot\in\Event\rightarrow\pset{\Value}$}                                                                                                                        \\
		\ctx_0\semlink\InitE\triangleq           & \: \{\ctx_0\}                                                                                                                             \\
		\ctx_0\semlink\ReadE(\event,x)\triangleq & \: \{v_+|\ctx_+\in\ctx_0\semlink\event\land\ctx_+(x)=v_+\}                                                                                \\
		\cup                                     & \: \{v_+[\mu\ell.v_+/\ell]|\ctx_+\in\ctx_0\semlink E\land\ctx_+(x)=\mu\ell.v_+\}                                                          \\
		\ctx_0\semlink\CallE(\event,v)\triangleq & \: \{v_+'|\langle\lambda x.e,\ctx_+\rangle\in\ctx_0\semlink E\land v_+\in\ctx_0\semlink v\land(x,v_+)\cons\ctx_+\vdash e\Downarrow v_+'\} \\
		\cup                                     & \: \{\CallE(\event_+,v_+)|\event_+\in\ctx_0\semlink E\land v_+\in\ctx_0\semlink v\}                                                       \\
		\fbox{$\ctx_0\semlink\cdot\in\Ctx\rightarrow\pset{\Ctx}$}                                                                                                                            \\
		\cdots                                                                                                                                                                               \\
		\fbox{$\ctx_0\semlink\cdot\in\Value\rightarrow\pset{\Value}$}                                                                                                                        \\
		\cdots                                                                                                                                                                               \\
		\fbox{$\ctx_0\semlink\cdot\in\Walue\rightarrow\pset{\Walue}$}                                                                                                                        \\
		\ctx_0\semlink\mu\ell.v\triangleq        & \: \{\mu\ell'.v_+|\ell'\not\in\FLoc(v)\cup\FLoc(\ctx_0)\land v_+\in\ctx_0\semlink v[\ell'/\ell]\}
	\end{align*}
\end{frame}
\begin{frame}
	\frametitle{Advance}
	{\small
		\[\eval(e,\ctx)\triangleq\{v|\ctx\vdash e\Downarrow v\}\quad\eval(e,\Sigma)\triangleq\bigcup_{\ctx\in\Sigma}\eval(e,\ctx)\quad\Sigma_0\semlink W\triangleq\bigcup_{\substack{\ctx_0\in\Sigma_0\\w\in W}}(\ctx_0\semlink w)\]
	}
	\begin{thm}[Advance]
		\[\eval(e,\Sigma_0\semlink\Sigma)\subseteq\Sigma_0\semlink\eval(e,\Sigma)\]
	\end{thm}
	\begin{itemize}
		\item 증명은 $\evjudg{\ctx}{e}{v}$에 대한 귀납법으로
		\item Coq으로 엄검증 완료! $\checkmark$
	\end{itemize}
\end{frame}
\begin{frame}[c,fragile]
	\frametitle{따로분석}
	\begin{enumerate}
		\item 증가하는 $\gamma\in\pset{\Walue}\rightarrow\Abs\Walue$
		\item 안전한 $\Abs\eval$: $\Sigma_0\subseteq\gamma(\Abs\ctx_0)\Rightarrow\eval(e,\Sigma_0)\subseteq\gamma(\Abs\eval(e,\Abs\ctx_0))$
		\item 안전한 $\Abs\semlink$: $\Sigma_0\subseteq\gamma(\Abs\ctx_0)\text{ 이고 }W\subseteq\gamma(\Abs{w})\Rightarrow\Sigma_0\semlink W\subseteq\gamma(\Abs\ctx_0\Abs\semlink\Abs{w})$
	\end{enumerate}
	\[\Sigma_0\subseteq\gamma(\Abs\ctx_0)\text{ 이고 }\Sigma\subseteq\gamma(\Abs\ctx)\Rightarrow\eval(e,\Sigma_0\semlink\Sigma)\subseteq\gamma(\Abs\ctx_0\Abs\semlink\Abs\eval(e,\Abs\ctx))\]
\end{frame}
\begin{frame}[c,fragile]
	\frametitle{특히}
	\[\Sigma_0\subseteq\gamma(\Abs\ctx_0)\]
	이고
	\[[\InitE]\in\gamma(\Abs\InitE)\]
	이면
	\[\eval(e_1\synlink e_2,\Sigma_0)\subseteq\gamma(\underbracket{\Abs\eval(e_1,\Abs\ctx_0)}_{\text{따로}}\Abs\semlink\underbracket{\Abs\eval(e_2,\Abs\InitE)}_{\text{따로}})\]
\end{frame}
\begin{frame}[c,fragile]
	\frametitle{예시: CFA}
	\small
	\begin{tabular}{rrcl}
		Program point        & $p$                            & $\in$         & $\mathbb{P}$                                                                    \\
		Abstract event       & $\Abs{\event}$                 & $\in$         & $\Abs{\Event}$                                                                  \\
		Abstract environment & $\Abs{\ctx}$                   & $\in$         & $\Abs{\Ctx}\triangleq(\fin{\ExprVar}{\pset{\mathbb{P}}})\times2^{\Abs{\Event}}$ \\
		Abstract closure     & $\langle\lambda x.p,p'\rangle$ & $\in$         & $\Abs{\text{Clos}}\triangleq\ExprVar\times\mathbb{P}\times\mathbb{P}$           \\
		Abstract value       & $\Abs{v}$                      & $\in$         & $\Abs{\Value}\triangleq\Abs{\Ctx}\times\pset{\Abs{\text{Clos}}}$                \\
		Abstract semantics   & $\Abs\ptree$                   & $\in$         & $\Abs{\mathbb{T}}\triangleq\mathbb{P}\rightarrow\Abs{\Ctx}\times\Abs{\Value}$   \\
		Abstract event       & $\Abs{\event}$                 & $\rightarrow$ & $\Abs{\InitE}$ | $\Abs{\ReadE}(p,x)$ | $\Abs{\CallE}(p,p)$
	\end{tabular}
\end{frame}
\begin{frame}[c,fragile]
	\frametitle{직관}
	\[\Abs\ptree=p\mapsto(\Abs\ctx,\Abs{v})\]
	모든 $p$마다, 입력 $\Abs\ctx$와 출력 $\Abs{v}$가 달려있음.
	\[\Abs{v}=(\underbracket{(x\mapsto\{p\},\overbracket{\{\Abs\InitE,\Abs\ReadE(p,x),\Abs\CallE(p_1,p_2)\}}^{\Abs\event\text{ 부분}})}_{\Abs\ctx\text{ 부분}},\underbracket{\{\langle\lambda x.p,p'\rangle\}}_{\Abs{\text{Clos}}\text{ 부분}})\]
	$x\mapsto\{p\}$: $\ctx(x)$가 $\llabel{p}{\ell}$이거나 $p$의 출력.

	$\Abs\ReadE(p,x)$: $p$에 들어오는 $[E]$에서 $x$를 읽음.

	$\Abs\CallE(p_1,p_2)$: $p_1$에서 나온 $E$를 $p_2$에서 나온 $v$에 적용.

	$\langle\lambda x.p,p'\rangle$: $p'$에 들어온 $\ctx$에서 실행되는 함수.
\end{frame}
\begin{frame}[c,fragile]
	\frametitle{$\gamma$ 정의하기}
	\scriptsize
	\begin{flushright}
		\fbox{$\ctx\preceq(\Abs\ctx,\Abs\ptree)$}
	\end{flushright}
	\[
		\begin{InfRule}{Conc-Nil}
			\infer0{\bullet\preceq\Abs\ctx}
		\end{InfRule}\:
		\begin{InfRule}{Conc-ENil}
			\hypo{\event\preceq(\Abs\ctx,\varnothing)}
			\infer1{[\event]\preceq\Abs\ctx}
		\end{InfRule}\:
		\begin{InfRule}{Conc-ConsLoc}
			\hypo{p\in\Abs\ctx.1(x)}
			\hypo{\ctx\preceq\Abs\ctx}
			\infer2{(x,\llabel{p}{\ell})\cons\ctx\preceq\Abs\ctx}
		\end{InfRule}\:
		\begin{InfRule}{Conc-ConsWVal}
			\hypo{p\in\Abs\ctx.1(x)}
			\hypo{w\preceq\Abs\ptree(p).2}
			\hypo{\ctx\preceq\Abs\ctx}
			\infer3{(x,w)\cons\ctx\preceq\Abs\ctx}
		\end{InfRule}
	\]
	\begin{flushright}
		\fbox{$w\preceq(\Abs{v},\Abs\ptree)$}
	\end{flushright}
	\[
		\begin{InfRule}{Conc-Clos}
			\hypo{\langle\lambda x.p,p'\rangle\in\Abs{v}.2}
			\hypo{\ctx\preceq\Abs\ptree(p').1}
			\infer2{\langle\lambda x.p,\ctx\rangle\preceq\Abs{v}}
		\end{InfRule}\:
		\begin{InfRule}{Conc-Rec}
			\hypo{v\preceq\Abs\ptree(p).2}
			\hypo{v\preceq\Abs{v}}
			\infer2{\mu\llabel{p}{\ell}.v\preceq\Abs{v}}
		\end{InfRule}\:
		\begin{InfRule}{Conc-Init}
			\hypo{\Abs\InitE\in\Abs{v}.1.2}
			\infer1{\InitE\preceq\Abs{v}}
		\end{InfRule}
	\]

	\[
		\begin{InfRule}{Conc-Read}
			\hypo{\Abs\ReadE(p,x)\in\Abs{v}.1.2}
			\hypo{[\event]\preceq\Abs\ptree(p).1}
			\infer2{\ReadE(\event,x)\preceq\Abs{v}}
		\end{InfRule}\:
		\begin{InfRule}{Conc-Call}
			\hypo{\Abs\CallE(p_1,p_2)\in\Abs{v}.1.2}
			\hypo{\event\preceq\Abs\ptree(p_1).2}
			\hypo{v\preceq\Abs\ptree(p_2).2}
			\infer3{\CallE(\event,v)\preceq\Abs{v}}
		\end{InfRule}
	\]
\end{frame}
\begin{frame}
	\frametitle{$\gamma$ 정의하기}
	\[\gamma(\Abs{v},\Abs\ptree)\triangleq\{v|v\preceq(\Abs{v},\Abs\ptree)\}\]
\end{frame}
\begin{frame}[c,fragile]
	\frametitle{안전한 $\Abs\Step$}
	\begin{itemize}
		\item $\Abs\ptree\sqsupseteq\Abs\Step(\Abs\ptree)$이면
		      \[\evjudg{\ctx}{p}{v}\Rightarrow\ctx\in\gamma(\Abs\ptree(p).1,\Abs\ptree)\Rightarrow v\in\gamma(\Abs\ptree(p).2,\Abs\ptree)\]
		      가 성립하는
	\end{itemize}
\end{frame}
\begin{frame}[c,fragile]
	\frametitle{안전한 $\Abs\Link$}
	\begin{itemize}
		\item $\ctx_0\in\gamma(\Abs\ctx_0,\Abs\ptree_0)$인 $\Abs\ctx_0,\Abs\ptree_0$와
		\item $\Abs\ptree$가 주어졌을 때,
    \item $\Abs\ptree_+\sqsupseteq\Abs\ptree_0\sqcup\Abs\ptree$이고 $\Abs\ptree_+\sqsupseteq\Abs\Link(\Abs\ctx_0,\Abs\ptree_+)$이면
	\end{itemize}
	\[w_+\in\ctx_0\semlink w\Rightarrow w\in\gamma(\Abs\ptree(p).1,\Abs\ptree)\Rightarrow w_+\in\gamma(\Abs\ptree_+(p).1,\Abs\ptree_+)\]
	와
	\[w_+\in\ctx_0\semlink w\Rightarrow w\in\gamma(\Abs\ptree(p).2,\Abs\ptree)\Rightarrow w_+\in\gamma(\Abs\ptree_+(p).2,\Abs\ptree_+)\]
	가 성립하는
\end{frame}
\begin{frame}
  \frametitle{안전한 $\Abs{\sembracket{p_0}}$와 $\Abs\semlink$}
  \[\Abs{\sembracket{p_0}}(\Abs\ctx_0,\Abs\ptree_0)\triangleq\lfp(\lambda\Abs\ptree.\Abs\Step(\Abs\ptree)\sqcup[p_0\mapsto(\Abs\ctx_0,\bot)]\sqcup\Abs\ptree_0)\]
  \[(\Abs\ctx_0,\Abs\ptree_0)\Abs\semlink\Abs\ptree\triangleq\lfp(\lambda\Abs\ptree_+.\Abs\Link(\Abs\ctx_0,\Abs\ptree_+)\sqcup\Abs\ptree_0\sqcup\Abs\ptree)\]
\end{frame}
\begin{frame}
	\begin{center}\LARGE
		감사합니다
	\end{center}
\end{frame}
\end{document}
