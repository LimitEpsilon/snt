%! TEX program = xelatex
\documentclass{article}
\usepackage{geometry}
\geometry{
  a4paper,
  total={170mm,257mm},
  left=20mm,
  top=20mm,
}
\usepackage{setspace}
\usepackage{graphicx}
\usepackage{xcolor}
\usepackage{kotex}
\usepackage{csquotes}
\usepackage{tabularray}
\usepackage{relsize}

%%% Typesetting for listings
\usepackage{listings}

\definecolor{dkgreen}{rgb}{0,0.6,0}
\definecolor{ltblue}{rgb}{0,0.4,0.4}
\definecolor{dkviolet}{rgb}{0.3,0,0.5}

%%% lstlisting coq style (inspired from a file of Assia Mahboubi)
\lstdefinelanguage{Coq}{ 
    % Anything betweeen $ becomes LaTeX math mode
    mathescape=true,
    % Comments may or not include Latex commands
    texcl=false, 
    % Vernacular commands
    morekeywords=[1]{Section, Module, End, Require, Import, Export,
        Variable, Variables, Parameter, Parameters, Axiom, Hypothesis,
        Hypotheses, Notation, Local, Tactic, Reserved, Scope, Open, Close,
        Bind, Delimit, Definition, Let, Ltac, Fixpoint, CoFixpoint, Add,
        Morphism, Relation, Implicit, Arguments, Unset, Contextual,
        Strict, Prenex, Implicits, Inductive, CoInductive, Record,
        Structure, Canonical, Coercion, Context, Class, Global, Instance,
        Program, Infix, Theorem, Lemma, Corollary, Proposition, Fact,
        Remark, Example, Proof, Goal, Save, Qed, Defined, Hint, Resolve,
        Rewrite, View, Search, Show, Print, Printing, All, Eval, Check,
        Projections, inside, outside, Def},
    % Gallina
    morekeywords=[2]{forall, exists, exists2, fun, fix, cofix, struct,
        match, with, end, as, in, return, let, if, is, then, else, for, of,
        nosimpl, when},
    % Sorts
    morekeywords=[3]{Type, Prop, Set, true, false, option},
    % Various tactics, some are std Coq subsumed by ssr, for the manual purpose
    morekeywords=[4]{pose, set, move, case, elim, apply, clear, hnf,
        intro, intros, generalize, rename, pattern, after, destruct,
        induction, using, refine, inversion, injection, rewrite, congr,
        unlock, compute, ring, field, fourier, replace, fold, unfold,
        change, cutrewrite, simpl, have, suff, wlog, suffices, without,
        loss, nat_norm, assert, cut, trivial, revert, bool_congr, nat_congr,
        symmetry, transitivity, auto, split, left, right, autorewrite},
    % Terminators
    morekeywords=[5]{by, done, exact, reflexivity, tauto, romega, omega,
        assumption, solve, contradiction, discriminate},
    % Control
    morekeywords=[6]{do, last, first, try, idtac, repeat},
    % Comments delimiters, we do turn this off for the manual
    morecomment=[s]{(*}{*)},
    % Spaces are not displayed as a special character
    showstringspaces=false,
    % String delimiters
    morestring=[b]",
    morestring=[d],
    % Size of tabulations
    tabsize=3,
    % Enables ASCII chars 128 to 255
    extendedchars=false,
    % Case sensitivity
    sensitive=true,
    % Automatic breaking of long lines
    breaklines=false,
    % Default style fors listings
    basicstyle=\small,
    % Position of captions is bottom
    captionpos=b,
    % flexible columns
    columns=[l]flexible,
    % Style for (listings') identifiers
    identifierstyle={\ttfamily\color{black}},
    % Style for declaration keywords
    keywordstyle=[1]{\ttfamily\color{dkviolet}},
    % Style for gallina keywords
    keywordstyle=[2]{\ttfamily\color{dkgreen}},
    % Style for sorts keywords
    keywordstyle=[3]{\ttfamily\color{ltblue}},
    % Style for tactics keywords
    keywordstyle=[4]{\ttfamily\color{dkblue}},
    % Style for terminators keywords
    keywordstyle=[5]{\ttfamily\color{dkred}},
    %Style for iterators
    %keywordstyle=[6]{\ttfamily\color{dkpink}},
    % Style for strings
    stringstyle=\ttfamily,
    % Style for comments
    commentstyle={\ttfamily\color{dkgreen}},
    %moredelim=**[is][\ttfamily\color{red}]{/&}{&/},
    literate=
    {\\forall}{{\color{dkgreen}{$\forall\;$}}}1
    {\\exists}{{$\exists\;$}}1
    {\ge -}{{$\geftarrow\;$}}1
    {=>}{{$\Rightarrow\;$}}1
    {==}{{\code{==}\;}}1
    {==>}{{\code{==>}\;}}1
    %    {:>}{{\code{:>}\;}}1
    {->}{{$\rightarrow\;$}}1
    {\ge ->}{{$\geftrightarrow\;$}}1
    {\ge ==}{{$\geq\;$}}1
    {\#}{{$^\star$}}1 
    {\\o}{{$\circ\;$}}1 
    {\@}{{$\cdot$}}1 
    {\/\\}{{$\text{ and }\;$}}1
    {\\\/}{{$\vee\;$}}1
    {++}{{\code{++}}}1
    {~}{{\ }}1
    {\@\@}{{$@$}}1
    {\\mapsto}{{$\mapsto\;$}}1
    {\\hline}{{\rule{\linewidth}{0.5pt}}}1
    %
}[keywords,comments,strings]

%%% Math settings
\usepackage{amssymb,amsmath,amsthm,mathtools}
\usepackage[math-style=TeX,bold-style=TeX]{unicode-math}
\theoremstyle{definition}
\newtheorem{definition}{Definition}[section]
\newtheorem{example}{Example}[section]
\newtheorem{lem}{Lemma}[section]
\newtheorem{thm}{Theorem}[section]
\newtheorem{cor}{Corollary}[section]
\newtheorem{clm}{Claim}[section]

%%% Font settings
%\setmainfont{Libertinus Serif}
%\setsansfont{Libertinus Sans}[Scale=MatchUppercase]
%\setmonofont{JuliaMono}[Scale=MatchLowercase]
%\setmainhangulfont{Noto Serif CJK KR}
%\setmonohangulfont{D2Coding}

%%% PL constructs
\usepackage{galois}
\usepackage{ebproof}
%% InfRule command
\ExplSyntaxOn
\NewTblrEnviron{@ruleenv}
\SetTblrInner[@ruleenv]{belowsep=0pt,stretch=0}
\SetTblrOuter[@ruleenv]{baseline=b}
\NewDocumentEnvironment { InfRule } { m +b }
  {
    \begin{@ruleenv}{l}
      \smaller{\textsc{#1}} \\
      \begin{prooftree} #2 \end{prooftree}
    \end{@ruleenv}
  }
  {}
\ExplSyntaxOff

%% Modification for \infer
\ExplSyntaxOn
\int_new:N \g__ebproof_sublevel_int
\box_new:N \g__ebproof_substack_box
\seq_new:N \g__ebproof_substack_seq

\cs_new:Nn \__ebproof_clear_substack:
  {
    \int_gset:Nn \g__ebproof_sublevel_int { 0 }
    \hbox_gset:Nn \g__ebproof_substack_box { }
    \seq_gclear:N \g__ebproof_substack_seq
  }

\cs_new:Nn \__ebproof_subpush:N
  {
    \int_gincr:N \g__ebproof_sublevel_int
    \hbox_gset:Nn \g__ebproof_substack_box
      { \hbox_unpack:N \g__ebproof_substack_box \box_use:c { \__ebproof_box:N #1 } }
    \seq_gput_left:Nv \g__ebproof_substack_seq
      { \__ebproof_marks:N #1 }
  }

\cs_new:Nn \__ebproof_subpop:N
  {
    \int_compare:nNnTF { \g__ebproof_sublevel_int } > { 0 }
      {
        \int_gdecr:N \g__ebproof_sublevel_int
        \hbox_gset:Nn \g__ebproof_substack_box {
          \hbox_unpack:N \g__ebproof_substack_box
          \box_gset_to_last:N \g_tmpa_box
        }
        \box_set_eq_drop:cN { \__ebproof_box:N #1 } \g_tmpa_box
        \seq_gpop_left:NN \g__ebproof_substack_seq \l_tmpa_tl
        \tl_set_eq:cN { \__ebproof_marks:N #1 } \l_tmpa_tl
      }
      { \PackageError{ebproof}{Missing~premiss~in~a~proof~tree}{} \__ebproof_clear:N #1 }
  }

\cs_new:Nn \__ebproof_append_subvertical:NN
  {
    \bool_if:NTF \l__ebproof_updown_bool
      { \__ebproof_append_above:NN #1 #2 }
      { \__ebproof_append_below:NN #1 #2 }
  }

\cs_new:Nn \__ebproof_join_subvertical:n
  {
    \group_begin:
    \__ebproof_subpop:N \l__ebproof_a_box
    \prg_replicate:nn { #1 - 1 }
      {
        \__ebproof_subpop:N \l__ebproof_b_box
        \__ebproof_enlarge_conclusion:NN \l__ebproof_b_box \l__ebproof_a_box

        \__ebproof_make_vertical:Nnnn \l__ebproof_c_box
          { \__ebproof_mark:Nn \l__ebproof_b_box {axis} - \__ebproof_mark:Nn \l__ebproof_b_box {left} }
          { \__ebproof_mark:Nn \l__ebproof_b_box {right} - \__ebproof_mark:Nn \l__ebproof_b_box {left} }
          { \skip_vertical:N \l__ebproof_rule_margin_dim }
        \__ebproof_vcenter:N \l__ebproof_b_box
        \__ebproof_append_subvertical:NN \l__ebproof_a_box \l__ebproof_c_box

        \__ebproof_append_subvertical:NN \l__ebproof_a_box \l__ebproof_b_box
      }
    \__ebproof_push:N \l__ebproof_a_box
    \group_end:
  }

\cs_new:Nn \__ebproof_renew_statement:nnn
  {
    \exp_args:Nc \RenewDocumentCommand { ebproof#1 }{ #2 } { #3 }
    \seq_gput_right:Nn \g__ebproof_statements_seq { #1 }
  }
\__ebproof_renew_statement:nnn { infer } { O{} m O{} m }
  {
    \group_begin:
    \__ebproof_restore_statements:
    \keys_set_known:nnN { ebproof / rule~style } { #1 } \l_tmpa_tl
    \keys_set:nV { ebproof } \l_tmpa_tl
    \tl_set:Nn \l__ebproof_right_label_tl { #3 }

    \__ebproof_clear_substack:
    \clist_map_inline:nn { #2 }
      {
        \__ebproof_join_horizontal:n { ##1 }
        \__ebproof_pop:N \l__ebproof_a_box
        \__ebproof_subpush:N \l__ebproof_a_box
      }
    \__ebproof_join_subvertical:n { \clist_count:n { #2 } }

    \__ebproof_push_statement:n { #4 }
    \__ebproof_join_vertical:
    \group_end:
  }
\ExplSyntaxOff
\ebproofset{left label template=\textsc{[\inserttext]}}
\ebproofset{center=false}

%%% Custom commands
\newcommand*{\vbar}{|}
\newcommand*{\finto}{\xrightarrow{\text{\textrm{fin}}}}
\newcommand*{\istype}{\mathrel{⩴}}
\newcommand*{\ortype}{\mathrel{|}}
\newcommand*{\cons}{::}
\newcommand*{\pset}{\mathscr{P}}
\newcommand*{\closeloc}[2]{\prescript{\text{\textbackslash}{#1}}{}{#2}}
\newcommand*{\openloc}[2]{{#1}^{#2}}

\def\ovbarw{1.2mu}
\def\ovbarh{1}
\newcommand*{\ovbar}[1]{\mkern \ovbarw\overline{\mkern-\ovbarw{\smash{#1}\scalebox{1}[\ovbarh]{\vphantom{i}}}\mkern-\ovbarw}\mkern \ovbarw}
\newcommand*{\A}[1]{\overset{\,_{\mbox{\Large .}}}{#1}}
\newcommand*{\Abs}[1]{{#1}^{\#}}
\newcommand*{\Expr}{\text{Expr}}
\newcommand*{\ExprVar}{\text{Var}}
\newcommand*{\Module}{\text{Module}}
\newcommand*{\ModVar}{\text{ModVar}}
\newcommand*{\modid}{d}
\newcommand*{\Time}{\mathbb{T}}
\newcommand*{\ATime}{\A{\Time}}
\newcommand*{\Ctx}{\text{Env}}
\newcommand*{\ctx}{\sigma}
\newcommand*{\Value}{\text{Val}}
\newcommand*{\Walue}{\text{WVal}}
\newcommand*{\Mem}{\text{Mem}}
\newcommand*{\Loc}{\text{Loc}}
\newcommand*{\FLoc}{\text{FLoc}}
\newcommand*{\Left}{\text{Left}}
\newcommand*{\Right}{\text{Right}}
\newcommand*{\Sig}{\text{Sig}}
\newcommand*{\mem}{m}
\newcommand*{\dom}{\text{dom}}
\newcommand*{\AMem}{\A{\text{Mem}}}
\newcommand*{\State}{\text{State}}
\newcommand*{\AState}{\A{\text{State}}}
\newcommand*{\Result}{\text{Result}}
\newcommand*{\AResult}{\A{\text{Result}}}
\newcommand*{\Tick}{\text{Tick}}
\newcommand*{\lfp}{\mathsf{lfp}}
\newcommand*{\Step}{\mathsf{Step}}
\newcommand*{\semarrow}{\Downarrow}
\newcommand*{\asemarrow}{\A{\rightsquigarrow}}
\newcommand*{\synlink}{\rtimes}
\newcommand*{\semlink}{\mathbin{\rotatebox[origin=c]{180}{$\propto$}}}
\newcommand*{\link}[2]{{#1}\rtimes{#2}}
\newcommand*{\mt}{\mathsf{empty}}
\newcommand*{\valid}{\mathsf{valid}}
\newcommand*{\Path}{\text{Path}}
\newcommand*{\equivalent}{\sim}

\newcommand*{\doubleplus}{\ensuremath{\mathbin{+\mkern-3mu+}}}
\newcommand*{\project}{\text{\texttt{:>} }}
\newcommand*{\Exp}{\mathsf{Exp}}
\newcommand*{\Imp}{\mathsf{Imp}}
\newcommand*{\Fin}{\mathsf{Fin}}
\newcommand*{\Link}{\mathsf{Link}}
\newcommand*{\sembracket}[1]{\lBrack{#1}\rBrack}
\newcommand*{\fin}[2]{{#1}\xrightarrow{\text{fin}}{#2}}
\newcommand*{\addr}{\mathsf{addr}}
\newcommand*{\tick}{\mathsf{tick}}
\newcommand*{\modctx}{\mathsf{ctx}}
\newcommand*{\mapinject}[2]{{#2}[{#1}]}
\newcommand*{\inject}[2]{{#2}\langle{#1}\rangle}
\newcommand*{\deletepre}[2]{{#2}\overline{\doubleplus}{#1}}
\newcommand*{\deletemap}[2]{{#1}\overline{[{#2}]}}
\newcommand*{\delete}[2]{{#2}{\langle{#1}\rangle}^{-1}}
\newcommand*{\filter}{\mathsf{filter}}
\newcommand*{\Lete}{\mathtt{val}}
\newcommand*{\Letm}{\mathtt{mod}}

\newcommand*{\ValRel}[1]{\mathcal{V}\sembracket{#1}}
\newcommand*{\ExprRel}[1]{\mathcal{E}\sembracket{#1}}
\newcommand*{\CtxRel}[1]{\mathcal{C}\sembracket{#1}}
\newcommand*{\ModRel}[1]{\mathcal{M}\sembracket{#1}}
\newcommand*{\TyEnv}{\text{TyEnv}}
\newcommand*{\TyVar}{\text{TyVar}}
\newcommand*{\Type}{\text{Type}}
\newcommand*{\Subst}{\text{Subst}}
\newcommand*{\external}{\Gamma_{\text{ext}}}

\newcommand*{\InitE}{\textsf{Init}}
\newcommand*{\ReadE}{\textsf{Read}}
\newcommand*{\CallE}{\textsf{Call}}
\newcommand*{\resolve}{\downarrow_{\ctx_0}}


\title{Modular Analysis}
\author{Joonhyup Lee}
\begin{document}
\maketitle
\section{Syntax and Semantics}
\subsection{Abstract Syntax}
\begin{figure}[htb]
  \centering
  \begin{tabular}{rrcll}
    Identifiers & $x$ & $\in$         & $\ExprVar$                                                     \\
    Expression  & $e$ & $\rightarrow$ & $x$ $\vbar$ $\lambda x.e$ $\vbar$ $e$ $e$ & $\lambda$-calculus \\
                &     & $\vbar$       & $\link{e}{e}$                             & linked expression  \\
                &     & $\vbar$       & $\varepsilon$                             & empty module       \\
                &     & $\vbar$       & $x=e$ ; $e$                               & binding
  \end{tabular}
  \caption{Abstract syntax of the language.}
  \label{fig:syntax}
\end{figure}
\subsection{Operational Semantics}
\begin{figure}[h!]
  \centering
  \begin{tabular}{rrcll}
    Environment     & $\ctx$ & $\in$         & $\Ctx$                                                                         \\
    Location        & $\ell$ & $\in$         & $\Loc$                                                                         \\
    de Bruijn Index & $n$    & $\in$         & $\mathbb{N}$                                                                   \\
    Value           & $v$    & $\in$         & $\Value \triangleq\Ctx+\ExprVar\times\Expr\times\Ctx$                          \\
    Weak Value      & $w$    & $\in$         & $\Walue\triangleq\Value+\underline\Value$                                      \\
    Environment     & $\ctx$ & $\rightarrow$ & $\bullet$                                             & empty stack            \\
                    &        & $\vbar$       & $(x,w)\cons \ctx$                                     & weak value binding     \\
                    &        & $\vbar$       & $(x,\ell)\cons\ctx$                                   & free location binding  \\
                    &        & $\vbar$       & $(x,n)\cons\ctx$                                      & bound location binding \\
    Value           & $v$    & $\rightarrow$ & $\ctx$                                                & exported environment   \\
                    &        & $\vbar$       & $\langle \lambda x.e, \ctx \rangle$                   & closure                \\
    Weak Value      & $w$    & $\rightarrow$ & $v$                                                   & value                  \\
                    &        & $\vbar$       & $\mu.v$                                               & recursive value
  \end{tabular}
  \caption{Definition of the semantic domains.}
  \label{fig:domain}
\end{figure}
\begin{figure}[h!]
  \begin{flushright}
    \fbox{$(e,\ctx)\Downarrow v$}
  \end{flushright}
  \centering
  \vspace{0pt} % -0.75em}
  \[
    \begin{InfRule}{Id}
      \hypo{v=\ctx(x)}
      \infer1{
        (x, \ctx)
        \Downarrow
        v
      }
    \end{InfRule}\:
    \begin{InfRule}{RecId}
      \hypo{\mu.v=\ctx(x)}
      \infer1{
        (x, \ctx)
        \Downarrow
        \openloc{v}{\mu.v}
      }
    \end{InfRule}\:
    \begin{InfRule}{Fn}
      \infer0{
        (\lambda x.e, \ctx)
        \Downarrow
        \langle\lambda x.e, \ctx\rangle
      }
    \end{InfRule}\:
    \begin{InfRule}{App}
      \hypo{
        (e_{1}, \ctx)
        \Downarrow
        \langle\lambda x.e, \ctx_1\rangle
      }
      \hypo{
        (e_{2}, \ctx)
        \Downarrow
        v_2
      }
      \hypo{
        (e, (x, v_2)\cons \ctx_1)
        \Downarrow
        v
      }
      \infer3{
        (e_{1}\:e_{2}, \ctx)
        \Downarrow
        v
      }
    \end{InfRule}
  \]

  \[
    \begin{InfRule}{Link}
      \hypo{
        (e_{1}, \ctx)
        \Downarrow
        \ctx_1
      }
      \hypo{
        (e_{2}, \ctx_1)
        \Downarrow
        v
      }
      \infer2{
        (\link{e_{1}}{e_{2}}, \ctx)
        \Downarrow
        v
      }
    \end{InfRule}\quad
    \begin{InfRule}{Empty}
      \infer0{
        (\varepsilon, \ctx)
        \Downarrow
        \bullet
      }
    \end{InfRule}\quad
    \begin{InfRule}{Bind}
      \hypo{
        \ell\not\in\FLoc(\ctx)
      }
      \hypo{
        (e_{1}, (x,\ell)\cons\ctx)
        \Downarrow
        v_1
      }
      \hypo{
        (e_{2}, (x, \mu.\closeloc{\ell}{v_1})\cons \ctx)
        \Downarrow
        \ctx_2
      }
      \infer3{
        (x=e_1; e_2, \ctx)
        \Downarrow
        (x,\mu.\closeloc{\ell}{v_1})\cons\ctx_2
      }
    \end{InfRule}
  \]
  \caption{The big-step operational semantics.}
  \label{fig:bigstep}
\end{figure}
We use the locally nameless representation, and enforce that all values be \emph{locally closed}.
As a consequence, the big-step operational semantics will be \emph{deterministic}, no matter what $\ell$ is chosen in the Bind rule.

\begin{figure}[h!]
  \centering
  \begin{tabular}{rclr}
    \fbox{$e,\ctx,K\rightarrow e,\ctx,K$}                     &               &                                                                                   &                          \\
    $e_1\:e_2,\ctx,K$                                         & $\rightarrow$ & $e_1,\ctx,K\circ(\_\:(e_2,\ctx))$                                                                            \\
    $\link{e_1}{e_2},\ctx,K$                                  & $\rightarrow$ & $e_1,\ctx,K\circ(\link{\_}{e_2})$                                                                            \\
    $x=e_1;e_2,\ctx,K$                                        & $\rightarrow$ & $e_1,(x,\ell)\cons\ctx,K\circ(x=\ell;(e_2,\ctx))$                                 & $\ell\not\in\FLoc(\ctx)$ \\
    \\
    \fbox{$v,K\rightarrow e,\ctx,K$}                          &               &                                                                                   &                          \\
    $\langle\lambda x.e,\ctx_1\rangle,K\circ(\_\:(e_2,\ctx))$ & $\rightarrow$ & $e_2,\ctx,K\circ(\langle\lambda x.e,\ctx_1\rangle\:\_)$                                                      \\
    $\ctx_1,K\circ(\link{\_}{e_2})$                           & $\rightarrow$ & $e_2,\ctx_1,K$                                                                                               \\
    $v_1,K\circ(x=\ell;(e_2,\ctx))$                           & $\rightarrow$ & $e_2,(x,\mu.\closeloc{\ell}{v_1})\cons\ctx,K\circ(x=\mu.\closeloc{\ell}{v_1};\_)$                            \\
    $v_2,K\circ(\langle\lambda x.e,\ctx_1\rangle\:\_)$        & $\rightarrow$ & $e,(x,v_2)\cons\ctx_1,K$                                                                                     \\
    \\
    \fbox{$v,K\rightarrow v,K$}                               &               &                                                                                   &                          \\
    $\ctx_2,K\circ(x=w_1;\_)$                                 & $\rightarrow$ & $(x,w_1)\cons\ctx_2,K$                                                                                       \\
    \\
    \fbox{$e,\ctx,K\rightarrow v,K$}                          &               &                                                                                   &                          \\
    $x,\ctx,K$                                                & $\rightarrow$ & $v,K$                                                                             & $v=\ctx(x)$              \\
    $x,\ctx,K$                                                & $\rightarrow$ & $v^{\mu.v},K$                                                                     & $\mu.v=\ctx(x)$          \\
    $\lambda x.e,\ctx,K$                                      & $\rightarrow$ & $\langle\lambda x.e,\ctx\rangle,K$                                                                           \\
    $\varepsilon,\ctx,K$                                      & $\rightarrow$ & $\bullet,K$
  \end{tabular}
  \caption{The equivalent small-step operational semantics.}
  \label{fig:smallstep}
\end{figure}

\subsection{Adding Memory}

The first step towards abstraction is reformulating the semantics into a version with memory.

\begin{figure}[h!]
  \centering
  \begin{tabular}{rrcll}
    Environment & $\ctx$ & $\in$         & $\Ctx$                                                                       \\
    Location    & $\ell$ & $\in$         & $\Loc$                                                                       \\
    Memory      & $\mem$ & $\in$         & $\Mem\triangleq\fin{\Loc}{\Value}$                                           \\
    Value       & $v$    & $\in$         & $\Value \triangleq\Ctx+\ExprVar\times\Expr\times\Ctx$                        \\
    Environment & $\ctx$ & $\rightarrow$ & $\bullet$                                             & empty stack          \\
                &        & $\vbar$       & $(x,\ell)\cons\ctx$                                   & location binding     \\
    Value       & $v$    & $\rightarrow$ & $\ctx$                                                & exported environment \\
                &        & $\vbar$       & $\langle \lambda x.e, \ctx \rangle$                   & closure
  \end{tabular}
  \caption{Definition of the semantic domains with memory.}
  \label{fig:memdomain}
\end{figure}
\begin{figure}[h!]
  \centering
  \begin{tabular}{rclr}
    \fbox{$e,\ctx,\mem,K\rightarrow e,\ctx,\mem,K$}                &               &                                                                 &                                  \\
    $e_1\:e_2,\ctx,\mem,K$                                         & $\rightarrow$ & $e_1,\ctx,\mem,K\circ(\_\:(e_2,\ctx))$                                                             \\
    $\link{e_1}{e_2},\ctx,\mem,K$                                  & $\rightarrow$ & $e_1,\ctx,\mem,K\circ(\link{\_}{e_2})$                                                             \\
    $x=e_1;e_2,\ctx,\mem,K$                                        & $\rightarrow$ & $e_1,(x,\ell)\cons\ctx,\mem,K\circ(x=\ell;(e_2,\ctx))$          & $\ell\not\in\dom(m)\cup\FLoc(K)$ \\
    \\
    \fbox{$v,\mem,K\rightarrow e,\ctx,\mem,K$}                     &               &                                                                 &                                  \\
    $\langle\lambda x.e,\ctx_1\rangle,\mem,K\circ(\_\:(e_2,\ctx))$ & $\rightarrow$ & $e_2,\ctx,\mem,K\circ(\langle\lambda x.e,\ctx_1\rangle\:\_)$                                       \\
    $\ctx_1,\mem,K\circ(\link{\_}{e_2})$                           & $\rightarrow$ & $e_2,\ctx_1,\mem,K$                                                                                \\
    $v_1,\mem,K\circ(x=\ell;(e_2,\ctx))$                           & $\rightarrow$ & $e_2,(x,\ell)\cons\ctx,\mem[\ell\mapsto v_1],K\circ(x=\ell;\_)$                                    \\
    $v_2,\mem,K\circ(\langle\lambda x.e,\ctx_1\rangle\:\_)$        & $\rightarrow$ & $e,(x,\ell)\cons\ctx_1,\mem[\ell\mapsto v_2],K$                 & $\ell\not\in\dom(m)\cup\FLoc(K)$ \\
    \\
    \fbox{$v,\mem,K\rightarrow v,\mem,K$}                          &               &                                                                 &                                  \\
    $\ctx_2,\mem,K\circ(x=\ell;\_)$                                & $\rightarrow$ & $(x,\ell)\cons\ctx_2,\mem,K$                                                                       \\
    \\
    \fbox{$e,\ctx,\mem,K\rightarrow v,\mem,K$}                     &               &                                                                 &                                  \\
    $x,\ctx,\mem,K$                                                & $\rightarrow$ & $v,\mem,K$                                                      & $\ell=\ctx(x),v=\mem(\ell)$      \\
    $\lambda x.e,\ctx,\mem,K$                                      & $\rightarrow$ & $\langle\lambda x.e,\ctx\rangle,\mem,K$                                                            \\
    $\varepsilon,\ctx,\mem,K$                                      & $\rightarrow$ & $\bullet,\mem,K$
  \end{tabular}
  \caption{The small-step operational semantics with memory.}
  \label{fig:memsmallstep}
\end{figure}
\subsection{Reconciling the Two Semantics}
We need to prove that the two semantics simulate each other.
Thus, we need to define a notion of equivalence between the two semantic domains.
\begin{figure}[h!]
  \centering
  \begin{flushright}
    \fbox{$w\sim_f v,\mem$}
  \end{flushright}
  \[
    \begin{InfRule}{Eq-Nil}
      \infer0{\bullet\sim_f\bullet}
    \end{InfRule}\:
    \begin{InfRule}{Eq-ConsFree}
      \hypo{\ell\not\in\dom(f)}
      \hypo{\ell\not\in\dom(\mem)}
      \hypo{\ctx\sim_f\ctx'}
      \infer3{(x,\ell)\cons\ctx\sim_f(x,\ell)\cons\ctx'}
    \end{InfRule}\:
    \begin{InfRule}{Eq-ConsBound}
      \hypo{f(\ell)=\ell'}
      \hypo{\ell'\in\dom(\mem)}
      \hypo{\ctx\sim_f\ctx'}
      \infer3{(x,\ell)\cons\ctx\sim_f(x,\ell')\cons\ctx'}
    \end{InfRule}
  \]

  \[
    \begin{InfRule}{Eq-ConsWVal}
      \hypo{\mem(\ell')=v'}
      \hypo{w\sim_f v'}
      \hypo{\ctx\sim_f\ctx'}
      \infer3{(x,w)\cons\ctx\sim_f(x,\ell')\cons\ctx'}
    \end{InfRule}\:
    \begin{InfRule}{Eq-Clos}
      \hypo{\ctx\sim_f\ctx'}
      \infer1{\langle\lambda x.e,\ctx\rangle\sim_f\langle\lambda x.e,\ctx'\rangle}
    \end{InfRule}\:
    \begin{InfRule}{Eq-Rec}
      \hypo{L\text{ finite}}
      \hypo{\mem(\ell')=v'}
      \hypo{\forall\ell\not\in L,\:\openloc{v}{\ell}\sim_{f[\ell\mapsto\ell']}v'}
      \infer3{\mu.v\sim_f v'}
    \end{InfRule}
  \]
  \caption{The equivalence relation between weak values in the original semantics and values in the semantics with memory.
  $f\in\fin{\Loc}{\Loc}$ tells what the free locations in $w$ should be mapped to in memory.}
  \label{fig:equivrel}
\end{figure}

\begin{lem}[Equivalence under Substitution]
  For all $w_1,w_2,\ell,f,v_1',v_2',\ell',\mem$,
  \[(w_1\sim_{f[\ell\mapsto\ell']} v_1',\mem-\ell')\land (v_2'=\mem(\ell'))\land(w_2\sim_f v_2',\mem)\Rightarrow w_1[w_2/\ell]\sim_f v_1',\mem\]
\end{lem}

\section{Typing (Without Recursive Bindings)}
The definitions for types are in Figure \ref{fig:exttypdom} and the typing rules are in Figure \ref{fig:exttypjudge}.
The definitions for subtyping are in Figure \ref{fig:subtyping}.

\begin{figure}[h!]
  \centering
  \begin{tabular}{rrcll}
    Types              & $\tau$   & $\rightarrow$ & $\Gamma$               & module type       \\
                       &          & $\vbar$       & $\tau\rightarrow\tau$  & function type     \\
    Typing Environment & $\Gamma$ & $\rightarrow$ & $\bullet$              & empty environment \\
                       &          & $\vbar$       & $(x,\tau)\cons \Gamma$ & type binding
  \end{tabular}
  \caption{Definition of types.}
  \label{fig:exttypdom}
\end{figure}
\begin{figure}[h!]
  \begin{flushright}
    \fbox{$\Gamma\vdash e:\tau$}
  \end{flushright}
  \centering
  \[
    \begin{InfRule}{T-Id}
      \hypo{\tau=\Gamma(x)}
      \infer1{
        \Gamma\vdash x:\tau
      }
    \end{InfRule}\qquad
    \begin{InfRule}{T-Fn}
      \hypo{(x,\tau_1)\cons\Gamma\vdash e:\tau_2}
      \infer1{
        \Gamma\vdash\lambda x.e:\tau_1\rightarrow\tau_2
      }
    \end{InfRule}\qquad
    \begin{InfRule}{T-App}
      \hypo{
        \Gamma\vdash e_1:\tau_1\rightarrow\tau
      }
      \hypo{
        \Gamma\vdash e_2:\tau_2
      }
      \hypo{
        \tau_1\ge\tau_2
      }
      \infer3{
        \Gamma\vdash e_{1}\:e_{2}:\tau
      }
    \end{InfRule}
  \]

  \[
    \begin{InfRule}{T-Link}
      \hypo{
        \Gamma\vdash e_1:\Gamma_1
      }
      \hypo{
        \Gamma_1\vdash e_2:\tau_2
      }
      \infer2{
        \Gamma\vdash\link{e_{1}}{e_{2}}:\tau_2
      }
    \end{InfRule}\qquad
    \begin{InfRule}{T-Empty}
      \infer0{
        \Gamma\vdash\varepsilon:\bullet
      }
    \end{InfRule}\qquad
    \begin{InfRule}{T-Bind}
      \hypo{
        \Gamma\vdash e_1:\tau_1
      }
      \hypo{
        (x, \tau_1)\cons\Gamma\vdash e_2:\Gamma_2
      }
      \infer2{
        \Gamma\vdash x=e_1;e_2:(x,\tau_1)\cons\Gamma_2
      }
    \end{InfRule}
  \]
  \caption{The typing judgment.}
  \label{fig:exttypjudge}
\end{figure}
\begin{figure}[h!]
  \begin{flushright}
    \fbox{$\tau\ge\tau$}
  \end{flushright}
  \centering
  \vspace{0pt} % -0.75em}
  \[
    \begin{InfRule}{Nil}
      \infer0{
        \bullet\ge\bullet
      }
    \end{InfRule}\qquad
    \begin{InfRule}{ConsFree}
      \hypo{
        x\not\in\mathsf{dom}(\Gamma)
      }
      \hypo{
        \Gamma\ge\Gamma'
      }
      \infer2{
        \Gamma\ge(x,\tau)\cons\Gamma'
      }
    \end{InfRule}\qquad
    \begin{InfRule}{ConsBound}
      \hypo{
        \Gamma(x)\ge\tau
      }
      \hypo{
        \Gamma-x\ge\Gamma'
      }
      \infer2{
        \Gamma\ge(x,\tau)\cons\Gamma'
      }
    \end{InfRule}\qquad
    \begin{InfRule}{Arrow}
      \hypo{
        \tau_2\ge\tau_1
      }
      \hypo{
        \tau_1'\ge\tau_2'
      }
      \infer2{
        \tau_1\rightarrow\tau_1'\ge
        \tau_2\rightarrow\tau_2'
      }
    \end{InfRule}
  \]
  \caption{The subtype relation.}
  \label{fig:subtyping}
\end{figure}

\subsection{Type Safety}
\begin{clm}[Type Safety]
  For all $e\in\Expr$, if $\bullet\vdash e:\tau$ for some $\tau$, then there exists some $v\in\Value$ such that $(e,\bullet)\semarrow v$.
\end{clm}
\begin{proof}[Proof sketch]
  We prove this through unary logical relations and induction on the typing judgment.

  \begin{tabular}{rclr}
    \textbf{Value Relation}            &              &                                                                                                        & \fbox{$\ValRel{\tau}$}       \\
    $\ValRel{\bullet}$                 & $\triangleq$ & $\Ctx$                                                                                                                                \\
    $\ValRel{(x,\tau)\cons\Gamma}$     & $\triangleq$ & $\{\ctx|\ctx(x)\in\ValRel{\tau}\land\ctx-x\in\ValRel{\Gamma-x}\}$                                                                     \\
    $\ValRel{\tau_1\rightarrow\tau_2}$ & $\triangleq$ & $\{\langle\lambda x.e,\ctx\rangle|\forall v\in\ValRel{\tau_1}:(e,(x,v)\cons\ctx)\in\ExprRel{\tau_2}\}$                                \\
    \\
    \textbf{Expression Relation}       &              &                                                                                                        & \fbox{$\ExprRel{\tau}$}      \\
    $\ExprRel{\tau}$                   & $\triangleq$ & $\{(e,\ctx)|\exists v\in\ValRel{\tau}:(e,\ctx)\semarrow v\}$                                                                          \\
    \\
    \textbf{Semantic Typing}           &              &                                                                                                        & \fbox{$\Gamma\vDash e:\tau$} \\
    $\Gamma\vDash e:\tau$              & $\triangleq$ & $\forall\ctx\in\ValRel{\Gamma}:(e,\ctx)\in\ExprRel{\tau}$
  \end{tabular}

  \vphantom{}

  We want to prove that:
  \[\Gamma\vdash e:\tau\Rightarrow\Gamma\vDash e:\tau\]
  by induction on $\vdash$.

  For the base case of $\bullet$, the proof is trivial.
  For inductive cases, we need to show \emph{compatibility} lemmas.
  That is, we must show that the typing rules for syntactic typing hold for semantic typing as well.
  For this, we need the \emph{subtyping} lemma:
  \[\tau_1\ge\tau_2\Rightarrow\ValRel{\tau_1}\supseteq\ValRel{\tau_2}\]
  Then by the inductive hypothesis and compatibility, the result follows.
\end{proof}

\subsection{Type Inference}

When modules are first-class, type variables can go in the place of type environments.

First we define the syntax for type constraints.
\begin{figure}[h!]
  \centering
  \begin{tabular}{rrcll}
    Type Variable      & $\alpha$ & $\in$         & $\TyVar$                                                               \\
    Path               & $p$      & $\rightarrow$ & $\epsilon$                       & empty string                        \\
                       &          & $\vbar$       & $p x$                            & concatenation with identifier       \\
    Types              & $\tau$   & $\rightarrow$ & $\Gamma$ | $\tau\rightarrow\tau$ & module/function types               \\
    Type Environment   & $\Gamma$ & $\rightarrow$ & $\bullet$                        & empty environment                   \\
                       &          & $\vbar$       & $(x,\tau)\cons \Gamma$           & binding                             \\
                       &          & $\vbar$       & $\alpha.p$                       & type variable                       \\
                       &          & $\vbar$       & $[].p$                           & types from the external environment \\
    Type Constraint    & $u$      & $\rightarrow$ & $\tau\A{=}\tau$                  & equality constraint                 \\
                       &          & $\vbar$       & $\tau\A{\ge}\tau$                & subtyping constraint                \\
    Set of Constraints & $U$      & $\subseteq$   & $\{u|u\text{ type constraint}\}$
  \end{tabular}
  \caption{Definition of type constraints.}
  \label{fig:typeqdom}
\end{figure}

Next we define the type access operation $\tau(x)$:
\begin{align*}
  \bullet(x)              & \triangleq\bot      &                      &  & (\alpha.p)(x)        & \triangleq\alpha.px \\
  ((x,\tau)\cons\_)(x)    & \triangleq\tau      &                      &  & ([].p)(x)            & \triangleq[].px     \\
  ((x',\_)\cons\Gamma)(x) & \triangleq\Gamma(x) & \text{when }x'\neq x &  & (\_\rightarrow\_)(x) & \triangleq\bot
\end{align*}

Now we can define the constraint generation algorithm $V(\Gamma,e,\alpha)$.
Note that the \textbf{let} $U$ \texttt{=} $\_$ \textbf{in} $\_$ notation returns $\bot$ if the right hand side is $\bot$.
Also note that we write $\alpha$ for $\alpha.\epsilon$ as well.

\begin{flushright}\fbox{$V(\Gamma,e,\alpha)=U$}\end{flushright}
\hspace{-2.5em}
\begin{tabular}{rclrcl}
  $V(\Gamma,\varepsilon,\alpha)$ & $\triangleq$ & $\{\alpha\A{=}\bullet\}$                                                             & $V(\Gamma,\link{e_1}{e_2},\alpha)$         & $\triangleq$ & \textbf{let} $\alpha_1$ \texttt{=} \textit{fresh} \textbf{in}                       \\
  $V(\Gamma,x,\alpha)$           & $\triangleq$ & \textbf{let} $\tau$ \texttt{=} $\Gamma(x)$ \textbf{in}                               &                                            &              & \textbf{let} $U_1$ \texttt{=} $V(\Gamma,e_1,\alpha_1)$ \textbf{in}                  \\
                                 &              & $\{\alpha\A{=}\tau\}$                                                                &                                            &              & \textbf{let} $U_2$ \texttt{=} $V(\alpha_1,e_2,\alpha)$  \textbf{in}                 \\
  $V(\Gamma,\lambda x.e,\alpha)$ & $\triangleq$ & \textbf{let} $\alpha_1,\alpha_2$ \texttt{=} \textit{fresh} \textbf{in}               &                                            &              & $U_1\cup U_2$                                                                       \\
                                 &              & \textbf{let} $U$ \texttt{=} $V((x,\alpha_1)\cons\Gamma,e,\alpha_2)$ \textbf{in}      & $V(\Gamma,\Lete\:\modid\:e_1\:e_2,\alpha)$ & $\triangleq$ & \textbf{let} $\alpha_1,\alpha_2$ \texttt{=} \textit{fresh} \textbf{in}              \\
                                 &              & $\{\alpha\A{=}\alpha_1\rightarrow\alpha_2\}\cup U$                                   &                                            &              & \textbf{let} $U_1$ \texttt{=} $V(\Gamma,e_1,\alpha_1)$ \textbf{in}                  \\
  $V(\Gamma,e_1\:e_2,\alpha)$    & $\triangleq$ & \textbf{let} $\alpha_1,\alpha_2,\alpha_3$ \texttt{=} \textit{fresh} \textbf{in}      &                                            &              & \textbf{let} $U_2$ \texttt{=} $V((x,\alpha_1)\cons\Gamma,e_2,\alpha_2)$ \textbf{in} \\
                                 &              & \textbf{let} $U_1$ \texttt{=} $V(\Gamma,e_1,\alpha_1)$ \textbf{in}                   &                                            &              & $\{\alpha\A{=}(x,\alpha_1)\cons\alpha_2\}\cup U_1\cup U_2$                          \\
                                 &              & \textbf{let} $U_2$ \texttt{=} $V(\Gamma,e_2,\alpha_2)$ \textbf{in}                                                                                                                                                                     \\
                                 &              & $\{\alpha_1\A{=}\alpha_3\rightarrow\alpha,\alpha_3\A{\ge}\alpha_2\}\cup U_1\cup U_2$
\end{tabular}

\vphantom{}

We want to prove that the constraint generation algorithm is correct.

First, for $\tau\in\Type$, define the access operation $\tau.p$ (which may fail):
\begin{align*}
  \tau.\epsilon & \triangleq\tau &  &  & \tau.px & \triangleq(\tau.p)(x)
\end{align*}
and define the injection operation $\tau[\external]$:
\begin{align*}
  (\bullet)[\external]                 & \triangleq\bullet                                       &  &  & ((x,\tau)\cons\Gamma)[\external] & \triangleq(x,\tau[\external])\cons\Gamma[\external] \\
  (\alpha.p)[\external]                & \triangleq\alpha.p                                      &  &  & ([].p)[\external]                & \triangleq\external.p                               \\
  (\tau_1\rightarrow\tau_2)[\external] & \triangleq\tau_1[\external]\rightarrow\tau_2[\external]
\end{align*}

Let $\Subst\triangleq\fin{\TyVar}{\Type}$ be the set of substitutions.
For $S\in\Subst$, define:
\begin{align*}
  S\bullet    & \triangleq\bullet  &                                  &  & S(\tau_1\rightarrow\tau_2) & \triangleq S\tau_1\rightarrow S\tau_2                                      \\
  S(\alpha.p) & \triangleq\alpha.p & \text{when }\alpha\not\in dom(S) &  & S(\alpha.p)                & \triangleq \tau.p                     & \text{when }\alpha\mapsto\tau\in S \\
  S([].p)     & \triangleq [].p
\end{align*}
Define:
\begin{align*}
  (S,\external)\vDash U\triangleq & \forall(\tau_1\A{=}\tau_2)\in U:(S\tau_1)[\external]= (S\tau_2)[\external]\text{ and} \\
                                  & \forall(\tau_1\A{\ge}\tau_2)\in U:(S\tau_1)[\external]\ge(S\tau_2)[\external]
\end{align*}
where subtyping rules are the same as Figure \ref{fig:subtyping} and subtyping between type variables are not defined.

Then we can show that:
\begin{clm}[Correnctness of $V$]
  For $e\in\Expr$, $\Gamma,\external\in\TyEnv$, $\alpha\in\TyVar$, $S\in\Subst$:
  \begin{align*}
    (S,\external)\vDash V(\Gamma,e,\alpha) \Leftrightarrow & (S\Gamma)[\external]\vdash e:(S\alpha)[\external]
  \end{align*}
\end{clm}
\begin{proof}[Proof sketch]
  Structural induction on $e$.
\end{proof}

Note that by including $[].p$ in type environments, we can naturally generate constraints about the external environment $[]$.
Also, by injection, we can utilize constraints generated \emph{in advance} to obtain constraints generated from a more informed environment.
We extend injection to the output of the constraint-generating algorithm:
\begin{align*}
  \bot[\external] \triangleq & \bot                                                                                                          \\
  U[\external]    \triangleq & \{\tau_1[\external]\A{=}\tau_2[\external]|(\tau_1\A{=}\tau_2)\in U\}\cup                                      \\
                             & \{\tau_1[\external]\A{\ge}\tau_2[\external]|(\tau_1\A{\ge}\tau_2)\in U\} & \text{when all injections succeed} \\
  U[\external]    \triangleq & \bot                                                                     & \text{when injection fails}
\end{align*}
Then we can prove:
\begin{clm}[Advance]
  For $e\in\Expr$, $\Gamma,\external\in\TyEnv$, $\alpha\in\TyVar$:
  \[V(\Gamma[\external],e,\alpha)=V(\Gamma,e,\alpha)[\external]\]
\end{clm}
\begin{proof}[Proof sketch]
  Structural induction on $\Gamma$.
\end{proof}

\end{document}
%%% Local Variables: 
%%% coding: utf-8
%%% mode: latex
%%% TeX-engine: xetex
%%% End:
