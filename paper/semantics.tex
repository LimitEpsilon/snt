\section{Syntax and Semantics}
\subsection{Abstract Syntax}
\begin{figure}[htb]
  \centering
  \begin{tabular}{rrcll}
    Identifiers & $x$ & $\in$         & $\ExprVar$                                                      \\
    Expression  & $e$ & $\rightarrow$ & $x$ $\vbar$ $\lambda x.e$ $\vbar$ $e$ $e$ & $\lambda$-calculus  \\
                &     & $\vbar$       & $\link{e}{e}$                             & linked expression   \\
                &     & $\vbar$       & $\varepsilon$                             & empty module        \\
                &     & $\vbar$       & $x=e$ ; $e$                               & (recursive) binding
  \end{tabular}
  \caption{Abstract syntax of the language.}
  \label{fig:syntax}
\end{figure}
\subsection{Operational Semantics}
\begin{figure}[h!]
  \centering
  \begin{tabular}{rrcll}
    Environment & $\ctx$ & $\in$         & $\Ctx$                                                                        \\
    Location    & $\ell$ & $\in$         & $\Loc\triangleq\{\text{infinite set of locations}\}$                          \\
    Value       & $v$    & $\in$         & $\Value \triangleq\Ctx+\ExprVar\times\Expr\times\Ctx$                         \\
    Weak Value  & $w$    & $\in$         & $\Walue\triangleq\Value+\Loc\times\Value$                                     \\
    Environment & $\ctx$ & $\rightarrow$ & $\bullet$                                             & empty stack           \\
                &        & $\vbar$       & $(x,\ell)\cons\ctx$                                   & free location binding \\
                &        & $\vbar$       & $(x,w)\cons \ctx$                                     & weak value binding    \\
    Value       & $v$    & $\rightarrow$ & $\ctx$                                                & exported environment  \\
                &        & $\vbar$       & $\langle \lambda x.e, \ctx \rangle$                   & closure               \\
    Weak Value  & $w$    & $\rightarrow$ & $v$                                                   & value                 \\
                &        & $\vbar$       & $\mu\ell.v$                                           & recursive value
  \end{tabular}
  \caption{Definition of the semantic domains.}
  \label{fig:domain}
\end{figure}
\begin{figure}[h!]
  \begin{flushright}
    \fbox{$\ctx\vdash e\Downarrow v$}
  \end{flushright}
  \centering
  \vspace{0pt} % -0.75em}
  \[
    \begin{InfRule}{Id}
      \hypo{\ctx(x)=v}
      \infer1{
        \ctx\vdash x
        \Downarrow
        v
      }
    \end{InfRule}\:
    \begin{InfRule}{RecId}
      \hypo{\ctx(x)=\mu\ell.v}
      \infer1{
        \ctx\vdash x
        \Downarrow
        v[\mu\ell.v/\ell]
      }
    \end{InfRule}\:
    \begin{InfRule}{Fn}
      \infer0{
        \ctx\vdash\lambda x.e
        \Downarrow
        \langle\lambda x.e, \ctx\rangle
      }
    \end{InfRule}\:
    \begin{InfRule}{App}
      \hypo{
        \ctx\vdash e_{1}
        \Downarrow
        \langle\lambda x.e, \ctx_1\rangle
      }
      \hypo{
        \ctx\vdash e_{2}
        \Downarrow
        v_2
      }
      \hypo{
        (x, v_2)\cons \ctx_1\vdash e
        \Downarrow
        v
      }
      \infer{2,1}{
        \ctx\vdash e_{1}\:e_{2}
        \Downarrow
        v
      }
    \end{InfRule}
  \]

  \[
    \begin{InfRule}{Link}
      \hypo{
        \ctx\vdash e_{1}
        \Downarrow
        \ctx_1
      }
      \hypo{
        \ctx_1\vdash e_{2}
        \Downarrow
        v
      }
      \infer2{
        \ctx\vdash \link{e_{1}}{e_{2}}
        \Downarrow
        v
      }
    \end{InfRule}\quad
    \begin{InfRule}{Empty}
      \infer0{
        \ctx\vdash \varepsilon
        \Downarrow
        \bullet
      }
    \end{InfRule}\quad
    \begin{InfRule}{Bind}
      \hypo{
        \ell\not\in\FLoc(\ctx)
      }
      \hypo{
        (x,\ell)\cons\ctx\vdash e_{1}
        \Downarrow
        v_1
      }
      \hypo{
        (x, \mu\ell.v_1)\cons \ctx\vdash e_{2}
        \Downarrow
        \ctx_2
      }
      \infer{2,1}{
        \ctx\vdash x=e_1; e_2
        \Downarrow
        (x,\mu\ell.v_1)\cons\ctx_2
      }
    \end{InfRule}
  \]
  \caption{The big-step operational semantics.}
  \label{fig:bigstep}
\end{figure}
\subsection{Reconciling with Conventional Backpatching}
\begin{figure}[h!]
  \centering
  \begin{tabular}{rrcll}
    Environment & $\ctx$ & $\in$         & $\Mtx$                                                                       \\
    Memory      & $\mem$ & $\in$         & $\Mem\triangleq\fin{\Loc}{\Malue}$                                           \\
    Value       & $v$    & $\in$         & $\Malue \triangleq\Mtx+\ExprVar\times\Expr\times\Mtx$                        \\
    Environment & $\ctx$ & $\rightarrow$ & $\bullet$                                             & empty stack          \\
                &        & $\vbar$       & $(x,\ell)\cons\ctx$                                   & location binding     \\
    Value       & $v$    & $\rightarrow$ & $\ctx$                                                & exported environment \\
                &        & $\vbar$       & $\langle \lambda x.e, \ctx \rangle$                   & closure
  \end{tabular}
  \caption{Definition of the semantic domains with memory.}
  \label{fig:memdomain}
\end{figure}
\begin{figure}[h!]
  \begin{flushright}
    \fbox{$\ctx,\mem,L\vdash e\Downarrow v,\mem',L'$}
  \end{flushright}
  \centering
  \vspace{0pt} % -0.75em}
  \[
    \begin{InfRule}{Id}
      \hypo{\ctx(x)=\ell}
      \hypo{\mem(\ell)=v}
      \infer2{
        \ctx,\mem,L\vdash x
        \Downarrow
        v,\mem,L
      }
    \end{InfRule}\:
    \begin{InfRule}{Fn}
      \infer0{
        \ctx,\mem,L\vdash \lambda x.e
        \Downarrow
        \langle\lambda x.e, \ctx\rangle,\mem,L
      }
    \end{InfRule}
  \]

  \[
    \begin{InfRule}{App}
      \hypo{
        \ctx,\mem,L\vdash e_{1}
        \Downarrow
        \langle\lambda x.e, \ctx_1\rangle,\mem_1,L_1
      }
      \hypo{
        \ctx,\mem_1,L_1\vdash e_{2}
        \Downarrow
        v_2,\mem_2,L_2
      }
      \hypo{
        \ell\not\in\dom(\mem_2)\cup L_2
      }
      \hypo{
        (x, \ell)\cons \ctx_1,\mem_2[\ell\mapsto v_2],L_2\vdash e
        \Downarrow
        v,\mem',L'
      }
      \infer{3,1}{
        \ctx,\mem,L\vdash e_{1}\:e_{2}
        \Downarrow
        v,\mem',L'
      }
    \end{InfRule}
  \]

  \[
    \begin{InfRule}{Link}
      \hypo{
        \ctx,\mem,L\vdash e_{1}
        \Downarrow
        \ctx_1,\mem_1,L_1
      }
      \hypo{
        \ctx_1,\mem_1,L_1\vdash e_{2}
        \Downarrow
        v,\mem',L'
      }
      \infer2{
        \ctx,\mem,L\vdash \link{e_{1}}{e_{2}}
        \Downarrow
        v,\mem',L'
      }
    \end{InfRule}\quad
    \begin{InfRule}{Empty}
      \infer0{
        \ctx,\mem,L\vdash \varepsilon
        \Downarrow
        \bullet,\mem,L
      }
    \end{InfRule}\quad
  \]

  \[
    \begin{InfRule}{Bind}
      \hypo{
        \ell\not\in\dom(\mem)\cup L
      }
      \hypo{
        (x,\ell)\cons\ctx,\mem,L\cup\{\ell\}\vdash e_{1}
        \Downarrow
        v_1,\mem_1,L_1
      }
      \hypo{
        (x, \ell)\cons \ctx,\mem_1[\ell\mapsto v_1],L_1\vdash e_{2}
        \Downarrow
        \ctx_2,\mem',L'
      }
      \infer{2,1}{
        \ctx,\mem,L\vdash x=e_1; e_2
        \Downarrow
        (x,\ell)\cons\ctx_2,\mem',L'
      }
    \end{InfRule}
  \]
  \caption{The big-step operational semantics with memory.}
  \label{fig:membigstep}
\end{figure}

The semantics in Figure \ref{fig:bigstep} makes sense due to similarity with a conventional backpatching semantics as presented in Figure \ref{fig:membigstep}.
We have defined a relation $\sim$ that satisfies:
\[\sim\subseteq\Walue\times(\Malue\times\Mem\times\pset(\Loc))\qquad\bullet\sim(\bullet,\varnothing,\varnothing)\]
and the following theorem:
\begin{theorem}[Equivalence of semantics]\normalfont
  For all $\ctx\in\Ctx,\ctx'\in\Mtx\times\Mem\times\pset(\Loc),v\in\Value,v'\in\Malue\times\Mem\times\pset(\Loc)$, we have:
  \begin{align*}
    \ctx\sim\ctx'\text{ and }\ctx\vdash e\Downarrow v   & \Rightarrow\exists v':v\sim v'\text{ and }\ctx'\vdash e\Downarrow v' \\
    \ctx\sim\ctx'\text{ and }\ctx'\vdash e\Downarrow v' & \Rightarrow\exists v:v\sim v'\text{ and }\ctx\vdash e\Downarrow v
  \end{align*}
\end{theorem}
The actual definition for $\sim$ can be found in the appendix.
