\section{Syntax and Semantics}
\subsection{Abstract Syntax}
\begin{figure}[htb]
  \centering
  \begin{tabular}{rrcll}
    Identifiers & $x$ & $\in$         & $\ExprVar$                                                      \\
    Expression  & $e$ & $\rightarrow$ & $x$ $\vbar$ $\lambda x.e$ $\vbar$ $e$ $e$ & $\lambda$-calculus  \\
                &     & $\vbar$       & $\link{e}{e}$                             & linked expression   \\
                &     & $\vbar$       & $\varepsilon$                             & empty module        \\
                &     & $\vbar$       & $x=e$ ; $e$                               & (recursive) binding
  \end{tabular}
  \caption{Abstract syntax of the language.}
  \label{fig:syntax}
\end{figure}
\subsection{Operational Semantics}
\begin{figure}[h!]
  \centering
  \begin{tabular}{rrcll}
    Environment & $\ctx$ & $\in$         & $\Ctx$                                                                        \\
    Location    & $\ell$ & $\in$         & $\Loc\triangleq\{\text{infinite set of locations}\}$                          \\
    Value       & $v$    & $\in$         & $\Value \triangleq\Ctx+\ExprVar\times\Expr\times\Ctx$                         \\
    Weak Value  & $w$    & $\in$         & $\Walue\triangleq\Value+\Loc\times\Value$                                     \\
    Environment & $\ctx$ & $\rightarrow$ & $\bullet$                                             & empty stack           \\
                &        & $\vbar$       & $(x,\ell)\cons\ctx$                                   & free location binding \\
                &        & $\vbar$       & $(x,w)\cons \ctx$                                     & weak value binding    \\
    Value       & $v$    & $\rightarrow$ & $\ctx$                                                & exported environment  \\
                &        & $\vbar$       & $\langle \lambda x.e, \ctx \rangle$                   & closure               \\
    Weak Value  & $w$    & $\rightarrow$ & $v$                                                   & value                 \\
                &        & $\vbar$       & $\mu\ell.v$                                           & recursive value
  \end{tabular}
  \caption{Definition of the semantic domains.}
  \label{fig:domain}
\end{figure}
\begin{figure}[h!]
  \begin{flushright}
    \fbox{$\evjudg{\ctx}{e}{v}$}
  \end{flushright}
  \centering
  \vspace{0pt} % -0.75em}
  \[
    \begin{InfRule}{Id}
      \hypo{\ctx(x)=v}
      \infer1{\evjudg{\ctx}{x}{v}}
    \end{InfRule}\:
    \begin{InfRule}{RecId}
      \hypo{\ctx(x)=\mu\ell.v}
      \infer1{\evjudg{\ctx}{x}{v[\mu\ell.v/\ell]}}
    \end{InfRule}\:
    \begin{InfRule}{Fn}
      \infer0{\evjudg{\ctx}{\lambda x.e}{\langle\lambda x.e, \ctx\rangle}}
    \end{InfRule}\:
    \begin{InfRule}{App}
      \hypo{\evjudg{\ctx}{e_1}{\langle\lambda x.e, \ctx_1\rangle}}
      \hypo{\evjudg{\ctx}{e_2}{v_2}}
      \hypo{\evjudg{(x, v_2)\cons \ctx_1}{e}{v}}
      \infer{2,1}{\evjudg{\ctx}{e_1\:e_2}{v}}
    \end{InfRule}
  \]

  \[
    \begin{InfRule}{Link}
      \hypo{\evjudg{\ctx}{e_1}{\ctx_1}}
      \hypo{\evjudg{\ctx_1}{e_2}{v}}
      \infer2{\evjudg{\ctx}{\link{e_1}{e_2}}{v}}
    \end{InfRule}\quad
    \begin{InfRule}{Empty}
      \infer0{\evjudg{\ctx}{\varepsilon}{\bullet}}
    \end{InfRule}\quad
    \begin{InfRule}{Bind}
      \hypo{\ell\not\in\FLoc(\ctx)}
      \hypo{\evjudg{(x,\ell)\cons\ctx}{e_1}{v_1}}
      \hypo{\evjudg{(x, \mu\ell.v_1)\cons \ctx}{e_1}{\ctx_2}}
      \infer{2,1}{\evjudg{\ctx}{x=e_1; e_2}{(x,\mu\ell.v_1)\cons\ctx_2}}
    \end{InfRule}
  \]
  \caption{The big-step operational semantics.}
  \label{fig:bigstep}
\end{figure}
\subsection{Reconciling with Conventional Backpatching}
\begin{figure}[h!]
  \centering
  \begin{tabular}{rrcll}
    Environment   & $\ctx$ & $\in$         & $\Mtx\triangleq\fin{\ExprVar}{\Loc}$                                         \\
    Memory        & $\mem$ & $\in$         & $\Mem\triangleq\fin{\Loc}{\Malue}$                                           \\
    Allocated set & $L$    & $\subseteq$   & $\Loc$                                                                       \\
    Value         & $v$    & $\in$         & $\Malue \triangleq\Mtx+\ExprVar\times\Expr\times\Mtx$                        \\
    Environment   & $\ctx$ & $\rightarrow$ & $\bullet$                                             & empty stack          \\
                  &        & $\vbar$       & $(x,\ell)\cons\ctx$                                   & location binding     \\
    Value         & $v$    & $\rightarrow$ & $\ctx$                                                & exported environment \\
                  &        & $\vbar$       & $\langle \lambda x.e, \ctx \rangle$                   & closure
  \end{tabular}
  \caption{Definition of the semantic domains with memory.}
  \label{fig:memdomain}
\end{figure}
\begin{figure}[h!]
  \begin{flushright}
    \fbox{$\evjudg{\ctx,\mem,L}{e}{v,\mem',L'}$}
  \end{flushright}
  \centering
  \vspace{0pt} % -0.75em}
  \[
    \begin{InfRule}{Id}
      \hypo{\ctx(x)=\ell}
      \hypo{\mem(\ell)=v}
      \infer2{\evjudg{\ctx,\mem,L}{x}{v,\mem,L}}
    \end{InfRule}\:
    \begin{InfRule}{Fn}
      \infer0{\evjudg{\ctx,\mem,L}{\lambda x.e}{\langle\lambda x.e,\ctx\rangle,\mem,L}}
    \end{InfRule}
  \]

  \[
    \begin{InfRule}{App}
      \hypo{\evjudg{\ctx,\mem,L}{e_1}{\langle\lambda x.e,\ctx_1\rangle,\mem_1,L_1}}
      \hypo{\evjudg{\ctx,\mem_1,L_1}{e_2}{v_2,\mem_2,L_2}}
      \hypo{\ell\not\in\dom(\mem_2)\cup L_2}
      \hypo{\evjudg{(x,\ell)\cons\ctx_1,\mem_2[\ell\mapsto v_2],L_2}{e}{v,\mem',L'}}
      \infer{3,1}{\evjudg{\ctx,\mem,L}{e_1\:e_2}{v,\mem',L'}}
    \end{InfRule}
  \]

  \[
    \begin{InfRule}{Link}
      \hypo{\evjudg{\ctx,\mem,L}{e_1}{\ctx_1,\mem_1,L_1}}
      \hypo{\evjudg{\ctx_1,\mem_1,L_1}{e_2}{v,\mem',L'}}
      \infer2{\evjudg{\ctx,\mem,L}{\link{e_1}{e_2}}{v,\mem',L'}}
    \end{InfRule}\quad
    \begin{InfRule}{Empty}
      \infer0{\evjudg{\ctx,\mem,L}{\varepsilon}{\bullet,\mem,L}}
    \end{InfRule}
  \]

  \[
    \begin{InfRule}{Bind}
      \hypo{\ell\not\in\dom(\mem)\cup L}
      \hypo{\evjudg{(x,\ell)\cons\ctx,\mem,L\cup\{\ell\}}{e_1}{v_1,\mem_1,L_1}}
      \hypo{\evjudg{(x,\ell)\cons\ctx,\mem_1[\ell\mapsto v_1],L_1}{e_2}{\ctx_2,\mem',L'}}
      \infer{2,1}{\evjudg{\ctx,\mem,L}{x=e_1;e_2}{(x,\ell)\cons\ctx_2,\mem',L'}}
    \end{InfRule}
  \]
  \caption{The big-step operational semantics with memory.}
  \label{fig:membigstep}
\end{figure}

The semantics in Figure \ref{fig:bigstep} makes sense due to similarity with a conventional backpatching semantics as presented in Figure \ref{fig:membigstep}.
We have defined a relation $\sim$ that satisfies:
\[\sim\subseteq\Walue\times(\Malue\times\Mem\times\pset(\Loc))\qquad\bullet\sim(\bullet,\varnothing,\varnothing)\]
and the following theorem:
\begin{theorem}[Equivalence of semantics]\normalfont
  For all $\ctx\in\Ctx,\ctx'\in\Mtx\times\Mem\times\pset(\Loc),v\in\Value,v'\in\Malue\times\Mem\times\pset(\Loc)$, we have:
  \begin{align*}
    \ctx\sim\ctx'\text{ and }\ctx\vdash e\evjudg{\ctx}{e}{v} v   & \Rightarrow\exists v':v\sim v'\text{ and }\ctx'\vdash e\evjudg{\ctx}{e}{v} v' \\
    \ctx\sim\ctx'\text{ and }\ctx'\vdash e\evjudg{\ctx}{e}{v} v' & \Rightarrow\exists v:v\sim v'\text{ and }\ctx\vdash e\evjudg{\ctx}{e}{v} v
  \end{align*}
\end{theorem}
The actual definition for $\sim$ can be found in the appendix.
