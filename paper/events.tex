\section{Generating and Resolving Events}
Now we formulate the semantics for generating events.

\begin{figure}[h!]
  \centering
  \begin{tabular}{rrcll}
    Event       & $\event$ & $\rightarrow$ & $\InitE$           & initial environment \\
                &          & $\vbar$       & $\ReadE(\event,x)$ & read event          \\
                &          & $\vbar$       & $\CallE(\event,v)$ & call event          \\
    Environment & $\ctx$   & $\rightarrow$ & $\cdots$                                 \\
                &          & $\vbar$       & $[\event]$         & answer to an event  \\
    Value       & $v$      & $\rightarrow$ & $\cdots$                                 \\
                &          & $\vbar$       & $\event$           & answer to an event
  \end{tabular}
  \caption{Definition of the semantic domains with events. All other semantic domains are equal to Figure \ref{fig:domain}.}
  \label{fig:eventdomain}
\end{figure}

We extend how to read weak values given an environment.
\begin{align*}
  \bullet(x)  & \triangleq\bot             &  &  & ((x',\ell)\cons\ctx)(x) & \triangleq (x=x'?\ell:\ctx(x)) \\
  [\event](x) & \triangleq\ReadE(\event,x) &  &  & ((x',w)\cons\ctx)(x)    & \triangleq (x=x'?w:\ctx(x))    \\
\end{align*}

Then we need to add only one rule to the semantics in Figure \ref{fig:bigstep} for the semantics to incorporate events.
\[
  \begin{InfRule}{AppEvent}
    \hypo{
      \ctx\vdash e_{1}
      \Downarrow
      \event
    }
    \hypo{
      \ctx\vdash e_{2}
      \Downarrow
      v
    }
    \infer2{
      \ctx\vdash e_{1}\:e_{2}
      \Downarrow
      \CallE(\event,v)
    }
  \end{InfRule}
\]

Now we need to formulate the \emph{concrete linking} rules.
The concrete linking rule $\ctx_0\semlink w$, given an answer $\ctx_0$ to the $\InitE$ event, resolves all events within $w$ to obtain a set of final results.

\begin{figure}
  \begin{align*}
    \fbox{$\semlink\in\Ctx\rightarrow\Event\rightarrow\pset(\Value)$}                                                                                                                               \\
    \ctx_0\semlink\InitE\triangleq                         & \{\ctx_0\}                                                                                                                             \\
    \ctx_0\semlink\ReadE(\event,x)\triangleq               & \{v_+|\ctx_+\in\ctx_0\semlink\event\land\ctx_+(x)=v_+\}                                                                                \\
    \cup                                                   & \{v_+[\mu\ell.v_+/\ell]|\ctx_+\in\ctx_0\semlink E\land\ctx_+(x)=\mu\ell.v_+\}                                                          \\
    \ctx_0\semlink\CallE(\event,v)\triangleq               & \{v_+'|\langle\lambda x.e,\ctx_+\rangle\in\ctx_0\semlink E\land v_+\in\ctx_0\semlink v\land(x,v_+)\cons\Sigma\vdash e\Downarrow v_+'\} \\
    \cup                                                   & \{\CallE(\event_+,v_+)|\event_+\in\ctx_0\semlink\event\land v_+\in\ctx_0\semlink v\}                                                   \\
    \fbox{$\semlink\in\Ctx\rightarrow\Ctx\rightarrow\pset(\Ctx)$}                                                                                                                                   \\
    \ctx_0\semlink\bullet\triangleq                        & \{\bullet\}                                                                                                                            \\
    \ctx_0\semlink(x,\ell)\cons\ctx\triangleq              & \{(x,\ell)\cons\ctx_+|\ctx_+\in\ctx_0\semlink\ctx\}                                                                                    \\
    \ctx_0\semlink(x,w)\cons\ctx\triangleq                 & \{(x,w_+)\cons\ctx_+|w_+\in\ctx_0\semlink w\land\ctx_+\in\ctx_0\semlink\ctx\}                                                          \\
    \ctx_0\semlink[E]\triangleq                            & \{\ctx_+|\ctx_+\in\ctx_0\semlink\event\}\cup\{[\event_+]|\event_+\in\ctx_0\semlink\event\}                                             \\
    \fbox{$\semlink\in\Ctx\rightarrow\Value\rightarrow\pset(\Value)$}                                                                                                                               \\
    \ctx_0\semlink\langle\lambda x.e,\ctx\rangle\triangleq & \{\langle\lambda x.e,\ctx_+\rangle|\ctx_+\in\ctx_0\semlink\ctx\}                                                                       \\
    \fbox{$\semlink\in\Ctx\rightarrow\Walue\rightarrow\pset(\Walue)$}                                                                                                                               \\
    \ctx_0\semlink\mu\ell.v\triangleq                      & \{\mu\ell.v_+|v_+\in\ctx_0\semlink v\}
  \end{align*}
  \caption{Definition for concrete linking.}
  \label{fig:conclink}
\end{figure}

Concrete linking makes sense because of the following theorem.
First define:
\[\eval(e,\ctx)\triangleq\{v|\ctx\vdash e\Downarrow v\}\qquad\eval(e,\Sigma)\triangleq\bigcup_{\ctx\in\Sigma}\eval(e,\ctx)\qquad\ctx_0\semlink W\triangleq\bigcup_{w\in W}(\ctx_0\semlink w)\]
Then the following holds:
\begin{theorem}[Soundness of concrete linking]\normalfont
  Given $e\in\Expr,\ctx\in\Ctx,v\in\Value$,
  \[\forall\ctx_0\in\Ctx:\eval(e,\ctx_0\semlink\ctx)\subseteq\ctx_0\semlink\eval(e,\ctx)\]
\end{theorem}
