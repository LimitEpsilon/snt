%! TEX program = xelatex
% \PassOptionsToPackage{draft}{graphicx}
\documentclass{beamer}

\usepackage{kotex}
\usepackage{xcolor}
\usepackage{tcolorbox}

\usepackage{setspace} % setstretch

\usepackage{tabularray}
\usepackage{relsize}
\UseTblrLibrary{booktabs}
\UseTblrLibrary{counter}

%%% Math settings
\usepackage{amsmath,amsthm,amssymb,mathtools} % Before unicode-math
\usepackage[math-style=TeX,bold-style=TeX]{unicode-math}
\theoremstyle{definition}
\newtheorem*{definition*}{Definition}
\newtheorem*{lemma*}{Lemma}
\newtheorem*{theorem*}{Theorem}
\newtheorem*{corollary*}{Corollary}
\newtheorem*{notation*}{Notation}

%%% Font settings
\setmainfont{Libertinus Serif}
\setmathfont{Libertinus Math} % Before set*hangulfont
\setmathfont{TeX Gyre Pagella Math}[range={\lbrace,\rbrace,\checkmark},Scale=1.1]
\setmainhangulfont{Noto Serif CJK KR}
\setsanshangulfont[BoldFont={* Bold}]{KoPubWorld Dotum.ttf}

%%% PL constructs
\usepackage{galois}
\usepackage{ebproof}
%% InfRule command
\ExplSyntaxOn
\NewTblrEnviron{@ruleenv}
\SetTblrInner[@ruleenv]{belowsep=0pt,stretch=0}
\SetTblrOuter[@ruleenv]{baseline=b}
\NewDocumentEnvironment { InfRule } { m +b }
  {
    \begin{@ruleenv}{l}
      \smaller{\textsc{#1}} \\
      \begin{prooftree} #2 \end{prooftree}
    \end{@ruleenv}
  }
  {}
\ExplSyntaxOff

%% Modification for \infer
\ExplSyntaxOn
\int_new:N \g__ebproof_sublevel_int
\box_new:N \g__ebproof_substack_box
\seq_new:N \g__ebproof_substack_seq

\cs_new:Nn \__ebproof_clear_substack:
  {
    \int_gset:Nn \g__ebproof_sublevel_int { 0 }
    \hbox_gset:Nn \g__ebproof_substack_box { }
    \seq_gclear:N \g__ebproof_substack_seq
  }

\cs_new:Nn \__ebproof_subpush:N
  {
    \int_gincr:N \g__ebproof_sublevel_int
    \hbox_gset:Nn \g__ebproof_substack_box
      { \hbox_unpack:N \g__ebproof_substack_box \box_use:c { \__ebproof_box:N #1 } }
    \seq_gput_left:Nv \g__ebproof_substack_seq
      { \__ebproof_marks:N #1 }
  }

\cs_new:Nn \__ebproof_subpop:N
  {
    \int_compare:nNnTF { \g__ebproof_sublevel_int } > { 0 }
      {
        \int_gdecr:N \g__ebproof_sublevel_int
        \hbox_gset:Nn \g__ebproof_substack_box {
          \hbox_unpack:N \g__ebproof_substack_box
          \box_gset_to_last:N \g_tmpa_box
        }
        \box_set_eq_drop:cN { \__ebproof_box:N #1 } \g_tmpa_box
        \seq_gpop_left:NN \g__ebproof_substack_seq \l_tmpa_tl
        \tl_set_eq:cN { \__ebproof_marks:N #1 } \l_tmpa_tl
      }
      { \PackageError{ebproof}{Missing~premiss~in~a~proof~tree}{} \__ebproof_clear:N #1 }
  }

\cs_new:Nn \__ebproof_append_subvertical:NN
  {
    \bool_if:NTF \l__ebproof_updown_bool
      { \__ebproof_append_above:NN #1 #2 }
      { \__ebproof_append_below:NN #1 #2 }
  }

\cs_new:Nn \__ebproof_join_subvertical:n
  {
    \group_begin:
    \__ebproof_subpop:N \l__ebproof_a_box
    \prg_replicate:nn { #1 - 1 }
      {
        \__ebproof_subpop:N \l__ebproof_b_box
        \__ebproof_enlarge_conclusion:NN \l__ebproof_b_box \l__ebproof_a_box

        \__ebproof_make_vertical:Nnnn \l__ebproof_c_box
          { \__ebproof_mark:Nn \l__ebproof_b_box {axis} - \__ebproof_mark:Nn \l__ebproof_b_box {left} }
          { \__ebproof_mark:Nn \l__ebproof_b_box {right} - \__ebproof_mark:Nn \l__ebproof_b_box {left} }
          { \skip_vertical:N \l__ebproof_rule_margin_dim }
        \__ebproof_vcenter:N \l__ebproof_b_box
        \__ebproof_append_subvertical:NN \l__ebproof_a_box \l__ebproof_c_box

        \__ebproof_append_subvertical:NN \l__ebproof_a_box \l__ebproof_b_box
      }
    \__ebproof_push:N \l__ebproof_a_box
    \group_end:
  }

\cs_new:Nn \__ebproof_renew_statement:nnn
  {
    \exp_args:Nc \RenewDocumentCommand { ebproof#1 }{ #2 } { #3 }
    \seq_gput_right:Nn \g__ebproof_statements_seq { #1 }
  }
\__ebproof_renew_statement:nnn { infer } { O{} m O{} m }
  {
    \group_begin:
    \__ebproof_restore_statements:
    \keys_set_known:nnN { ebproof / rule~style } { #1 } \l_tmpa_tl
    \keys_set:nV { ebproof } \l_tmpa_tl
    \tl_set:Nn \l__ebproof_right_label_tl { #3 }

    \__ebproof_clear_substack:
    \clist_map_inline:nn { #2 }
      {
        \__ebproof_join_horizontal:n { ##1 }
        \__ebproof_pop:N \l__ebproof_a_box
        \__ebproof_subpush:N \l__ebproof_a_box
      }
    \__ebproof_join_subvertical:n { \clist_count:n { #2 } }

    \__ebproof_push_statement:n { #4 }
    \__ebproof_join_vertical:
    \group_end:
  }
\ExplSyntaxOff
\ebproofset{left label template=\textsc{[\inserttext]}}
\ebproofset{center=false}

% For simplebnf
\newfontfamily{\fallbackfont}{EB Garamond}
\DeclareTextFontCommand{\textfallback}{\fallbackfont}
\usepackage{newunicodechar}
\newunicodechar{⩴}{\textfallback{⩴}}

%%% Custom commands
\newcommand*{\vbar}{|}
\newcommand*{\finto}{\xrightarrow{\text{\textrm{fin}}}}
\newcommand*{\istype}{\mathrel{⩴}}
\newcommand*{\ortype}{\mathrel{|}}
\newcommand*{\cons}{:\hskip-2pt:}
\newcommand*{\pset}{\mathscr{P}}

\def\ovbarw{1.2mu}
\def\ovbarh{1}
\newcommand*{\ovbar}[1]{\mkern \ovbarw\overline{\mkern-\ovbarw{\smash{#1}\scalebox{1}[\ovbarh]{\vphantom{i}}}\mkern-\ovbarw}\mkern \ovbarw}
\newcommand*{\A}[1]{\overset{\,_{\mbox{\Large .}}}{#1}}
\newcommand*{\Abs}[1]{{#1}^{\#}}
\newcommand*{\Expr}{\text{Expr}}
\newcommand*{\ExprVar}{\text{Var}}
\newcommand*{\Module}{\text{Module}}
\newcommand*{\ModVar}{\text{ModVar}}
\newcommand*{\modid}{d}
\newcommand*{\Time}{\mathbb{T}}
\newcommand*{\ATime}{\A{\Time}}
\newcommand*{\Ctx}{\text{Env}}
\newcommand*{\ctx}{\sigma}
\newcommand*{\Value}{\text{Val}}
\newcommand*{\Mem}{\text{Mem}}
\newcommand*{\Left}{\text{Left}}
\newcommand*{\Right}{\text{Right}}
\newcommand*{\Sig}{\text{Sig}}
\newcommand*{\mem}{m}
\newcommand*{\AMem}{\A{\text{Mem}}}
\newcommand*{\State}{\text{State}}
\newcommand*{\AState}{\A{\text{State}}}
\newcommand*{\Result}{\text{Result}}
\newcommand*{\AResult}{\A{\text{Result}}}
\newcommand*{\Tick}{\text{Tick}}
\newcommand*{\lfp}{\mathsf{lfp}}
\newcommand*{\Step}{\mathsf{Step}}
\newcommand*{\semarrow}{\Downarrow}
\newcommand*{\asemarrow}{\A{\rightsquigarrow}}
\newcommand*{\synlink}{\rtimes}
\newcommand*{\semlink}{\mathbin{\rotatebox[origin=c]{180}{$\propto$}}}
\newcommand*{\link}[2]{{#1}\rtimes{#2}}
\newcommand*{\mt}{\mathsf{empty}}
\newcommand*{\valid}{\mathsf{valid}}
\newcommand*{\Path}{\text{Path}}
\newcommand*{\equivalent}{\sim}

\newcommand*{\dblplus}{{+\hskip-2pt+}}
\newcommand*{\project}{\text{\texttt{:>} }}
\newcommand*{\Exp}{\mathsf{Exp}}
\newcommand*{\Imp}{\mathsf{Imp}}
\newcommand*{\Fin}{\mathsf{Fin}}
\newcommand*{\Link}{\mathsf{Link}}
\newcommand*{\sembracket}[1]{\lBrack{#1}\rBrack}
\newcommand*{\fin}[2]{{#1}\xrightarrow{\text{fin}}{#2}}
\newcommand*{\addr}{\mathsf{addr}}
\newcommand*{\tick}{\mathsf{tick}}
\newcommand*{\modctx}{\mathsf{ctx}}
\newcommand*{\mapinject}[2]{{#2}[{#1}]}
\newcommand*{\inject}[2]{{#2}\langle{#1}\rangle}
\newcommand*{\deletepre}[2]{{#2}\overline{\doubleplus}{#1}}
\newcommand*{\deletemap}[2]{{#1}\overline{[{#2}]}}
\newcommand*{\delete}[2]{{#2}{\langle{#1}\rangle}^{-1}}
\newcommand*{\filter}{\mathsf{filter}}
\newcommand*{\Lete}{\mathtt{val}}
\newcommand*{\Letm}{\mathtt{mod}}

\newcommand*{\ValRel}[1]{V\sembracket{#1}}
\newcommand*{\ExprRel}[1]{E\sembracket{#1}}
\newcommand*{\CtxRel}[1]{\mathcal{C}\sembracket{#1}}
\newcommand*{\ModRel}[1]{\mathcal{M}\sembracket{#1}}
\newcommand*{\TyEnv}{\text{TyEnv}}
\newcommand*{\TyVar}{\text{TyVar}}
\newcommand*{\Type}{\text{Type}}
\newcommand*{\Subst}{\text{Subst}}
\newcommand*{\external}{\Gamma_{\text{ext}}}

%%%%%%%%%%%%%%%%%%%%%
%  Beamer Settings  %
%%%%%%%%%%%%%%%%%%%%%
\usetheme[numbering=fraction,progressbar=none]{metropolis}
\useoutertheme[subsection=false]{miniframes}
\usecolortheme{rose}

\setbeamertemplate{itemize item}[square]
\setbeamertemplate{itemize subitem}[triangle]
\setbeamertemplate{itemize subsubitem}[circle]
\setbeamertemplate{headline}{}

\usepackage{listings}
\definecolor{dkgreen}{rgb}{0,0.6,0}
\definecolor{ltblue}{rgb}{0,0.4,0.4}
\definecolor{dkviolet}{rgb}{0.3,0,0.5}

%%% lstlisting coq style (inspired from a file of Assia Mahboubi)
\lstdefinelanguage{Coq}{
    % To insert blank lines, write %
    escapechar=\%,
    % Anything betweeen $ becomes LaTeX math mode
    mathescape=true,
    % Comments may or not include Latex commands
    texcl=false, 
    % Vernacular commands
    morekeywords=[1]{Section, Module, End, Require, Import, Export, Include,
        Variable, Variables, Parameter, Parameters, Axiom, Hypothesis,
        Hypotheses, Notation, Local, Tactic, Reserved, Scope, Open, Close,
        Bind, Delimit, Definition, Let, Ltac, Fixpoint, CoFixpoint, Add,
        Morphism, Relation, Implicit, Arguments, Unset, Contextual,
        Strict, Prenex, Implicits, Inductive, CoInductive, Record,
        Structure, Canonical, Coercion, Context, Class, Global, Instance,
        Program, Infix, Theorem, Lemma, Corollary, Proposition, Fact,
        Remark, Example, Proof, Goal, Save, Qed, Defined, Hint, Resolve,
        Rewrite, View, Search, Show, Print, Printing, All, Eval, Check,
        Projections, inside, outside, Def},
    % Gallina
    morekeywords=[2]{forall, exists, exists2, fun, fix, cofix, struct,
        match, with, end, as, in, return, let, if, is, then, else, for, of,
        nosimpl, when},
    % Sorts
    morekeywords=[3]{Type, Prop, Set, true, false, option},
    % Various tactics, some are std Coq subsumed by ssr, for the manual purpose
    morekeywords=[4]{pose, set, move, case, elim, apply, clear, hnf,
        intro, intros, generalize, rename, pattern, after, destruct,
        induction, using, refine, inversion, injection, rewrite, congr,
        unlock, compute, ring, field, fourier, replace, fold, unfold,
        change, cutrewrite, simpl, have, suff, wlog, suffices, without,
        loss, nat_norm, assert, cut, trivial, revert, bool_congr, nat_congr,
        symmetry, transitivity, auto, split, left, right, autorewrite},
    % Terminators
    morekeywords=[5]{by, done, exact, reflexivity, tauto, romega, omega,
        assumption, solve, contradiction, discriminate},
    % Control
    morekeywords=[6]{do, last, first, try, idtac, repeat},
    % Comments delimiters, we do turn this off for the manual
    morecomment=[s]{(*}{*)},
    % Spaces are not displayed as a special character
    showstringspaces=false,
    % String delimiters
    morestring=[b]",
    morestring=[d],
    % Size of tabulations
    tabsize=3,
    % Enables ASCII chars 128 to 255
    extendedchars=false,
    % Case sensitivity
    sensitive=true,
    % Automatic breaking of long lines
    breaklines=false,
    % Default style fors listings
    basicstyle=\footnotesize,
    % Position of captions is bottom
    captionpos=b,
    % flexible columns
    columns=[l]flexible,
    % Style for (listings') identifiers
    identifierstyle={\ttfamily\color{black}},
    % Style for declaration keywords
    keywordstyle=[1]{\ttfamily\color{dkviolet}},
    % Style for gallina keywords
    keywordstyle=[2]{\ttfamily\color{dkgreen}},
    % Style for sorts keywords
    keywordstyle=[3]{\ttfamily\color{ltblue}},
    % Style for tactics keywords
    keywordstyle=[4]{\ttfamily\color{dkblue}},
    % Style for terminators keywords
    keywordstyle=[5]{\ttfamily\color{dkred}},
    %Style for iterators
    %keywordstyle=[6]{\ttfamily\color{dkpink}},
    % Style for strings
    stringstyle=\ttfamily,
    % Style for comments
    commentstyle={\ttfamily\color{dkgreen}},
    %moredelim=**[is][\ttfamily\color{red}]{/&}{&/},
    literate=
    {\\forall}{{\color{dkgreen}{$\forall\;$}}}1
    {\\exists}{{$\exists\;$}}1
    {<-}{{$\leftarrow\;$}}1
    {=>}{{$\Rightarrow\;$}}1
    {==}{{\code{==}\;}}1
    {==>}{{\code{==>}\;}}1
    %    {:>}{{\code{:>}\;}}1
    {->}{{$\rightarrow\;$}}1
    {<->}{{$\leftrightarrow\;$}}1
    {<==}{{$\leq\;$}}1
    {\#}{{$^\star$}}1 
    {\\o}{{$\circ\;$}}1 
    {\@}{{$\cdot$}}1 
    {\/\\}{{$\wedge\;$}}1
    {\\\/}{{$\vee\;$}}1
    {++}{{\code{++}}}1
    {~}{{\ }}1
    {\@\@}{{$@$}}1
    {\\mapsto}{{$\mapsto\;$}}1
    {\\hline}{{\rule{\linewidth}{0.5pt}}}1
    %
}[keywords,comments,strings]

\usepackage{pgfplots}
\usetikzlibrary{shapes,arrows}

\usepackage{adjustbox}
\newcommand{\cfbox}[1]{\adjustbox{cfbox=#1}}

\title{모듈이 있는 언어에서 타입 안전성 증명}
\subtitle{Logical Relations and Subtyping}
\author{이준협}
\date{2024년 1월 12일}
\institute{ROPAS{@}SNU}

\titlegraphic{%
  \begin{tikzpicture}[overlay,remember picture]
    \node at (current page.145) [xshift=3em, yshift=-1.3em] {
      \includegraphics[width=3em]{snu-symbol.png}
    };
    \node at (current page.35) [xshift=-3em, yshift=-1.3em] {
      \includegraphics[width=2.5em]{ropas-symbol.png}
    };
  \end{tikzpicture}%
}

\begin{document}
\maketitle
\begin{frame}[c,fragile]
  \frametitle{동기: 타입의 안전성 증명}
  \begin{itemize}
    \item 따로분석을 타입분석의 경우로 확장하고 싶다.
    \item 그 전, \underline{언어에 맞는 타입규칙}을 만들고, \underline{안전성 증명}을 해야한다.
  \end{itemize}
\end{frame}
\begin{frame}[c, fragile]
  \frametitle{문제: 의미구조와 아래타입}
  \begin{enumerate}
    \item 의미구조가 초기 상태와 최종 결과를 바로 연관시킨다. (큰 보폭)\\
          $\rightarrow$ 직관적이지만\dots\\
          $\rightarrow$ Progress \& Preservation 스타일의 증명 어려움
    \item 풍부한 타입규칙을 위해 \underline{아래타입} (subtype) 도입\\
          $\rightarrow$ 예시: \texttt{x}만 쓰는 함수에게 \texttt{x,y}를 내보내는 모듈이 주어져도 안전\\
          $\rightarrow$ 타입 안전성 증명을 어떻게 해야할까?
  \end{enumerate}
\end{frame}
\begin{frame}[c, fragile]
  \frametitle{해답: Logical Relations}
  \begin{itemize}
    \item Simply Typed $\lambda$의 예시: $\tau,A,B\in\textsf{Type},e\in\textsf{Expr},v\in\textsf{Val}$\\
          \fbox{$R_\tau\subseteq\textsf{Val}^n$}는 다음과 같을 때 $n$\textbf{-ary logical relation}이다:\\
          \begin{center}
            \begin{tabular}{|l|}
              \hline
              \textbf{모든} $(v_1,\dots,v_n)\in R_A$에 대해                                              \\
              \textbf{어떤} $(v_1',\dots,v_n')\in R_B$가 존재해 $e_i[v_i/x_i]\Downarrow v_i'$\textbf{이면이} \\
              $(\lambda x_1.e_1,\dots,\lambda x_n.e_n)\in R_{A\rightarrow B}$\textbf{이다.}           \\
              \hline
            \end{tabular}
          \end{center}
    \item ``연관된'' 입력을 ``연관된'' 출력으로 보내는 함수들의 집합.
    \item $R\in\textsf{Type}\rightarrow \mathcal{P}(\textsf{Val}^n)$: 타입의 ``구체화''로 이해
  \end{itemize}
\end{frame}
\begin{frame}[c, fragile]
  \frametitle{타입 안전성}
  \begin{itemize}
    \item $n=1$인 경우(unary) 사용.\\
          목표: \fbox{$\vdash e:\tau\Rightarrow\exists v\text{ s.t }e\Downarrow v\text{ and } v\in R_\tau$}
    \item \textbf{겉모습}만 보고 추론된 타입이 실제로 프로그램의 \textbf{의미}를 포섭한다
  \end{itemize}
\end{frame}
\begin{frame}[c, fragile]
  \frametitle{모듈이 있는 언어: 겉모습}
  \begin{figure}[htb]
    \centering
    \footnotesize
    \begin{tabular}{rrcll}
      Identifiers & $x$ & $\in$         & $\ExprVar$                                                     \\
      Expression  & $e$ & $\rightarrow$ & $x$ $\vbar$ $\lambda x.e$ $\vbar$ $e$ $e$ & $\lambda$-calculus \\
                  &     & $\vbar$       & $\link{e}{e}$                             & linked expression  \\
                  &     & $\vbar$       & $\varepsilon$                             & empty module       \\
                  &     & $\vbar$       & $x=e$ ; $e$                               & binding
    \end{tabular}
  \end{figure}
\end{frame}
\begin{frame}[c, fragile]
  \frametitle{의미공간}
  \begin{figure}[h!]
    \centering
    \footnotesize
    \begin{tabular}{rrcll}
      Environment & $\ctx$ & $\in$         & $\Ctx$                                                                       \\
      Value       & $v$    & $\in$         & $\Value \triangleq\Ctx+\ExprVar\times\Expr\times\Ctx$                        \\
      Environment & $\ctx$ & $\rightarrow$ & $\bullet$                                             & empty stack          \\
                  &        & $\vbar$       & $(x,v)\cons\ctx$                                      & value binding        \\
      Value       & $v$    & $\rightarrow$ & $\ctx$                                                & exported environment \\
                  &        & $\vbar$       & $\langle \lambda x.e, \ctx \rangle$                   & closure
    \end{tabular}
  \end{figure}
\end{frame}
\begin{frame}[c,fragile]
  \frametitle{큰 보폭 실행의미}
  \begin{figure}[h!]
    \footnotesize
    \begin{flushright}
      \fbox{$(e,\ctx)\Downarrow v$}
    \end{flushright}
    \centering
    \vspace{0pt} % -0.75em}
    \[
      \begin{InfRule}{Id}
        \hypo{v=\ctx(x)}
        \infer1{
          (x, \ctx)
          \Downarrow
          v
        }

      \end{InfRule}\qquad
      \begin{InfRule}{Fn}
        \infer0{
          (\lambda x.e, \ctx)
          \Downarrow
          \langle\lambda x.e, \ctx\rangle
        }
      \end{InfRule}
    \]

    \[
      \begin{InfRule}{App}
        \hypo{
          (e_{1}, \ctx)
          \Downarrow
          \langle\lambda x.e_{\lambda}, \ctx_{\lambda}\rangle
        }
        \hypo{
          (e_{2}, \ctx)
          \Downarrow
          v
        }
        \hypo{
          (e_{\lambda}, (x, v)\cons \ctx_{\lambda})
          \Downarrow
          v'
        }
        \infer3{
          (e_{1}\:e_{2}, \ctx)
          \Downarrow
          v'
        }
      \end{InfRule}
    \]

    \[
      \begin{InfRule}{Link}
        \hypo{
          (e_{1}, \ctx)
          \Downarrow
          \ctx'
        }
        \hypo{
          (e_{2}, \ctx')
          \Downarrow
          v
        }
        \infer2{
          (\link{e_{1}}{e_{2}}, \ctx)
          \Downarrow
          v
        }
      \end{InfRule}\qquad
      \begin{InfRule}{Empty}
        \infer0{
          (\varepsilon, \ctx)
          \Downarrow
          \bullet
        }
      \end{InfRule}\qquad
      \begin{InfRule}{Bind}
        \hypo{
          (e_{1}, \ctx)
          \Downarrow
          v
        }
        \hypo{
          (e_{2}, (x, v)\cons \ctx)
          \Downarrow
          \ctx'
        }
        \infer2{
          (x=e_1; e_2, \ctx)
          \Downarrow
          (x,v)\cons\ctx'
        }
      \end{InfRule}
    \]
  \end{figure}
\end{frame}
\begin{frame}[c,fragile]
  \frametitle{타입}
  \begin{figure}[h!]
    \centering
    \footnotesize
    \begin{tabular}{rrcll}
      Types            & $\tau$   & $\rightarrow$ & $\Gamma$               & module type       \\
                       &          & $\vbar$       & $\tau\rightarrow\tau$  & function type     \\
      Type Environment & $\Gamma$ & $\rightarrow$ & $\bullet$              & empty environment \\
                       &          & $\vbar$       & $(x,\tau)\cons \Gamma$ & type binding
    \end{tabular}
  \end{figure}
\end{frame}
\begin{frame}[c,fragile]
  \frametitle{타입규칙}
  \begin{figure}[h!]
    \footnotesize
    \begin{flushright}
      \fbox{$\Gamma\vdash e:\tau$}
    \end{flushright}
    \centering
    \footnotesize
    \[
      \begin{InfRule}{T-Id}
        \hypo{\tau=\Gamma(x)}
        \infer1{
          \Gamma\vdash x:\tau
        }
      \end{InfRule}\qquad
      \begin{InfRule}{T-Fn}
        \hypo{(x,\tau_1)\cons\Gamma\vdash e:\tau_2}
        \infer1{
          \Gamma\vdash\lambda x.e:\tau_1\rightarrow\tau_2
        }
      \end{InfRule}\qquad
      \begin{InfRule}{T-App}
        \hypo{
          \Gamma\vdash e_1:\tau_1\rightarrow\tau
        }
        \hypo{
          \Gamma\vdash e_2:\tau_2
        }
        \hypo{
          \tau_1\ge\tau_2
        }
        \infer3{
          \Gamma\vdash e_{1}\:e_{2}:\tau
        }
      \end{InfRule}
    \]

    \[
      \begin{InfRule}{T-Link}
        \hypo{
          \Gamma\vdash e_1:\Gamma_1
        }
        \hypo{
          \Gamma_1\vdash e_2:\tau_2
        }
        \infer2{
          \Gamma\vdash\link{e_{1}}{e_{2}}:\tau_2
        }
      \end{InfRule}\qquad
      \begin{InfRule}{T-Mt}
        \infer0{
          \Gamma\vdash\varepsilon:\bullet
        }
      \end{InfRule}\qquad
      \begin{InfRule}{T-Bind}
        \hypo{
          \Gamma\vdash e_1:\tau_1
        }
        \hypo{
          (x, \tau_1)\cons\Gamma\vdash e_2:\Gamma_2
        }
        \infer2{
          \Gamma\vdash x=e_1;e_2:(x,\tau_1)\cons\Gamma_2
        }
      \end{InfRule}
    \]
  \end{figure}
\end{frame}
\begin{frame}[c,fragile]
  \frametitle{아래타입}
  \begin{figure}[h!]
    \footnotesize
    \begin{flushright}
      \fbox{$\tau\ge\tau$}
    \end{flushright}
    \centering
    \footnotesize
    \vspace{0pt} % -0.75em}
    \[
      \begin{InfRule}{Nil}
        \infer0{
          \bullet\ge\bullet
        }
      \end{InfRule}\qquad
      \begin{InfRule}{ConsFree}
        \hypo{
          x\not\in\mathsf{dom}(\Gamma)
        }
        \hypo{
          \Gamma\ge\Gamma'
        }
        \infer2{
          \Gamma\ge(x,\tau)\cons\Gamma'
        }
      \end{InfRule}\qquad
      \begin{InfRule}{ConsBound}
        \hypo{
          \Gamma(x)\ge\tau
        }
        \hypo{
          \Gamma-x\ge\Gamma'
        }
        \infer2{
          \Gamma\ge(x,\tau)\cons\Gamma'
        }
      \end{InfRule}
    \]

    \[
      \begin{InfRule}{Arrow}
        \hypo{
          \tau_2\ge\tau_1
        }
        \hypo{
          \tau_1'\ge\tau_2'
        }
        \infer2{
          \tau_1\rightarrow\tau_1'\ge
          \tau_2\rightarrow\tau_2'
        }
      \end{InfRule}
    \]
  \end{figure}
  \begin{itemize}
    \item $\tau_1\ge\tau_2$: $\tau_1$ 자리에 $\tau_2$를 넣어도 안전하다.
    \item Reflexivity, Transitivity 만족함을 증명할 수 있음.
  \end{itemize}
\end{frame}
\begin{frame}[c,fragile]
  \frametitle{(Unary) Logical Relation}
  \begin{figure}
    \footnotesize
    \hspace{-3em}
    \begin{tabular}{rclr}
      \textbf{Value Relation}            &              &                                                                                                        & \fbox{$\ValRel{\tau}$}  \\
      $\ValRel{\bullet}$                 & $\triangleq$ & $\Ctx$                                                                                                                           \\
      $\ValRel{(x,\tau)\cons\Gamma}$     & $\triangleq$ & $\{\ctx|\ctx(x)\in\ValRel{\tau}\land\ctx-x\in\ValRel{\Gamma-x}\}$                                                                \\
      $\ValRel{\tau_1\rightarrow\tau_2}$ & $\triangleq$ & $\{\langle\lambda x.e,\ctx\rangle|\forall v\in\ValRel{\tau_1}:(e,(x,v)\cons\ctx)\in\ExprRel{\tau_2}\}$                           \\
      \\
      \textbf{Expression Relation}       &              &                                                                                                        & \fbox{$\ExprRel{\tau}$} \\
      $\ExprRel{\tau}$                   & $\triangleq$ & $\{(e,\ctx)|\exists v\in\ValRel{\tau}:(e,\ctx)\semarrow v\}$
    \end{tabular}
  \end{figure}
  \begin{itemize}
    \item 타입에 대한 귀납법으로 정의됨(well-founded)
    \item $\bullet$에 대한 해석: $\ValRel{\bullet}=\Ctx$ vs $\ValRel{\bullet}=\{\bullet\}$
  \end{itemize}
\end{frame}
\begin{frame}[c,fragile]
  \frametitle{타입 안전성}
  \begin{figure}
    \footnotesize
    \begin{tabular}{rclr}
      \textbf{Semantic Typing} &              &                                                           & \fbox{$\Gamma\vDash e:\tau$} \\
      $\Gamma\vDash e:\tau$    & $\triangleq$ & $\forall\ctx\in\ValRel{\Gamma}:(e,\ctx)\in\ExprRel{\tau}$
    \end{tabular}
  \end{figure}
  \begin{theorem*}[Type Safety]
    \[\Gamma\vdash e:\tau\Rightarrow\Gamma\vDash e:\tau\]
  \end{theorem*}
  \begin{itemize}
    \item 타입규칙에 대한 귀납법으로 증명\\
          + 보조정리 2개
  \end{itemize}
\end{frame}
\begin{frame}[c,fragile]
  \frametitle{보조정리: 타입규칙이 잘 정의됨}
  \begin{itemize}
    \item $\vdash$를 정의하는 규칙들은 $\vDash$의 경우에도 성립한다
    \item 즉, $\vdash$는 잘 정의되었다.
  \end{itemize}
  예시:
  \begin{figure}[h!]
    \footnotesize
    \centering
    \footnotesize
    \[
      \begin{InfRule}{T-Id}
        \hypo{\tau=\Gamma(x)}
        \infer1{
          \Gamma\vDash x:\tau
        }
      \end{InfRule}\qquad
      \begin{InfRule}{T-Fn}
        \hypo{(x,\tau_1)\cons\Gamma\vDash e:\tau_2}
        \infer1{
          \Gamma\vDash\lambda x.e:\tau_1\rightarrow\tau_2
        }
      \end{InfRule}\qquad
      \begin{InfRule}{T-App}
        \hypo{
          \Gamma\vDash e_1:\tau_1\rightarrow\tau
        }
        \hypo{
          \Gamma\vDash e_2:\tau_2
        }
        \hypo{
          \tau_1\ge\tau_2
        }
        \infer3{
          \Gamma\vDash e_{1}\:e_{2}:\tau
        }
      \end{InfRule}
    \]
  \end{figure}
\end{frame}
\begin{frame}[c,fragile]
  \frametitle{보조정리: 아래타입이 잘 정의됨}
  \begin{lemma*}[Subtyping is well-defined]
    \[\tau_1\ge\tau_2\Rightarrow\ValRel{\tau_1}\supseteq\ValRel{\tau_2}\]
  \end{lemma*}
  \begin{itemize}
    \item 성립하는 이유: $\ValRel{\bullet}=\Ctx$로 정의했기 때문
    \item 모든 $\Gamma$에 대해, $\bullet\ge\Gamma$임.
  \end{itemize}
\end{frame}
\begin{frame}[c,fragile]
  \frametitle{만약 $\ValRel{\bullet}=\{\bullet\}$이었다면?}
  \begin{itemize}
    \item 아래타입 관계를 해석할 수 없게 됨
    \item $\sigma\in\ValRel{\Gamma}$의 의미가 강력해짐:\\
          $\Gamma$이상을 내보내는$\rightarrow$정확히 $\Gamma$만큼 내보내는
    \item 예시: $\sigma_1\in\ValRel{\Gamma_1}\land\sigma_2\in\ValRel{\Gamma_2}\Rightarrow\sigma_1\dblplus\sigma_2\in\ValRel{\Gamma_1\Abs{\dblplus}\Gamma_2}$인 $\Abs\dblplus$?\\
          \begin{enumerate}
            \item$\ValRel{\bullet}=\Ctx$일 때: $\Gamma_1\Abs\dblplus\Gamma_2\triangleq\Gamma_1$
            \item$\ValRel{\bullet}=\{\bullet\}$일 때: $\Gamma_1\Abs\dblplus\Gamma_2\triangleq\Gamma_1\dblplus\Gamma_2$
          \end{enumerate}
  \end{itemize}
\end{frame}
\begin{frame}[c,fragile]
  \frametitle{재귀적 모듈을 지원하려면?}
  \begin{center}
    {\footnotesize
      \[
        \begin{InfRule}{T-Bind}
          \hypo{
            (x,\tau_1)\cons\Gamma\vdash e_1:\tau_1
          }
          \hypo{
            (x, \tau_1)\cons\Gamma\vdash e_2:\Gamma_2
          }
          \infer2{
            \Gamma\vdash x=e_1;e_2:(x,\tau_1)\cons\Gamma_2
          }
        \end{InfRule}
      \]이면?
      
      \[
        \begin{InfRule}{T-Bind}
          \hypo{
            (x,\tau)\cons\Gamma\vdash x:\tau
          }
          \hypo{
            (x, \tau)\cons\Gamma\vdash \varepsilon:\bullet
          }
          \infer2{
            \Gamma\vdash x=x;\varepsilon:(x,\tau)\cons\bullet
          }
        \end{InfRule}
      \]이 모든 $\Gamma$와 $\tau$에 대해 성립}
  \end{center}

  \begin{itemize}
    \item ``어떤 위치''에서 재귀적 정의된 값이 쓰이는지 중요
    \begin{itemize}
      \item ``A practical mode system for recursive definitions'', POPL 2021
    \end{itemize}
  \end{itemize}
\end{frame}
\end{document}
%%% Local Variables:
%%% coding: utf-8
%%% mode: latex
%%% TeX-engine: xetex
%%% End:
